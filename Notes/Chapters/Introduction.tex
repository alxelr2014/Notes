\chapter{Introduction}
An \textbf{agent} is an entity that \textit{perceives} and \textit{acts}. An agent is a function from percept histories to actions:
\begin{equation}
    f : \mathcal{P}^\ast \to \mathcal{A}
\end{equation}
A rational agent chooses whichever action maximizes the expected value of performance measure given the precept sequence to date.

Types of agents
\begin{description}
    \item[Reflex agent] choose action based on current percept (and maybe memory). Act oh how the world is.
    \item[Goal-based agent] decision based of hypothesized consequence of action. Act on how the world would be.
    \item[Utility-based agent] Trade off multiple goals. Act on how the world will likely be.    
\end{description}