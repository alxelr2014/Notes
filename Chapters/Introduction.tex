\chapter{Introduction}

There four components of economy that we will investigate:
\begin{enumerate}
    \item Households
    \item Product market
    \item Labor market
    \item Financial intermediaries
\end{enumerate}


There are two aspects of financial analysis:
\begin{enumerate}
    \item valuating assets
    \item managing assets : Objective + Valuation = Decision
\end{enumerate}


The two factors that make finance interesting are time and risk.
Six principle of finance:
\begin{enumerate}
    \item No free lunches.
    \item Other things equal, individuals:
          \begin{itemize}
              \item Want more money than less (non-satiation)
              \item Prefer money now to later (impatience)
              \item Prefer to avoid risk (risk aversion)
          \end{itemize}
    \item All agents act to further their own self-interest
    \item Financial market prices shift to equalize supply and demand
    \item Financial markets are highly adaptive and competitive
    \item Risk-sharing and frictions are central to financial innovation
\end{enumerate}


\section{Corporate finance}
To carry on business, a corporation needs \textbf{assets} that needs to be paid for. Corporations pays for these assets by selling claims on them, which are called \textbf{financial assets} or \textbf{securities}. An \textit{investment decision} involves purchase of assets, managing assets , rist management and etc. A \textit{Financial decision} involves sale of securities, paying those obligation, paying the shareholde and etc. 
\subsection{Investment decision}
Capital investments generate future cash returns that might last for a short amount of time or decades. Most investment decisions are as small as, for example, buying a new equipment. Corporation prepare an annual \textbf{capital budget}, listing the major projects approved for investment. These investment decision are called \textbf{capital budgeting} or \textbf{capital expenditure (CAPEX)}.
\subsection{Financing decision}
A corporation can raise money from lender or from shareholders. If it borrows from the lender then it has to pay back the debt with interest. When the money comes from shareholders, they a get a share of stock and therefore get a fraction of future profits. The shareholders are \textit{equity investors}, who contribute \textit{equity financing}. The choice between debt and equity financing is called the \textbf{capital structure} decision. 

The decision to pay dividends or share buyback is called the \textit{payout decision}.

\textit{Market capitalization} is equal to number share outstanding times the price of each share.

\section{Corporation}
A corporation is legal entity. A corporation is owned by its shareholde but is legally distince from them. Therefore, the shareholders have \textbf{limited liability}. 

\begin{description}
    \item[sole proprietorship] usually a small business where the owner face unlimited liability.
    \item[partnership] A small business that have many owners each may face unlimited liability, or in case of \textit{limited liability partnership} or \textit{limited liability companies} all limited liability.
\end{description}

Larger companies with hundreds of thousands shares and many shareholders, it is impossible to manage directly. Therefore, they need to have professional manager to control the business. This is the \textit{seperation of ownership and control}. The goal of managers is to make the shareholders richer and a common way to do this is  maximizing the market value of the company. 

An investment is prefered to dividend if the rate return of the investment is higher than the rate of return dividend when invested by the shareholders. The minimum rate of return is called \textit{hurdle rate} or \textit{cost of capital}. It is an \textbf{opportunity cost of capital} as it depends on the investment opportunities available to investors in the market.