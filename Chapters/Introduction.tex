\chapter{Introduction}
\begin{definition}
    A statiscal model is parametric if it can be determined using a set of parameters. A statiscal model that can not be adquately determined by a set of parameters is called a non-parametric model. Models that have both components are called semi-parametric. 
\end{definition}

\begin{definition}
    A parametric model is identifiable if 
    \begin{equation}
        \theta_1 \neq \theta_2 \implies P_{\theta_1} \neq P_{\theta_2}
    \end{equation}
\end{definition}

\begin{definition}
    A statistic is a function from sample space \(\calX\) to some spave of values, \(\calT\).
\end{definition}

\begin{definition}
    Any parametric model that either
    \begin{enumerate}
        \item All of \(P_{\theta}\) are continuous with denisities \(\func{p}{x,\theta}\)
        \item All of \(P_{\theta}\) are discrete with frequency functions \(\func{p}{x,\theta}\), and the support set \(\set{x_1, x_2, \dots} \equiv \set<x>{\func{p}{x,\theta} > 0}\)
    \end{enumerate}
    are called regular parametric models.
\end{definition}

