\chapter{Introduction}
Cryptography is the art and science of encrypting and decrypting a message. 
\section{Symmetric cipher}
A symmetric cipher scheme \(\Pi\) can be viewed as a triplet \(\bracket{\Gen, \Enc , \Dec}\) of algorithms. Suppose \(\Messages\) be the set of all possible messages and \(\Keys\) be the set of all keys. \(\Gen\) chooses a key \(k \in \Keys\) and then \(\Enc : \Messages \times \Keys \to \Ciphers\) encrypts the message \(m\) with key \(k\) and returns the cipher \(c\). Lastly, \(\Dec: \Ciphers \times \Keys \to \Messages \cup \but\) decrypts the cipher \(c\) with key \(k\) and returns either a message or an error, denoted as \(\but\). Without loss of generality we can assume that \(\Gen\) picks \(k\) uniformly from \(\Keys\). Futhermore, \(\Enc\) can be randomized, however \(\Dec\) is deterministic and for every message \(m\) and key \(k\) we must have 
\begin{equation*}
    \func{\Dec_k}{\func{\Enc_k}{m}} = m
\end{equation*}

\section{Kerckhoff's principle}
Kerckhoff's principle assumes the following for every encryption scheme 
\begin{enumerate}
    \item The encryption and decryption is known to everyone.
    \item The security of the scheme is only dependent on the key.
\end{enumerate}


\section{Attacks}
Some possible attacks include (in increasing power)
\begin{description}
    \item [Cihpertext only] Attacker only knows the ciphertexts.
    \item [Known-plaintext] Attacker knows one or more plaintext/ciphertext generated by the key.
    \item [Chosen-plaintext] Attacker can obtain encryption of plaintexts of his choice.
    \item [Chosen-ciphertext] Attacker can obtain decryption of ciphertexts of his choise.
\end{description}