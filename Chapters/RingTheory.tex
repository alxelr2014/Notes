\chapter{Ring Theory}
\begin{definition}
    A non-empty set \(R\) is an \textbf{associative ring} if in \(R\) there are defined two operations \((+, \cdot)\) such that for all \(a,b,c \in R\)
    \begin{enumerate}
        \item \(R\) is closed under \(+\).
        \item \(+\) is commutative.
        \item \(+\) is associative.
        \item There exists an element \(0 \in R\), which is the identity element of \(+\).
        \item For each \(a\), there exists \(b\) such that \(a + b = b + a = 0\).
        \item \(R\) is closed under \(\cdot\).
        \item \(\cdot\) is associative.
        \item \(\cdot\) is distributive over \(+\). That is, \(a\cdot(b+c) = a\cdot b + a \cdot c\)  and \((a + b) \cdot c = a \cdot c + b \cdot c\).
    \end{enumerate}
    If there is an element \(1 \in R\) such that \(a \cdot 1 = 1 \cdot a = a\) for all \(a \in R\), \(R\) is said to be a \textbf{ring with unity}. If \(\cdot\) is commutative, \(R\) is said to be a \textbf{commutative ring}. If the non-zero elements of \(R\) form an abelian group under \(\cdot\), \(R\) is said to be a \textbf{field}.
\end{definition}

\begin{example}
    Consider the \textbf{real quaternions}, \(Q = \set<\alpha_0 + \alpha_1 i + \alpha_2 j + \alpha_3 k>{\alpha_1,\alpha_2,\alpha_3 \in \Reals}\) with multiplication rules; \(i^2 = j^2 = k^2 = ijk = 1\), \(ij = -ji = k\), \(jk = -kj = i\), \(ki = -ik = j\). Then, \(Q\) is a non-commutative ring and its non-zero elements form a non-commutative group under multiplication.
\end{example}
\section{Some special classed of ring}
\begin{definition}
    If \(R\) is a commutative ring, then a non-zero element \(a \in R\) is a \textbf{zero-divisor} if there exists another non-zero element \(b\) such that \(ab = 0\).
\end{definition}
\begin{definition}
    A commutative ring is an \textbf{integral domain} if it has no zero-divisors.
\end{definition}
\begin{definition}
    A ring in which all non-zero elements form a group under multiplication is called a \textbf{division ring} or \textbf{skew-field}.
\end{definition}

\begin{definition}
    A field is a commutative division ring.
\end{definition}

\begin{lemma}
    for all \(a,b,c \in R\)
    \begin{enumerate}
        \item \(a \cdot 0 = 0 \cdot a = 0\).
        \item \(a(-b) = (-a)b = -ab\).
        \item \((-a)(-b) = ab\).
    \end{enumerate}
    If \(1 \in R\)
    \begin{enumerate}
        \item \((-1)a = -a\).
        \item \((-1)(-1) = 1\).
    \end{enumerate}
\end{lemma}

\begin{lemma}
    A finite integral domain is a field.
\end{lemma}

\begin{corollary}
    If \(p\) is a prime, \(\Integers_p\) is a field.
\end{corollary}

\begin{definition}
    An integral domain \(D\) is said to be of characteristic \(0\) if the relation \(ma  = 0\) where \(a \neq 0\) and \(m \in \Integers\) holds only if \(m = 0\). \(D\) is of finite characteristic if there exists a positive integer \(m\) such that for all \(a \in D\), \(ma = 0\). The characteristic of \(D\) is the samllest such integer. We say that a ring \(R\) has \textbf{\(n\)-torsion} if there exists \(a \neq 0\) in \(R\) such that \(na = 0\) and \(ma \neq 0\) for \(0 < m < n\).
\end{definition}

\section{Homomorphisms}
\begin{definition}
    A mapping \(\phi\) from the ring \(R\) into the ring \(R'\) is a homomorphism if 
    \begin{equation*}
        \func{\phi}{a + b} = \func{\phi}{a} + \func{\phi}{b}
    \end{equation*}
    and 
    \begin{equation*}
        \func{\phi}{ab} = \func{\phi}{a} \func{\phi}{b}
    \end{equation*}
    for all \(a,b\in R\).
\end{definition}

\begin{lemma}
   If \(\phi : R \to R'\) is a homomorphism
   \begin{enumerate}
    \item \(\func{\phi}{0} = 0\).
    \item \(\func{\phi}{-a}= - \func{\phi}{a}\).
   \end{enumerate}
\end{lemma}

\begin{definition}
    Suppose \(\phi:R \to R'\) is a homomorphism. The kernel \(\func{I}{\phi} = \set{ a \in R}{\func{\phi}{a} = 0}\).
\end{definition}

\begin{lemma}
    If \(\phi: R \to R'\) is a homomorphism
    \begin{enumerate}
        \item \(\func{I}{\phi}\) is a subgroup of \(R\) under addition.
        \item If \(a \in \func{I}{\phi}\) and \(r \in R\), then \(ra, ar \func{I}{\phi}\).
    \end{enumerate}
\end{lemma}

\begin{definition}
    A homomorphism \(R\) into \(R'\) is an isomorphism of it is one-to-one. \(R\) and \(R'\) are isomorphic if there is an onto isomorphism between them.
\end{definition}

\begin{lemma}
    The homomorphism \(\phi: R \to R'\) is an isomorphism if and only if \(\func{I}{\phi} = \set{0}\).
\end{lemma}

\section{Ideals and quotient ring}
\begin{definition}
    A non-empty subset \(U\) of \(R\) is a \textbf{two-sided ideal} of \(R\) if 
    \begin{enumerate}
        \item \(U\) is a subgroup of \(R\) under addition.
        \item For all \(u \in U\) and \(r \in R\), \(ur , ru \in U\).
    \end{enumerate}
\end{definition}

\(R/U\) is the set of distinct cosets of \(U\) in \(R\) as a group under addition. \(R/U\) is a ring with \((a + U)(b + U) = ab + U\).

If \(R\) is commutative or it has unit element, then \(R/U\) is commutative or has unit element. But the converse is not necessarily true.
--- give an example.

\begin{lemma}
    If \(U\) is an ideal of the ring \(R\). then \(R/U\) is a ring and is a homomorphic image of \(R\).
\end{lemma}

\begin{theorem}
    Suppose \(\phi:R \to R"\) is a homomorphism and let \(U = \func{I}{\phi}\). Then, \(R' \approx R/U\). Moreover, there is a one-to-one correspondence between the set of ideals of \(R'\) and the set of ideals of \(R\) that contain \(U\). This correspondence can be achieved by associating with an ideal \(W'\) of \(R'\), the ideal \(W \) in \(R\) defined by \(W = \set<x \in R>{\func{\phi}{x} \in W'}\), then \(W' \approx R/W\).
\end{theorem}
\section{More ideals and quotient rings}
\begin{lemma}
    Let \(R\) be a commutative ring with unit element whose only ideals are \(\bracket{0}\) and \(R\). Then, \(R\) is a field.
\end{lemma}

\begin{definition}
    An ideal \(M \neq R\) is said to be \textbf{maximal ideal} of \(R\) whenever \(U\) is an ideal of \(R\) such that \(M \subset U \subset R\), then either \(U R\) or \(U = M\).
\end{definition}

If a ring has unit element, then using axiom of choice it can be shown that there is a maximal ideal.

\begin{theorem}
    If \(R\) is a commutative ring with unit element and \(M\) is an ideal of \(R\), then \(M\) is maximal ideal if and only if \(R/M\) is a field.
\end{theorem}

\section{The field of quotients of integral domain}
\begin{definition}
    A ring \(R\) can be \textbf{imbedded} in ring \(R'\) if there is an isomorphism of \(R\) inot \(R'\). If \(R\) and \(R'\) have unit elements, this isomorphism should take \(1\) onto \(1'\). \(R'\) will be called an \textbf{over ring or extension } of \(R\).
\end{definition}

\begin{theorem}
    Every integral domain can be imbedded in a field.
\end{theorem}
\begin{proof}
    Take a look at quotients \(\frac{a}{b}\). \(M = \set<(a,b)>{a,b\in D, b \neq 0}\). \((a,b) \sim (c,d)\) if \(ad = bc\). \(F\) be the set of equivalence classes. \(F\) is a field and \(D\) can be imbedded in \(F\).
\end{proof}
\(F\) is caled the \textbf{field of quotients} of \(D\).

\section{Euclidean ring}
\begin{definition}
    An integral domain \(R\) is an \textbf{Euclidean ring} if for every \(a \neq 0\) in \(R\) there exists a non-negative integer \(\func{d}{a}\) such that 
    \begin{enumerate}
        \item For all non-zero \(a,b \in R\), \(\func{d}{a} \leq \func{d}{ab}\).
        \item For all non-zero \(a,b \in R\), there exists \(t,r \in R\) such that \(a = tb + r\) where either \(r = 0\) or \(\func{d}{r} < \func{d}{b}\).
    \end{enumerate}
\end{definition}

\(\angleBracket{a} = \set<xa>{x \in R}\).

\begin{theorem}
    Let \(R\) be a Euclidean ring and let \(A\) be an ideal of \(R\). Then, there exists \(a_0\in A\) such that \(A\) consists exactly of \(a_0 x\) as \(x\) ranges over \(R\).
\end{theorem}

\begin{definition}
    An integral domain \(R\) with unit element is a \textbf{principle ideal ring} if every ideal \(A\) of \(R\) is of the form \(A = \angleBracket{a}\) for some \(a \in R\)
\end{definition}

\begin{corollary}
    A Euclidean ring possesses a unit element.
\end{corollary}

\begin{definition}
    If \(a \neq 0\) and \(b\) are in a commutative ring \(R\), then \(a\) is said to divide \(b\) there exists \(c \in R\) such that  \(b = ac\) denoted by \(a \mid b\).
\end{definition}

\begin{remark}
    \ 
    \begin{enumerate}
        \item \(a \mid b,\ b\mid c \implies a \mid c\).
        \item \(a \mid b,\ a \mid c \implies a \mid (b \pm c)\).
        \item \(a \mid b \implies a \mid bx\) for all \(x \in R\).
    \end{enumerate}
\end{remark}

\begin{definition}
    If \(a,b \in R\), then \(d \in R\) is the \textbf{greatest common divisor} of \(a\) and \(b\) if 
    \begin{enumerate}
        \item \(d \mid a, \ d\mid b\).
        \item \(c \mid a, \ c \mid b \implies c \mid d\).
    \end{enumerate}
    It is denoted as \(d = (a,b) = \func{\gcd}{a,b}\).
\end{definition}

\begin{lemma}
    Let \(R\) be a Euclidean ring. Then, any two elements \(a\) and \(b\) in \(R\) have a greatest common divisor \(d\). Moreover, \(d = \lambda a + \mu b\) for some \(\lambda, \mu \in R\).
\end{lemma}

\begin{definition}
    Let \(R\) be a commutative ring with unit element. An element \(a \in R\) is a \textbf{unit} in \(R\) if there exists an element \(b\) such that \(ab = 1\).
\end{definition}

A unit is an element whose multiplicative inverse exists in \(R\).

\begin{lemma}
    Let \(R\) be an integral domain with unit element and suppose that for \(a,b \in R\) both \(a \mid b\) and \(b \mid a\) are true. Then, \(a = ub\), where \(u\) is a unit in \(R\).
\end{lemma}

\begin{definition}
    In a commutative ring \(R\) with unit element, two elements \(a\) and \(b\) are \textbf{associates}  if \(b = ua\) for some unit \(u \in R\).
\end{definition}

\begin{lemma}
    Let \(R\) be a Euclidean ring and \(a,b \in R\) be non-zero elements. If \(b\) is not a unit in \(R\), then \(\func{d}{a} < \func{d}{ab}\).
\end{lemma}

\begin{definition}
    Let \(R\) be a Euclidean. A non-unit elemnt \(\pi \in R\) is \textbf{prime} if whenever \(\pi = ab\), one of \(a\) or \(b\) is a unit in \(R\).
\end{definition}

\begin{theorem}\label{lm:factorizationTheorem}
    Let \(R\) be a Euclidean ring. Then, every element is either a unit in \(R\) or can be written as a product of finite number prime elements.
\end{theorem}

\begin{definition}
    Let \(R\) be a Euclidean ring. Two elements \(a\) and \(b\) in \(R\) are \textbf{relatively prime} if their greatest common divisor is a unit in \(R\).
\end{definition}

\begin{lemma}
    Let \(R\) be a Euclidean ring. If \(a \mid bc\) but \(a\) and \(b\) are relatively prime, then \(a \mid c\).
\end{lemma}

\begin{lemma}
    If \(\pi\) is a prime element in a Euclidean ring \(R\), then \(\pi \mid ab \implies \pi \mid a\) or \(\pi \mid b\).
\end{lemma}

\begin{theorem}[Unique factorization theorem]
    Let \(R\) be a Euclidean ring and \(a \neq 0\) be non-unit element of \(R\). Suppose that \(a = \pi_1 \dots \pi_n = \pi'_1 \dots \pi'_m\) where \(\pi_i\) and \(\pi'_j\) are prime elements. Then, \(n = m\) and each \(\pi_i\) is an associate of a \(\pi_j'\) and each \(\pi'_j\) is an associate of a \(\pi_i\).
\end{theorem}

Combining unique factorization theorem with \ref{lm:factorizationTheorem} gives that every non-zero element in \(R\) can be written uniquely up to associates as a product of primes in \(R\).

\begin{lemma}
    The ideal \(A = \angleBracket{a_0}\) is a maximal ideal of the Euclidean ring \(R\) if and only if \(a_0\) is a prime element.
\end{lemma}

\section{A particular Euclidean ring}
The domain of \textbf{Gaussian integers} \(\squareFunc{\Integers}{i} = \set<a + bi>{a,b \in \Integers, i = \sqrt{-1}}\) is a Euclidean ring, with \(\func{d}{a + bi} = a^2 + b^2\).
\begin{theorem}
    \(\squareFunc{\Integers}{i}\) is a Euclidean ring.
\end{theorem}

\begin{lemma}
    Let \(p\) be a prime integer and suppose for integer \(c\) relatively prime to \(p\) we can find integers \(x\) and \(y\) such that \(x^2 + y^2 = cp\). Then, \(p\) can be written as a sum of two squares of integers. \textit{i.e.} there exists integers \(a\) and \(b\) such that \(a^2 + b^2 = p\).
\end{lemma}

\begin{lemma}
    If \(p \equiv 1 \mod{4}\), we can solve the congruence \(x^2 \equiv -1 \mod{p}\).
\end{lemma}

\begin{theorem}
    If \(p\) is a prime of form \(4n + 1\), then \(p = a^2 + b^2\) for some integers \(a\) and \(b\).
\end{theorem}

\section{Polynomial rings}
Let \(F\) be a field. \(\squareFunc{F}{x} = \set<a_0 + a_1 x + \dots + a_nx^n>{n \geq 0, a_i \in F}\) is the ring of polynomials in the indeterminate \(x\).

\begin{definition}
    If \(\func{p}{x} = a_0 + a_1 x + \dots + a_m x^m\) and \(\func{q}{x} = b_0 + \dots + b_nx^n\) are in \(\squareFunc{F}{x}\), then \(\func{p}{x} = \func{q}{x}\) if \(m = n\) and for each \(i \geq 0\), \(a_i = b_i\).
\end{definition}

\begin{definition}
    \(\func{p}{x} + \func{q}{x} = c_0 + \dots + c_k x^k\) where \(c_i = a_i + b_i\).
\end{definition}

\begin{definition}
    \(\func{p}{x} \func{q}{x} = c_0 + \dots + c_k x^k\) where \(c_i = \sum_{t = 0}^i a_t b_{i - t}\).
\end{definition}

Therefore, \(\squareFunc{F}{x}\) is a commutative ring with unit element.

\begin{definition}
    If \(\func{f}{x} = a_0 + a_1 x + \dots + a_n x^n \neq 0\) and \(a_n \neq 0\), then the \textbf{degree} of \(f\) is \(n\). \textit{i.e.} the degree of \(f\), \(\deg f = \min\set<n \geq 0>{a_k =0,\ \forall k > n}\). The zero polynomial can be defined to be of infinite degree. 
 \end{definition}

 \begin{lemma}
    If \(\func{f}{x}, \func{g}{x} \neq 0\) are two polynomials in \(\squareFunc{F}{x}\), then 
    \begin{equation*}
        \func{\deg}{fg}= \func{\deg}{f}  + \func{\deg}{g}
    \end{equation*}
 \end{lemma}

 \begin{corollary}
    \(\func{f}{x}, \func{g}{x} \neq 0\), then \( \func{\deg}{f}\leq \func{\deg}{fg}\).
 \end{corollary}

 \begin{corollary}
    \(\squareFunc{F}{x}\) is an integral domain.
 \end{corollary}

 Since \(\squareFunc{F}{x}\) is an integeral domain, we can construct its field of quotients which is the field of rational functions in \(x\) over \(F\).

 \begin{lemma}[The division algorithm]
    Given two polynomials \(\func{f}{x}\) and \(\func{g}{x} \neq 0\), there exists two polynomials \(\func{t}{x},\func{r}{x} \in \squareFunc{F}{x}\) such that \(\func{f}{x} = \func{t}{x} \func{g}{x} + \func{r}{x}\) where \(\func{r}{x} =0\) or \(\deg r < \deg g\).
 \end{lemma}

 \begin{theorem}
    \(\squareFunc{F}{x}\) is a Euclidean ring.
 \end{theorem}

 \begin{theorem}
    \(\squareFunc{F}{x}\) is a principle ideal group.
 \end{theorem}

 \begin{lemma}
    Given two polynomials \(\func{f}{x}, \func{g}{x} \in \squareFunc{F}{x}\), the greatest common divisor \(\func{d}{x} = (\func{f}{x}, \func{g}{x})\) can be realized as \(\func{d}{x} = \func{\lambda}{x} \func{f}{x} + \func{\mu}{x} \func{g}{x}\) for some \(\func{\lambda}{x}, \func{\mu}{x} \in \squareFunc{F}{x}\).
 \end{lemma}
 \begin{definition}
    A polynomial \(\func{p}{x} \in \squareFunc{F}{x}\) is \textbf{irreducible} over \(F\) if whenever \(\func{p}{x} = \func{a}{x} \func{b}{x}\) with \(\func{a}{x}, \func{b}{x} \in \squareFunc{F}{x}\), one of \(\func{a}{x}\) or \(\func{b}{x}\) has degree \(0\).
 \end{definition}

 \begin{lemma}
    Any polynomial in \(\squareFunc{F}{x}\) can be written in a unique manner as product of irreducible polynomials in \(\squareFunc{F}{x}\).
 \end{lemma}

 \begin{lemma}
    The ideal \(A = \angleBracket{\func{p}{x}}\) in \(\squareFunc{F}{x}\) is a maximal ideal if and only \(\func{p}{x}\) is irreducible.
 \end{lemma}

 \section{Polynomials over field of rationals}

\begin{definition}
    The polynomial \(\func{f}{x} = a_0 + a_1x + \dots + a_nx^n\) where \(a_i \in \Integers\) is said to be \textbf{primitive} if the greatest common divisor of \(a_0, \dots, a_n\) is \(1\).
\end{definition}

\begin{lemma}
    If \(\func{f}{x}\) and \(\func{g}{x}\) are primitive, then \(\func{f}{x} \func{g}{x}\) is a primitive polynomial.
\end{lemma}

\begin{definition}
    The \textbf{content} of a polynomial \(\func{f}{x} =  a_0 + a_1x + \dots + a_nx^n\) where \(a_i \in \Integers\) is the \(\func{\gcd}{a_0, \dots, a_n}\).
\end{definition}

\begin{theorem}[Guass' lemma]
    If primitive polynomial \(\func{f}{x}\) can be factored as a product of two polynomials with rational coefficients, it can be factored as the product of two polynomials with integer coefficients.
\end{theorem}

\begin{definition}
    A polynomial is said to be \textbf{integer monic} if all of its coefficients are integers and its highest coefficient is \(1\).
\end{definition}

\begin{corollary}
    If an integer monic polynomial \(\func{f}{x}\) can be factored as a product of two polynomials with rational coefficients, it can be factored as a product of two integer monic polynomials.
\end{corollary}

\begin{theorem}[The Eisenstein criterion]
    Let \(\func{f}{x}=  a_0 + a_1x + \dots + a_nx^n\) with \(a_i \in \Integers\). Suppose that for some \(p\), \(p \nmid a_n\), \(p \mid a_{n-1}, \dots, p \mid a_1, \ p \mid a_0\), but \(p^2 \nmid a_0\). Then, \(\func{f}{x}\) is irreducible over rationals.
\end{theorem}

\section{Polynomial rings over commutative rings}
\(\squareFunc{R}{x} = \set<a_0 + a_1x + \dots + a_nx^n>{a_i \in R}\).
For the rest of this section \(R\) is assumed to be commutative and have unit element.
\(\squareFunc{R}{x_1, \dots, x_n}\) is the ring of polynomials in the indeterminate \(x_1, \dots, x_n\). It can be constructed as \(\squareFunc{\squareFunc{\squareFunc{R}{x_1}}{x_2}\dots}{x_n} = \set{\sum a_{i_1, \dots, i_n} x_1^{i_1} \dots x_n^{i_n}}\).

\begin{lemma}
    If \(R\) is an integral domain, so is \(\squareFunc{R}{x}\) and by induction, \(\squareFunc{R}{x_1, \dots, x_n}\) is an integral domain.
\end{lemma}