\chapter{Time Frequency Characterization}
\section{Magnitude-phase representation of Fourier Transform}
\(\angle \func{X}{j\omega}\) is the relative phase and \(\frac{1}{2\pi}\abs{\func{X}{j\omega}}^2 \diffOperator \omega \) is the energy of \(\func{x}{t}\) in \(\opop{\omega}{ \omega + \diffOperator \omega}\). Obviously, the magnitude matters but also phase matters depending on the context. 
\section{Magnitude-phase representation of the frequency response of LTI systems}
In an LTI system 
\begin{equation*}
    \func{Y}{j\omega} = \func{H}{j\omega} \func{X}{j\omega} \implies \begin{cases}
        \abs{Y} &= \abs{H} \abs{X} \\
        \angle Y &= \angle H + \angle X
    \end{cases}
\end{equation*}
where the \(\abs{H}\) is called the gain and \(\angle H\) is the phase shift. 
\subsection*{Linear and non-linear phase}
For unit gain all-pass \(\func{H}{j\omega} = 1\) with linear phase \(\angle \func{H}{j\omega} = t_0 \omega\). Then, 
\begin{equation*}
    \func{y}{t} = \func{x}{t - t_0}
\end{equation*}
\subsubsection*{Group delay}
Suppose \(\func{X}{j\omega}\) is zero outside a narrow band around \(\omega = \omega_0\). Then, a non-linear phase can be approximated by a linear phase 
\begin{align*}
    \angle \func{H}{j\omega} &\simeq - \phi - \alpha \\
    \implies \func{Y}{j\omega} &= \func{X}{j\omega} \abs{\func{H}{j\omega}} e^{j\phi} e^{j \omega \alpha}
\end{align*}
\(\alpha\) seconds delay. The group delay is defined as 
\begin{equation*}
    \func{\tau}{\omega} = - \dfrac{\diffOperator}{\diffOperator \omega} \angle \func{H}{j\omega}
\end{equation*}
but this might be discontinuous at \(2\pi k\). So we use the un-wrapped phase 
\begin{equation*}
    \func{\tau}{\omega} = - \dfrac{\diffOperator}{\diffOperator \omega} \squareBracket{\angle \func{H}{j\omega}}
\end{equation*}

\subsection*{Log-magnitude and Bode plots}
\(dB = 20 \log_{10} \abs{\func{H}{j\omega}}\). If \(\omega\)-axis is logarithmic \(\log_{10} \omega\) as well then it is called Bode plot.

\section{Time-Domain properties of ideal frequency selective filter}
a frequency selective filter 
\begin{equation*}
    \abs{\func{H}{j \omega}} = \begin{cases}
        1 & \abs{\omega} \leq \omega_c \\
        0 & \abs{\omega} > \omega_c
    \end{cases}
\end{equation*}
With linear phase (\(\alpha\) seconds delay)
\begin{equation*}
    \angle \func{H}{j\omega} = -\alpha \omega 
\end{equation*}

\section{Time-domain and frequency aspects of non-ideal filters}
add images in pg 175

\section{First and second order CT system}
\subsection*{First order continuous-time system}

\begin{equation*}
    \tau \dfrac{\diffOperator y}{\diffOperator t} + u = \func{x}{t} \implies \func{H}{j\omega} = \dfrac{1}{j \omega \tau + 1}
\end{equation*}
which then implies that 
\begin{align*}
    \func{h}{t} &= \dfrac{1}{\tau} e^{-\frac{t}{\tau}} \func{u}{t}\\
    \func{s}{t} &= \bracket{ 1 - e^{-\frac{t}{\tau}}} \func{u}{t}
\end{align*}

\subsection*{Second order continuous-time system}
\begin{equation*}
    \dfrac{\diffOperator^2 y}{\diffOperator t^2} + 2 \xi \omega_n  \dfrac{\diffOperator y}{\diffOperator t} + \omega_n^2 \func{y}{t} = \omega_n^2 \func{x}{t}
\end{equation*}
then 
\begin{align*}
    \func{H}{j\omega} &= \dfrac{\omega_n^2}{\bracket{j\omega}^2 + 2 \xi \omega_n j \omega + \omega_n^2}\\
    &= \dfrac{\omega_n^2}{\bracket{j\omega - c_1} \bracket{j\omega - c_2}}
\end{align*}
where \(c_1 = - \xi \omega_n + \omega_n \sqrt{\xi^2 - 1}\) and \(c_2 = - \xi \omega_n + \omega_n \sqrt{\xi^2 - 1}\). If \(\xi \neq 1\) then 
\begin{align*}
    \func{H}{j\omega} &= \dfrac{M}{j\omega - c_1} - \dfrac{M}{j\omega - c_2} \qquad \qquad M = \dfrac{\omega_n}{2 \sqrt{\xi^2 - 1}}\\
    \implies \func{h}{t} &= M \bracket{e^{c_1 t} - e^{c_2 t}}\func{u}{t}
\end{align*}
and if \(\xi = 1\) then 
\begin{equation*}
    \func{H}{j\omega} = \dfrac{\omega_n^2}{\bracket{j\omega + \omega_n}^2} \implies \func{h}{t} = \omega_n^2 t e^{-\omega_n t} \func{u}{t}
\end{equation*}
\(\xi\) is called the damping ratio and \(\omega_n\) is the undamped natural frequency. 
\begin{itemize}
    \item For \(0 < \xi < 1\) the response is underdamped which will overshoot and rings in step function. 
    \item For \(\xi = 1\) the response is critically damped which will have the fastest settling time. 
    \item For \(\xi > 1\) the response is overdamped which will imply it has slow settling time.
\end{itemize}

\section{First order and second order discrete-time system}
\section{First order}
\begin{align*}
    \squareFunc{y}{n} - a \squareFunc{y}{n - 1} = \squareFunc{x}{n} \implies \func{H}{e^{j\omega}} &= \dfrac{1}{1 - a e^{j\omega}} \\
    \squareFunc{h}{n} &= a^n \func{u}{n} \\
    \squareFunc{s}{n} &= \dfrac{1 - a^{n+1}}{1 - a} \func{u}{n}
\end{align*}
\section{Second order}
\begin{equation*}
    \squareFunc{y}{n} - 2r \cos \theta \squareFunc{y}{n - 1} + r^2 \squareFunc{y}{n - 2} = \squareFunc{x}{n}
\end{equation*}
\begin{align*}
    \func{H}{e^{j\omega}} &= \dfrac{1}{1 - 2r \cos \theta e^{-j\omega} + r^2 e^{-2j\omega}} \\
    &= \dfrac{1}{\bracket{1 - re^{j\theta} e^{-j\omega}}\bracket{1 - re^{=j\theta} e^{-j\omega}}}
\end{align*}

If \(\theta \neq 0,\pi\) then 
\begin{equation*}
    \func{H}{e^{j\omega}} = \dfrac{A}{1 -re^{j\theta} e^{-j\omega}} + \dfrac{B}{1 -re^{-j\theta} e^{-j\omega}}
\end{equation*}
where
\begin{equation*}
    A = \dfrac{e^{j\theta}}{2j \sin \theta} \qquad B = \dfrac{e^{-j\theta}}{2j \sin\theta}
\end{equation*}
if \(\theta = 0\)
\begin{equation*}
    \func{H}{e^{j\omega}} = \dfrac{1}{\bracket{1 - re^{-j\omega}}^2} \implies \squareFunc{h}{n} = (n + 1)r^n \squareFunc{u}{n} 
\end{equation*}
and for \(\theta = \pi\)
\begin{equation*}
    \func{H}{e^{j\omega}} = \dfrac{1}{\bracket{1 + re^{-j\omega}}^2} \implies \squareFunc{h}{n} = (n + 1)(-r)^n \squareFunc{u}{n} 
\end{equation*}