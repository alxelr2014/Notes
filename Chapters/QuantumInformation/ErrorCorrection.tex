\chapter{Quantum Error Correction}
There are four major differences between quantum and classical error.

\begin{enumerate}
    \item Classical bits and errors are macroscopic but the quantum ones are microscopic. As a result, quantum information is more prone to errors.
    \item Classical information can be copied, however due to non-cloning theorem, quantum information can not be copied.
    \item Classical errors are discrete, especially when considering the digital information. Again, quantum information are subject to continuous errors.
    \item Classical bits may be measured but quantum measurement will result in collapse of wave function. 
\end{enumerate}
Despite these challenges that make quantum error correction look impossible, we will show that we can reliable quantum compution and communication.

\section{One qubit error correction}
Suppose we have a qubit \(\ket{\psi}\) that is subject to error. Let \(\rho_{env}\) be the denisty operator corresponding to that of the environment. Then, there exists a unitary operator \(U\) that describes how \(\ket{\psi}\) and \(\rho_{env}\) evolve.
\begin{equation*}
    \ket{\psi}\otimes \rho_{env} \mapsto U(\ket{\psi}\otimes \rho_{env})
\end{equation*}
We may formulate \(U\) in terms of Pauli matrices
\begin{equation*}
    U = I \otimes U_0 + X \otimes U_1 + Y \otimes U_2 + Z \otimes U_3
\end{equation*}
Where \(X= \sigma_x, Y = i\sigma_Y,\) and \(Z = \sigma_Z\). Now suppose \(\ket{\psi}\) is such that 
\begin{align*}
    \ket{\psi} && X \ket{\psi} && Y\ket{\psi} && Z\ket{\psi}
\end{align*}
are perpendicular to each other. Thus, there is a measurement that enables to distinguish which error happened and then correct it. Typically, this is not the case for \(\ket{\psi}\). As result, we need a mapping \(\ket{\psi} \mapsto \ket{\psi'}\) such that 
\begin{equation*}
    \bra{\psi'}X \ket{\psi'} = \bra{\psi'}Y \ket{\psi'} = \bra{\psi'}Z\ket{\psi'} = 0
\end{equation*}
You may also note that it is enough that only \(X\ket{\psi'}\) and \(Z\ket{\psi'}\) be perpendicular to each other and \(\ket{\psi'}\). Because,
\begin{equation*}
    \bra{\psi'}Y\ket{\psi'} = \bra{\psi'}XZ\ket{\psi'} = 0
\end{equation*}
\section{Simple quantum error correction codes}

