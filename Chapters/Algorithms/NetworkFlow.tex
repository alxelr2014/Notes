\chapter{Network Flow}
A match \(M \subset E\) in a graph \(G = (V,E)\) is a set of edges uch that each node appears at most once. In a perfect matching each node appears exactly once. In a bipartite graph, there is a perfect matching if and only if the number of vertices in each side are equal. A problem is to find the largest matching in a bipartite graph.
\section{Flow network and Maximum cut}
\begin{definition}
    Flow network is directed graph \(G\) with 
\begin{enumerate}
    \item each edge \(e\) has non-negative capacity \(\func{c}{e}\).
    \item There is a single source node \(s \in V\).
    \item There is a single sink node \(t \in V\).
\end{enumerate}
The other nodes are called internal nodes.
\end{definition}
\begin{definition}
    An \(s-t\) flow is a function \(f : E \to \Reals^+\) such that 
    \begin{enumerate}
        \item for all edges \(e \in E\), \(0 \leq \func{f}{e} \leq \func{c}{e}\).
        \item for all vertices \(v \neq s,t\)
        \begin{equation*}
            \sum_{e \text{ into } v} \func{f}{e} = \sum_{e \text{ out of } v} \func{f}{e}
        \end{equation*}
    \end{enumerate}
    The value of the flow is defined by 
    \begin{equation*}
        \func{v}{f} = \sum_{e  \text{ out of } s} \func{f}{e} = \func{f^{out}}{s} = \func{f^{in}}{t}
    \end{equation*}
    More generally, for any \(S \subset V\)
    \begin{align*}
        \func{f^{in}}{S} &= \sum_{e \text{ into } S} \func{f}{e} & \func{f^{out}}{S} &=  \sum_{e \text{ out of } S} \func{f}{e}
    \end{align*}
\end{definition}
The problem is given a flow network, find the maximum flow. 
\begin{definition}
    Given flow network \(G\) and flow \(f\) on \(G\), the residual graph \(G_f\) is 
    \begin{enumerate}
        \item the nodes are the same as \(G\).
        \item for all \(e =(u,v) \in E\) which \(\func{f}{e} < \func{c}{e}\) there is an edge from \(u\) to \(v\) with capacity \(\func{c}{e}  - \func{f}{e}\). These are called forward edges.
        \item for all \(e =(u,v) \in E\) which \(0 < \func{f}{e}\) there is an edge from \(v\) to \(u\) with capacity \(\func{f}{e}\). These are called backward edges.
    \end{enumerate}
\end{definition}
The bottleneck of a path \(P\) in \(G_f\) is defined as 
\begin{equation*}
    \func{bottleneck}{P,f} = \min_{e \in P} \func{c_f}{e}
\end{equation*}
where \(\func{c_f}{e}\) is the residual capacity.
Suppose \(P\) is a \(s-t\) path in \(G_f\) that visits each node at most once.
\begin{algorithm}
    \DontPrintSemicolon
    \(b = \func{bottleneck}{P,f}\)\; 
    \For{\(e = (u,v) \in P\)}{
        \If{\(e\) is a forward edge}{
            \(\func{f'}{e} = \func{f}{e} + b\)\;
        }
        \If{\(e\) is a backward edge}{
            \(\func{f'}{(v,u)} = \func{f}{(v,u)} - b\)\;
        }
    }
    \Return{\(f'\)}
    \caption{Augment\((P,f)\)}
\end{algorithm}

\begin{theorem}
    Augment\((P,f)\) is a flow in \(G\). 
\end{theorem}
The Ford-Fulkerson algorithm for maximum is given below.
\begin{algorithm}
    \(f \equiv 0\)\; 
    \While{there is an \(s-t\) path \(P\) in \(G_f\)}{
        \(f' = \)Augment\((P,f)\)\;
        \(f \gets f'\)\;
        update \(G_f\)\; 
    }
    \Return{\(f\)}
    \caption{Maxflow}
\end{algorithm}
Let assume that each node has at least one incident edge.
\begin{proposition}
    Assuming capacities are integers, then at every step of the algorithm, the flow values and the residual capacities are integers. Moreover, \(\func{v}{f'} = \func{v}{f} + \func{bottleneck}{P,f}\) and hence \(\func{v}{f'} \geq \func{v}{f}\).
\end{proposition}

\begin{proposition}
    Let \(v\) be the value of maximum flow. Then, \(v \leq \sum_{e \text{ out of} s} c_e = C\). 
\end{proposition}

\begin{proposition}
    The complexity of the algorithm is \(\bigO{mC}\).
\end{proposition}

\begin{definition}
    An \(s-t\) cut \((A,B)\) is such that \(s \in A\) and \(t \in B\). The capacity of a cut \((A,B)\) is given by 
    \begin{equation*}
        \func{c}{A,B} = \sum_{e \text{ out of } A} c_e
    \end{equation*}
    A minimum cut is a cut with minimum capacity.
\end{definition}
\begin{definition}
    For any flow \(f\) and any cut \((A,B)\) the flow of cut is 
    \begin{equation*}
        \func{f}{A,B}  = \sum_{e \text{ out of } A} \func{f}{e} - \sum_{e \text{ into } A} \func{f}{e}
    \end{equation*}
\end{definition}

\begin{proposition}
    For any flow \(f\) and any cut \((A,B)\) the flow of cut 
    \begin{enumerate}
        \item \(\func{v}{f} = \func{f^{out}}{A} - \func{f^{in}}{A} = \func{f}{A,B}\).
        \item \(\func{v}{f} \leq \func{c}{A,B}\).
    \end{enumerate}
\end{proposition}

\begin{lemma}
    A flow \(f\) is maximum if there exists a cut \((A,B)\) such that \(\func{v}{f} = \func{c}{A,B}\) and \((A,B)\) is minimum cut. Also there always exist such matching.
\end{lemma}

