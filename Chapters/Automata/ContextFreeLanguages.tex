\chapter{Context Free Languages}
\section{Context free languages}
\subsection{Formal definition}
\begin{definition}
    A context-free grammar is a 4-tuple \((V,\Sigma,R,S)\) where 
    \begin{enumerate}
        \item \(V\) is a finite set called the \textbf{variables}.
        \item \(\Sigma\) is a finite set disjoint from \(V\), called the \textbf{terminals}.
        \item \(R\) is a finite set of rule, with each rule being a variable and a string of variables and terminals.
        \item \(S \in V\) is the start state. 
    \end{enumerate}
\end{definition}

We say \(u\) derives \(v\), denoted by \(u \xRightarrow{\ast} v\), if \(u = v\) or if a sequence \(u_1,u_2, \dots, u_k\) exists for \(k\geq 0\) such that 
\begin{equation*}
    u \Rightarrow u_1 \Rightarrow u_2 \Rightarrow \dots \Rightarrow u_k \Rightarrow v
\end{equation*}
The language of the grammar is \(\set<w \in \Sigma^{\ast}>{S \xRightarrow{\ast} w }\).
\subsection{Designing CFL}
Suppose \(G_i\) for \(i=1,\dots,k\) are context free grammars with start state \(S_i\). Then, the union of these grammar \(G\) can be obtained by the following rule 
\begin{equation*}
    S \rightarrow S_1 | S_2 | \dots | S_k
\end{equation*}
where \(S\) is the start state of \(G\).

Moreover we can construct a context free grammar equivalent of a DFA. For each state \(q_i\), consider a variable \(R_i\). Then, \(R_i \rightarrow aR_j\) if \(\func{\delta}{q_i,a} = q_j\).

\subsection{Ambguity}
A string \(w\) is derived \textbf{ambiguously} in a context free grammar \(G\) if it has two or more distince leftmost derivation. Grammar \(G\) is \textbf{ambiguous} if it generate some strings ambiguously. In leftmost derivation, at each step the left most variable is replace.

Some grammars can only be generated by ambiguous CFLs. These are said to be inherently ambiguous. 
\subsection{Chomsky normal formal}
\begin{definition}
    A context free grammar is in \textbf{Chomsky normal form} if every rule is of the form 
    \begin{equation*}
        A \to BC \qquad \qquad A \to a
    \end{equation*}
    where \(a\) is any terminal and \(A,B,\) and \(C\) are any terminals -- except that \(B\) and \(C\) may not be the start state. In addition, we permit \(S \rightarrow \epsilon\).
\end{definition}

\begin{theorem}
    Any context free language is generated by a context free grammar in Chomsky normal form.
\end{theorem}

\begin{proof}
    Let \(u\) and \(v\) be any strings of terminals and variables.
    \begin{itemize}
        \item Add a new start variable.
        \item For \(\epsilon\)-rules, such as \(A \rightarrow \epsilon\), remove the rule and replace any \(R \rightarrow uAv\) with \(R \rightarrow uv\). If \(R \rightarrow A\), then add \(R \rightarrow \epsilon\) unless it had been previously removed. 
        \item For unit rules, such as \(A \rightarrow B\), remove the rule and replace any \(B \rightarrow u\) with \(A \rightarrow u\) unless this was a unit rule previously removed.
        \item Lastly, consider \(A \rightarrow u_1\dots u_k\) where \(u_i\) are either a variable or a terminal. We can replace this rule with the following 
        \begin{align*}
            A &\to U_1 A_1\\
            A_i &\to U_{i+1} A_{i+1}\qquad \qquad i = 1, \dots , k-3\\
            A_{k-2} &\to U_{k-1}U_{k}
        \end{align*} 
        where \(U_i = u_i\) if \(u_i\) is a variable, otherwise we must add \(U_i \rightarrow u_i\).
    \end{itemize}
\end{proof}

\section{Pushdown automata}
A pushdown automaton is a nondeterminstic finite automaton with a stack. 
\subsection{Formal definition}
\begin{definition}
    A pushdown automaton is a 6-tuple \((Q,\Sigma,\Gamma, \delta, q_0,F)\) where 
    \begin{enumerate}
        \item \(Q\) is a finite set of states.
        \item \(\Sigma\) is a fintie set of input alphabet.
        \item \(\Gamma\) is a finite set of stack alphabet.
        \item \(\delta: Q \times \Sigma_{\epsilon} \times \Gamma_{\epsilon} \to \powerSet{Q \times \Gamma_{\epsilon}}\) is the transition function.
        \item \(q_0 \in Q\) is the start state.
        \item \(F \subset Q\) is the set of accept states.
    \end{enumerate}
\end{definition}

It accepts \(w = w_1 \dots w_n\) where \(w_i \in \Sigma_{\epsilon}\) if there exists a sequence of states \(r_0, \dots , r_n \in Q\) and string \(s_0, \dots , s_n \in \Gamma^{\ast}\) exists that satisfy the following 
\begin{enumerate}
    \item \(r_0 = q_0\) and \(s_0 = \epsilon\).
    \item \((r_{i+1},b) \in \func{\delta}{r_i, w_{i+1},a}\) where \(s_i = at\) and \(s_{i+1} = bt\) for some \(a,b \in \Gamma_{\epsilon}\) and \(t \in \Gamma^{\ast}\).
    \item \(r_n \in F\).
\end{enumerate}
\subsection{Equivalence with context free grammars}
\begin{theorem}
    A language is context free if and only if some pushdown automaton recognizes it.
\end{theorem}

\begin{corollary}
    Every regular language is context free. 
\end{corollary}

\section{Non-context free languages}
\begin{theorem}[Pumping lemma for CFL]
    If \(G\) is a context free language, then there is a number \(p\) where if \(s\) is any string in \(G\) of length at least \(p\), then \(s\) may be divided into five pieces \(s =uvxyz\) satisfying 
    \begin{enumerate}
        \item \(\abs{vy} > 0\).
        \item \(\abs{vxy} \leq p\).
        \item \(uv^i x y^i z \in A\) for all \(i \geq 0\).
    \end{enumerate}   
\end{theorem}

\section{Deterministic context free languages}
\begin{definition}
    A nondeterminstic pushdown automaton is a 6-tuple \((Q,\Sigma,\Gamma,\delta,q_0,F)\) where 
    \begin{enumerate}
        \item \(Q\) is a finite set of states.
        \item \(\Sigma\) is a fintie set of input alphabet.
        \item \(\Gamma\) is a finite set of stack alphabet.
        \item \(\delta: Q \times \Sigma_{\epsilon} \times \Gamma_{\epsilon} \to Q \times \Gamma_{\epsilon} \cup \set{\emptyset}\) is the transition function.
        \item \(q_0 \in Q\) is the start state.
        \item \(F \subset Q\) is the set of accept states.
    \end{enumerate}
    and for all \(q \in Q\), \(a \in \Sigma\), and \(x \in \Gamma\) exactly one the 
    \begin{equation*}
        \func{\delta}{q,a,x} \quad \func{\delta}{q,\epsilon,x} \quad \func{\delta}{q,a,\epsilon} \quad \func{\delta}{q,\epsilon,\epsilon} 
    \end{equation*}
    is not \(\emptyset\).
\end{definition}

\begin{lemma}
    Every DPDA has an equivalent DPDA that always reads the entire string.
\end{lemma}
\subsection{Closure properties}
\begin{theorem}
    The class of DFCL is closed under complementation.
\end{theorem}

Let \(A \dashv = \set<w \dashv>{w \in A}\). 
\begin{theorem}
     \(A\) is DFCL if and only if \(A \dashv \) is DCFL.
\end{theorem}
% TODO: ambiguity, quotient of CFL with regular languages
% TODO: add proofs, add examples and problems 
% TODO: add some more about DCFC