\chapter{Baseband Data Transmission}
\begin{remark}
    Most efficient: PAM,PDM,PPM.
\end{remark}

--add image of page 43

\(r_b\) is bit rate, \(T_b\) is bit duration.
\begin{equation*}
    \func{X}{t} = \sum_{k = -\infty}^{\infty} a_k \func{p_g}{t - kT_b}
\end{equation*}
where we can assume that 
\begin{equation*}
    \func{p_g}{0} = 1 \qquad \qquad a_k = \begin{cases}
        -a & d_k = 0\\
        a & d_k = 1
    \end{cases}
\end{equation*}
Then, 
\begin{equation*}
    \func{Y}{t} = \sum_{k = -\infty}^{\infty} A_k \func{p_r}{t - t_d - kT_b} + \func{n_0}{t}
\end{equation*}
where \(A_k = K_c a_k\), \(K_c\) is the normalizing constant that yields \(\func{p_r}{0} = 1\), and \(K_c\func{p_r}{t-t_d}\) is the response of the system to \(\func{p_g}{t}\).

\begin{remark}
    Equalizing filter.
\end{remark}

\(\func{Y}{t}\) is sampled at rate of \(t_m = mT_b + t_d\) and \(m_{\cardinalTH}\) is generated by comparing \(\func{Y}{t_m}\) to some threshold.
\begin{align*}
    \func{Y}{t} &= A_m + \underbrace{\sum_{k \neq m} A_k \func{p_r}{(m -k )T_b}}_{\text{Intersymbol Interference}} + \underbrace{\func{n_0}{t_m}}_{\text{channel noise}}
\end{align*}

The goals are 
\begin{itemize}
    \item minimize errors introduced by noise and ISI.
    \item maximize \(r_b\) for a given bandwidth.
    \item minimize bandwidth for a given \(r_b\).
\end{itemize}

\section{Baseband binary PAM system}
For design purposes we will assume that input data rate, overall bit error probability, characteristics of the channel are given. Channel noise is modeled by a AWGN with known spectral density \(\func{\calG_n}{f}\). Source output is assumed be equiprobable sequence of independent bits.
\begin{remark}
    PAM: specify pulse shapes, \(\func{p_g}{t}, \func{p_r}{t},\func{H_R}{f},\func{H_T}{f}\).
\end{remark}

\subsection{Baseband pulse shaping}
IN the equation for \(\func{Y}{t}\), to remove ISI we must have 
\begin{equation*}
    \func{p_r}{nT_b} = \begin{cases}
        1 & \text{for } n = 0\\
        0 & \text{for } n \neq 0
    \end{cases}
\end{equation*}
\begin{theorem}
    If \(\func{P_r}{f}\) satisfies Nyquist criterion.
    \begin{equation*}
        \sum_{k = -\infty}^{\infty} \func{P_r}{f + \frac{k}{T_b}} = T_b \qquad \qquad \text{for } \abs{f} \leq \frac{1}{2T_b}
    \end{equation*}
    then 
    \begin{equation*}
        \func{p_r}{nT_b} = \begin{cases}
            1 & \text{for } n = 0\\
            0 & \text{for } n \neq 0
        \end{cases}
    \end{equation*}
\end{theorem}

hence, ISI can be removed if and only if bandwidth of \(P_r\), is \(\abs{f} > \frac{r_b}{2}\). ??

In practical systems for \(r_b\) rate the available bandwidth is between \(\frac{r_b}{2}\) to \(r_b\) Hz. and a class of \(\func{P_r}{f}\) with a this bandwidth are \textbf{raised cosine frequence} which are commonly used. (parameter \(\beta\))
\begin{enumerate}
    \item Bandwidth \(= \func{r_b}{2} + \beta\).
    \item larger \(\beta\) implies faster decaying pules, hence less ISI due to timing errors.
    \item \(\func{P_r}{f}\) is real, non-negative and \(\int_{-\infty}^{\infty} \func{P_r}{f} \diffOperator f = 1\).
    \item \(\beta =0\) produces zero ISI at a data rate of \(r_b\).
    \item Practically impossible since time response must be zero prior to a time \(t_0 > 0\). However, a delayed version \(\func{p_r}{t-t_d}\) may be chose so that \(\func{p_r}{t-t_d} = 0\) for \(t < t_0\).
    \item One may want to use the whole bandwidth to get a faster decay of \(\func{p_r}{t}\).
\end{enumerate}
-- add image page 44.


\subsection{Optimum transmitting and receiving filter}
A design constraint 
\begin{equation*}
    \func{P_g}{f} \func{H_T}{f} \func{H_c}{f}\func{H_R}{f} = K_c \func{P_r}{f} e^{-2\pi j ft_d}
\end{equation*}
where \(P_g,H_c,\) and \(P_r\) are assumed to be known. If \(P_r\) is chosen to have zero ISI, then \(P_g\) is a delay version of \(P_r\). Therefore, we need to choose \(H_T\) and \(H_R\) such that thee effect of noise is minimized. Lets derive the probability of error. 
\begin{equation*}
    \func{Y}{t_m} = A_m + \func{n_0}{t_m}
\end{equation*}
where \(\func{n_0}{t_m} \sim \func{\NormalDist}{0,N_0}\). The threshold is assumed to be zero.
\begin{remark}
    To minimize error threshold should be set \(\frac{N_0}{2A} \func{\ln}{\frac{\prob{d_m = 0}}{\prob{d_m =1 }}}\).
\end{remark}
\begin{align*}
    P_e = \prob{\hat{d} \neq d} &= \condProb{\func{Y}{t_m} > 0}{d_m = 0} \prob{d_m = 0} + \condProb{\func{Y}{t_m} < 0}{d_m = 1} \prob{d_m = 1}\\
    &= \dfrac{1}{2} \squareBracket{\prob{\func{n_0}{t_m} < -A} + \prob{\func{n_0}{t_m} > A}}\\
    &= \dfrac{1}{2} \prob{\abs{\func{n_0}{t_m}} > A} 
\end{align*}
Since \(\func{n_0}{t_m}\) is assumed to be zero mean gaussian at the input to \(\func{H_R}{f}\), then 
\begin{equation*}
    N_0 = \int_{-\infty}^{\infty} \func{\calG_n}{f} \abs{\func{H_R}{f}}^2 \diffOperator f
\end{equation*}
Hence 
\begin{align*}
    P_e &= \dfrac{1}{2} \int_{\abs{x} > A} \dfrac{1}{\sqrt{2\pi N_0}} \func{\exp}{-\dfrac{x^2}{2N_0}} \diffOperator x\\
    &= \func{\Phi}{\dfrac{A}{\sqrt{N_0}}} = 1 - \func{Q}{\dfrac{A}{\sqrt{N_0}}}
\end{align*}
Thus to minimize \(P_e\) we should maximize \(\frac{A}{\sqrt{N_0}}\). To do this, we express \(\frac{A^2}{N_0}\) in terms of \(H_T\) and \(H_R\). Recall
\begin{equation*}
    \func{X}{t} = \int_{k = -\infty}^{\infty} a_k \func{p_g}{t - kT_b}
\end{equation*}
equiprobable and independent imples \(\func{X}{t}\) is a random waveform with psd 
\begin{align*}
    \func{\calG_X}{f} &= \dfrac{\abs{\func{P_g}{f}}^2}{T_b} \expected{a_k^2}\\
    &= a^2\dfrac{\abs{\func{P_g}{f}}^2}{T_b} 
\end{align*}
psd of the transmitted signal is 
\begin{equation*}
    \func{\calG_Z}{f} = \abs{\func{H_T}{f}}^2 \func{\calG_X}{f}
\end{equation*}
The average power 
\begin{equation*}
    \calS_T = \dfrac{a^2}{T_b} \int_{-\infty}^{\infty} \abs{\func{P_g}{f}}Y2 \abs{\func{H_T}{f}}^2 \diffOperator f
\end{equation*}
setting \(A_k = k_ca_k\) and \(A = K_ca\)
\begin{equation*}
    \calS_T = \dfrac{A^2}{K_c^2T_b} \int_{-\infty}^{\infty} \abs{\func{P_g}{f}}Y2 \abs{\func{H_T}{f}}^2 \diffOperator f
\end{equation*}
Let \(I = \int_{-\infty}^{\infty} \abs{\func{P_g}{f}}Y2 \abs{\func{H_T}{f}}^2 \diffOperator f\) and \(N_0 = = \int_{-\infty}^{\infty} \func{\calG_n}{f} \abs{\func{H_R}{f}}^2 \diffOperator f\)
\begin{equation*}
    A^2 = K_c^2 T_b S_T I^{-1}
\end{equation*}
thus 
\begin{equation*}
    \frac{A^2}{N_0} = \dfrac{K_c^2 T_b S_T}{IJ}
\end{equation*}
hence we should minimize \(\gamma^2 = IJ\)