\chapter{Arithmetical Functions and Dirichlet Multiplication}
\begin{definition}
    A function \(f: \Naturals \to \Complex\) is an arithmetical function.
\end{definition}
\section{Mobius function}
The Mobius function \(\mu\), is defined as \(\func{\mu}{1} = 1\) and for \(n > 1\) if \(n = p_1^{\alpha_1} \dots p_k^{\alpha_k}\)
\begin{equation*}
    \func{\mu}{n} = \begin{cases}
        (-1)^k & \alpha_1 = \dots = \alpha_k = 1\\
        0 & \mathrm{otherwise}
    \end{cases}
\end{equation*}
\begin{theorem}
    If \(n\geq 1\), 
    \begin{equation*}
        \sum_{d \mid n} \func{\mu}{d} = \floor{\frac{1}{n}} = \begin{cases}
            1 & n = 1\\
            0 & \mathrm{otherwise}
        \end{cases}
    \end{equation*}
\end{theorem}
\begin{proof}
    Suppose \(n > 1\) and \(n = p_1^{\alpha_1} \dots p_k^{\alpha_k}\), then 
    \begin{equation*}
        \sum_{d \mid n} \func{\mu}{d} = \sum_{i = 0}^k (-1)^i\binom{k}{i} = (1 - 1)^{k} = 0
    \end{equation*}
    If \(n = 1\), then \(\sum_{d \mid n} \func{\mu}{d} = \func{\mu}{1} = 1\).
\end{proof}
\section{The Euler totient function}
The Euler totient function \(\phi\) is defined as 
\begin{equation*}
    \func{\phi}{n} = \sum_{\substack{k  = 1 \\ (k,n) = 1}}^n 1 = \abs{\set[[\Big]]<1 \leq k \leq n>{(k,n) = 1}}
\end{equation*}
\begin{theorem}
    If \(n \geq 1\), 
    \begin{equation*}
        \sum_{d \mid n} \func{\phi}{d} = n 
    \end{equation*}
\end{theorem}

\begin{proof}
    Define the equivalence relation \(i \sim j\) whenever \((n,i) = (n,j)\) on the numbers \(\leq n\). The divisors of \(n\) can be taken as class representatives. We claim that the size of the class \(d\) is equal to \(\func{\phi}{\frac{n}{d}}\). Note that, if \((n,i) = d\), then \((n/d, i/d) = 1\) and vice versa. That is, there is a bijection between elements of the class \(d\) and numbers that are coprime to \(n/d\) less than \(n/d\). Therefore,
    \begin{equation*}
        n = \sum_{d \mid n} \abs{\text{class}_d} = \sum_{d \mid n } \func{\phi}{\frac{n}{d}} = \sum_{d \mid n} \func{\phi}{d}
    \end{equation*}
\end{proof}

\begin{theorem}
    If \(n \geq 1\), 
    \begin{equation*}
        \func{\phi}{n} = \sum_{d \mid n} \func{\mu}{d} \frac{n}{d}
    \end{equation*}
\end{theorem}

\begin{proof}
    The statement is clearly true for \(n = 1\). Suppose \(n > 1\) and \(n = p_1^{\alpha_1} \dots p_k^{\alpha_k}\). Let \(A_i\) denote the set of all numbers \(k\) less than or equal to \(n\) such that \(p_i \mid (n,k)\). Then, 
    \begin{align*}
        \func{\phi}{n} &= \abs{\bracket{\bigcup_{i = 1}^k A_i}^c} \\
        &= n - \abs{\bigcup_{i = 1}^k A_i}\\
        &= n - \sum_{j = 1}^n (-1)^{j-1}\sum_{i_1 < i_2 < \dots < i_j}  \abs{A_{i_1} \cap \dots \cap A_{i_j}}\\
        &= n + \sum_{j = 1}^n \sum_{i_1 < i_2 < \dots < i_j}  (-1)^{j} \dfrac{n}{p_{i_1} \dots p_{i_j}}\\
        &= n + \sum_{j = 1}^n \sum_{i_1 < i_2 < \dots < i_j} \func{\mu}{p_{i_1} \dots p_{i_j}}  \dfrac{n}{p_{i_1} \dots p_{i_j}}\\
        &= \sum_{d \mid n} \func{\mu}{d}  \dfrac{n}{d}
    \end{align*}
\end{proof}

\subsection{The product formular for \(\func{\phi}{n}\)}
\begin{theorem}
    For any \(n \geq 1\),
    \begin{equation*}
        \func{\phi}{n} = n \prod_{p \mid n}\bracket{1 - \dfrac{1}{p}}
    \end{equation*}
\end{theorem}

\begin{proof}
    If \(n = p_1^{\alpha_1} \dots p_k^{\alpha_k}\) let \(m = p_1 \dots p_k\).
    \begin{align*}
        \func{\phi}{n} &= \sum_{d \mid n} \func{\mu}{d} \dfrac{n}{d}\\
        &=n \sum_{d \mid m}  \dfrac{\func{\mu}{d}}{d} \\
        &= n \bracket{ \sum_{\substack{d \mid m \\ p_1 \mid d}}  \dfrac{\func{\mu}{d}}{d} + \sum_{\substack{d \mid m \\ p_1 \nmid d}}  \dfrac{\func{\mu}{d}}{d}} \\  
        &= n \bracket{\sum_{\substack{d \mid m \\ p_1 \nmid d}}  \dfrac{\func{\mu}{p_1 d}}{p_1 d} + \sum_{\substack{d \mid m \\ p_1 \nmid d}}  \dfrac{\func{\mu}{d}}{d}}\\
        &= n  \bracket{\bracket{1- \dfrac{1}{p_1}} \sum_{\substack{d \mid m \\ p_1 \nmid d}} \dfrac{\func{\mu}{d}}{d}} \\  
        &= n \prod_{p \mid n} \bracket{1 - \frac{1}{p}}    
    \end{align*}
\end{proof}

\begin{corollary}
    \ 
    \begin{enumerate}
        \item \(\func{\phi}{p^{\alpha}} = (p-1) p^{\alpha - 1}\).
        \item \(\func{\phi}{mn} = \func{\phi}{m} \func{\phi}{n} \frac{d}{\func{\phi}{d}}\) where \(d = (m,n)\).
        \item If \(a \mid b\), then \(\func{\phi}{a} \mid \func{\phi}{b}\).
        \item \(\func{\phi}{n}\) is even for \(n \geq 3\). Moreover, if \(n\) has \(r\) distinct odd prime factors, then \(2^r \mid \func{\phi}{n}\).
    \end{enumerate}
\end{corollary}

\begin{proof}
    \ 
    \begin{enumerate}
        \item \(\func{\phi}{p^{\alpha}} = p^{\alpha} \bracket{\frac{p-1}{p}}= (p - 1)p ^{\alpha - 1}\).
        \item 
        \begin{align*}
            \func{\phi}{mn} &= mn \prod_{p \mid mn} \bracket{1 - \dfrac{1}{p}}\\
            &= mn \prod_{\substack{p \mid n \\ p \nmid m}} \bracket{1 - \dfrac{1}{p}} \prod_{\substack{p \mid m \\ p \nmid n}} \bracket{1 - \dfrac{1}{p}} \prod_{p \mid n, m} \bracket{1 - \dfrac{1}{p}} \\
            &= mn \dfrac{\prod_{p \mid n }  \bracket{1 - \dfrac{1}{p}} }{\prod_{p \mid n,m} \bracket{1 - \dfrac{1}{p}}} \dfrac{\prod_{p \mid m }  \bracket{1 - \dfrac{1}{p}} }{\prod_{p \mid n,m} \bracket{1 - \dfrac{1}{p}}}   \prod_{p \mid n,m} \bracket{1 - \dfrac{1}{p}} \\
            &= \func{\phi}{m} \func{\phi}{n} \dfrac{1}{\prod_{p \mid n,m} \bracket{1 - \dfrac{1}{p}}}\\
            &= \func{\phi}{m} \func{\phi}{n} \dfrac{d}{\func{\phi}{d}}
        \end{align*}
        \item Note that if \(p \mid a\), then \(p \mid b\).
        \item If \(n\) has an odd prime factor, then \(\func{\phi}{n}\) is even. If \(n\) is power of \(2\) greater than \(4\), then \(\func{\phi}{n}\) is also even. If \(n\) has \(r\) distinct odd prime factors, each contribute at least one factor of \(2\) in \(\func{\phi}{n}\) and thus \(2^r \mid \func{\phi}{n}\).
    \end{enumerate}
\end{proof}

\section{The Dirichlet product}
\begin{definition}
    Let \(f\) and \(g\) be two arithmetical functions, their \textbf{Dirichlet product} is defined as 
    \begin{equation*}
        \func{(f \ast g)}{n} = \sum_{d \mid n}\func{f}{d} \func{g}{\frac{n}{d}}
    \end{equation*}
\end{definition}
Then, we can write \(\phi = \mu \ast N\) where \(\func{N}{n}  = n\).
\begin{theorem}
    \ 
    \begin{enumerate}
        \item \(f \ast g = g \ast f\).
        \item \((f \ast g) \ast h = f \ast ( g \ast h)\).
    \end{enumerate}
\end{theorem}

\begin{proof}
    \ 
    \begin{enumerate}
        \item 
        \begin{equation*}
            \func{(f \ast g)}{n} = \sum_{d \mid n} \func{f}{d} \func{g}{\frac{n}{d}}  = \sum_{n/d \mid n} \func{f}{\frac{n}{d}} \func{g}{d} \sum_{d \mid n} \func{f}{\frac{n}{d}} \func{g}{d} = \func{(g \ast f)}{n}
        \end{equation*}
        \item 
        \begin{align*}
            \func{((f \ast g) \ast h)}{n} &= \sum_{d \mid n} \func{(f \ast g)}{d} \func{h}{\frac{n}{d}}\\
            &= \sum_{d \mid n} \sum_{k \mid d}  \func{f}{k} \func{g}{\frac{d}{k}} \func{h}{\frac{n}{d}}\\
            &= \sum_{k \mid n} \sum_{k \mid d, d \mid n} \func{f}{k} \func{g}{\frac{d}{k}} \func{h}{\frac{n}{d}}\\
            &= \sum_{k \mid n} \sum_{d \mid n/k} \func{f}{k} \func{g}{\frac{kd}{k}} \func{h}{\frac{n}{kd}}\\
            &= \sum_{k \mid n} \sum_{d \mid n/k} \func{f}{k} \func{g}{d} \func{h}{\frac{n}{kd}}\\
            &= \sum_{k \mid n} \sum_{d \mid n/k} \func{f}{k} \func{(g \ast h)}{\frac{n}{k}}\\
            &= \func{(f \ast (g \ast h))}{n}
        \end{align*}
    \end{enumerate}
\end{proof}

\begin{definition}
    The identity function, \(\func{I}{n} = \floor{\frac{1}{n}}\).
\end{definition}
\begin{theorem}
    For any arithmetical function \(f\), \(I \ast f = f \ast I = f\).
\end{theorem}
\begin{proof}
    Trivial.
\end{proof}

\begin{theorem}
    If \(f\) is an arithmetical function with \(\func{f}{1} \neq 0\), there is a unique arithmetical function \(f^{-1}\), called the Dirichlet inverse of \(f\) such that
    \begin{equation*}
        f \ast f^{-1} = f^{-1} \ast f = I
    \end{equation*}
    Moreover, \(f^{-1}\) is given by \(\func{f^{-1}}{1} = \frac{1}{\func{f}{1}}\) and for \(n > 1\)
    \begin{equation*}
        \func{f^{-1}}{n} = -\frac{1}{\func{f}{1}} \sum_{\substack{d \mid n \\ d < n}} \func{f}{\frac{n}{d}} \func{f^{-1}}{d}
    \end{equation*}
\end{theorem}

\begin{proof}
    It can be easily shown that the given function is a Dirichlet inverse of \(f\). That is, 
    \begin{equation*}
        f \ast f^{-1} = f^{-1} \ast f = I
    \end{equation*}
    Suppose \(g\) is also a Dirichlet inverse of \(f\). Then, 
    \begin{align*}
        g \ast f \ast f^{-1} &= (g \ast f) \ast f^{-1} = I \ast f^{-1} = f^{-1}\\
        &= g \ast (f \ast f^{-1}) = g \ast I = g 
    \end{align*}
    Therefore, \(g= f^{-1}\) and \(f^{-1}\) is unique.
\end{proof}

\begin{remark}
    The set of all arithmetical functions \(f\) with \(\func{f}{1} \neq 0\) is an Abelian group under Dirichlet multiplication.
\end{remark}

\begin{proposition}
    Suppose \(f\) and \(g\) are invertible arithmetical functions, then \((f \ast g)^{-1} = f^{-1} \ast g^{-1}\).
\end{proposition}
\begin{proof}
    We can readily deduct this from the fact that invertible functions form an Abelian group under Dirichlet multiplication. 
\end{proof}
\begin{definition}
    The unit function \(\func{u}{n} = 1\) for all \(n \geq 1\). Since \(\sum_{d \mid n} \func{\mu}{d} = \func{I}{n}\), then \(\mu \ast u = I\) and thus by uniqueness of inverse \(\mu^{-1} = u\).
\end{definition}

\begin{theorem}[Mobius inversion formula]
    If 
    \begin{equation*}
        \func{f}{n} = \sum_{d \mid n} \func{g}{n} 
    \end{equation*}
    then, 
    \begin{equation}
        \func{g}{n} = \sum_{d \mid n} \func{f}{d} \func{\mu}{\frac{n}{d}}
    \end{equation}
\end{theorem}
\begin{proof}
    Since \(f = g \ast u\), then \(g = f \ast u^{-1} = f \ast \mu\).
\end{proof}
\section{The Mangoldt function \(\Lambda\)}
\begin{definition}
    For every integer \(n \geq 1\), we define 
    \begin{equation*}
        \func{\Lambda}{n} = \begin{cases}
            \log p & \mathrm{if\ } n = p^m \mathrm{\ for\ some\ prime\ } p \mathrm{\ and\ } m \geq 1\\
            0 & \mathrm{otherwise}
        \end{cases}
    \end{equation*}
\end{definition}
\begin{theorem}
    For \(n \geq 1\),
    \begin{equation*}
        \func{\log}{n} = \sum_{d \mid n} \func{\Lambda}{d}
    \end{equation*}
    and 
    \begin{equation*}
        \func{\Lambda}{n} = \sum_{d \mid n} \func{\mu}{d} \func{\log}{\frac{n}{d}} = - \sum_{d \mid n} \func{\mu}{d} \func{\log}{d}
    \end{equation*}
\end{theorem}

\begin{proof}
    For the first identity we have 
    \begin{equation*}
        \sum_{d \mid n} \func{\Lambda}{d} = \sum_{p^{\alpha} \mid n} \func{\Lambda}{p^{\alpha}} = \sum_{p^{\alpha} \mid n} \log p = \sum_{p^{\alpha} \mid n} \alpha \log p = \log n
    \end{equation*}
    Hence, \(\log = \Lambda \ast u\). Therfore, \(\Lambda = \log \ast u^{-1} = \log \ast \mu\).
\end{proof}
\section{Multiplicative functions}
\begin{definition}
    An arithmetical function \(f\) is \textbf{multiplicative} if \(f \not\equiv 0\) and 
    \begin{equation*}
        \func{f}{mn} = \func{f}{m} \func{f}{n}
    \end{equation*}
    whenver \((m,n) = 1\). The function \(f\) is said to be \textbf{completely multiplicative} if for all \(m,n\)
    \begin{equation*}
        \func{f}{mn} = \func{f}{m} \func{f}{n}
    \end{equation*}
\end{definition}
\begin{remark}
    Multiplicative functions from a subgroup under \(\ast\). The ordinary multiplication and division of two (completely) multiplicative functions are (completely) multiplicative.
\end{remark}
\begin{proposition}
    If \(f\) is multiplicative, then \(\func{f}{1} = 1\).
\end{proposition}
\begin{proof}
    Since \(f\) is multiplicative, \(\func{f}{1} = \func{f}{1}\func{f}{1}\) thus, \(\func{f}{1} = 0,1\). If \(\func{f}{1} = 0\), then \(f \equiv 0\) which contradicts our assumption hence \(\func{f}{1}\) must be \(1\).
\end{proof}
\begin{theorem}
    Given an arithmetical function \(f\) with \(\func{f}{1} = 1\)
    \begin{enumerate}
        \item  \(f\) is multiplicative if and only if \(\func{f}{\prod p_i^{\alpha_i}} = \prod \func{f}{p_i^{\alpha_i}}\)
        \item If \(f\) is multiplicative, then \(f\) is completely multiplicative if and only if \(\func{f}{p^{\alpha}} = \bracket{\func{f}{p}}^{\alpha}\).
    \end{enumerate}
\end{theorem}

\begin{proof}
    \
    \begin{enumerate}
        \item If \(f\) is multiplicative, then the formula is obviously true. Suppose the formula holds and the integers \(m,n\) are relatively prime. Let \(m = p_1^{\alpha_1} \dots p_k^{\alpha_k}\) and \(n = q_1^{\beta_1} \dots q_r^{\beta_r}\) with no \(p\) equal to a \(q\).
        \begin{equation*}
            \func{f}{mn} = \func{f}{\prod p_i^{\alpha_i} \prod q_i^{\beta_i}} = \prod_{i,j} \func{f}{p_i^{\alpha_i}} \func{f}{q_j^{\beta_j}} = \prod_i \func{f}{p_i^{\alpha_i }} \prod_j \func{f}{q_j^{\beta_j}} = \func{f}{m} \func{f}{n}
        \end{equation*}
        Therefore, \(f\) is multiplicative.
        \item If \(f\) is completely multiplicative, then the formula holds trivially. Suppose the formula holds and \(m,n\) are integers with prime decomposition \(m = p_1^{\alpha_1} \dots p_k^{\alpha_k}\) and \(n = p_1^{\gamma_1} \dots p_k^{\gamma_k} q_1^{\beta_1} \dots q_r^{\beta_r}\) with no \(p\) equal to a \(q\).
        \begin{align*}
            \func{f}{mn} &= \func{f}{\prod p_i^{\alpha_i + \gamma_i} \prod q_i^{\beta_i}}\\
            &= \prod_{i,j} \func{f}{p_i^{\alpha_i + \gamma_i}} \func{f}{q_j^{\beta_j}}\\
            &= \prod_i \bracket{\func{f}{p_i}}^{\alpha_i + \gamma_i} \prod_j \func{f}{q_j^{\beta_j}} \\
            &= \prod_i \bracket{\func{f}{p_i}}^{\alpha_i} \prod_i \bracket{\func{f}{p_i}}^{\gamma_i} \prod_j \func{f}{q_j^{\beta_j}} \\
            &= \prod_i \func{f}{p_i^{\alpha_i}} \prod_i \func{f}{p_i^{\gamma_i}} \prod_j \func{f}{q_j^{\beta_j}}\\
            &= \func{f}{m} \func{f}{n}
        \end{align*}
        
    \end{enumerate}
\end{proof}

\begin{theorem}
    If \(f\) and \(g\) are both multiplicative, then \(f \ast g\) is multiplicative. If \(g\) and \(f \ast g\) are both multiplicative, then \(f\) is multiplicative.
\end{theorem}

\begin{proof}
    Suppose \(f\) and \(g\) are two multiplicative functions and \(m,n\) are two relatively prime integers. Then, 
    \begin{align*}
        \func{f \ast g}{m n} &= \sum_{d \mid mn} \func{f}{d} \func{g}{\frac{mn}{d}}\\
        &= \sum_{\substack{d_m \mid m \\ d_n \mid n}} \func{f}{d_m d_n} \func{g}{\frac{m}{d_m} \frac{n}{d_n}}\\
        &= \sum_{d_m \mid m} \sum_{d_n \mid n} \func{f}{d_m} \func{f}{d_n} \func{g}{\frac{m}{d_m}} \func{g}{\frac{n}{d_n}}\\
        &= \func{f \ast g}{m} \func{f \ast g}{n}
    \end{align*} 
    Let \(g\) be a multiplicative function. We show that \(g^{-1}\) is multiplicative as well. Since \(\func{g}{1} = 1\), then \(\func{g^{-1}}{1} = 1\). 
    % Moreover, suppose \(m,n\) are two relatively prime integers and suppose for all integers less than \(mn\), \(g^{-1}\) is multiplicative. 
    % \begin{align*}
    %     \func{g^{-1}}{mn} &= - \sum_{\substack{d \mid mn \\ d < mn}} \func{g}{\frac{mn}{d}} \func{g^{-1}}{d}\\
    %     &= - \sum_{\substack{d_m \mid m, d_n \mid n \\ d_m < m, d_n < n}} \func{g}{\frac{mn}{d_m d_n}} \func{g^{-1}}{d_m d_n} - \sum_{\substack{d_m \mid m\\ d_m < m}} \func{g}{\frac{m}{d_m}} \func{g^{-1}}{n d_m} \\
    %     & \qquad - \sum_{\substack{d_n \mid n\\ d_n < n}} \func{g}{\frac{n}{d_n}} \func{g^{-1}}{m d_n}\\
    %     &= - \sum_{\substack{d_m \mid m, d_n \mid n \\ d_m < m, d_n < n}} \func{g}{\frac{m}{d_m}} \func{g}{\frac{n}{d_n}} \func{g^{-1}}{d_m} \func{g^{-1}}{d_n} \\
    %     & \qquad + \func{g^{-1}}{m}\func{g^{-1}}{n} + \func{g^{-1}}{n}\func{g^{-1}}{m}\\
    %     &= - \func{g^{-1}}{m} \func{g^{-1}}{n} + 2\func{g^{-1}}{m}\func{g^{-1}}{n}\\
    %     &= \func{g^{-1}}{m}\func{g^{-1}}{n}
    % \end{align*}
    Note that if \(p\) is a prime for \(k \geq 1\) we have, 
    \begin{align*}
        \func{g^{-1}}{p^k} = - \sum_{i = 0}^{k-1} \func{g}{p^{k-i}} \func{g^{-1}}{p^{i}}
    \end{align*}
    Let \(h\) be the multiplicative function that agrees with \(g^{-1}\) on prime powers. Consider the Dirichlet multiplication \(g \ast h\) for \(p_1^{\alpha_1} \dots p_k^{\alpha_k}\) with \(\alpha_i \geq 1\).
    \begin{align*}
        \func{g \ast h}{p_1^{\alpha_1} \dots p_k^{\alpha_k}} &= \sum_{ 0 \leq  i_j \leq \alpha_j} \func{h}{p_1^{i_1} \dots p_k^{i_k} } \func{g}{p_1^{\alpha_1 - i_1} \dots p_k^{\alpha_k  - i_k} }\\
        &= \sum_{ 0 \leq  i_j \leq \alpha_j} \func{h}{p_1^{i_1}} \dots\func{h}{ p_k^{i_k} } \func{g}{p_1^{\alpha_1 - i_1}} \dots \func{g}{p_k^{\alpha_k  - i_k} }\\
        &=  \prod_j \sum_{ 0 \leq  i_j \leq \alpha_j} \func{h}{p_j^{i_j}}\func{g}{p_j^{\alpha_j - i_j}}\\
        &=  \prod_j \sum_{ 0 \leq  i_j \leq \alpha_j} \func{g^{-1}}{p_j^{i_j}}\func{g}{p_j^{\alpha_j - i_j}}\\
        &=  \prod_j \bracket{\sum_{ 0 \leq  i_j < \alpha_j} \func{g^{-1}}{p_j^{i_j}}\func{g}{p_j^{\alpha_j - i_j}} + \func{g^{-1}}{p_j^{\alpha_j}}}\\
        &=  \prod_j \bracket{\sum_{ 0 \leq  i_j < \alpha_j} - \func{g^{-1}}{p_j^{\alpha_j}} + \func{g^{-1}}{p_j^{\alpha_j}}}\\
        &= 0
    \end{align*}
    Also, \(\func{g \ast h}{1} = \func{g}{1} \func{h}{1} = 1\). That is, \(g \ast h = I\) and since Dirichlet inverse is unique it must be that \(g^{-1} = h\).
\end{proof}

\subsection{Inverse of completely multiplicative functions}
\begin{theorem}
    Let \(f\) be a multiplicative function. Then, \(f\) is completely multiplicative if and only if 
    \begin{equation*}
        \func{f^{-1}}{n} = \func{\mu}{n} \func{f}{n}
    \end{equation*}
\end{theorem}

\begin{proof}
    Suppose \(f\) is completely multiplicative and \(\func{g}{n} = \func{\mu}{n} \func{f}{n}\)
    \begin{equation*}
        \func{f \ast g}{n} = \sum_{d \mid n} \func{f}{d} \func{\mu}{d} \func{f}{\frac{n}{d}} = \func{f}{n} \sum_{d \mid n} \func{\mu}{d} = \func{f}{n} \func{I}{n} = \func{I}{n}
    \end{equation*}
    Thus, \(f^{-1} = g\). Suppose \(f\) is a multiplicative function such that \(f^{-1} = \mu f\). Let \(p\) be prime and \(\alpha \geq 1\) be such that \(\func{f}{p^{\alpha}} = \bracket{\func{f}{p}}^{\alpha}\). Then, note 
    \begin{equation*}
        \func{f}{p^{\alpha + 1}} = -\sum_{i = 0}^{\alpha} \func{f}{p^i} \func{f^{-1}}{p^{\alpha + 1 - i}} =- \func{f}{p^{\alpha}} \func{f^{-1}}{p} =\bracket{\func{f}{p}}^{\alpha} \func{f}{p} = \bracket{\func{f}{p}}^{\alpha + 1}
    \end{equation*}
\end{proof}

\begin{remark}
    Note that \(N = \phi \ast u\) and \(\phi = N \ast \mu\) therefore, \(\phi^{-1}= \mu^{-1} \ast N^{-1} = u \ast N^{-1}\). Since \(N\) is completely multiplicative, \(\phi^{-1} = u \ast \mu N\). That is, 
    \begin{equation*}
        \func{\phi^{-1}}{n} = \sum_{d \mid n} d \func{\mu}{d}
    \end{equation*}
\end{remark}
\begin{theorem}
    If \(f\) is multiplicative,
    \begin{equation*}
        \sum_{d \mid n} \func{\mu}{d} \func{f}{d} = \prod_{p \mid n} \bracket{1 - \func{f}{p}}
    \end{equation*}
\end{theorem}

\begin{proof}
    Let \(\func{g}{n} = \sum_{d \mid n} \func{\mu}{d} \func{f}{d}\). Note that \(g = \mu f \ast u\) and thus it is multiplicative. Then, to determine \(g\) we need to evaluate \(\func{g}{p^{\alpha}}\) for prime \(p\) and \(\alpha \geq 1\).
    \begin{equation*}
        \func{g}{p^{\alpha}} = \sum_{d \mid p^{\alpha}} \func{\mu}{d} \func{f}{d} =
         \sum_{d \mid p} \func{\mu}{d} \func{f}{d} = 1 - \func{f}{p}
    \end{equation*}
    As a result, 
    \begin{equation*}
        \func{g}{n} = \prod_{p^{\alpha} \mid\mid n} \func{g}{p^{\alpha}} = \prod_{p \mid n} \bracket{1 - \func{f}{p}}
    \end{equation*}
\end{proof}

\section{Liouville's function \(\lambda\)}
\begin{definition}
    The Liouville function \(\lambda\) is defined as \(\func{\lambda}{1} = 1\) and if \(n = p_1^{\alpha_1} \dots p_k^{\alpha_k}\), then 
    \begin{equation*}
        \func{\lambda}{n} = (-1)^{\alpha_1 + \dots + \alpha_k}
    \end{equation*}
\end{definition}
\begin{theorem}
    For \(n \geq 1\), 
    \begin{equation*}
        \sum_{d \mid n} \func{\lambda}{d} = \begin{cases}
            1 & n \text{ is a perfect square}\\
            0 & \text{otherwise}
        \end{cases}
    \end{equation*}
    and also \(\func{\lambda^{-1}}{n} = \abs{\func{\mu}{n}}\).
\end{theorem}
\begin{proof}
    Note that \(g = \lambda \ast u\) is multiplicative since \(\lambda\) is completely multiplicative. Hence, for a prime \(p\) and \(\alpha \geq 1\) we have 
    \begin{equation*}
        \func{g}{p^{\alpha}} = \sum_{i = 0}^{\alpha} \func{\lambda}{p^i} = \sum_{i = 0}^{\alpha} (-1)^i = \frac{1 - (-1)^{\alpha + 1}}{1 - (-1)} =  \frac{1 + (-1)^{\alpha}}{2} = \begin{cases}
            1 & \alpha \text{ is even}\\
            0 & \alpha \text{ is odd}\\
        \end{cases} 
    \end{equation*}
    Therefore, 
    \begin{equation*}
        \func{g}{n} = \prod_{p^{\alpha} \mid\mid n} \func{g}{p^{\alpha}} = \begin{cases}
            1 & n \text{ is a perfect square}\\
            0 & \text{otherwise}
        \end{cases} 
    \end{equation*}
    Since \(\lambda\) is completely multiplicative, \(\lambda^{-1} = \mu \lambda\). If there is a prime \(p\) such that \(p^2 \mid n\), then \(\func{\mu}{n} = 0\) and \(\func{\mu}{n} \func{\lambda}{n}  = \abs{\func{\mu}{n}}\). If \(n = p_1\dots p_k\), then \(\func{\lambda}{n} = \func{\mu}{n}\) and thus \(\func{\lambda}{n} \func{\mu}{n} = \bracket{\func{\mu}{n}}^2 = \abs{\func{\mu}{n}}\).
\end{proof}
\section{The divisor function \(\sigma_{\alpha}\)}
\begin{definition}
    For all \(\alpha \in \Complex\), \(\func{\sigma_{\alpha}}{n} = \sum_{d \mid n} d^{\alpha} =  N^{\alpha} \ast u\)
\end{definition}
\begin{proposition}
    The divisor function \(\sigma_{\alpha}\) is multiplicative and
    \begin{equation*}
        \func{\sigma_{\alpha}}{p^k} = 1 + p^{\alpha} + \dots + p^{k \alpha} = \begin{cases}
            \dfrac{p^{(k+1)\alpha} - 1}{p^{\alpha} - 1} & \alpha \neq 0\\
            k + 1 & \alpha = 0
        \end{cases}
    \end{equation*}
\end{proposition}

\begin{proof}
    Trivial.
\end{proof}

\begin{theorem}
    For \(n \geq 1\), we have 
    \begin{equation*}
        \func{\sigma_{\alpha}^{-1}}{n} = \sum_{d \mid n} d^{\alpha} \func{\mu}{d} \func{\mu}{\frac{n}{d}}
    \end{equation*}
\end{theorem}
\begin{proof}
    Since \(N^{\alpha}\) is completely multiplicative we have
    \begin{equation*}
        \sigma_{\alpha}^{-1} = (N^{\alpha})^{-1} \ast \mu = N^{\alpha} \mu \ast \mu
    \end{equation*}
\end{proof}

\section{Generalized convolution}
Let \(F: \opop{0}{\infty} \to \Complex\) such that \(\func{F}{x} = 0\) for \(0 < x < 1\). Let \(f\) be an arithmetical function 
\begin{equation*}
    \func{f \circ F}{x} = \sum_{n \leq x} \func{f}{n} \func{F}{\frac{x}{n}}
\end{equation*}
is a function such that \(\func{f \circ F}{x} = 0\) for \(0 < x < 1\) and defined on \(\opop{0}{\infty}\). 
\begin{remark}
    In general, \(\circ\) is not commutative nor associative.
\end{remark}
\begin{theorem}
    Let \(f\) and \(g\) be two arithmetical functions
    \begin{equation*}
        f \circ (g \circ F) = (f \ast g) \circ F
    \end{equation*}
\end{theorem}
\begin{theorem}[Inverse formula] 
    Let \(f\) have inverse \(f^{-1}\), then the equation 
    \begin{equation*}
        \func{G}{x} = \sum_{n \leq x} \func{f}{x} \func{F}{\frac{x}{n}}
    \end{equation*}
    implies 
    \begin{equation*}
        \func{F}{x} = \sum_{n \leq x} \func{f^{-1}}{x} \func{G}{\frac{x}{n}}
    \end{equation*}
\end{theorem}

\begin{proof}
    \begin{align*}
        \func{f \circ (g \circ F)}{x} &= \sum_{n \leq x} \func{f}{n} \func{g \circ F}{\frac{x}{n}}\\
        &= \sum_{n \leq x} \func{f}{n} \sum_{k \leq x/n} \func{g}{k} \func{F}{\frac{x}{nk}}\\
        &= \sum_{n \leq x} \sum_{nk \leq x} \func{f}{n}\func{g}{k} \func{F}{\frac{x}{nk}}\\
        &= \sum_{nk \leq x} \func{f}{n}\func{g}{k} \func{F}{\frac{x}{nk}}\\
        &= \sum_{m \leq x} \sum_{d \mid m} \func{f}{d}\func{g}{\frac{m}{d}} \func{F}{\frac{x}{m}}\\
        &= \sum_{m \leq x} \func{f \ast g}{m} \func{F}{\frac{x}{m}}\\
        &= \func{(f \ast g) \circ F}{x}
    \end{align*}
\end{proof}

\begin{theorem}[Generalized Mobius inversion]
    Let \(f\) be completely multiplicative 
    \begin{equation*}
        \func{G}{x} = \sum_{n \leq x} \func{f}{n}\func{F}{\frac{x}{n}} \iff \func{F}{x} = \sum_{n \leq x} \func{\mu}{n} \func{f}{n} \func{G}{\frac{x}{n}}
    \end{equation*}    
\end{theorem}
\begin{proof}
    We have 
    \begin{equation*}
        \mu f \circ G= f^{-1} \circ G  =  f^{-1}  \circ (f \circ F) = ( f^{-1} \ast f )\circ F = F
    \end{equation*}
\end{proof}
\section{Formal power series}
Definiton of formal power series as usual with equality, sum, and multiplication. Therefore, formal power series form a ring with \(0\) and \(1\). If the leading coefficient is non-zero, then the formal power series is invertible. 
\begin{definition}
    Let \(f\) be an arithmetical function and \(p\) be a prime 
    \begin{equation*}
        \func{f_p}{x} = \sum_{n = 0}^{\infty} \func{f}{p^n}x^n
    \end{equation*}
    is the \textbf{Bell series of \(f\) modulo \(p\)}.
\end{definition}

\begin{theorem}
    If \(f\) and \(g\) are multiplicative, then \(f = g\) if and only if \(f_p = g_p\) for all \(p\). 
\end{theorem}
\begin{proof}
    Trivial.
\end{proof}
\begin{example}
    \begin{align*}
        \func{\mu_p}{x} &= 1-x & \func{I_p}{x} &= 1 & \func{\lambda_p}{x} &= \frac{1}{1 + x}\\
        \func{\phi_p}{x} &= \frac{1-x}{1 - px} & \func{u_p}{x} &= \frac{1}{1 - x} & \func{N^{\alpha}_p}{x} &= \frac{1}{1 - p^{\alpha} x} 
    \end{align*}
\end{example}
\begin{theorem}
    Let \(f\) and \(g\) be two arithmetical functions and \(h = f \ast g\), then \(h_p = f_p g_p\) for all \(p\).
\end{theorem}
\begin{proof}
    We have,
    \begin{align*}
        \func{h_p}{x} &= \sum_{n = 0}^{\infty} \func{h}{p^n} x^n\\
        &= \sum_{n = 0}^{\infty} \sum_{i = 0}^{n} \func{f}{p^i} \func{g}{p^{n-i}} x^n\\
        &= \sum_{i = 0}^{\infty} \sum_{n = i}^{\infty} \func{f}{p^i} \func{g}{p^{n-i}} x^n\\
        &= \sum_{i = 0}^{\infty}  \func{f}{p^i} x^i \sum_{n = i}^{\infty} \func{g}{p^{n-i}} x^{n-i}\\
        &= \sum_{i = 0}^{\infty}  \func{f}{p^i} x^i \sum_{n = 0}^{\infty} \func{g}{p^{n}} x^{n}\\
        &= \func{f_p}{x} \func{g_p}{x}
    \end{align*}
\end{proof}
As a result, 
\begin{equation*}
    \func{\bracket{\sigma_{\alpha}}_p}{x} = \func{N^{\alpha}_p}{x} \func{u_p}{x} = \frac{1}{1- p^{\alpha}x} \frac{1}{1- x} = \frac{1}{1 - (p^{\alpha} + 1)x + p^{\alpha}x^2} = \frac{1}{1 - \func{\sigma_{\alpha}}{p} + p^{\alpha} x^2}
\end{equation*}

\begin{definition}
    The derivative arithmetical function \(f\) is defined by 
    \begin{equation*}
        \func{f'}{n} = \func{f}{n} \func{\log}{n}
    \end{equation*}
\end{definition}

\begin{theorem}
    \ 
    \begin{enumerate}
        \item \((f + g)' = f' + g'\).
        \item \((f \ast g)' = f' \ast g + f \ast g'\).
        \item \((f^{-1})' = -f' \ast (f \ast f)^{-1}\) provided that \(\func{f}{1} \neq 0\). 
    \end{enumerate}
\end{theorem}

\begin{proof}
    \ 
    \begin{enumerate}
        \item \((f + g)' = (f + g)\log = f \log + g \log \).
        \item 
        \begin{align*}
            \func{(f \ast g)'}{n} &= \func{f \ast g}{n} \log n\\
            &= \sum_{d \mid n} \func{f}{d} \func{g}{\frac{n}{d}} \log n\\
            &= \sum_{d \mid n} \func{f}{d} \func{g}{\frac{n}{d}} \bracket{\log d + \log \frac{n}{d}} \\
            &= \sum_{d \mid n} \func{f}{d}\log d \func{g}{\frac{n}{d}}+ \sum_{d \mid n} \func{f}{d} \func{g}{\frac{n}{d}} \log \frac{n}{d}\\
            &= \func{f' \ast g}{n} + \func{f \ast g'}{n}
        \end{align*}
        \item Note that, \((f \ast f^{-1})' = I' = I \log \equiv 0\). From the previous part we have 
        \begin{equation*}
            (f \ast f^{-1})' = f' \ast f^{-1} + f \ast (f^{-1})' = 0 \implies (f^{-1})' = - f^{-1} \ast f' \ast f^{-1} = -f' \ast (f \ast f)^{-1}
        \end{equation*}
    \end{enumerate}
\end{proof}

\section{The Selberg theorem}
\begin{theorem}
    For \(n \geq 1\), 
    \begin{equation*}
        \func{\Lambda}{n} \func{\log}{n} + \sum_{d \mid n} \func{\Lambda}{d} \func{\Lambda}{\frac{n}{d}} = \sum_{d \mid n} \func{\mu}{d} \func{\log^2}{\frac{n}{d}}
    \end{equation*}
\end{theorem}

\begin{proof}
    Recall that \(\Lambda = \mu \ast \log\) and \(\Lambda' = \Lambda \log\) by definition.
    \begin{align*}
        \Lambda \log + \Lambda \ast \Lambda &= \Lambda' + (\mu \ast \log) \ast \Lambda\\
        &= (\mu \ast \log)' + (\mu \ast u')\ast \Lambda\\
        &= \mu' \ast \log + \mu \ast \log' + \squareBracket{(\mu \ast u)' - \mu' \ast u} \ast \Lambda\\
        &= \mu \log \ast \log + \mu \ast \log^2 - \mu\log \ast u \ast \Lambda\\
        &= \mu \log \ast \log + \mu \ast \log^2 - \mu\log \ast \log\\
        &= \mu \ast \log^2 \\
    \end{align*}
\end{proof}

\begin{exercise}
    \item Prove that
    \begin{equation*}
        \frac{n}{\func{\phi}{n}} = \sum_{d \mid n} \dfrac{\func{\mu^2}{d}}{\func{\phi}{d}}
    \end{equation*}
    \begin{solution}
        Note that, both the left hand side \(N/\phi\) and the right hand side \(\mu^2/\phi \ast u\) are multiplicative therefore, it suffices to show that they are equal on prime powers.
        \begin{align*}
            LHS &= \dfrac{p^{\alpha}}{\func{\phi}{p^{\alpha}}} = \dfrac{p^{\alpha}}{p^{\alpha - 1}(p - 1)} = \dfrac{p}{p-1} \\
            RHS &= \sum_{d \mid p^{\alpha}} \frac{\func{\mu^2}{d}}{\func{\phi}{d}} = \dfrac{1}{\func{\phi}{1}} + \frac{1}{\func{\phi}{p}} = \dfrac{p}{p-1} \\
            \implies& LHS = RHS 
        \end{align*}
    \end{solution}
    \item Let \(\func{\nu}{n}\) be the number of distinct prime factors of \(n\) with \(\func{\nu}{1} = 1\). Let \(f = \mu \ast \nu\) and prove that \(\func{f}{n}\) is either \(0\) or \(1\).
    \begin{solution}
        Let \(m,k\) be an integer with \(m,k\geq 1\) and \(p\) a prime such that \((m,p) = 1\). Then, 
        \begin{align*}
            \func{\mu \ast \nu}{p^k m} &= \sum_{d \mid p^k m} \func{\mu}{d} \func{\nu}{\dfrac{p^km}{d}}\\
            &= \sum_{d \mid m}\sum_{l \mid p^k} \func{\mu}{l d} \func{\nu}{\dfrac{p^km}{ld}}\\
            &=  \sum_{d \mid m}\func{\mu}{d} \func{\nu}{\dfrac{p^km}{d}} + \func{\mu}{pd} \func{\nu}{\dfrac{p^{k-1}m}{d}}\\
            &= \sum_{d \mid m}\func{\mu}{d} \bracket{1 + \func{\nu}{\dfrac{m}{d}}} - \func{\mu}{d} \bracket{(1 - \func{I}{k}) + \func{\nu}{\dfrac{m}{d}}}\\
            &= \func{I}{k}\sum_{d \mid m}  \func{\mu}{d}\\
            &= \func{I}{k} \func{I}{m}
        \end{align*}
        Therefore, the value of the function is either \(0\) or \(1\). Moreover, it is only \(1\) for prime numbers.
    \end{solution}
    \item Prove that 
    \begin{equation*}
        \sum_{d^k \mid n} \func{\mu}{d} = \begin{cases}
            0 & \text{if } m^k \mid n \text{ for some } m > 1\\
            1 & \text{otherwise}
        \end{cases}
    \end{equation*}
    \begin{solution}
        Let \(n = m^k r\) with \(m \geq 1\) and \(r\) is \(k_{\cardinalTH}\) power free. That is, there is no integer whose \(k_{\cardinalTH}\) power divides \(r\). Therefore,
        \begin{equation*}
            \sum_{d^k \mid n} \func{\mu}{d} = \sum_{d^k \mid m^k} \func{\mu}{d} = \sum_{d \mid m} \func{\mu}{d} = \func{I}{m}
        \end{equation*}
    \end{solution}
    \item Prove that 
    \begin{equation*}
        \sum_{d \mid n} \func{\mu}{d} \func{\log^m}{d} = 0
    \end{equation*}
    if \(m \geq 1\) and \(n\) has more than \(m\) distinct prime factors.
    \begin{solution}
        Let \(n = p_1^{\alpha_1} \dots p_k^{\alpha_k}\) has \(k\) distinct prime factors.
        \begin{align*}
            \sum_{d \mid n} \func{\mu}{d} \func{\log^m}{d} &= \sum_{d \mid p_1\dots p_k}\func{\mu}{d} \func{\log^m}{d}\\
            &= \sum_{d \mid p_1\dots p_{k-1}} \func{\mu}{d} \func{\log^m}{d} + \func{\mu}{dp_k} \func{\log^m}{dp_k}\\
            &= \sum_{d \mid p_1\dots p_{k-1}} \func{\mu}{d} \func{\log^m}{d} - \func{\mu}{d} \bracket{\log d + \log p_k}^m\\
            &=- \sum_{d \mid p_1\dots p_{k-1}}\sum_{j = 0}^{m-1} \binom{m}{j} \func{\mu}{d}\func{\log^j}{d} \func{\log^{m-j}}{p_k}\\
            &=- \sum_{j = 0}^{m-1} \binom{m}{j} \func{\log^{m-j}}{p_k}  \sum_{d \mid p_1\dots p_{k-1}} \func{\mu}{d} \func{\log^j}{d} 
        \end{align*}
            Assuming that the induction base is true and \(k > m\), then we are done by induction. The base case is when \(m = 1\). Let \(n = p_1^{\alpha_1} \dots p_k^{\alpha_k}\) and \(k \geq 2\), 
            \begin{align*}
                \sum_{d \mid n} \func{\mu}{d} \log d &=-  \func{\log}{p_k}  \sum_{d \mid p_1\dots p_{k-1}} \func{\mu}{d} \\
                &= - \log p_k \func{I}{p_1 \dots p_{k-1}}= 0
            \end{align*}
    \end{solution}
    \item Let \(\func{f}{x}\) be defined for all rational \(x\) in \(0 \leq x \leq 1\) and let 
    \begin{align*}
        \func{F}{n} &= \sum_{k = 1}^n \func{f}{\frac{k}{n}} & \func{F^{\ast}}{n} = \sum_{\substack{k = 1\\ (k,n) = 1}} \func{f}{\frac{k}{n}}
    \end{align*}
    \begin{enumerate}
        \item Show that \(F^{\ast} = F \ast \mu\).
        \item Show that 
        \begin{equation*}
            \func{\mu}{n} =\sum_{\substack{k = 1\\ (k,n) = 1}} e^{2\pi ik/n}
        \end{equation*}
    \end{enumerate}
    \begin{solution}
        \begin{enumerate}
            \item We have,
        \begin{align*}
            \func{F^{\ast}}{n} &= \sum_{k = 1}^n \func{I}{(n,k)} \func{f}{\frac{k}{n}}\\
            &= \sum_{k = 1}^n \sum_{d \mid (n,k)} \func{\mu}{d} \func{f}{\frac{k}{n}}\\
            &= \sum_{d \mid n} \sum_{k = 1}^{n/d} \func{\mu}{d} \func{f}{\frac{dk}{n}}\\
            &= \sum_{d \mid n} \func{\mu}{d} \func{F}{\frac{n}{d}}\\
            &= \func{\mu \ast F}{n}
        \end{align*}
            \item Let \(\func{f}{x} = e^{2 \pi i x}\), then 
            \begin{equation*}
                \func{F}{n} = \sum_{k = 1}^n e^{2 \pi i k/n} = \func{I}{n}
            \end{equation*}
            and thus 
            \begin{equation*}
                \mu \ast F = \mu = F^{\ast} = \sum_{\substack{k = 1\\(k,n) = 1}}^n e^{2 \pi i k/n}
            \end{equation*}
    \end{enumerate}
    \end{solution}
    \item Prove that,
    \begin{equation*}
        \func{\sigma_1}{n} = \sum_{d \mid n} \func{\phi}{d} \func{\sigma_0}{\frac{n}{d}}
    \end{equation*}
    And try to generalize it for \(\sigma_{\alpha}\)
    \begin{solution}
        For integer \(\alpha \geq 1\)
        \begin{align*}
            \sigma_{\alpha} &= N^{\alpha} \ast u = (N^{\alpha - 1} N) \ast u\\
            &=  (N^{\alpha - 1} N)  \ast (N^{\alpha- 1} \mu) \ast (N^{\alpha - 1} \mu)^{-1} \ast u\\
            &= (N^{\alpha - 1} \phi) \ast N^{\alpha - 1} \ast u\\
            &= (N^{\alpha - 1} \phi) \ast \sigma_{\alpha - 1}
        \end{align*}
    \end{solution}
    \item 
\end{exercise}