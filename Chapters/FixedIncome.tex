\chapter{Fixed Income Security}
\textbf{Fixed-income securities} are financial claim with promised cashflows of known fixed amount paid at fixed dates. They include
\begin{description}
    \item[Treasury securities] bills, notes, and bonds.
    \item[Federal agency securities] issued by federal agencies. 
    \item[Corporate securites] Commercial paper, medium-term notes, corporate bonds, \(\dots\) 
    \item[Municipal securities]
    \item[Mortgage-backed securites]
    \item[Derivatives] CDO's , CDS's, \(\dots\)   
\end{description}

As it is not easy to figure out the value of these securites every minute, they trade less often and thus are less liquid.

A fixed-income security is issued by an \textbf{issuer} which typically is a government, corporation , bank , or etc. An issuer issues these securities and get funding, in return they are obliged to pay the security with interest. An \textbf{investor} buys the bonds and loan the fund needed. \textbf{Intermediaries} are institution that facilitate trade of these bonds.