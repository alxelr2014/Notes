\chapter{Micro-Architecture}
Micro-Architecture is the way a given ISA is implemented on a processor.
Basic blocks of a Micro-Architecture:
\begin{description}
    \item [Cache] A high-speed unit to keep code and data.
          \begin{description}
              \item [ICache] holds instructions.
              \item [DCache] holde data.
          \end{description}
    \item [IFU] A unit to fetch instruction from cache.
    \item [IDU] A unit to decode instruction after fetch.
    \item [EU] A unit to execute instruction:
          \begin{description}
              \item [ALU] for arithmetic and logic.
              \item [Branch Unit] for branching instruction.
              \item [L/S Unit] for loading and saving instruction
          \end{description}
    \item [Register File] A unit to save temporary results.
    \item [Program Counter] A unit to locate next instruction.
    \item [Control Unit] A unit to schedule all data movement.
\end{description}

\section{Standard Benchmarks}
\subsection{Performance Metrics}
\textbf{Latency} is the time between start and finish of a single task. The number of taks finished in a give unit of time \textbf{Throughput}.
\textbf{Response time} is the total time to complete a task, also called.
Response time consistes of
\begin{enumerate}
    \item CPU time
          \begin{enumerate}
              \item User CPU time
              \item System CPU time: time spent in OS doing tasks on behalf of a program.
          \end{enumerate}
    \item I/O time
\end{enumerate}
Then \textbf{System Performance} is the inverse time elapsed and \textbf{CPU performance} is the inverse user CPU time. From user perespective response time is more important and from system admin perespective throughput is more important.

For CPU time we have:
\begin{align}
    \text{CPU time } & = \text{CPU clock cycle} \times \text{Clock cycle time} \label{eq:CPUTime1} \\
                     & = \text{CPU clock cycle} / \text{Clock rate} \label{eq:CPUTime2}
\end{align}

To have an estimation for CPU clock cycle we define \textbf{CPI}, \textit{cycle per instruction}, to be the average clock cycles per instruction. Thus the \Cref{eq:CPUTime1,,eq:CPUTime2} become

\begin{align*}
    \text{CPU time } & = \text{Instruction count} \times \text{CPI} \times \text{Clock cycle time} \\
                     & = (\text{Instruction count} \times \text{CPI}) / \text{Clock rate}
\end{align*}

Therefore, CPU time is dependent on:
\begin{enumerate}
    \item Instruction count
          \begin{itemize}
              \item Program (e.g. Algorithm)
              \item ISA
              \item Compiler
          \end{itemize}
    \item CPI
          \begin{itemize}
              \item \(\mu\)Arch
              \item Code CPI also depends on program
          \end{itemize}
    \item Clock cycle time
          \begin{itemize}
              \item \(\mu\)Arch and technology
          \end{itemize}
\end{enumerate}

%TODO:complete from slides
TO evaluate the performance of a computer we use \textbf{benchmarks}.

\begin{itemize}
    \item SPEC Benchmarks
          \begin{itemize}
              \item CPU/Memory intensive benchmarks
              \item some stress CPU
              \item some stress memory subsystem
          \end{itemize}
    \item SPEC Web
          \begin{itemize}
              \item I/O intensive benchmarks
              \item mostly stress I/O subsystem
          \end{itemize}
\end{itemize}

\subsection{Speedup}
\begin{equation*}
    \text{Speedup} = \text{Time}_{\text{original}} / \text{Time}_{\text{improved}}
\end{equation*}

\textbf{Amdahl's law} states that speedup is limited to the improved fraction. That is
\begin{align*}
    \text{Time}_{\text{improved}}   & = \text{Time}_{\text{original}} \times \left[ \left( 1 - \text{Fraction}_{\text{improved}} \right) + \text{Fraction}_{\text{improved}} / \text{Speedup}_{\text{improved}} \right] \\
    \text{Speedup}_{\text{overall}} & = \text{Time}_{\text{original}} / \text{Time}_{\text{improved}}                                                                                                                   \\
                                    & = \dfrac{1}{ \left( 1 - \text{Fraction}_{\text{improved}} \right) + \text{Fraction}_{\text{improved}} / \text{Speedup}_{\text{improved}}}
\end{align*}

\textbf{MIPS}, \textit{million instruction per second}, is a performance metric. The drawbacks of this metric are
\begin{enumerate}
    \item not taking into account capabilites of instructions
    \item not realistic metric even on the same CPU
\end{enumerate}

\textbf{FLOPS}, \textit{floating point operations per second}, is another performance metric that measures floating point operations per unit time.
\begin{equation*}
    \text{FLOPS} = (\text{FP ops}/\text{program}) \times (\text{program} / \text{time})
\end{equation*}
The drawbacks of this metric are
\begin{enumerate}
    \item ignores other instruction (e.g. load/store)
    \item not all floating point operations have common format
    \item depends on how FP-instensive program is
\end{enumerate}

\section{RTL}
Data movement is characterized in terms of
\begin{itemize}
    \item Registers
    \item Operations done on registers
\end{itemize}
\textbf{RTL}, \textit{Register Transfer Language}, describes data movement at register level.
\textbf{Micro-operations} is the functions performed on registers. For example, shift, clear, load, increment, etc.
One can describe a \(\mu\)Arch in terms of RTL and \(\mu\)Ops and control signals that initiate sequence of \(\mu\)Ops to perfrom functions.
