\chapter{Complexity Theory}
\section{Models of computation}
\subsection{Turing machines}
is defined by the tuple \((Q,\Sigma, \Gamma, \delta, q_{acc}, q_{rej})\) where \(\delta: Q \times  \Gamma \to Q \times \Gamma \times \set{L,R,S}\).
\subsection{Circuits}
is defined by gates \(f: \set{0,1}^{k} \to \set{0,1}^l\).
\section{Analysis of computation problems}
\begin{definition}[Strong Church-Turing thesis:] 
    Any model of computation can be simulated  on a probabilistic Turing machine with at most a polynomial increase in the number of elementary operations required.
\end{definition}
The language \(L\) is decided by a Turing machine if the machine is able to decide whether the input \(x\) is a member of \(L\)  or not. That is, on any string it halts. For example, if \(L \in \func{TIME}{\func{f}{n}}\), then a Turing machine can decide \(x\) with \(\abs{x} = n\) in time \(\bigO{\func{f}{n}}\). 
\begin{equation*}
    P = \set<L>{L \in \func{TIME}{n^k}, \; \text{for some finite} \ k}
\end{equation*}
NP are the set of problems not in \(P\) but it can be checked efficiently.
\begin{equation*}
    NP = \set<L>{}
\end{equation*}
%\exists TM \; \suchThat \; \text{if} \(x \in L\) \text{there exists a witness} \(w\) \; \suchThat \; \(TM\) \text{halts at } \(q_{acc}\) \text{starting at} x. \text{Else for all} w, M \text{halts at } \(q_{rej}\)
coNP is the complement of NP. NP-complete if solves in time \(t\) allows any other problem in NP to be solved in \(\bigO{\func{p}{t}}\).

A language \(B\) is reducible to \(A\), if there exists a Turing machine \(TM\), running in polynomial time given \(x\) outputs \(\func{R}{x}\) such that \(x \in B\) if and only if \(\func{R}{x} \in A\).

\begin{proposition}
    If \(L_1\) is reducible to \(L_2\) and \(L_2\) is reducible to \(L_3\), then \(L_1\) is reducible to \(L_3\).
\end{proposition}

A language \(L\) is complete for a class, if \(L\) is in that class and all other languages in that class are reducible to \(L\).
\begin{example}
    Circuit satisfiability is complete for NP. Cook-Levin problem. Given a Boolean circuit with AND, OR, NOT, determine if there is an assignment which output \(1\).
\end{example}

The focus of quantum computer is NPI problems.
\subsection{Space}
PSPACE is the set of all problems that use polynomial nymber of working bits on a Turing machine. \(P \subset NP \subset PSPACE\). If \(P = PSPACE\), then quantum computers are technically worthless. 
\begin{equation*}
    L \subset P \subset NP \subset PSPACE \subset EXP
\end{equation*}
since \(P \subsetneq EXP\), then at least one the of the inequalities is strict. 
\subsection{Approximate algorithms and MASNP}
Random (bound-error probabilistic) BPP and BPQ. done repeatedly gives correct asnwer using Chernoff bound.
\subsection{Energy}
\begin{theorem}[Landauer's first principle] Suppose a computer erases a single bit. The amount of energy dissipated into environment is at least \(k_B T \ln 2\), \(k_B\) is the Boltzmann constant and \(T\) is the temperature.
\end{theorem}

\begin{theorem}[Landauer's second principle] Suppose a computer erases a single bit of information. The entropy of the environment is increased by at least \(k_B \ln 2\)
\end{theorem}
\subsubsection{Reversible and conservative computation}
Fredkin, Toffoli gates.
Reversible computation is highly sensitive to noise and thus we must use an error-correcting code and then need to delete the redundant information which uses energy by Landauer's principles.

Analog computers comput based on continuous degree of freedom. Therefore, are sensitive to noise and thus we need to reduce the number of states from continuous to discrete and finite.

Kau97,98a,98b, Pap94, Min67, Con72,86.