\chapter{Group Theory}
\section{Introduction}
\begin{definition}
    A set \(S\) equipped with an associative binary operation is a \textbf{semigroup}.
\end{definition}
A semigroup can have multiple left or right identities. However, if it has both left identity, \(e\), and right identity, \(f\), then those two are equal since \(e = ef = f\). Two sided identity are unique. We have the same story with inverses.

\begin{definition}
    A non-empty set of elements \(G\) together with a binary operation \(\circ\) are said to be a \textbf{group} if 
    \begin{description}
        \item[Closure:] \(\forall a,b \in G, a \circ b \in G\).
        \item[Associative:] \(\forall a,b,c \in G, (a \circ b ) \circ c = a \circ (b \circ c)\).
        \item[Identity:] \(\exists e \in G\) such that \(\forall a\in G, a \circ e = e \circ a = a\).  
        \item[Inverse:] \(\forall a \in G \; \exists b \in G\) such that \(a \circ b = b \circ a = e\).  
    \end{description}
\end{definition}

\begin{definition}
    A group \(G\) is said to be \textbf{abelian} or \textbf{commutative} if for any two element \(a\) and \(b\) commute. i.e. \(a \circ b = b \circ a\).
\end{definition}

\begin{definition}
    The number of elements in a group is called the \textbf{order} of the group and it is denoted by \(\func{o}{G}\).
\end{definition}

\begin{definition}
    Let \(\angleBracket{a} = \set<a^n>{n \in \Integers}\). If for some choice of \(a\), \(G = \angleBracket{a}\), then \(G\) is said to be a \textbf{cyclic group}. More generally, for a set \(W \subset G\), \(\angleBracket{W} = \bigcap{W \subset H \subset G} H\) where \(H\) is a subgroup of \(G\).
\end{definition}

\begin{lemma}
    Given \(a,b \in G\) the equation \(ax = b\) and \(ya = b\) have unique solutions for \(x,y \in G\).
\end{lemma}

\begin{prooflemma}
    Note that \(a^{-1}\) and \(b^{-1}\) are unique. Therefore, \(x = a^{-1}b\) and \(y = ba^{-1}\) are unique.
\end{prooflemma}
\ \\ 
{\Large{\textbf{Exercises}}}
\begin{enumerate}
    \item Let \(S\) be a finite semi-group. Prove that there exists \(e \in S\) such that \(e^2 = e\).
    \begin{proof}
        Pick \(a \in S\) and consider \(a_i = a^{2^i}\) for \(i \geq 1\). After some point, \(a_i\)s repeat, by the pigeon hole principle. Let that point be \(a_j\). Therefore, for some \(m \geq 1\).
        \begin{equation*}
            a_j = (a_j)^{2^m}
        \end{equation*}
        Let \(e = a_j^{2^m - 1}\), then
        \begin{equation*}
            e^2 = a_j^{2^{m+1} - 2} = a_j^{2^{m}} a_j^{2^m - 2} = a_j a_j^{2^m - 2} = e
        \end{equation*}
        we are done.
    \end{proof}
    \item Show that if a group \(G\) is abelian, then for \(a,b \in G\) and any integer \(n\), \((ab)^n = a^n b^n\).
    \begin{proof}
        Induct over positive \(n\). It is trivially true for \(n = 1\). Suppose it is true for \(n = k\), then 
        \begin{equation*}
            (ab)^{k+1} = (ab)^k ab = a^k b^k a b = a^{k} a b^k b = a^{k+1}b^{k+1}
        \end{equation*}
        For negative \(n\), note that 
        \begin{equation*}
            (ab)^{-1} = b^{-1} a^{-1} = a^{-1} b^{-1} \implies (ab)^{n} = ((ab)^{-1})^{-n} = (a^{-1} b^{-1})^{-n} = a^{n} b^{n}
        \end{equation*}
        hence it is true for all integers \(n\).
    \end{proof}
    \item If a group has an even order, then there exists \(a \neq e\) such that \(a^2 = e\).
    \begin{proof}
        Let \(A = \set<g>{g \neq g^{-1}}\) and \(B = \set<g>{g = g^{-1}}\). Note that, \(\abs{A}\) is even since  \(g  \in A \implies g^{-1} \in A\). Moreover, \(\func{o}{G} = \abs{A} + \abs{B}\), therefore \(\abs{B}\) must be even and since \(e \in B\), \(\abs{B} \geq 2\).
    \end{proof}
    \item For any \(n > 2\) construct a non-abelian group of order \(2n\).
    \begin{proof}
        Consider \(\phi, \psi\) where \(\psi^n = \phi^2 = e \) and \(\psi \phi = \phi \psi^{-1}\). Then
        \begin{equation*}
            G = \set{I, \phi ,\psi, \psi^2, \dots ,\psi^{n-1}, \phi \psi, \dots , \phi \psi^{n-1}}
        \end{equation*}
        is a group of order \(2n\). Because, by the product rules defined, any combination of \(\psi\) and \(\phi\) can be reduced to \(\phi^b \psi^k\) where \(b = 0,1\) and \(k = 0,1 , \dots, n-1\). It is cleary non-abelian as well.
    \end{proof}
    \item Find the order of \(\func{\GL_2}{\Integers_p}\) and \(\func{\SL_2}{\Integers_p}\) for a prime \(p\).
    \begin{proof}
        \begin{align*}
            \func{o}{\func{\GL_2}{\Integers_p}} &= (p+1) p (p-1)^2\\
            \func{o}{\func{\SL_2}{\Integers_p}} &= (p+1) p (p-1)
        \end{align*}
        which be can be calculate with some basic casing.
    \end{proof}
\end{enumerate}

\section{Subgroup}
\begin{definition}
    A non-empty subset \(H\) of a group \(G\) is called a \textbf{subgroup} if under the product in \(G\), \(H\) itself forms a group.
\end{definition}

\begin{lemma}\label{lm:subgroupConditions}
    \(H\) is a subgroup of \(G\) if and only if 
    \begin{enumerate}
        \item \(\forall a,b \in H, ab \in H\).
        \item \(\forall a \in H, a^{-1} \in H\).
    \end{enumerate}
\end{lemma}

\begin{prooflemma}
  If \(H\) is a subgroup, then the conditions hold. Suppose \(H\) is a subset of \(G\) that satisfies the conditions. Then, 
  \begin{enumerate}
      \item \(e \in H\) since \((a \in H \implies a^{-1} \in H) \implies e = a a^{-1} \in H\).
      \item Associativity is inherited from \(G\).
  \end{enumerate}
  invertibility and closure are given from the conditions. Therefore, \(H\) is a subgroup.
\end{prooflemma}

\begin{lemma}
    If \(H\) is a non-empty finite subset of a group \(G\) and \(H\) is closed under multiplication, then \(H\) is a subgroup of \(G\).
\end{lemma}

\begin{prooflemma}
    Since \(H\) is non-empty there exists a \(a \in H\). By closure, \(a^n\) for positive integer \(n\), are also in \(H\). We know that for some \(N\), \(a^N = e\) and therefore \(a^{-1} = a^{N - 1} \in H\). By \href{lm:subgroupConditions}, \(H\) is a subgroup.
\end{prooflemma}

\begin{definition}
    Let \(G\) be a group and \(H\) a subgroup of \(G\). For \(a,b \in G\) we say that \(a\) is congruent to \(b \mod{H}\), written as \(a \equiv b \mod{H}\) if \(ab^{-1} \in H\).
\end{definition}

\begin{lemma}
    The relation \(a \equiv b \mod{H}\) is an equivalence relation.
\end{lemma}

\begin{prooflemma}
    We show the equivalence axioms:
    \begin{enumerate}
        \item for any \(a\), \(a \equiv a \mod H\) becuase, \(aa^{-1} = e \in H\).
        \item for any \(a,b\), \(a \equiv b \mod H \implies b \equiv a \mod H\) since \(ab^{-1} \in H \) because of invertibility implies that \( (ab^{-1})^{-1} = ba^{-1} \in H\).
        \item for any \(a,b,c\), \(a \equiv b \mod H, b \equiv c \mod H \implies a \equiv c \mod H\) since \(ab^{-1}, bc^{-1} \in H \) because of closure implies that \( ab^{-1} bc^{-1} = bc^{-1} \in H\).
    \end{enumerate}
\end{prooflemma}

\begin{definition}
    If \(H\) is a subgroup of \(G\) and \(a \in G\), then \(Ha = \set<ha>{h \in H}\) is a \textbf{right coset} of \(H\) in \(G\). Similary, \(aH = \set<ah>{h \in H}\) is a \textbf{left coset} of \(H\) in \(G\).
\end{definition}

\begin{lemma}\label{lm:cosetsAreEquivalenceClasses}
    For all \(a \in G\), 
    \begin{equation*}
        Ha = \set<x \in G>{a \equiv x \mod{H}}
    \end{equation*}
\end{lemma}

\begin{prooflemma}
    Suppose \(x \in G\) and \(x \equiv a \mod{H}\). That is, \(xa^{-1} = h\) for some \(h \in H\). Then, \(x = ha\). Suppose \(h \in H\) and \(x = ha\). Then, \(xa^{-1} = h\) and hence \(x \equiv a \mod{H}\).
\end{prooflemma}

This implies, two right/left coset of \(H\) are either identical or disjoint.

\begin{lemma}\label{lm:cosetsHaveSameCardinality}
    There is a one-to-one correspondence between any two right/left cosets of \(H\).
\end{lemma}

\begin{prooflemma}
    Let \(R_1,R_2\) be two right cosets of \(H\) with \(a_1 \in R_1\) and \(a_2 \in R_2\). Note that, \(R_1 = Ha_1\) and \(R_2 = Ha_2\), therefore the map \(g \mapsto ga_1^{-1} a_2\) is a bijective map from \(R_1\) to \(R_2\).
\end{prooflemma}

\begin{theorem}[Lagrange's theorem]
    If \(G\) is a finite group and \(H\) is a subgroup of \(G\), then \(\func{o}{H} \mid \func{o}{G}\).
\end{theorem}

\begin{proof}
    By \href{lm:cosetsAreEquivalenceClasses} and \href{lm:cosetsHaveSameCardinality}, and from finiteness of \(G\), the order of \(G\) is equal to the number of right cosets multiplied by the cardinality of a right coset which is equal to the order of \(H\). Hence, \(\func{o}{H} \mid \func{o}{G}\)
\end{proof}

\begin{definition}
    If \(H\) is a subgroup of \(G\), the \textbf{index} of \(H\) in \(G\) is the number of distince right cosets of \(H\), denoted by \([G:H]\) or \(\func{i_G}{H}\).
\end{definition}

\begin{definition}
    Let \(G\) be a group and \(a \in G\), then the \textbf{order} or \textbf{period} of \(a\) is the least positive integer \(m\) such that \(a^m = e\). If no such integer exists we say that \(a\) is of infinite order. The order of \(a\) is denoted by \(\func{\ord_G}{a}\).
\end{definition}

\begin{corollary}
    If \(G\) is a finite group, then 
    \begin{enumerate}
        \item \(\func{o}{G} = \func{i_G}{H} \func{o}{H}\).
        \item \(\func{\ord_G}{a} \mid \func{o}{G}\).
        \item \(a^{\func{o}{G}} = e\).
        \item If \(\func{o}{G}\) is a prime, then \(G\) is cyclic.
    \end{enumerate}
\end{corollary}

\section{A counting principle}
Let \(H\) and \(K\) be two subgroups of \(G\), then 
\begin{equation*}
    HK = \set<hk>{h \in H, k \in K}
\end{equation*}

\begin{lemma}
    \(HK\) is a subgroup of \(G\) if and only if \(HK = KH\).
\end{lemma}
\begin{proof}
    Suppose \(HK\) is a subgroup. If \(hk \in HK\), then
    \begin{equation*}
        k^{-1}h^{-1} \in HK \implies k^{-1} \in H, h^{-1} \in K \implies k \in H,h \in K \implies hk  \in KH 
    \end{equation*}
    hence \(HK \subset KH\). If \(kh \in KH\), then 
    \begin{equation*}
        hk \in HK  \implies k^{-1} \in H, h^{-1} \in K \implies k \in H,h \in K \implies kh \in HK 
    \end{equation*}
    thus \(HK = KH\). Suppose \(HK = KH\) with \(h_1k_1, h_2k_2 \in HK\).
    \begin{enumerate}
        \item for closure we have 
        \begin{equation*}
            h_1k_1 h_2 k_2 = h_1 k_1(k_2' h_2') = h_1 (k_1k_2') h_2' = h_1 (k^{\ast} h_2') = h_1 h_2'' k^{\ast'}
        \end{equation*}
        \item for inverse 
        \begin{equation*}
            (h_1k_1)^{-1} = k_1^{-1} h_1^{-1} = h_1' k_1'
        \end{equation*}
    \end{enumerate}
\end{proof}
\begin{corollary}
    If \(H\) and \(K\) are subgroups of an abelian group \(G\), then \(HK\) is a subgroup of \(G\).
\end{corollary}

\begin{lemma}
    If \(H\) and \(K\) are finite subgroups \(G\), then 
    \begin{equation*}
        \abs{HK} = \dfrac{\func{o}{H} \func{o}{K}}{\func{o}{H \cap K}}
    \end{equation*}
\end{lemma}

\begin{prooflemma}
    If \(h_1 \in H \cap K\) then \(hk = (hh_1)(h_1^{-1}k)\). Therefore, \(hk\) appears at least \(\func{o}{H \cap K}\) times. If \(hk = h'k'\), then \(h'^{-1}h = k' k^{-1} \in H \cap K\). Let \(u = h'^{-1}h\) then \(h' = hu^{-1}\) and \(k' = uk\). Thus, all duplicates are accounted for.
\end{prooflemma}

\begin{corollary}
    If \(H\) and \(K\) are subgroups of \(G\) and \(\func{o}{H},\func{o}{K} > \sqrt{\func{o}{G}}\), then \(H \cap K \neq \set{e}\).
\end{corollary}

\begin{proof}
    \(HK \subset G\) therefore, \(\abs{HK} \leq \func{o}{G}\) and 
    \begin{equation*}
        \func{o}{G} \geq \abs{HK}  = \dfrac{\func{o}{H} \func{o}{K}}{\func{o}{H \cap K}} > \dfrac{\func{o}{G}}{\func{o}{H \cap K}} 
    \end{equation*}
    which implies that \(\func{o}{H \cap K} > 1\).
\end{proof}
\ \\ 
{\Large{\textbf{Exercises}}}
\begin{enumerate}
    \item Let \(G\) be a group such that the intersection of all of its subgroups that are different from \(\set{e}\) is different from \(\set{e}\). Prove that every element in \(G\) has finite order.
    \begin{proof}
        For the sake of contradiction, suppose \(a \in G\) has infinite order. Then, \(a^k\) are all different and 
        \begin{equation*}
            \bigcup_{k = 1}^{\infty} \angleBracket{a^k} = \set{e}
        \end{equation*}
        which is a contradiction.
    \end{proof}
    \item Show that there is one-to-one correspondence between the right and left cosets of a subgroup.
    \item Suppose \(H\) and \(K\) are finite index subgroups in \(G\). Show that \(H \cap K\) is a finite subgroup in \(G\).
    \begin{proof}
        Let \(Ha_1, \dots , Ha_n\) be the right cosets of \(H\) in \(G\) and \(Kb_1, \dots , Kb_m\) be the right costs of \(K\) in \(G\). Then, 
        \begin{equation*}
            G = G \cap G = \bigcap_i Ha_i \cap \bigcap_j Kb_j = \bigcap_{i,j} Ha_i \cap Kb_j
        \end{equation*}
        Suppose \(Ha_i \cap Kb_j\) is not empty. Let \(g \in Ha_i \cap Kb_j\), then \(Hg = Ha_i\) and \(Kg = Kb_j\). Thus, 
        \begin{equation*}
            Ha_i \cap Kb_j = Hg \cap Kg = (H \cap K)g
        \end{equation*}
        Therefore, \(Ha_i \cap Kb_j\) are either empty or a right coset of \(H \cap K\). Since there finitely many \(Ha_i \cap Kb_j\), there finitely many right cosets of \(H \cap K\) in \(G\). Moreover, \(\squareBracket{G:H \cap K} \leq \squareBracket{G:H} \squareBracket{G:K}\) by this construction. Note that, \(H \cap K\) is finite index in \(H\), and let \((H \cap K)c_1, \dots , (H \cap K)c_l\) be the right cosets of \(H \cap K\) in \(H\). We claim that \((H \cap K)c_ra_i\) are the right cosets of \(H \cap K\) in \(G\). By definition, for each \(x \in G\), there exists \(i\) such that \(x \in Ha_i\)and hence \(x = ha_i\) for some \(h \in H\). Similary, there exists \(r\) such that \(h \in (H \cap K)c_r\) and hence \(h = fc_r\) for some \(f \in H \cap K\). Therefore, \(x = fc_ra_i\) and \(x \in (H \cap K)c_ra_i\). Lastly, we must show that \((H \cap K)c_ra_i\) are disjoint. Consider \((H \cap K)  c_{r_1}a_{i_1}\) and \((H \cap K)c_{r_2}a_{i_2}\). Since \((H \cap K)c_{r_1}, (H \cap K)c_{r_2} \subset H\), then 
        \begin{align*}
            (H \cap K)c_{r_1}a_{i_1} = (H \cap K)  c_{r_2}a_{i_2} &\implies a_{i_1} = a_{i_2}, (H \cap K)  c_{r_1} = (H \cap K)  c_{r_2}\\
            &\implies a_{i_1} = a_{i_2}, c_{r_1} = c_{r_2}
        \end{align*}
        As a result, \(\squareBracket{G:H \cap K} = \squareBracket{G:H} \squareBracket{H:H\cap K}\).
        \end{proof}
    \item Let \(H\) be a finite index subgroup in \(G\). Show that there is only finitely many subgroups of form \(aHa^{-1}\) in \(G\).
    \begin{proof}
        Let \(a_1H, \dots, a_nH\) be left cosets of \(H\). Then, \(Ha_1^{-1}, \dots, Ha_n^{-1}\) are right cosets of \(H\). Suppose \(aH = a_iH\), then \(Ha^{-1} = Ha_i^{-1}\) and therefore, \(aHa^{-1} = a_{i} H a_{i}^{-1}\). Since there are finitely many \(a_i H a_i^{-1}\), then there are finitely many \(aHa^{-1}\).
    \end{proof}
    \item If an abelian group has subgroups of orders \(m\) and \(n\), respectively, then show it has a subgroup whose order is the least common multiple of \(m\) and \(n\).
    \item Let \(G\) be a finite (abelian) group in which the number of solutions in \(G\) of the equation \(x^n = e\) is at most \(n\) for every positive integer \(n\). Prove that \(G\) must be a cyclic group.
\end{enumerate}
\section{Normal subgroups}
\begin{definition}
    A subgroup \(N\) of \(G\) is \textbf{normal} if \(\forall g \in G,n \in N,\ gng^{-1} \in N\).
\end{definition}

\begin{lemma}\label{lm:normalEquivalency}
    \(N\) is normal if and only if \(gNg^{-1} = N\) for every \(g \in G\).
\end{lemma}

\begin{prooflemma}
    By definition, \(gNg^{-1} \subset N\). Let \(n \in N\), then \(g^{-1}ng = n'\) for some \(n' \in N\). Hence, \(n \in gNg^{-1}\) for all \(n \in N\).
\end{prooflemma}

\begin{lemma}
    \(N\) is a normal subgroup if and only if every left coset of \(N\) is a right coset.
\end{lemma}

\begin{prooflemma}
    If \(N\) is normal, then by \ref{lm:normalEquivalency}, \(gN = Ng\) for all \(g\). Suppose, for all \(g \in G\), \(gN = Nh\) for some \(h \in G\). Then, \(h = gn \implies gN = Ngn\) for \(n \in N\). This implies, \(gNn^{-1} = gN = Ng\) and therefore, \(gNg^{-1} = N\) which by \ref{lm:normalEquivalency} means that \(N\) is normal.
\end{prooflemma}

\begin{lemma}
    \(N\) is a normal subgroup if and only if the product of two right cosets of \(N\) is a right coset as well.
\end{lemma}

\begin{prooflemma}
    If \(N\) is normal, then 
    \begin{equation*}
        Na Nb = N (aN)b = N (Na)b = Nab
    \end{equation*}
    Then, suppose \(NaNb = Nc\) for all \(a,b \in G\) and some \(c \in G\). This implies \(NaNb = Nab\) and therefore, \(NaNa^{-1} = N \implies NaN = Na\).
    \begin{align*}
        NaN = Na &\implies \forall n, an \in Na \implies  aN \subset Na\\
        Na^{-1}N = Na^{-1} &\implies \forall n \exists n', a^{-1}n = n'a^{-1} \implies na = an' \implies Na \subset aN \\
    \end{align*}
    therefore, \(aN = Na\).
\end{prooflemma}

\begin{definition}
    \(G/N\) is called a \textbf{quotient group} is the set of all right cosets of \(N\).
\end{definition}

\begin{theorem}
    If \(N\) is normal in \(G\), then \(G/N\) is a group. Furthermore, for finite \(G\), \(\func{o}{G/N} = \frac{\func{o}{G}}{\func{o}{N}}\).
\end{theorem}

\begin{proof}
    Checking axioms is pretty easy. Note that, \(\func{o}{G/N} = \func{i_G}{N}\).
\end{proof}
\ \\ 
{\Large{\textbf{Exercises}}}
\begin{enumerate}
    \item The groups in which all subgroups are normal are called \textbf{Dedekind groups}. Non-abelian dedekind groups are called \textbf{Hamiltonian groups}. Show that quaternion group is a Hamiltonian group.
    \item Show that if \(K\) is a normal subgroup of \(N\) and \(N\) is a normal subgroup of \(G\), then \(K\) is not necessarily a subgroup of \(G\).
    % TODO: Dihedral group
\end{enumerate}

\section{Homomorphism}
\begin{definition}
    A mapping \(\phi\) from a group \(G\) to another group \(\bar{G}\) is a \textbf{homomorphism} if for all \(a,b \in G\)
    \begin{equation*}
        \func{\phi}{ab} = \func{\phi}{a} \func{\phi}{b}
    \end{equation*}
\end{definition}

\begin{lemma}
    Suppose \(G\) is a group, \(N\) a normal subgroup of \(G\), \(\phi: G \to G/N\) given by \(\func{\phi}{x} = Nx\) for all \(x \in G\). Then, \(\phi\) is a homomorphism.
\end{lemma}

\begin{prooflemma}
    Note that \(\func{\phi}{xy} = Nxy\) and \(\func{\phi}{x} \func{\phi}{y} = NxNy = Nxy\).
\end{prooflemma}

\begin{definition}
    If \(\phi\) is a homomorphism of \(G\) into \(\bar{G}\), the \textbf{kernel} of \(\phi\), \(K_{\phi}\) is defined as \(K_{\phi} = \set<x \in G>{\func{\phi}{x} = \bar{e}}\).
\end{definition}

\begin{lemma}
    If \(\phi: G \to \bar{G}\) is a homomorphism, then  
    \begin{enumerate}
        \item \(\func{\phi}{e} = \bar{e}\).
        \item \(\func{\phi}{x^{-1}} = \bracket{\func{\phi}{x}}^{-1}\).
    \end{enumerate}
\end{lemma}

\begin{prooflemma}
    \begin{equation*}
        \func{\phi}{xe} = \func{\phi}{x} = \func{\phi}{x}\func{\phi}{e} \implies \func{\phi}{e} = \bar{e}
    \end{equation*}
    and 
    \begin{equation*}
        \func{\phi}{x^{-1}} \func{\phi}{x} = \func{\phi}{x^{-1}x} = \bar{e} \implies \func{\phi}{x^{-1}} = \bracket{\func{\phi}{x}}^{-1}
    \end{equation*}
    \
\end{prooflemma}

\begin{lemma}
    If \(\phi\) is a homomorphism, then \(K_{\phi}\) is a normal subgroup of \(G\).
\end{lemma}

\begin{prooflemma}
    Pick an arbitray \(x \in G\) and \(y \in K_{\phi}\). Then, 
    \begin{equation*}
        \func{\phi}{x y x^{-1}} = \func{\phi}{x} \func{\phi}{y} \func{\phi}{x^{-1}} = \bar{e}
    \end{equation*}
    hence, \(xyx^{-1} \in K_{\phi}\).
\end{prooflemma}

\begin{lemma}\label{lm:homomorphismInverse}
    If \(\phi\) is a homomorphism, then the set all iverse images of \(\bar{g} \in \bar{G}\) under \(\phi\) is given by \(K_{\phi} x\) for any particular inverse image of \(\bar{g}\).
\end{lemma}

\begin{prooflemma}
    Suppose \(y\) is another inverse image of \(\bar{g}\). 
    \begin{align*}
        \func{\phi}{y} &= \bar{g} & \func{\phi}{x} &=\bar{g}\\
        \implies & \func{\phi}{yx^{-1}} = \bar{e} &\implies& yx^{-1} \in K_{\phi}
    \end{align*}
    which means \(y \in K_{\phi}{x}\). Also, clearly each \(y \in K_{\phi}{x}\) is an inverse image of \(\bar{g}\).
\end{prooflemma}

\begin{definition}
    A homomorphism \(\phi: G \to \bar{G}\) is an \textbf{isomorphism} if \(\phi\) is \underline{one-to-one}.
\end{definition}

\begin{definition}
    Two groups \(G\) and \(\bar{G}\) are \textbf{isomorphic} if there exists an isomorphism of \(G\) \underline{onto} \(\bar{G}\). Isomorphic groups are denoted by \(G \approx \bar{G}\).
\end{definition}

\begin{corollary}
    Let \(\phi\) be a homomorphism. Then, \(\phi\) is an isomorphism if and only if \(K_{\phi} = \set{e}\).
\end{corollary}

\begin{prooflemma}
    If \(\phi\) is an isomorphism, then it is injective and hence only \(e \in K_{\phi}\). Suppose \(K_{\phi} = \set{e}\), then we must show that \(\phi\) is a injective function. Suppose \(\func{\phi}{x} = \func{\phi}{y}\), then by \ref{lm:homomorphismInverse}, \(y x^{-1} \in K_{\phi}\). Thus, \(y = x\) and \(\phi\) is injective.
\end{prooflemma}

\begin{theorem}\label{thm:surjectiveHomomorphism}
    If \(\phi: G \to \bar{G}\) is a surjective homomorphism, then \(G/K_{\phi} \approx \bar{G}\)
\end{theorem}

\begin{prooflemma}
    Consider the following mapping, \(\psi: G/K_{\phi} \to \bar{G}\). For any \(X \in K/\phi\), \(\func{\psi}{X} = \func{\phi}{g}\) for some \(g \in X\). This is well-defined since if \(g,g' \in X\), then \(g' = xg\) for some \(x \in K_{\phi}\) and hence 
    \begin{equation*}
        \func{\phi}{g'} = \func{\phi}{g} \func{\phi}{x} = \func{\phi}{g}
    \end{equation*}
    Furthermore, \(\psi\) is injective. Suppose \(xK_{\phi},yK_{\phi} \in G/K_{\phi}\). Then,
    \begin{equation*}
        \func{\psi}{xK_{\phi}} = \func{\psi}{yK_{\phi}} \implies \func{\phi}{x} = \func{\phi}{y} \implies xy^{-1} \in K_{\phi}
    \end{equation*}
    which implies that \(x \in K_{\phi} y\) and hence \(K_{\phi} y = K_{\phi} x\). Moreover, this map is surjective. Let \(\bar{g} \in \bar{G}\). Since \(\phi\) is surjective, then there exists an inverse image \(g\). Therefore, \(\func{\psi}{gK_{\phi}} = \bar{g}\). Finally, we must show that \(\psi\) is a homomorphism. Since \(K_{\phi}\) is normal in \(G\) we have 
    \begin{equation*}
        \func{\psi}{xK_{\phi} y K_{\phi}} = \func{\psi}{xy K_{\phi}}= \func{\phi}{xy} = \func{\phi}{x}\func{\phi}{y} = \func{\psi}{xK_{\phi}} \func{\psi}{yK_{\phi}}
    \end{equation*}
    which concludes the proof.
\end{prooflemma}

Thus, we can find all homomorphic images of \(G\) by going through normal subgroups of \(G\).

\begin{definition}
    A group is \textbf{simple} if it has no non-trivial homomorphic images. i.e. it has no non-trivial normal subgroup.
\end{definition}

\begin{theorem}[Cauchy's theorem for finite abelian groups]\label{thm:CauchyForAbelian}
    Suoppose \(G\) is a finite abelian group, and \(p \mid \func{o}{G}\) where \(p\) is a prime number. Then, there is an element \(a \neq e\) such that \(a^p = e\).
\end{theorem}

\begin{proof}
    We induct over \(\func{o}{G}\). For \(G\) with a single element, the theorem is true trivially. If \(G\) has non-trivial subgroup \(H\), then \(G\) is cyclic and hence all its elements satisfy the condition. Suppose \(H\) is a non-trivial group of \(G\). Since \(G\) is abelian, then \(H\) is normal in \(G\). If \(p \mid \func{o}{H}\) then by induction we are done. Suppose otherwise, then \(p \mid \func{o}{G/H}\). Consder a set \(S\) where each element correspond to a right coset of \(H\). Clearly, there is a isomorphism between \(G/H\) and \(S\). Since \(S\) is a subgroup of \(G\) and \(p \mid \func{o}{S}\) by induction hypothesis we are done.
\end{proof}

\begin{theorem}[Sylow's theorem for finite abelian groups]
    Suppose the group \(G\) is a finite abelian group and \(p^{\alpha} \mid\mid \func{o}{G}\), then \(G\) has a unique subgroup of order \(p^{\alpha}\).
\end{theorem}

\begin{proof}
    We first prove the existence of such group. For \(\alpha = 0\), the claim holds trivially as \(\set{e}\) is a subgroup of order 1. . Suppose \(H = \set<x \in G>{x^{p^n} = e}\) is a subgroup of \(G\). Since \(p \mid \func{o}{G}\) there is a non identity element \(g\) such that \(g^p = e\). Hence \(g \in H\). We show that \(q \not \mid \func{o}{H}\) for any other prime \(q \neq p\). Since otherwise there is a an element \(h \in H\) where \(h \neq e \) and \(h^q = e\) by \ref{thm:CauchyForAbelian}. Since \(q\) and \(p^n\) are coprime, then \(h = e\) which is a contradiction. Lastly, we claim that \(p^{\alpha} \mid\mid \func{o}{H}\). Suppose the contrary that \(p^{\beta} \mid\mid \func{o}{H}\) for some \(\beta < \alpha\). Then, the quotient group of \(H\), \(p \mid \func{o}{G/H}\). By \ref{thm:CauchyForAbelian}, there is a right coset \(Hx \neq H\) such that \((Hx)^p = Hx^p = H\). This implies that \(x^p \in H\) which means \((x^p)^{p^n} = e\) for some \(n\). \(x^{p^{n+1}} = e \implies x \in H\). which is a contradtion. Thus, \(\func{o}{H} = p^{\alpha}\).

    Finally, suppose \(K \neq H\) is another subgroup of \(G\) such that \(\func{o}{K} = p^{\alpha}\). Then, note that 
    \begin{equation*}
        \abs{HK} = \dfrac{\func{o}{H} \func{o}{K}}{\func{o}{H \cap K}} = \dfrac{p^{2\alpha}}{\func{o}{H \cap K}} \implies p^{\gamma} \mid\mid \abs{HK}
    \end{equation*}
    However, this is a contradiction since  \(HK\) is a subgroup in \(G\). Therefore \(H\) is unique in \(G\).
\end{proof}

\begin{lemma}
    Suppose \(\phi: G \to \bar{G}\) is a surjective homomorphism and \(\bar{H}\) is a subgroup of \(\bar{G}\). Let \(H = \set<x \in G>{\func{\phi}{x} \in \bar{H}}\). Then, \(H\) is a subgroup of \(G\) and \(H \supset K_{\phi}\). If \(\bar{H}\) is normal in \(\bar{G}\), then \(H\) is normal. Moreover, this association sets up a one-to-one mapping from the set of all subgroups \(\bar{G}\) onto the set of all subgroups of \(G\) which contain \(K_{\phi}\).
\end{lemma}

\begin{prooflemma}
    Since \(\bar{e} \in \bar{H}\), then \(K_{\phi} \subset H\). Let \(x,y \in H\). \(xy \in H\) since \(\func{\phi}{xy} = \func{\phi}{x} \func{\phi}{y} \in \bar{H}\) and \(x^{-1} \in H\) since \(\func{\phi}{x^{-1}} = \bracket{\func{\phi}{x}}^{-1} \in \bar{H}\). Thus, \(H\) is a subgroup in \(G\). Assume that \(\bar{H}\) is normal and pick arbitray elements \(g \in G\) and \(h \in H\). 
    \begin{equation*}
        \func{\phi}{ghg^{-1}} = \func{\phi}{g} \func{\phi}{h} \bracket{\func{\phi}{g}}^{-1} \in \bar{H } \implies ghg^{-1} \in H
    \end{equation*}
    hence \(H\) is normal in \(G\). Let \(\bar{H},\bar{H'}\) be two subgroups of \(\bar{G}\) and \(H = \func{\phi^{-1}}{\bar{H}},H'= \func{\phi^{-1}}{\bar{H'}}\). Thus far we have proved that \(H,H' \supset K_{\phi}\) are subgroups of \(G\) and \(\phi^{-1}\) is surjective. If \(\bar{H} \neq \bar{H'}\), then there is an element \(x \in \bar{H}\) but \(x \notin \bar{H'}\). We should see that for any \(y = \func{\phi^{-1}}{x}\), \(y \subset H\) but \(y \notin H'\). Since \(\func{\phi}{y} = x \in \bar{H}\), then \(y \in H\). If \(y \in H'\), then \(\func{\phi}{y} = x \in \bar{H'}\) which is a contradiction. Therefore, \(\phi^{-1}\) is a injective as well. So \(\phi^{-1}\) is a bijection between the subgroups of \(\bar{G}\) and subgroups of \(G\) that contain \(K_{\phi}\).
\end{prooflemma}

\begin{theorem}
    Let \(\phi:G \to \bar{G}\) be a surjective homomorphism, \(\bar{N}\) a normal subgroup of \(\bar{G}\), and \(N = \set<x \in G>{\func{\phi}{x} \in N}\). Then, \(G/N \approx \bar{G}/\bar{N}\) and equivalently \(G/N \approx (G/K_{\phi})/(N/K_{\phi})\).
\end{theorem}

\begin{proof}
    The last equivalency results immediately from \ref{thm:surjectiveHomomorphism}. 
\end{proof}
\ \\ 
{\Large{\textbf{Exercises}}}
\begin{enumerate}
    \item Let \(U\) be a subset of a group \(G\). The subgroup generated by \(U\), denoted by \(\angleBracket{U}\) is the smallest subgroup that contains \(U\). Show that \(\angleBracket{U}\) exists and give a construction for it.
    \item Let \(U = \set<xyx^{-1}y^{-1}>{x,y \in G}\). In this case,\(\angleBracket{U}\) is usually written as \(\hat{G}\) and is called the \textbf{commutator subgroup} of \(G\).
    \begin{enumerate}
        \item Prove \(\hat{G}\) is normal in \(G\).
        \item Prove \(G/\hat{G}\) is abelian.
        \item If \(G/N\) is abelian, prove that \(N \supset \hat{G}\).
        \item Prove that if \(H\) is a subgroup of \(G\) and \(H \supset \hat{G}\), then \(H\) is normal in \(G\).
        \item Let \(G = \func{\GL_2}{\Reals}\) and \(N = \func{\SL_2}{\Reals}\). Show that \(N = \hat{G}\).
    \end{enumerate}
\end{enumerate}

\section{Automorphism}
\begin{definition}
    An isomorphism of a group onto iteslf is called an \textbf{automorphism}.
\end{definition}

\begin{lemma}
    If \(G\) is a group, then \(\func{\scrA}{G}\), the set of all automorphisms of \(G\) is also a group. The \(\func{\scrA}{G}\) is also denoted by \(\aut{G}\).
\end{lemma}

\begin{prooflemma}
    The \(\aut{G}\) is a group under composition. Suppose \(\theta,\phi, \psi \in \aut{G}\).
    \begin{enumerate}
        \item It is closed under composition. Since \(\phi, \theta\) are both bijective, then their composition is a bijection as well. Moreover, it is a homomorphisms
        \begin{equation*}
            \func{\phi}{\func{\psi}{xy}} = \func{\phi}{\func{\psi}{x} \func{\psi}{y}} = \func{\phi}{\func{\psi}{x}}\func{\phi}{\func{\psi}{y}}
        \end{equation*}
        therefore, \(\phi \circ \psi \in \aut{G}\).
        \item The identity is the identity transformation \(I\).
        \begin{equation*}
            I \circ \phi = \phi \circ I = \phi
        \end{equation*}
        \item the inverse of each automorphisms is its inverse map. Suppose \(\phi^{-1}\) is inverse of \(\phi\) 
        \begin{equation*}
            xy = \func{\phi}{\func{\phi^{-1}}{x}} \func{\phi}{ \func{\phi^{-1}}{y} } = \func{\phi}{\func{\phi^{-1}}{x} \func{\phi^{-1}}{x}} \implies \func{\phi^{-1}}{xy} = \func{\phi^{-1}}{x} \func{\phi^{-1}}{y} 
        \end{equation*}
        \item composition is associative 
        \begin{equation*}
            \phi \circ (\psi \circ \theta) = (\phi \circ \psi) \circ \theta
        \end{equation*}
    \end{enumerate} 
    for any maps \(\phi,\psi, \theta \) from \(G\) to \(G\).
\end{prooflemma}

\begin{example}
\(T_g:G \to G\) with \(xT_g = g^{-1}xg\). \(T_g\) is an automorphisms. \(T_g\) is called the \textbf{inner automorphism corresponding to \(g\)}. Let \(\func{\scrT}{G} = \set<T_g \in \aut{G}>{g \in G}\) is the \textbf{inner automorphism group} and is also denoted by \(\inn{G}\). \(\Psi: G \to \aut{G}\) given by \(g\Psi = T_g\) is a homomorphism. The kernel of \(\Psi\) is the \textbf{center} of \(G\), \(\func{Z}{G}\), the set of the elements that commute with all other elements. Note that, if \(g_o \in K_{\Psi}\), then \(T_{g_0} = I\), hence \(g_0^{-1}xg_0 = x\) implying \(g_0x = xg_0\) for all \(x \in G\). If \(g_0 \in \func{Z}{G}\), then \(xg_0 = g_0x\) for all \(x\), thus \(T_{g_0} = I\) and \(g_0 \in K_{\Psi}\).
\end{example}

\begin{lemma}
    \( G/Z\approx \inn{G}\).
\end{lemma}

\begin{prooflemma}
    Since \(K_{\psi} = Z\), this is an immediate result of \ref{thm:surjectiveHomomorphism}, by considering \(\Psi:G \to \inn{G}\).
\end{prooflemma}

\begin{lemma}
    Let \(G\) be a group and \(\phi\) be an automorphism of \(G\). If \(a \in G\) is of order \(\func{o}{a} > 0\), then \(\func{o}{\func{\phi}{a}} = \func{o}{a}\).
\end{lemma}

\begin{prooflemma}
    For any homomorphism \(\phi:G \to \bar{G}\), \(\func{o}{\func{\phi}{a}} \mid \func{o}{a}\) since 
    \begin{equation*}
        \func{\phi}{a}^{\func{o}{a}}= \func{\phi}{a^{\func{o}{a}}} = \func{\phi}{e} = \bar{e}
    \end{equation*}
    since both \(\phi\) and \(\phi^{-1}\) are homomorphism from \(G\) to \(G\), then 
    \begin{align*}
        &\func{o}{\func{\phi}{a}} \mid \func{o}{a}\\
        &\func{o}{\func{\phi^{-1}}{\func{\phi}{a}}} = \func{o}{a} \mid \func{o}{\func{\phi}{a}}\\
        \implies & \func{o}{\func{\phi}{a}} = \func{o}{a}
    \end{align*}
    \
\end{prooflemma}
\ \\ 
{\Large{\textbf{Exercises}}}
\begin{enumerate}
    \item A subgroup \(C\) of \(G\) is said to be a \textbf{characteristics subgroup} of \(G\) if \(CT \subset C\) for all automorphisms \(T\) of \(G\).    For any group \(G\), prove that the commutator subgroup \(\hat{G}\) is a characteristic subgroup of \(G\).
    \item Let \(G\) be a finite group, \(T\) an automorphism of \(G\) with property that \(xT=x\) if and only if \(x = e\). Suppose futher that \(T^2 = I\). Prove that \(G\) must be abelian.
    \item Let \(G\) be a finite group, \(T\) an automorphism of \(G\) that sends more than three-quarters of the elements of \(G\) onto their inverses. Prove that  \(xT = x^{-1}\) and that \(G\) is abelian.
    \item Let \(G\) be a group of order \(2n\). Suppose that half of the elements of \(G\) are of order \(2\), and the other half form a subgroup \(H\) of order \(n\). Prove that \(H\) is of odd order and is an abelian subgroup of \(G\).
\end{enumerate}
\section{Cayley's theorem}
\begin{theorem}[Cayley]
    Every group is isomorphic to a subgroup of \(\func{A}{S}\) for some set \(S\).
\end{theorem}

\begin{proof}
    Take \(S = G\) and let \(\tau_g: S \to S\) be given by \(\tau_g: x \mapsto xg\) for a \(g \in G\). We claim that \(\theta: G \to \func{A}{S}\) given by \(\theta: g \mapsto \tau_g\) is an isomorphism. First, we must show that \(\theta\) is well defined. That is, for all \(g \in G\), \(\tau_g \in \func{A}{S}\). Note that, if \(xg = yg\), then \(x = y\), hence \(\tau_g\) is injective. For every \(y \in G\), \(y = yg^{-1} \tau_g\), hence \(\tau_g\) is surjective. Thus, \(\tau_g \in \func{A}{S}\). Second, we show that \(\theta\) is a homomorphism. For all \(g,h,x \in G\), \(x(gh) = (xg)h\) therefore, \(\tau_{gh} = \tau_g \tau_h\). Finally, to show that \(\theta\) is an isomorphism, we must show that it is injective. If for all \(x \in G\), \(x\tau_g = x\tau_h\), then \(g = h\). Which was what was wanted.
\end{proof}

The construction above, describes a group \(G\) as a subgroup of \(\func{A}{G}\) that for finite \(G\), is of order \(\func{o}{G}!\). Too BIG. We wish to make it smaller. Consider the following results.

\begin{theorem}
    If \(G\) is a group, \(H\) a subgroup of \(G\), and \(S\) is the set of all right cosets of \(H\) in \(G\), then there is a homomorphism \(\theta: G \to \func{A}{S}\) and the kernel of \(\theta\) is the largest normal subgroup of \(G\) which is contained in \(H\).
\end{theorem}

\begin{proof}
    Let \(\tau_g: S \to S\) be given by \(Hx \tau_g = Hxg\) and then let \(\theta: G \to \func{A}{S}\) be given by \(\theta: g \mapsto \tau_g\). One can easily check that, \(\tau_g \in \func{A}{S}\) for all \(g\) and that \(\theta\) is a homomorphism. Suppose \(K\) is the kernel of \(\theta\). Since \(K\) is a kernel of a homomorphism, it is normal. Moreover, if \(g \in K\), then \(Hxg = Hx\) for all \(x \in G\). In particular, \(Hg = H\) which implies that \(g \in H\). As a result, \(K \subset H\). Lastly, suppose \(K'\) is another normal subgroup of \(G\) which is contained in \(H\). If \(g' \in K'\), then for all \(x \in G\), \(xg' x^{-1} \in K' \subset H'\). That is, there exists a \(h_x \in H\) such that \(xg' = hx\) which implies \(Hxg' = Hx\) for all \(x\). Therefore, \(g' \in K\) and \(K' \subset K\). Which was what was wanted.
\end{proof}

Given the above theorem, if \(H\) has no non-trivial normal subgroup of \(G\) inside it, then \(\theta\) is an isomorphism.

\begin{lemma}
    If \(G\) is a finite group, and \(H \neq G\) is a subgroup of \(G\) such that \(\func{o}{G} \nmid \func{i}{H}!\), then \(H\) must contain a non-trivial normal subgroup of \(G\). In particular, \(G\) is not simple.
\end{lemma}

\begin{proof}
    Suppose \(H\) contains no non-trivial normal subgroup of \(G\). Then, by preceding theorem, \(\theta \) is an isomorphism and \(G\) is isomorphic to a subgroup of \(\func{A}{S}\), where \(\func{A}{S} = \func{i}{H}!\). By Lagrange, theorem, \(\func{o}{G} \mid \func{i}{H}!\) which was what was wanted.
\end{proof}
\ \\ 
{\Large{\textbf{Exercises}}}
\begin{enumerate}
    \item Let \(\func{o}{G} = pq\), \(p > q\) are primes, prove 
    \begin{enumerate}
        \item \(G\) has a subgroup of order \(p\) and a subgroup of order \(q\).
        \item If \(q \nmid p - 1\), then \(G\) is cyclic.
        \item Given two primes, \(p\) and \(q\) with \(q \mid p - 1\), there exists a non-abelian group of order \(pq\).
        \item Any two non-abelian groups of order \(pq\) are isomorphic.
    \end{enumerate}
\end{enumerate}
\section{Permutation group}
Suppose \(S\) is a finite set having \(n\) elements \(x_1, \dots, x_n\). If \(\phi \in \func{A}{S}\), then \(\phi\) is a one-to-one correspondence and it can be represented as 
\begin{align*}
    \phi:& \begin{pmatrix*}
        x_1 & x_2 & \dots & x_n\\
        x_{i_1} & x_{i_2} & \dots & x_{i_n}
    \end{pmatrix*}
\intertext{ where \( x_{i_j} = \func{\phi}{x_j}\). More simply}
    & \begin{pmatrix*}
        1 & 2& \dots & n\\
        i_1 &i_2& \dots & i_n
    \end{pmatrix*} 
\end{align*}
By considering composition of \(\theta,\psi \in \func{A}{S}\), we can define multiplication on their representation. 

For \(\theta \in \func{A}{S}\) and \(a,b \in S\), \(a \overset{\theta}{\equiv} b \iff a = b \theta^i\) for some \(i \in \Integers\). This defines an equivalence relation. 

\begin{enumerate}
    \item \(a \overset{\theta}{\equiv} a \) for all \(a\), since \(a = a \theta^0\).
    \item \(a \overset{\theta}{\equiv} b\) implies \(b \overset{\theta}{\equiv} a\), since \(a = b \theta^{i} \implies b = a\theta^{-1}\).
    \item \(a \overset{\theta}{\equiv} b\) and \(b \overset{\theta}{\equiv} c\) implies \(a \overset{\theta}{\equiv} c\), since \(a = b \theta^i\) and \(b = c \theta^j\) implies \(a = c \theta^{i + j}\).
\end{enumerate}

We call the equivalence classes of \(s \in S\), the \textbf{orbit} of \(s\) under \(\theta\). The orbit of \(s\) consists of all elements in form of \(s \theta^i\), \(i \in \Integers\). If \(S\) is finite, then there is a smallest positive integer \(l = \func{l}{s}\) such that \(s\theta^l = s\). By \textbf{cycle} of \(\theta\) we mean the ordered set \((s,s\theta, \dots , s\theta^{l-1})\).

\begin{lemma}
    Every permutation is a product of its cycles.
\end{lemma}

\begin{prooflemma}
    Note that the cycles of a permuation are disjoint, and each is a permuation, hence their product is a permuation. Suppose \(\psi\) is the permuation of the product of cycles of \(\theta\). \(\psi\) is well-defined since the product of disjoint permuation is commutative. Futhermore, for each \(s \in S\), \(s\psi = \theta s\) thus, \(\theta = \psi\).
\end{prooflemma}

\begin{lemma}
    Every cycle can be written as a product of 2-cycle or \textbf{transpositions}.
\end{lemma}

\begin{prooflemma}
    Every \(m\)-cycle can be written as a product of 2-cycles. 
    \begin{equation*}
        \begin{pmatrix*}
            1 & 2& \dots & m
        \end{pmatrix*} = \begin{pmatrix*}
            1 & 2
        \end{pmatrix*}
        \begin{pmatrix*}
            2& 3
        \end{pmatrix*} \dots 
        \begin{pmatrix*}
            m-1& m
        \end{pmatrix*}
    \end{equation*}
\end{prooflemma}

\begin{definition}
    A permutation \(\theta \in S_n\) is said to be an \textbf{even permuation} if it can be represented as a product of an even number of transpositions, 
\end{definition}

The proof of well-definition of even permuation involves the polynomial \(\func{p}{x_1, \dots,x_n}\)
\begin{equation*}
    \func{p}{x_1,\dots, x_n} = \prod_{i < j} (x_i - x_j)
\end{equation*}
Define the action of \(\theta \in A(S_n)\) on the polynomial \(p\)
\begin{equation*}
    \theta \cdot p = \prod_{i < j} (x_{\func{\theta}{i}} - x_{\func{\theta}{j}})
\end{equation*}
It can be easily seen that \(\theta \cdot p = \pm p\). In fact, if \(\theta\) is a transposition, then \(\theta \cdot p = -p\). Since this is an action on \(p\), if \(\theta\) is the product of \(m\) transposition, \(\theta \cdot p = (-1)^m p\). Therefore, even permuations are well-defined. That is, no permuation can be written as a product of even number of transpositions and odd number of transpositions simultaneously.

Let \(A_n \subset S_n\) be the set of even permutations. \(A_n\) is a subgroup of \(S_n\) and it is called the \textbf{alternating group}. 
\begin{lemma}
    The alternating group is a normal subgroup of \(S_n\) of index \(2\), .
\end{lemma}
\begin{prooflemma}
    A way to prove this lemma, is to show that every odd permuation is in one coset of \(A_n\).
    
    Another way, is to show that \(\Psi:S_n \to W\) given by 
    \begin{equation*}
        \theta \Psi = \begin{cases}
            1 & \theta \text{ is even}\\
            -1 & \theta \text{ is odd}\\
        \end{cases}
    \end{equation*}
    is an onto homomorphism. \(W\) is the group of \(\set{1,-1}\) under multiplication. Then \(A_n\) is the kernel of \(\Psi\). Since \(S_n/A_n \approx W\), then 
    \begin{equation*}
        \dfrac{\func{o}{S_n}}{\func{o}{A_n}} = \func{o}{W} = 2
    \end{equation*}
    Which was what was wanted.
\end{prooflemma}
\ \\ 
{\Large{\textbf{Exercises}}}
\begin{enumerate}
    \item  
    \begin{enumerate}
        \item What is the order of an \(n\)-cycle.
        \item What is the order of the product of disjoint cycles of length \(m_1,m_2, \dots , m_k\).
        \item How do you find the order of a given permutation?
    \end{enumerate}
    \item Prove that \(A_5\) has no non-trivial normal subgroups.
    \item If \(n \geq 5\) prove that \(A_n\) is the only non-trivial normal subgroup in \(S_n\).
\end{enumerate}
\section{Another counting principle} 
\begin{definition}
    If \(a,b \in G\), then \(b\) is said to be a \textbf{conjugate} of \(a\) in \(G\), denoted by \(a \sim b\), if there exists an element \(c \in G\) such that \(b = c^{-1}ac\)
\end{definition}

\begin{lemma}
    Conjugacy is an equivalence relation on \(G\).
\end{lemma}

\begin{prooflemma}
    \begin{enumerate}
        \item \(a \sim a\) for all \(a \in G\), \(a = e^{-1}ae\).
        \item \(a \sim b \implies b \sim a\) for all \(a,b \in G\), since \(a = c^{-1}bc \) implies that \(b = cac^{-1}\).
        \item \(a \sim b, b \sim c \implies a \sim c\) for all \(a,b,c \in G\), since \(a = d^{-1}b d = d^{-1}e^{-1}ced = (ed)^{-1}c(ed)\).
    \end{enumerate}
\end{prooflemma}

For \(a\in G\) let \(\func{C}{a} = \set<x \in G>{x \sim a}\). \(\func{C}{a}\) is called the \textbf{conjugate class} of \(a \) in \(G\). It consists all elements in form of \(y^{-1}ay\) for \(y \in G\). Suppose \(G\) is a finite group and \(A\) is a set of representative of conjugacy classes. Then, 
\begin{equation*}
    \func{o}{G} = \sum_{a \in A} \abs{\func{C}{a}}
\end{equation*}

\begin{definition}
    Suppose \(a \in G\). The \textbf{normalizer} of \(a\) in \(G\), denoted by \(\func{N}{a}\), is the set of all elements that commute with \(a\), \(\func{N}{a} = \set<x \in G>{ax = xa}\).
\end{definition}

\begin{lemma}
    \(\func{N}{a}\) is a subgroup of \(G\).
\end{lemma}

\begin{prooflemma}
    Suppose \(x,y \in \func{N}{a}\), then \(a(xy) = (ax)y = (xa)y = x(ay) = x(ya) = (xy)a\). And \(x^{-1}a = ax^{-1}\) holds. Therefore, \(\func{N}{a}\) is a subgroup of \(G\).
\end{prooflemma}

\begin{theorem}\label{thm:SizeofConjugacyClass}
    If \(G\) is a finite group, then \(\abs{\func{C}{a}} = \func{i_G}{\func{N}{a}}\). i.e. the number of elements conjugate to \(a\) in \(G\) is the index of normalized of \(a\) in \(G\).
\end{theorem}

\begin{proof}
    Let \(S\) be the set of right cosets of \(\func{N}{a}\) in \(G\). Consider \(\varphi : S \to \func{C}{a}\) given by \(\varphi: \func{N}{a}g \mapsto g^{-1}ag\). This, function is well-defined since if \(\func{N}{a}g = \func{N}{a}h\), then \(g = nh\) for some \(n \in \func{N}{a}\). Then, \(g^{-1}ag = h^{-1}n^{-1}anh = h^{-1}ah\). Similary, it is injective. If \(\func{N}{a}g \varphi = \func{N}{a}h \varphi\), then \(g^{-1}ag = h^{-1}a h \implies a=(gh^{-1})a (hg^{-1}) \implies hg^{-1} \in \func{N}{a}\) hence \(\func{N}{a}g = \func{N}{a}h\). \(\varphi\) is clearly surjective. Suppose \(x \in \func{C}{a}\), then there exists \(g \in G\) such that \(x = g^{-1}ag\). Then, \(\func{N}{a}g \varphi = g^{-1}ag = x\). Therefore, \(\varphi\) is a bijection and \(\abs{\func{C}{a}} = \func{i_G}{\func{N}{a}}\).
\end{proof}
\begin{corollary}\label{cor:classEquation}
    The \textit{class equation} of \(G\)
    \begin{equation*}
        \func{o}{G} = \sum_{a \in A} \dfrac{\func{o}{G}}{\func{o}{\func{N}{a}}}
    \end{equation*}
\end{corollary}

Recall that the center \(\func{Z}{G}\) of a group \(G\) is the set of all \(a \in G\) such that \(ax = xa\) for all \(x \in G\). 

\begin{lemma}
    \(a \in \func{Z}{G}\) if and only if \(\func{N}{a} = G\). If \(G\) is finite, \(a \in \func{Z}{G}\) if and only if \(\func{o}{\func{N}{a}} = \func{o}{G}\).
\end{lemma}

\begin{prooflemma}
    It can be readily proven by applying the definitions.
\end{prooflemma}

\subsection{Applications of \ref{thm:SizeofConjugacyClass}}

\begin{theorem}
    If \(\func{o}{G} = p^n\) where \(p\) is a prime number, then \(\func{Z}{G} \neq \set{e}\).
\end{theorem}

\begin{proof}
    Let \(z = \func{o}{\func{Z}{G}}\). For each \(a \in \func{Z}{G}\), \(\abs{\func{C}{a} }= 1\). For each \(b \notin \func{Z}{G}\), \(\func{N}{a} \neq G\), hence \(\func{o}{\func{N}{a}} = p^{k}\) for some \(0 < k < n\). Therefore, \(\abs{\func{C}{a}} = p^{n - k}\) with \(n - k \geq 1\). Hence, 
    \begin{align*}
        p^n &= \sum_{a \in A} \abs{\func{C}{a}} \\
        &= \sum_{A \cap \func{Z}{G}} \abs{\func{C}{a}} + \sum_{A \cap (\func{Z}{G})^c} \abs{\func{C}{a}}\\
        &= z + \sum_{A \cap (\func{Z}{G})^c} \abs{\func{C}{a}}
    \end{align*}
    We have shown that, for \(a \notin \func{Z}{G}\), then \(p \mid \abs{\func{C}{a}}\), thus \(p \mid z\). Since \(e \in \func{Z}{G}\), then \(\func{Z}{G}\) contains at least \(p\) elements.
\end{proof}

\begin{corollary}
    If \(\func{o}{G} = p^2\) where \(p\) is a prime number, then \(G\) is abelian.
\end{corollary}

\begin{proof}
    Based on the proof last theorem, \(\func{o}{\func{Z}{G}}= p,p^2\). Suppose \(\func{o}{\func{Z}{G}} = p\) and \(a \notin \func{Z}{G}\). Then, \(\func{Z}{G} \subsetneq \func{N}{a}\). By Lagrange's theorem, \(\func{o}{\func{N}{a}} \mid \func{o}{G}\), thus \(\func{o}{\func{N}{a}} = p^2\) which means \(a \in \func{Z}{G}\), a contradiction. Therefore, \(\func{o}{\func{Z}{G}} = p^2\) and \(G\) is abelian.
\end{proof}

\begin{theorem}[Cauchy]
    If \(p\) is a prime number and \(p \mid \func{o}{G}\), then \(G\) has an element of order \(p\).
\end{theorem}

\begin{proof}
    If \(\func{o}{G} = p\), then \(G\) is cyclic and the theorem holds. Suppose, the statement is true for all groups with \(\func{o}{G} = pk\) for \(1 \leq k \leq n-1\), we will show that it is also true for \(\func{o}{G} = np\). That is, we will prove the theorem by induction. If \(G\) has a non-trivial subset \(H\) where \(p \mid \func{o}{H}\), then we would be done. Suppose, that \(p\) divides the order of no non-trivial subgroup of \(H\). Consider the normalizer subgroups, \(\func{N}{a}\). If a normalizer subgroup is trivial, then \(\func{N}{a} = G\) and hence \(a \in \func{Z}{G}\). If it is not trivial, then its index divides \(p\). 
    \begin{equation*}
        p^n = z + \sum_{A \cap (\func{Z}{G})^c} \abs{\func{C}{a}} \implies p \mid z
    \end{equation*} 
    That is \(p \mid \func{o}{\func{Z}{G}}\). Therefore, \(\func{Z}{G} = G\) which means \(G\) is abelian. By Cauchy's theorem for abelian groups, there exists \(a \neq e\) such that \(a^p = e\).
\end{proof}

Recall that every permutation in \(S_n\) can be decomposed into disjoint cycles. We shall say a permutation \(\sigma \in S_n\) has the \textbf{cycle decomposition} \(\set{n_1, \dots , n_r}\) if it can be written as product of disjoint cycles of length \(n_1, \dots ,n _r\) with \(n_1 \leq n_2 \leq \dots \leq n_r\).

\begin{lemma}
    Two permuations in \(S_n\) are conjugate if and only if they have the same cycle decomposition.
\end{lemma}

\begin{prooflemma}
    Conjugation in \(S_n\) leaves the cyclic decomposition unchanged. Also, for any two permuations with the same cyclic decomposition, we can find a \(\theta \in S_n\) such that \(\sigma_1 = \theta^{-1} \sigma_2 \theta\).
\end{prooflemma}
\begin{corollary}
    The number of conjugate classes in \(S_n\) is \(\func{p}{n}\), the number of partitions of \(n\).
\end{corollary}
\begin{prooflemma}
    Every conjugate class corresponds to a partition of \(n\).
\end{prooflemma}

\ \\ 
{\Large{\textbf{Exercises}}}
\begin{enumerate}
    \item 
\end{enumerate}
\section{Sylow's theorem}

\begin{theorem}[Sylow]
    If \(p\) is a prime number and \(p^{\alpha} \mid \func{o}{G}\), then \(G\) has a subgroup of order \(p^{\alpha}\).
\end{theorem}
We give three proofs for this theorem.
\begin{proof}
    Let \(\func{o}{G} = p^{\alpha}m\) where \(p^r \mid\mid m\) for some \(r \geq 0\). Consider \(\calM\), the set of all \(p^{\alpha}\)-element subsets of \(G\). Clearly, \(\abs{\calM} = \binom{p^{\alpha} m}{p^{\alpha}}\). Let \(\func{e_p}{n}\) be \(p^{\func{e_p}{n}} \mid\mid n\).
     We claim that \(p^r \mid \mid \abs{\calM}\). Note that 
    \begin{equation*}
        \func{e_p}{\abs{\calM}} =  \func{e_p}{(p^{\alpha}m)!} -  \func{e_p}{(p^{\alpha})!} -  \func{e_p}{(p^{\alpha}(m - 1))!}
    \end{equation*}
    For any \(m\) and \(\alpha\)
    \begin{equation*}
        \func{e_p}{(p^{\alpha}m)!} = m \func{e_p}{(p^{\alpha})!} + \func{e_p}{m!}
    \end{equation*}
    therefore,
    \begin{align*}
        \func{e_p}{\abs{\calM}} &=  \func{e_p}{(p^{\alpha}m)!} -  \func{e_p}{(p^{\alpha})!} -  \func{e_p}{(p^{\alpha}(m - 1))!}\\
        &= \func{e_p}{m!} - \func{e_p}{(m-1)!} \\
        &= \func{e_p}{\dfrac{m!}{(m-1)!}} \\
        &= \func{e_p}{m}
    \end{align*}
    which proves the claim. Define the equivalence relation \(\sim\) on \(\calM\) as following. \(M_1, M_2 \in \calM\) are equivalent if there exists a \(g \in G\) such that  \(M_1 = M_2 g\). There is at least one equivalence class that the number of elements in that class does not divide \(p^{r+1}\). As otherwise, \(p^{r+1}\mid \abs{\calM}\) which is a contradiction. Suppose \(\set{M_1, \dots, M_n}\) where \(p^{r+1} \nmid n\) is that equivalence class. Let \(H = \set<g \in G>{M_1 g = M_1}\). It can be easily shown that \(H\) is a subgroup of \(G\). We will show that \(\func{i_G}{H} = n\). Let \(\phi: Hg \mapsto M_1 g\)
    \begin{itemize}
        \item \(\phi\) is well-defined. Let \(Hg_1 = Hg_2\), then \(g_2 = hg_1\) where \(h \in H\). Hence 
        \begin{equation*}
            M_1 g_2 = M_1 hg_1 = M_1 g_1
        \end{equation*}
        \item \(\phi\) is injective. Suppose \(M_1 g_1 = M_1 g_2\), then \(M_1 g_1g^{-1}_2 = M_2\) thus \(g_1g_2^{-1} \in H \implies H g_1 = H g_2\).
        \item \(\phi\) is clearly surjective. 
    \end{itemize}
    Note that \(\set<M_1g>{g \in G} = \set{M_1, \dots, M_n}\) by definition. Then, \(\func{i_G}{H} = n\). which implies \(p^{\alpha} \mid \func{o}{H}\). For each \(m_1 \in M_1\), \(m_1 H_1 \subset M_1\), therefore, \(H\) has at most \(p^{\alpha}\) distinct elements. Thus \(\func{o}{H} = p^{\alpha}\).
\end{proof}

\begin{corollary}
    If \(p^m \mid \func{o}{G}\), \(p^{m+1} \nmid \func{o}{G}\), then \(G\) has a subgroup of order \(p^m\).
\end{corollary}

The second proof is by induction. 
\begin{proof}
    For \(\func{o}{G} = 2\), the only prime divisor is \(2\) and \(G\) itself is a subgroup of \(G\) with order \(2\). Suppose for all groups with order less than \(\func{o}{G}\), the theorem holds and suppose \(p^{\alpha} \mid \func{o}{G}\). If \(G\) has a non-trivial subgroup \(H\) where \(p^{\alpha} \mid \func{o}{H}\), then by induction hypothesis there exists a subgroup \(T\) of \(H\) with \(p^{\alpha}\) elements. We are done, since \(T\) is a subgroup of \(G\) as well. Suppose, \(G\) does not have a non-trivial subgroup whose order is divisible by \(p^{\alpha}\). Consider the normalizer groups \(\func{N}{a}\). If \(\func{N}{a} = G\), then \(a \in \func{Z}{G}\). Otherwise, \(p^{\alpha}  \nmid \func{o}{\func{N}{a}}\), hence \(p \mid \func{i_G}{\func{N}{a}}\). By class equation, \ref{cor:classEquation},
    \begin{equation*}
        \func{o}{G} = \func{o}{\func{Z}{G}} + \sum_{A \cap (\func{Z}{G})^c}  \func{i_G}{\func{N}{a}}
    \end{equation*}
    which implies that \(p \mid \func{o}{\func{Z}{G}}\). By Cauchy's theorem, there exists an element \(b \in \func{Z}{G}\) with order \(p\). Let \(B = \angleBracket{b}\). Since \(B \subset \func{Z}{G}\) it commutes with all elements of \(G\) and hence it is a normal subgroup. Let \(\bar{G} = G/B\), then \(\func{o}{\bar{G}} = \func{o}{G}/\func{o}{B} = \func{o}{G}/p\). Therefore, \(p^{\alpha - 1} \mid \func{o}{\bar{G}}\) and by the induction hypothesis, there exists a subgroup \(\bar{P}\) with order of \(p^{\alpha}\). Let \(P = \set<x \in G>{Bx \in \bar{P}}\), then \(P/B\) is isomorphic to \(\bar{P}\) and hence \(\func{o}{P} = \func{o}{\bar{P}} \func{o}{B} = p^{\alpha}\). Which was what was wanted.
\end{proof}

A subgroup of \(G\) of order \(p^m\) where \(p^m \mid\mid \func{o}{G}\) is called a \textbf{\(p\)-Sylow group}.

For the third proof of Sylow's theorem, consider the following lemmas.

\begin{lemma}
    \(S_{p^k}\) has a \(p\)-Sylow group.
\end{lemma}

\begin{proof}
    For \(k = 1\), the order of \(p\)-Sylow group is \(p\). Therefore, \(H = \angleBracket{\begin{pmatrix}1 & 2 & \dots & p \end{pmatrix}}\) is a \(p\)-Sylow group.
    Suppose that \(S_{p^{k-1}}\) has a \(p\)-Sylow group. Consider the permuation \(\sigma \in S_{p^k}\) defined as following 
    \begin{align*}
        \sigma = \begin{pmatrix}
            1 & p^{k-1} + 1 & \dots & (p-1)p^{k-1} + 1
        \end{pmatrix} & \begin{pmatrix}
            2 & p^{k-1} + 2 & \dots & (p-1)p^{k-1} + 2
        \end{pmatrix} \\
        \dots& \begin{pmatrix}
            p^{k-1} & 2 p^{k-1} & \dots & p^k
        \end{pmatrix}
    \end{align*}
    Let \(A_n = \set<\tau \in S_{p^k}>{i \tau = i \text{ for } i \leq (n-1)p^{k-1}  \text{ and } i > np^{k-1}}\) for \(n = 1, \dots,p \) the set of all permuations that only change the elements \((n-1)p^{k-1} + 1, \dots,np^{k-1}\). It can be easily shown that \(A_n\) is a subgroup of \(S_{p^k}\). Futhermore, \(A_n = \sigma^{-n} A_1 \sigma^{n}\) and \(\func{o}{A_1} = (p^{k-1})!\), in fact \(A_1 \approx S_{p^{k-1}}\). Therefore, \(A_n\) has a \(p\)-Sylow group \(P_n\), where \(P_n = \sigma^{-n} P_1 \sigma^{n}\). Let \(T = P_1 P_2 \dots P_n\). Since \(P_i \subset A_i\) and \(A_i\) are disjoint, then \(P_i\) are disjoint and hence they commute. Thus \(T\) is a subgroup of \(S_{p^k}\) with order \(\func{o}{P_1}^p = p^{p\func{e_p}{p^{k-1}!}}\). Which means \(T\) is a not a \(p\)-Sylow group. Note that \(\sigma \notin T\) and \(P_{i}\sigma^j = \sigma^j P_{i+j}\). Consider \(P = \set<\sigma^j t>{t \in T, 0 \leq j < p}\), we claim that \(P\) is a subgroup of \(S_{p^k}\). 
    \begin{enumerate}
        \item Let \(t = q_1 \dots q_p\) where \(q_i \in P_1\). Then, 
        \begin{align*}
            \sigma^{j} t \sigma^{k} t' &= \sigma^{j}  q_1 \dots q_{p-1} q_p \sigma^{k}\ t'\\
            &= \sigma^{j}  q_1 \dots q_{p-1} \sigma^{k} q'_p\  t'\\
            &= \sigma^{j+k}  q_1^{'} \dots q_{p-1}^{'}  q_p^{'} \ t'
        \end{align*}
        where \(q_i^{'} \in P_{i + j}\). Since \(P_i\) are commutative, then \(  q_1^{'} \dots  q_p^{'}  t' \in T\).
        \item The inverse of \(\sigma^j t\) can be easily found.
    \end{enumerate}
    The order of \(P\) is \(p\; \func{o}{T} = p^{p\func{e_p}{p^{k-1}!} + 1} = p^{\func{e_p}{p^k!}}\). Which means, \(P\) is a \(p\)-Sylow subgroup of \(S_{p^k}\).
\end{proof}

\begin{definition}
    Let \(G\) be a group, \(A,B\) subgroups of \(G\). If \(x,y \in G\) define \(x \sim^A_B y\) if \(y = axb\) for some \(a \in A\) and \(b \in B\). 
\end{definition}

\begin{lemma}
    The relation \( \sim^A_B \) defines an equivalence relation on \(G\). The equivalence class of \(x \in G\) is the set \(AxB = \set<axb>{a \in A, b \in B}\).
\end{lemma}

\begin{prooflemma}
    \
    \begin{enumerate}
        \item For all \(x \in G\), \(x = exe\) and hence \(x \sim_B^A x\).
        \item For all \(x,y \in G\), if \(x \sim^A_B y\), then \(y = axb\) for some \(a \in A\) and \(b \in B\), hence \(x = a^{-1}y b^{-1}\), therefore, \(y \sim^A_B x\).
        \item For all \(x,y,z \in G\), if \(x \sim^A_B y\) and \(y \sim^A_B z\), then \(y = a_1xb_1\) and \(z = a_2yb_2\) for some \(a_1,a_2 \in A\) and \(b_1,b_2 \in B\), hence \(z = a_2a_1 x b_1b_2\), therefore, \(x \sim^A_B z\).
    \end{enumerate}

\end{prooflemma}

\begin{lemma}
    If \(A,B\) are finie subgroups of \(G\) then 
    \begin{equation*}
        \abs{AxB} = \dfrac{\func{o}{A} \func{o}{B}}{\func{o}{A \cup xBx^{-1}}}
    \end{equation*}
\end{lemma}

\begin{prooflemma}
    Note that \(\abs{AxB} = \abs{AxBx^{-1}}\)
    \begin{equation*}
        \abs{AxB} = \abs{AxBx^{-1}} = \dfrac{\func{o}{A} \func{o}{xBx^{-1}}}{\func{o}{A \cap xBx^{-1}}} =  \dfrac{\func{o}{A} \func{o}{B}}{\func{o}{A \cap xBx^{-1}}}
    \end{equation*}
    which proves the lemma.
\end{prooflemma}

\begin{lemma}
    Let \(G\) be a finite group and suppose \(G\) is a subgroup of the finite group \(M\). Suppose further that \(M\) has a \(p\)-Sylow group subgroup \(Q\). Then \(G\) has a \(p\)-Sylow subgroup \(P\). In fact, \(P = G \cap xQx^{-1}\) for some \(x \in M\).
\end{lemma}

\begin{prooflemma}
    Let \(p^m \mid\mid \func{o}{M}\) and \(p^n \mid \mid \func{o}{G}\) with \(n \leq m\). Therefore, \(\func{o}{Q} = p^m\) and since \(G \cap xQx^{-1} \overset{\mathrm{gp}}{\subset} xQx^{-1}\) for all \(x \in M\), then \(\func{o}{G \cap xQx^{-1}} = p^{m_x}\) for some \(m_x \leq n\). Note that by the above's lemma
    \begin{equation*}
        \abs{GxQ} =  \dfrac{\func{o}{G} \func{o}{Q}}{\func{o}{G \cup xPx^{-1}}} = \dfrac{p^n \alpha p^m}{p^{m_x}} = p^{n + m - m_x} \alpha
    \end{equation*}
    We claim that there exists \(x \in M\) such that \(m_x = n\). As otherwise, \(m_x\) would be strictly smaller than \(n\), hence \(n - m_x \geq 1\). Thus, 
    \begin{equation*}
        \func{o}{M} = \sum_{x \in A} \abs{GxQ}
    \end{equation*}
    would divide \(p^{m+1}\) which is a contradiction. Therefore, let \(x\) be such that \(m_x  = n\) and \(P = G \cap xQx^{-1}\)
    \begin{equation*}
        \func{o}{P} = \dfrac{\func{o}{G} \func{o}{Q}}{\abs{G \cap xQx^{-1}}} = \dfrac{p^n \alpha p^m}{p^m \alpha} = p^n
    \end{equation*}
    which means that \(P\) is a \(p\)-Sylow group of \(G\).
\end{prooflemma}

We now present the thrid proof.
\begin{proof}
    Let \(\func{o}{G} = n\). By the Cayley's theorem, we can isomorphically embed \(G\) in \(S_n\). Let \(p^k > n\). Then, \(S_n\) is a subgroup of \(S_{p^k}\) and therefore \(G\) is a subgroup of \(S_{p^k}\). By the last lemma, \(G\) has a \(p\)-Sylow group.
\end{proof}

\begin{theorem}[Second part of Sylow's theorem]
    If \(G\) is a finite group, \(p\) is a prime and \(p^n \mid\mid \func{o}{G}\), then any two subgroups of \(G\) of order \(p^n\) are conjugate.
\end{theorem}

\begin{proof}
    Let \(A\) and \(B\) be two \(p\)-Sylow groups of \(G\) with order \(p^n\). Consider the double coset decomposition of \(G\) with respected to \(A\) and \(B\).
    \begin{equation*}
        \abs{AxB} = \dfrac{\func{o}{A} \func{o}{B}}{\func{o}{A \cap xBx^{-1}}} = p^{2n - m_x}
    \end{equation*}
    where \(m_x = \func{o}{A \cap xBx^{-1}}\). If \(A \neq xBx^{-1}\) for any \(x \in G\), then \(m_x < n\) for all \(x \in G\). Therefore, \(2n -m_x \geq n + 1\) for all \(x \in G\). Particularly, if \(A\) is the set of representatives of equivalence classes of \(\sim^A_B\), 
    \begin{equation*}
        \func{o}{G} = \sum_{x \in A} \abs{AxB}
    \end{equation*}
    which means \(p^{n+1} \mid \func{o}{G}\) which is a contradiction. Therefore, there exists a \(x \in G\) such that \(A = xBx^{-1}\).
\end{proof}

\begin{definition}
    Suppose \(H\) is a subgroup of \(G\). The \textbf{normalizer} of \(H\) is the subgroup \(\func{N}{H}= \set<x \in G>{x^{-1}Hx = H}\).
\end{definition}

\begin{lemma}
    Let \(H\) be a subgroup of \(G\). Then, the number of distinct conjugates of \(H\) is \(\func{i_G}{\func{N}{H}}\).
\end{lemma}

\begin{prooflemma}
    Let \(S\) be the set of right cosets of \(\func{N}{H}\) in \(G\) and \(T\) be the set of conjugates of \(H\). Consider \(\varphi : S \to T\) given by \(\varphi: \func{N}{H}g \mapsto g^{-1}Hg\). This, function is well-defined since if \(\func{N}{H}g = \func{N}{H}h\), then \(g = nh\) for some \(n \in \func{N}{H}\). Then, \(g^{-1}Hg = h^{-1}n^{-1}Hnh = h^{-1}Hh\). Similary, it is injective. If \(\func{N}{H}g \varphi = \func{N}{H}h \varphi\), then \(g^{-1}Hg = h^{-1}H h \implies H =(gh^{-1})H (hg^{-1}) \implies hg^{-1} \in \func{N}{H}\) hence \(\func{N}{H}g = \func{N}{H}h\). \(\varphi\) is clearly surjective. Suppose \(x^{-1}Hx\in T\) then, \(\func{N}{H}x \varphi = x^{-1}Hx\). Therefore, \(\varphi\) is a bijection and \(\abs{T} =\abs{S} =  \func{i_G}{\func{N}{H}}\).
\end{prooflemma}

\begin{corollary}
    The number of \(p\)-Sylow subgroups in \(G\) equals \(\func{o}{G}/\func{o}{\func{N}{P}}\) where \(P\) is any \(p\)-Sylow subgroup of \(G\). In particular, this number is a divisor of \(\func{o}{G}\).
\end{corollary}

\begin{prooflemma}
    \(p\)-Sylow subgroups are conjugates.
\end{prooflemma}

\begin{theorem}[Second part of Sylow's theorem]
    The number of \(p\)-Sylow subgroups in \(G\), is of the form \(1 + kp\).
\end{theorem}

\begin{proof}
    Let  \(p^n \mid\mid G\) and consider the double coset decomposition of \(G\) with respect to \(P\) and \(P\). 
    \begin{equation*}
        \abs{PxP} = \dfrac{(\func{o}{P})^2}{\func{o}{P \cap xPx^{-1}}}
    \end{equation*}
    if \(x \in \func{N}{P}\), then \(P \cap xPx^{-1} = P\) and hence \(\func{o}{P \cap xPx^{-1}} = p^n\). Otherwise, \(P \cap xPx^{-1} \subsetneq P\) and hence \(\func{o}{P \cap xPx^{-1}} = p^{m_x}\) for some \(m_x < n\). Therefore, 
    \begin{align*}
        \func{o}{G} = \sum_{x \in \func{N}{P}} \abs{PxP}  + \sum_{x \notin \func{N}{P}} \abs{PxP} 
    \end{align*}
    If \(x \in \func{N}{P}\), then \(xPx^{-1} = P \implies PxP = Px\). Hence, the first summation is 
    \begin{equation*}
        \sum_{x \in \func{N}{P}} \abs{Px}  = \func{o}{P} \func{i_{\func{N}{P}}}{P} = \func{o}{\func{N}{P}}
    \end{equation*}
    and the second summation is divisible by \(p^{n+1}\) hence there exists an intger \(u\) such that
    \begin{equation*}
        \sum_{x \notin \func{N}{P}} \abs{PxP} = p^{n+1}u
    \end{equation*}
    therefore 
    \begin{align*}
        \func{o}{G} = \func{o}{\func{N}{P}} + p^{n+1}u \implies \func{i_G}{\func{N}{P}} = 1 + \dfrac{p^{n+1} u}{ \func{o}{\func{N}{P}}}
    \end{align*}
    Moreover, \(p^{n+1}\) does not divide \(G\) and hence it does not divide \(\func{N}{P}\). Thus, \(p^{n+1} u /\func{o}{\func{N}{P}}\) is an integer divisible by 
    \(p\).
\end{proof}
\ \\ 
{\Large{\textbf{Exercises}}}
\begin{enumerate}
    \item Let \(N\) be a subgroup of of finite group \(G\) such that \(\func{i_G}{N}\) is the smallest prime factor of \(\func{o}{G}\). Prove \(N\) is normal.
    \item 
\end{enumerate}

\section{Direct product}
Let \(A\) and \(B\) be any two groups and \(G = A \times B\). Define the operation \(\circ_G\) as \((a_1,b_1) \circ_G (a_2,b_2) = (a_1\circ_A a_2, b_1 \circ_B b_2)\). It can be readily verified that \(G\) is group under the operation \(\circ_G\). We call \((G,\circ_G)\) the \textbf{external direct product} of \(A\) and \(B\).

Now suppose \(G = A \times B\) and consider \(\bar{A} = \set<(a,f) \in G>{a \in A}\) where \(f\) is the unit element of \(B\). Then, \(\bar{A}\) is a normal subgroup in \(G\) and is isomorphic to \(A\). We claim that \(G = \bar{A}\bar{B}\) and every \(g \in G\) has a unique decomposition in the form of \(g = \bar{a}\bar{b}\) where \(\bar{a} \in \bar{A}\) and \(\bar{b} \in \bar{B}\). Thus we have realized \(G\) as an \textbf{internal product} \(\bar{A}\bar{B}\) of two normal subgroups.

\begin{definition}
    Let \(G\) be a group and \(N_1, \dots , N_n\) normal subgroups of \(G\) such that 
    \begin{enumerate}
        \item \(G = N_1 \dots N_n\).
        \item Any \(g \in G\) can be uniquely represented as \(g = n_1n_2 \dots n_n\) where \(n_i \in N_i\).
    \end{enumerate}
    We then say that \(G\) is the \textbf{internal direct product} of \(N_1, \dots, N_n\).
\end{definition}

\begin{lemma}
    Suppose that \(G\) is the internal product of \(N_1, \dots , N_n\). Then for \(i \neq j\), \(N_i \cap N_j = \set{e}\) and if \(a \in N_i\) and \(b \in N_j\) then \(ab = ba\).
\end{lemma}

\begin{theorem}
    Suppose that \(G\) is the internal product of \(N_1, \dots , N_n\) and let \(T = N_1 \times \dots \times N_n\). Then \(G\) and \(T\) are isomorphic.
\end{theorem}
\section{Finite abelian groups}

\begin{theorem}[The fundamental theorem on finite abelian groups]
    Every finite abelian group is the direct product of cyclic groups.
\end{theorem}

\begin{definition}
    If \(G\) is an abelian group of order \(p^n\), \(p\) a prime, and \(G = A_1 \times \dots \times A_k\) where \(A_i\) is cyclic of order \(p^{n_i}\) with \(n_1 \geq n_2 \geq \dots \geq n_k > 0\), then the integers \(n_1,n_2, \dots ,n_k\) are called the \textbf{invariants} of \(G\).
\end{definition}

\begin{definition}
    Ig \(G\) is an abelian group and \(s\) is any integer, then \(\func{G}{s} = \set<x \in G>{x^s = e}\).
\end{definition}

\begin{lemma}
    If \(G\) and \(G'\) are isomorphic abelian groups, then for every integer \(s\), \(\func{G}{s}\) and \(\func{G'}{s}\) are isomorphic.
\end{lemma}