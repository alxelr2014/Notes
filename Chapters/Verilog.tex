\chapter{Verilog}
\section{Behavioral and structural design}
\lstset{language = Verilog}

Behavioral design describes circuit as an algorithm and structural design describes explicit circuit elements. A behavioral example for a full adder.
\begin{lstlisting}
    module adder(a,b,cin,s,cout);
        output s,cout;
        input a,b,cin;
        reg s, cout;
        always @(a or b or cin)
        begin
            s = a^b^cin;
            cout = a&b | a& cin | b& cin;
        end
    endmodule
\end{lstlisting}

And an example for a structural design of the same full adder.
\begin{lstlisting}
    module adder(a,b,cin,s,cout)
        output s,cout;
        input a,b,cin;

        wire w1,w2,w3,w4,w5;
        xor g1(w1,a,b);
        xor g2(s,w1,cin);
        and g3(w2,a,b);
        and g4(w3,a,cin);
        and g5(w4,b,cin);
        or g6(w5,w2,w3);
        or g7(cout,w4,w5);
    endmodule
\end{lstlisting}
Number are specified by:
\begin{itemize}
    \item Radix ('b,'h,'o,'d)
    \item Bit-length
    \item Value
\end{itemize}
for example
\begin{itemize}
    \item 4'b1011
    \item 12'habc
    \item 16'd255
\end{itemize}



\section{4-value logic}
The set of value are \(\{0,1,x,z\}\). \(x\) is for unknown and \(z\) is for high impedence.
Signal strengths are (in order):
\begin{enumerate}
    \item supply (Driving)
    \item strong (Driving)
    \item pull (Driving)
    \item large (Storage)
    \item weak (Driving)
    \item medium (Storage)
    \item small (Storage)
    \item highz (High impedence)
\end{enumerate}

wire is used to represent connections between hardware elements (default = z).
reg for hardware as well but it retains its value until next assignment (default = x).

\section{Wire and Reg}
\begin{enumerate}
    \item Syntax

          \begin{lstlisting}
    wire/reg [msb_index : lsb_index] data_id;
\end{lstlisting}
    \item Example
          \begin{lstlisting}
    wire a;
    wire [7:0] bus;
    wire [31:0] busA, busB, busC;
    reg clock;
    reg [0:40] virtual_addr;
\end{lstlisting}
\end{enumerate}

modules are models of hardware. internals not visible to enviroment. internals can be changed as long as the interface (ports) is not changed.
\begin{lstlisting}
    module fulladd4 (sum, c_out,a,b,c_in)
    ... 
    endmodule  
\end{lstlisting}
Ports are the terminals (pins) which have three types (in, out, inout). For port declaration
\begin{lstlisting}
    input a,b,sel;
    input signed [15:0] a,b;
    output signed [31:9] result;
    output reg signed [32:1] sun;
    inout [15:12] addr;
\end{lstlisting}
In a module, inputs are always of type net but output can be reg as well. inout can be of type net only.
We can implement delays as such
\section{Delay}
\begin{lstlisting}
    and #(delay-time) a1 (out,i1,i2);
    and #(rise-val,fall-val) a1 (out,i1,i2);
    and #(min:typ:max,min:typ:max) a1 (out,i1,i2);
    bufif0 #(rise-val,fall-val,turnoff-val) a1 (out,in,control);
\end{lstlisting}

\section{Data types}
\subsection{Integer}
signed numbers.
\begin{lstlisting}
    integer [7:0] tmp;
\end{lstlisting}

\subsection{Real}
default value is zero
\begin{lstlisting}
    real delta;
    delta=4e10;
    delta=2.13;
\end{lstlisting}

\subsection{Time}
at least 64 bit and \textit{\$time} is a system function that gives current simulation time.
\begin{lstlisting}
    time save_sim_time;
    initial 
        save_sim_time = $time;
\end{lstlisting}

\section{Vector and array}
\subsection{Vectors}
\begin{lstlisting}
    wire/reg [msb_index : lsb_index] data_id;

    wire a; //single bit 
    wire [7:0] bus; //8-bit vector
    wire [31:0] busA, busB, busC;
    reg clock;
    reg [0:40] virtual_addr;
\end{lstlisting}

\subsection{Arryas}
only one-dimensional and allowed for \textbf{reg,integer,time}
\begin{lstlisting}
    <data_type> <var_name> [start_idx : end_idx];

    integer count[0:7];
    reg bool[31:0];
    time chk_point[1:100];
    reg [4:0] port_id[0:7];
    integer matrix[4:0][4:0]; // illegal
\end{lstlisting}

\section{Assignments}
\subsection{Continuous assignments}
assigns values to nets, happens whenever simulation causes the value of the right-hand side to change. used for combinational logic. can happen outside of any blocks and cant be used in \textbf{always} or \textbf{initial}

\begin{lstlisting}
    wire [15:0] sum, a, b; // declaration of 16-bit vector nets
    wire cin, cout; // declaration of 1-bit (scalar) nets
    assign {cout, sum} = a + b + cin;
\end{lstlisting}

\subsection{Procedural Assignments}
drive value to \textbf{reg}. holds value of the assignment until the next assignment. occur in blocks. In \textbf{initial} block it is only executed once, but in \textbf{always} is repeated continuously. for combinational and sequential.

\begin{lstlisting}
    reg Clock, Enable, Load, Reset;
    reg [7:0] Data;
    parameter HalfPeriod = 5;
    initial
        begin 
            Clock = 0;
            #(HalfPeriod) Clock = ~Clock;
        end
\end{lstlisting}

\section{Delay}
\subsection{Inertial Delay}
Like the delay induced by combinational elements. delayed evaluation
\begin{lstlisting}
module testbench;
    reg a,b;
    always @(a)
    begin
        $display($time);
        #10 b=~a;
    end
    initial
    begin
        a=1'b0;
        #15 a=1'b1;
        #3 a=1'b0;
        #15 a=1'b1;
        #11 a=1'b0;
    end
endmodule
\end{lstlisting}

\subsection{Transport Delay}
Delayed assignment

\begin{lstlisting}
module testbench;
    reg a,b,c;
    always @(a)
    begin
        $display($time);
        b <= #10 ~a;
    end
    initial
    begin
        a=1'b0;
        #15 a=1'b1;
        #3 a=1'b0;
        #15 a=1'b1;
        #11 a=1'b0;
    end
endmodule
\end{lstlisting}

In combinational if
\begin{definition}
    \item [No dalay] use blocking
    \item [Interial delay] use blocking
    \item [Transport delay] use non-blocking
\end{definition}

In sequential logic if
\begin{definition}
    \item [No dalay] use non-blocking
    \item [with delay] use non-blocking and transport delay.
\end{definition}

\section{More keywords}
\subsection{wait}
wait for a certain condition to be true and is level-sensitive.
\begin{lstlisting}
always
    wait (CountEnable) #20 count = count+1;
\end{lstlisting}

\subsection{case}

\begin{lstlisting}
reg [1:0] alu_control;
case (alu_control)
    2'd0 : y = x + z;
    2'd1 : y = x - z;
    2'd2 : y = x * z;
    default : $display("Invalid ALU control signal");
endcase
\end{lstlisting}

\begin{lstlisting}
module mux4-to-1 (out, i0, i1, i2, i3, s1, s0);
    output out;
    input i0, i1, i2, i3;
    input s1, s0;
    reg out;
    always @(s1 or s0 or i0 or i1 or i2 or i3)
        case ({s1, s0}) 
        2'd0 : out = i0;
        2'd1 : out = i1;
        2'd2 : out = i2;
        2'd3 : out = i3;
        default: $display("Invalid control signals");
        //default is selected when s1(or s2)=x or z 
    endcase
endmodule
\end{lstlisting}

casez treats all z values as don't cares. casex treats all z and x values as don't cares.

\begin{lstlisting}
module test;
    reg s1, s0;
    reg out;
    initial
    begin
        s1 = 1’b0;
        s0 = 1’bx;
        casex ({s1, s0}) 
        2'b01 : $display("01"); 
        2'bx1 : $display("x1");
        2'b0x : $display("0x"); 
        //this alternative is executed
        2'b00 : $display("00");
        default: $display("Default");
        endcase
    end 
endmodule
\end{lstlisting}

\subsection{repeat and forever}
used only for modeling

\begin{lstlisting}
initial
begin
    count = 0;
    repeat (128)
    begin
    $display("Count = %d", count);
    count = count + 1;
    // count: from 0 to 127
    end
end
initial
begin
    clock = 1'b0;
    forever #10 clock = ~clock; //best clock practice
en
\end{lstlisting}

\subsection{function}
no timing control and returns a single value and has at least an input. can be called in behavioral and structural models.
\begin{lstlisting}
module parity;
    reg [31:0] addr;
    reg parity;
    always @(addr)
    begin
        parity = calcparity(addr); 
        $display("Parity calculated = %b", calcparity(addr));
    end
    function calcparity;
        input [31:0] address;
        begin
            calcparity = ^address; 
        end
    endfunction
endmodule   
\end{lstlisting}
\begin{example}
    A recursive example which needs the keyword \underline{automatic}.
    \begin{lstlisting}
module function_auto ();
    function automatic [7:0] factorial;
        input [7:0] i_Num; 
        begin
            if (i_Num == 1)
            factorial = 1; 
            else
            factorial = i_Num * factorial(i_Num-1);
        end
    endfunction
    initial
    begin
        $display("Factorial of 1 = %d", factorial(1));
        $display("Factorial of 2 = %d", factorial(2));
        $display("Factorial of 3 = %d", factorial(3));
        $display("Factorial of 4 = %d", factorial(4));
        $display("Factorial of 5 = %d", factorial(5));
    end
endmodule
    \end{lstlisting}
\end{example}

\subsection{task}
must be used if the procdure has any timing control, zero or more than one output argument. must be called within behavioral bodies.
\begin{lstlisting}
module operation;
    reg [15:0] A, B;
    reg [15:0] AB_AND, AB_OR, AB_XOR;
    always @(A or B) 
    begin
        bitwise_oper(AB_AND, AB_OR, AB_XOR, A, B);
        //Argument passing by name not yet supported
    end
    //define task bitwise_oper
    task bitwise_oper;
    output [15:0] ab_and, ab_or, ab_xor; 
    input [15:0] a, b; 
        begin
        #10 ab_and = a & b;
        ab_or = a | b;
        ab_xor = a ^ b;
        end
    endtask
endmodule
\end{lstlisting}
\section{Generate Statement}
\subsection{Generate Conditional Statement}
synthesize different hardware based on conditions that resolve in compile. only one hardware gets synthesized.
\begin{lstlisting}
//This module implements a parametrized multiplier
module multiplier (product, a0, a1);
    // Parameter Declaration. This can be redefined
    parameter a0_width = 8; // 8-bit bus by default
    parameter a1_width = 8; // 8-bit bus by default
    // Local Parameter declaration.
    // This parameter cannot be modified with defparam or
    // with module instance # statement.
    localparam product_width = a0_width + a1_width;
    // Port declarations
    output [product_width -1:0] product;
    input [a0_width-1:0] a0;
    input [a1_width-1:0] a1;
    // Instantiate the type of multiplier conditionally.
    // Depending on the value of the a0_width and a1_width
    // parameters at the time of instantiation, the appropriate
    // multiplier will be instantiated.
    generate
        if (a0_width <8) || (a1_width < 8)
            cla_multiplier #(a0_width, a1_width) m0 (product, a0, a1);
        else
            tree_multiplier #(a0_width, a1_width) m0 (product, a0, a1);
    endgenerate //end of the generate block
endmodule
\end{lstlisting}
\subsection{Generate a unit multiple times}