\chapter{Propositional Logic}

A language is formed from an alphabet and a set of principles to construct words and sentences.

\begin{definition}
    The language \(\calL\) of a propositional logic contains
    \begin{enumerate}
        \item A countable alphabet, also known as propositional symbol, \(Q = \set{p_0, p_1, p_2,\dots}\).
        \item Logical connectives \(\land, \lor, \neg, \to,\top\), and \(\bot\).
        \item Paranthesis \((,)\).
    \end{enumerate}
\end{definition}
\section{Syntax}
The words in a mathematical language are called expressions.

\begin{definition}
    An expression is a finite sequence of propositional symbols, logical connectives, and paranthesis. The set of all expressions in \(\calL\) is denoted by \(\LEXP_{\calL}\), however, we usually omit the subscript.
\end{definition}

However, not all strings are acceptable in a language. In propositional logic, propisitions or well-formed formulea are the acceptable expression.

\begin{definition}
    The set of well-formed formulea, \(\LPR_{\calL} \subset \LEXP_{\calL}\) is the smallest subset with the following properties.
    \begin{enumerate}
        \item \(\top,\bot \in \LPR\),
        \item The atomic propisition \(p_i\) are in \(\LPR\), \(P \subset \LPR\).
        \item For all propositions \(A,B \in \LPR\), the expressions \((A \land B)\), \((A \lor B),\) and \((A \to B)\) are propositions.
        \item For all propositions \(A, \in \LPR\), the expression \((\neg A) \in \LPR\).
    \end{enumerate}
\end{definition}
The set of \(P = Q \cup \set{\top,\bot}\) is called the atmoic symbols.

The majority of results in propsitional logic are proved with induction.

\begin{theorem}[Principle of Induction]
    Suppose \(\calP\) is a property of propositions that;
    \begin{enumerate}
        \item \(\calP\) holds all propositional symbol \(p_i\) and \(\top,\bot\).
        \item If \(\calP\) holds propositions \(A\) and \(B\), then \((A \lor B), (A \land B)\), and \((A \to B)\) satisfy \(\calP\).
        \item If \(\calP\) holds for a proposition \(A\), then it holds for \((\neg A)\).
    \end{enumerate}
    Then, \(\calP\) holds for all propsitions.
\end{theorem}

\begin{proof}
    Suppose \(X \subset \LPR\) is the set of all propositions that satisfy \(\calP\). Then, by definition, \(P,\bot \subset X\) and thus \(X\) is non-empty. Moreover, from the second and third statement, we conclude that \(\LPR \subset X\) hence, \(X = \LPR\).
\end{proof}

In these notes, we consider the symbol \(\bot\) as an atomic proposition.

\begin{theorem}[Uniqueness of Function Extension] \label{thm:uniqueext}
    Suppose there are functions, \(f_P: P \to S\), \(f_{\land},f_{\lor}, f_{\to} : S^2 \to S\) and \(f_{\neg}: S \to S\), where \(S\) is an arbitrary set, then there exists a unique function \(F: \LPR \to S\) such that;
    \begin{enumerate}
        \item \(\func{F}{p} = \func{f_p}{p}\) for all \(p \in P\).
        \item \(\func{F}{(A \circ B)} = \func{f_{\circ}}{\func{F}{A}, \func{F}{B}}\) for all propisitions \(A\) and \(B\), where \(\circ \in \set{\land,\lor,\to}\).
        \item \(\func{F}{(\neg A)} = \func{f_{\neg}}{\func{F}{A}}\) for all propositions \(A\).
    \end{enumerate}
\end{theorem}

\begin{proof}
    We call a function \(f_P \subset F \subset \LPR \times S\) a good function if it satisfies the following conditions.
    \begin{enumerate}
        \item When \((A \circ B)\) is in its domain, then \(A\) and \(B\) are in its domain as well, and \(\func{F}{(A \circ B)} = \func{f_{\circ}}{\func{F}{A}, \func{F}{B}}\).
        \item When \((\neg A)\) is in its domain, then \(A\) is in its domain as well, and \(\func{F}{(\neg A)} = \func{f_{\neg}}{\func{F}{A}}\).
    \end{enumerate}
    Consider the set of such good function. This set is non-empty as \(f_P\) is a good function. Let \(F\) be the union of all good functions. Obviously, such a set \(F\) exists.
    \begin{enumerate}
        \item First we must show that \(F\) is a function. That is, for each \(A\) in its domain there is at most one \(s \in S\). Let \(X\) be the set of all propositions that have at most one value \(s\) such that \((A,s) \in F\). By construction, all atomic propositions are in \(X\). Suppose \(A,B \in X\) and \((A \circ B)\) is in the domain \(F\). Then, there must some good functions such as \(G_0,G_1,\dots\) that have \((A \circ B)\) in their domain. Since \(A,B \in X\), then we must have \(\func{G_i}{A} = \func{G_0}{A}\) and \(\func{G_i}{B} = \func{G_0}{B}\) for any \(i\). As a result, 
        \begin{equation}
            \func{G_i}{(A \circ B)} = \func{f_{\circ}}{\func{G_i}{A},\func{G_i}{B}} =  \func{f_{\circ}}{\func{G_0}{A},\func{G_0}{B}} =  \func{G_0}{(A \circ B)} 
        \end{equation}
        hence, \((A \circ B) \in X\). Similarly, when \(A \in X\) and \((\neg A)\) is in the domain \(F\), there must some good functions such as \(G_0,G_1,\dots\) that have \((\neg A)\) in their domain. Since \(A \in X\), then we must have \(\func{G_i}{A} = \func{G_0}{A}\) for any \(i\). As a result,
        \begin{equation}
            \func{G_i}{(\neg A)} = \func{f_{\neg}}{\func{G_i}{A}} =  \func{f_{\neg}}{\func{G_0}{A}} =  \func{G_0}{(\neg A)} 
        \end{equation}
        hence, \((\neg A) \in X\). By the principle of induction \(X = \LPR\) and thus, \(F\) is a function.
        \item  Moreover, \(F\) must be a good function. This, follows immediately from the definition.
        \item The domain \(F\) is \(\LPR\). Let \(X\) be the domain of \(F\). By definition, \(P \subset X\). Moreover, if \(A,B \in X\), then there must exists a good function \(G\) with \(A\) and \(B\) in its domain. Let \(G^{\ast} = G \cup \set{((A \circ B), \func{f_{\circ}}{\func{G}{A}, \func{G}{B}})}\). Clearly, \(G^{\ast}\) is a good function, hence \((A \circ B) \in X\). A similiar argument can be for \((\neg A) \in X\) when \(A \in X\). Thus, \(X = \LPR\).
        \item Lastly, we must show that \(F\) is unique.  Suppose, there is another function \(G\) that satisfies the criteria and let \(X\) be the set of propositions \(A\) such that \(\func{F}{A} = \func{G}{A}\). From the principle of induction we must have \(F \equiv G\) thus, \(F\) is unique.
    \end{enumerate}
\end{proof}
-- priority of operations.
-- examples of induction

For all proposition \(A\) we can define the set of all its sub-proposition, \(\mathrm{SP} {A}\), and it can be defined inductively. Order of operation.
\section{Semantic}
The semantic of a propositional logic considers the truth and falsity of propositions.
\begin{definition}
    An interpretation is a function \(I : \LPR \to \set{0,1}\) such that;
    \begin{enumerate}
        \item \(\func{I}{\perp} = 0 \) and \(\func{I}{\top} = 1\).
        \item \(\func{I}{A \land B} = 1\) if and only if \(\func{I}{A} = 1\) and \(\func{I}{B} = 1\).
        \item \(\func{I}{A \lor B} = 0\) if and only if \(\func{I}{A} = 0\) and \(\func{I}{B} = 0\).
        \item \(\func{I}{A \to B} = 0\) if and only if \(\func{I}{A} = 1\) and \(\func{I}{B} = 0\).
        \item \(\func{I}{\neg A} = 1\) if and only if \(\func{I}{A} = 0\).
    \end{enumerate}
    The last two properties are points of contention among logicians. we say \(I\) models \(A\), denoted by \(I \models A\), when \(\func{I}{A} = 1\) and \(I \not\models A\) when \(\func{I}{A} = 0\), for some proposition \(A\). Generally, if \(\Gamma\) is a set of propositions, \(I \models \Gamma\) means that \(I\) models every proposition \(B \in \Gamma\). 
\end{definition}

\begin{definition}
    A valuation is an interpretation function restricted to the atoms, \(v : P \to \set{0,1}\), with \(\func{v}{\top} = 1\) and \(\func{v}{\bot} = 0\).
\end{definition}


\begin{theorem}
    For each valuation function \(v\) there is unique extension to an interpretation function \(I\).
\end{theorem}

\begin{proof}
    Let \(I\) be defined as follows.
    \begin{enumerate}
        \item \(\func{I}{p} = \func{v}{p}\) for atomic propositions.
        \item \(\func{I}{A \land B} = \func{I}{A} \cdot \func{I}{B}\).
        \item \(\func{I}{A \lor B} = \func{I}{A} + \func{I}{B} - \func{I}{A}\cdot \func{I}{B}\).
        \item \(\func{I}{A \to B} = \func{I}{A}\cdot \func{I}{B} + 1 - \func{I}{A}\).
        \item \(\func{I}{\neg A} = 1- \func{I}{A}\).
    \end{enumerate}
    The existence and uniqueness of \(I\) is proved from \cref{thm:uniqueext}.
\end{proof}

\begin{definition}
    If for all interpretation functions \(I\) that models \(\Gamma\), \(I\) models \(A\) as well, we write \(\Gamma \models A\). When \(\Gamma = \emptyset\), we say \(A\) is a tautology, denoted by \(\models A\). Moreover, two propositions \(A\) and \(B\) are equivalent, denoted by \(A \equiv B\), if \(\models (A \to B) \land (B \to A)\).
\end{definition}
We introduce the logical connective \(A \leftrightarrow B\) as a shorthand for \((A \to B) \land (B \to A)\).

Let \(A[p/B]\) denote the substituition of every propositional symbol \(p\) in \(A\) with the proposition \(B\). For example, if \(A = p_0 \land p_1\) and \(B = p_2 \lor p_0\), then \(A[p_0/B] = (p_2 \lor p_0) \land p_1\).

\begin{theorem}
    If \(\models A\), then \(\models A[p/B]\) for all choices of \(p\) and \(B\).
\end{theorem}

\begin{proof}
    Suppose the interpretation function \(I\) is generated by valuation \(v\). Define the valuation function \(u\) such that \(\func{u}{p} = \func{I}{B}\) and \(\func{u}{q} = \func{v}{q}\) for all propositional symbols \(q\) other than \(p\). Let \(J\) be the interpretation function generated by \(u\). Then, \(J \models A\) if and only if \(I \models A[p/B]\). Since \(\models A\), then \( \models A[p/B]\)
\end{proof}

\begin{lemma}
    We have \(\models A \to B\) if and only if \(\func{I}{A} \leq \func{I}{B}\) for all interpretations \(I\).
\end{lemma}

\begin{proof}
    For any interpretation \(I\), we must have \(\func{I}{A \to B} = 1\). Thus,
    \begin{align}
        \func{I}{A \to B} - 1 &= \func{I}{A} \func{I}{B} - \func{I}{A}\\
        &= \func{I}{A}(\func{I}{B} -1)
    \end{align}
    If \(\func{I}{A} = 0\), then the lemma follows trivially. If \(\func{I}{A} =1\), then we must have \(\func{I}{B} = 1\) and hence, \(\func{I}{A}\leq \func{I}{B}\).
\end{proof}

\begin{lemma}
    Suppose \(\models A \to B\), then if \(\models A\), then \(\models B\). 
\end{lemma}
\begin{proof}
    An immediate corrolary of the previous lemma.
\end{proof}

\begin{lemma}
    For all choices of \(p\), \(\func{I}{B_1 \leftrightarrow B_2} \leq \func{I}{A[p/B_1] \leftrightarrow A[p/B_2]}\).
\end{lemma}

\begin{proof}
    Without loss of generality suppose \(\func{I}{B_1 \leftrightarrow B_2} = 1\) and suppose \(A\) contains the propositional symbol \(p\). Then, \(A[p/B_1] = B_1\) and \(A[p/B_2] = B_2\) and thus, the substituition holds trivially, \(\func{I}{A[p/B_1] \leftrightarrow A[p/B_2]} = 1\).  Let \(A = A_1 \circ A_2\), then \(\func{I}{A_i[p/B_j]} = \func{I}{A_i[p/B_k]}\) for \(j\neq k\) and as a result, \(\func{I}{A[p/B_j]} = \func{I}{A[p/B_k]}\). A similiar argument holds for \(\neg\). 
\end{proof}

\begin{theorem}[Substituition]
    If \(\models B_1 \leftrightarrow B_2\), then \(\models A[p/B_1] \leftrightarrow A[p/B_2]\).
\end{theorem}

\begin{proof}
    Immediately follows from the preceeding lemmas and theorems. 
\end{proof}

\subsection{Truth Table}
The simplest method determining whether a proposition is a tautology is constructing a truth table. In each row, we consider a valuation function on the propsitional symbols of that proposition. As a result, for proposition containing \(n\) symbols, we need a table of \(2^n\) rows. We thus have the following theorem.

\begin{theorem}
    The set of tautologies in the set of propositions is decidable.
\end{theorem}

-- some tautologies + deMorgans

From the above tautologies and the deMorgan's ralations, it is apparent that we describe all connectives with \(\set{\land, \neg}, \set{\lor,\neg}\), \(\set{\land,\to}\), etc. We call a set of connective functionally complete if all other connectives can be written in terms that set. The Sheffer connective \(|\), defined as \(A | B = \neg (A \land B)\), is complete by itself. One might ask if there is another connective that can not be written with Sheffer connective. Suppose \(O\) is an \(n\)-ary functional connective, that is there exists a function \(f_O\) such that for all interpretations \(I\) and symbols \(p_1,\dots,p_n\).
\begin{equation*}
    \func{I}{\func{O}{p_1,\dots, p_n}} = \func{f_O}{\func{I}{p_1}, \dots, \func{I}{p_n}}
\end{equation*} 
The following theorem states that all functional connectives can be written in terms of Sheffer or any other complete set of connectives.

\begin{theorem}
    For all functional connectives \(O\), there exists a proposition only composed of \(p_1,\dots,p_n\) and connectives \(\set{\lor,\neg}\) such that \(\models \func{O}{p_1,\dots,p_n} \leftrightarrow A\).
\end{theorem}

When the constrained to \(\set{\land,\lor,\neg}\) we can writte every proposition in a conjunctive or distinjuctive normal form. 
\begin{enumerate}
    \item A proposition \(A\) is in the conjunctive normal form if \(A\) is 
    \begin{equation*}
        (A_{1,1} \lor \dots \lor A_{1,i_1}) \land \dots \land (A_{m,1} \lor \dots \lor A_{m,i_m})
    \end{equation*}
    where all \(A_{i,j}\) are either a propositional symbol or its negate.
    \item A proposition \(A\) is in the disjunctive normal form if \(A\) is 
    \begin{equation*}
        (A_{1,1} \land \dots \land A_{1,i_1}) \lor \dots \lor (A_{m,1} \land \dots \land A_{m,i_m})
    \end{equation*}
    where all \(A_{i,j}\) are either a propositional symbol or its negate.
\end{enumerate}

\begin{theorem}
    For a propisition \(A\) with symbols \(p_1,\dots,p_n\);
    \begin{enumerate}
        \item there exists an equivalent conjunctive normal form with the same symbols.
         \item there exists an equivalent disjunctive normal form with the same symbols.
    \end{enumerate}
\end{theorem}

\section{Proofs}
We consider proofs in propositional logic, where proof is a logical derivation of the assumptions. Note that, there are some tautologies in propositional logic. We would like find all tautologies with proofs and inferences instead of semantics. In this section, we study three proof systems.
\subsection{Hilbert System}
In the Hilbert system, we have 10 axioms and an inference rule, which as as follows.
\begin{enumerate}
    \item \(A \to (B \to A)\).
    \item \((A \to (B \to C)) \to ((A \to B) \to (A \to C))\).
    \item \(A \to (B \to (A \land B))\).
    \item \((A \land B) \to A\).
    \item \((A \land B) \to B\).
    \item \((A \to B) \to ((C\to B) \to ((A \lor C) \to B))\).
    \item \(A \to (A \lor B)\).
    \item \(B \to (A \lor B)\).
    \item \(\neg \neg A \to A\).
    \item \(\neg A \leftrightarrow (A \to \bot)\).
\end{enumerate}
The inference rule is just Modus Ponens.
\begin{equation}
    \begin{array}{ccc} 
        A  & & A \to B  \\
        \cline{1-3}
        & B &
    \end{array}
\end{equation}
We may also use substituition.
\begin{equation}
    \begin{array}{c} 
        A[p_i] \\
        \cline{1-1}
        A[p_i/B]
    \end{array}
\end{equation}
We say \(A\) is proved from \(\Gamma\), denoted by \(\Gamma \sststile{HP}{} A\), if there exists a sequence of propositions \(A_1,\dots,A_n\) such that \(A_n = A\) and for \(i \leq n\).
\begin{enumerate}
    \item  \(A_i \in \Gamma\).
    \item \(A_i\) is an axiom.
    \item There exists \(j,k < i\) such that \(A_k = A_j \to A_i\).
\end{enumerate}
When \(\Gamma = \emptyset\), we say \(A\) is a theorem, denoted by \(\sststile{HP}{} A\).
\begin{theorem}
    Suppose \(\Gamma \cup \set{A,B}\) is a set of propositions. Then, \(\Gamma,A \sststile{HP}{} B\) if and only if \(\Gamma \sststile{HP}{} A \to B\).
\end{theorem}
 \subsection{Natural Deduction} 
 There are 10 inference rules, introduction rules and elimination rules. 

\begin{table}[t]
    \SetTblrInner{rowsep=2ex}
    \centering
    \begin{tblr}{c | c | c}
        & Introduction Rules & Elimination Rules \\\hline
        \(\land\) & \(\begin{array}{ccc}
            \calD & & \calD'\\
            A & & B\\
            \cline{1-3}
            & A \land B & 
            \end{array} \)  &  \(\begin{array}{ccc} 
            \calD & & \calD\\
            A \land B & & A \land B\\
            \cline{1-1} \cline{3-3}
            A & & B
            \end{array}\)\\\hline
        \(\lor\) & \(\begin{array}{ccc}
            \calD & & \calD'\\
            A & & B\\
            \cline{1-1} \cline{3-3}
            A \lor B & & A \lor B 
            \end{array}\) & \( \begin{array}{cccc} 
            [A] & [B] & & \\
            \calD & \calD' & & \\
            C & C & & A \lor  B\\
            \cline{1-4} 
            & & C & 
            \end{array}\)\\\hline
        \(\to\) & \(\begin{array}{c}
            [A]\\
            \calD\\
            B\\
            \cline{1-1}
            A \to B
            \end{array}\) &  \(\begin{array}{ccc} 
            \calD & & \calD'\\
            A & & A \to B\\
            \cline{1-3} 
            & B & 
            \end{array}\)\\\hline
        \(\neg\)& \(\begin{array}{c}
            [A]\\
            \calD\\
            \bot\\
            \cline{1-1}
            \neg A
            \end{array} \)&  \(\begin{array}{ccc} 
            \calD & & \calD'\\
            A & & \neg A\\
            \cline{1-3} 
            & \bot & 
        \end{array}\)\\\hline
        \(\bot\)& \(\begin{array}{c}
            \calD\\
            \bot\\
            \cline{1-1}
            A
            \end{array}\) &  \(\begin{array}{c} 
            [\neg A]\\
            \calD\\
            \bot\\
            \cline{1-1} 
            A 
            \end{array}\)
    \end{tblr}
\end{table}

\begin{definition}
    An inference set/tree is the smallest set \(X\) such that;
    \begin{enumerate}
        \item for all proposition \(A\), the one-node tree of \(A\) is in \(X\).
        \item For all rules.   
    \end{enumerate}
    If such a tree exists with leaves \(\Gamma\) and root \(A\), then we say \(A\) is proved from \(\Gamma\), denoted by \(\Gamma \sststile{NP}{} A\). When \(\Gamma\) is empty, we call \(A\) a theorem. 
\end{definition}

\begin{lemma}
    \begin{enumerate}
        \item If \(A \in \Gamma\), \(\Gamma \sststile{NP}{} A\).
        \item If \(\Gamma \sststile{NP}{} A\) and \(\Delta \sststile{NP}{} A\), then \(\Gamma \cup \Delta \sststile{NP}{} A \land B\).
        \item If \(\Gamma \sststile{NP}{} A \land B\), then \(\Gamma \sststile{NP}{} A\) and \(\Gamma \sststile{NP}{} B\).
        \item If \(\Gamma \cup \set{A} \sststile{NP}{} B \), then \(\Gamma \sststile{NP}{} A \to B\).
        \item If \(\Gamma \sststile{NP}{} A\) and \(\Gamma \sststile{NP}{} A \to B\), then \(\Gamma \sststile{NP}{} B\).
        \item If \(\Gamma \sststile{NP}{} \bot\), then \(\Gamma \sststile{NP}{} A\).
        \item If \(\Gamma \cup \set{\neg A} \sststile{NP}{} \bot\), then \(\Gamma \sststile{NP}{} A\).
        \item If \(\Gamma \sststile{NP}{} A\), then \(\Gamma \sststile{NP}{} A \lor B\).
        \item If \(\Gamma \sststile{NP}{} B\), then \(\Gamma \sststile{NP}{} A \lor B\).
        \item If \(\Gamma \cup \set{A} \sststile{NP}{} C\) and \(\Delta \cup \set{B} \sststile{NP}{} C\), then \(\Gamma \cup \Delta \cup \set{A \lor B} \sststile{NP}{} C\).
        \item If \(\Gamma \cup \set{A} \sststile{NP}{} \bot\), then \(\Gamma \sststile{NP}{} \neg A\).
        \item If \(\Gamma \sststile{NP}{} A\) and \(\Gamma \sststile{NP}{} \neg A\), then \(\Gamma \sststile{NP}{} \bot\).
    \end{enumerate}
\end{lemma}
 
\begin{theorem}
    \(\Gamma \sststile{NP}{} A\) if and only if \(\Gamma \sststile{HP}{} A\).
\end{theorem}

\subsection{Sequent Calculus}
Suppose \(\Gamma\) and \(\Delta\) are finite sets of propositions. By \(\Gamma \Rightarrow \Delta\) we mean 
\begin{equation}
   ( \gamma_1 \land \gamma_2 \land \dots \land \gamma_n) \to (\delta_1 \lor \delta_2 \lor \dots \lor \delta_m)
\end{equation}
If \(\Gamma = \emptyset\) or \(\Delta = \emptyset\), then \(\Gamma \Rightarrow \Delta\) is \(\top\).

\begin{definition}[Axiom system]
    The axiomatic principles is 
    \begin{enumerate}
        \item \(A \Rightarrow A\).
        \item \(\bot \Rightarrow A\).
    \end{enumerate}
    The structural principles
    \begin{enumerate}
        \item If \(\Gamma \Rightarrow \Delta\), then \(\Gamma \Rightarrow \Delta,A\).
        \item If \(\Gamma \Rightarrow \Delta\), then \(A,\Gamma \Rightarrow \Delta\).
        \item If \(\Gamma \Rightarrow \Delta,A,A\), then \(\Gamma \Rightarrow \Delta,A\).
        \item If \(A,A,\Gamma \Rightarrow \Delta\), then \(A,\Gamma \Rightarrow \Delta,A\).
    \end{enumerate}
    The logical principles.
    \begin{enumerate}
        \item If \(\Gamma \Rightarrow \Delta,A\) and \(\Gamma \Rightarrow \Delta,B\), then \(\Gamma \Rightarrow \Delta,A\land B\).
        \item If \(A,\Gamma \Rightarrow \Delta\) and \(B,\Gamma \Rightarrow \Delta,B\), then \(A\land B,\Gamma \Rightarrow \Delta\).
        \item If \(\Gamma \Rightarrow \Delta,A\), then \(\Gamma \Rightarrow \Delta,A\lor B\).
        \item If \(A,\Gamma \Rightarrow \Delta\) and \(B,\Gamma \Rightarrow \Delta,B\), then \(A\lor B,\Gamma \Rightarrow \Delta\).
        \item If \(A,\Gamma \Rightarrow \Delta,B\), then \(\Gamma \Rightarrow \Delta,A\to B\).
        \item If \(\Gamma \Rightarrow \Delta,A\) and \(B,\Gamma \Rightarrow \Delta,B\), then \(A\to B,\Gamma \Rightarrow \Delta\).
    \end{enumerate}
\end{definition}

\begin{definition}
    A proof for \(\Gamma \Rightarrow \Delta\) in sequent calculus is a finite tree whose root is \(\Gamma \Rightarrow \Delta\) and the leaves are axioms of the system.
\end{definition}

In sequent calculus, the assumptions in a inference is simpler than the result. Thus, a bruteforce search for proof will terminate eventually. This is called subformula property.

The cut elimination property describes the transitivity.
\begin{equation}
    Cut: \begin{array}{c c c}
        \Gamma \Rightarrow \Delta,A & & A,\Gamma'  \Rightarrow \Delta'\\\cline{1-3}
        &  \Gamma, \Gamma'  \Rightarrow \Delta,\Delta' &
    \end{array}
\end{equation}
The addition cut elimination destroys the subformula property. However, we can show that the cut elimination does not grant us any additional power.

\begin{theorem}
    If \(GP + Cut\vdash \Gamma \Rightarrow \Delta \), then \(GP \vdash \Gamma \Rightarrow \Delta\).
\end{theorem}

