\chapter{Integeration}
\begin{definition}
    A \textbf{measure space} is a triplet \((X,\scrF,\mu)\) where \(\scrF\) is a \(\sigma\)-field of subsets of \(X\) and \(\mu\) is a measure defined on \(\scrF\). A \textbf{measurable space} is a pair \((X,\scrF)\).
\end{definition}
\section{Measurable functions}
Let \((X,\scrF)\) and \((Y,\scrS)\) be two measurable spaces. The function \(f:X\to Y\) is measurable if for all \(B \in \scrS\), \(\func{f^{-1}}{B} \in \scrF\). That is, the \(\sigma\)-algebra generated by \(f\), \(\func{\sigma}{f} = \set<\func{f^{-1}}{B}>{B \in \scrS}\) is a subset of \(\scrF\). Moreover, if \(\scrF = \func{\scrB}{X}\) and \(\scrS = \func{\scrB}{Y}\) are the Borel set of \(X\) and \(Y\), respectively, \(f\) is called a \textbf{Borel measurable} function. 

Let \(\ExtReals\) denote the set of the \textit{extended real numbers}, \(\Reals \cup \set{\pm \infty} = \clcl{-\infty}{+\infty}\). We may define addition and multiplication as follows 
\begin{enumerate}
    \item \(\forall a \in R, -\infty < a < \infty\).
    \item \(\forall a \in R, a + (\pm \infty) = \pm \infty\).
    \item \(\forall a \in \Reals^+, a(\pm \infty) = \pm \infty\).
    \item \((-1)(\pm \infty) = \mp \infty\).
\end{enumerate}
The extended Borels sets, \(\func{\scrB}{\ExtReals}\) are collection of subsets having the following form 
\begin{equation*}
    A, \quad A \cup \set{\pm \infty}, \quad A \cup \set{-\infty, + \infty}
\end{equation*}
where \(A\) is a Borel set. The extended Borel set make a \(\sigma\)-field. 

For the rest of this text, we may assume a measurable function \(f: X \to \ExtReals\) where \(\ExtReals\) is equipped with \(\func{\scrB}{\ExtReals}\). 

\begin{lemma}
    Suppose \(f: X \to \ExtReals\) is a function. The followings are equivalent 
    \begin{enumerate}
        \item \(f\) is measurable. 
        \item For all \(a \in \Reals\), \(\set<x \in X>{\func{f}{x} > a} \in \scrF\).
        \item For all \(a \in \Reals\), \(\set<x \in X>{\func{f}{x} \geq a} \in \scrF\).
        \item  For all \(a \in \Reals\), \(\set<x \in X>{\func{f}{x} < a} \in \scrF\).
        \item For all \(a \in \Reals\), \(\set<x \in X>{\func{f}{x} \leq a} \in \scrF\).
    \end{enumerate}
\end{lemma}

\begin{example}
    Let \(f:\Reals^n \to \Reals\) and \(\scrF = \scrM\) the Lebesgue measurable sets. If \(f\) is continuous, then \(f\) is measurable. 
\end{example}

Random variables are measurable functions from a measure space. 

\begin{theorem}
    If \(f\) and \(g\) are measurable functions, then \(\func{\max}{f,g}\) and \(\func{\min}{f,g}\) are also measurable. 
\end{theorem}

\begin{corollary}
    Suppose \(f\) is a measurable function, then 
    \begin{equation*}
        \func{f^+}{x} = \begin{cases}
            \func{f}{x} & \func{f}{x} \geq 0 \\
            0 & \func{f}{x} < 0
        \end{cases}
        \qquad 
        \func{f^-}{x} = \begin{cases}
            -\func{f}{x} & \func{f}{x} \leq 0 \\
            0 & \func{f}{x} > 0
        \end{cases}
    \end{equation*}
    are measurable function. Since \(f = f^+ - f^-\), then every function is the difference of two non-negative measurable functions.
\end{corollary}

\begin{definition}
    Let \(f_i\) be functions of \(X\) to \(\ExtReals\). Then 
    \begin{equation*}
        \func{\inf f_i}{x} = \inf \set{\func{f_i}{x}} \qquad \func{\sup f_i}{x} = \sup \set{\func{f_i}{x}}
    \end{equation*}
\end{definition}

\begin{theorem}
    If \(\set{f_i}\) are measurable functions, then \(\sup f_i\) and \(\inf f_i\) are measurable functions.
\end{theorem}

Furthermore, we may define \(\limsup\) and \(\liminf\) as follows
\begin{align*}
    \limsup f_i &= \lim_{n \to \infty} \sup_{i \geq n} f_i & \liminf f_i &= \lim_{n \to \infty} \inf_{i \geq n}\\
    &= \inf_n \sup_{i \geq n} f_i & \sup_n \inf_{i \geq n } f_i    
\end{align*}
\begin{corollary}
    If \(\set{f_i}\) is a collection of measurable functions, then \(\limsup f_i\) and \(\liminf f_i\) are measurable. 
\end{corollary}

\begin{corollary}
    Suppose \(\set{f_i}\) are measurable and converge pointwise to \(f\). Then, \(f\) is measurable.
\end{corollary}

\begin{remark}
    The restriction of a measurable function \(f:(X,\scrF)\to(Y,\scrS)\) to a measurable set \(A \in \scrF\), is measurable as well. Since 
    \begin{equation*}
        \func{f|_A^{-1}}{B} = A \cap \func{f^{-1}}{B} \in \scrF
    \end{equation*}
    Hence, if a sequence of measurable function \(f_n\) converge pointwise to \( f\) on a measurable set \(A\), then \(f:A \to Y\) is measurable.
\end{remark}

Measurablity of sum/multiplication/inverse of two measurable functions is requires some care. Particulary, to avoid situation like \(+\infty + (- \infty)\).

\begin{theorem}
    Let \(f_i : X \to \Reals\) be some measurable functions. Then, for a continuous function \(G:\Reals^n \to \Reals\), \(\func{G}{f_1, \dots , f_n}\) is measurable. 
\end{theorem}

\begin{definition}
    Suppose \(X\) and \(Y\) are two spaces equipped with their respective Borel sets, \(\func{\scrB}{X}\) and \(\func{\scrB}{Y}\). A function \(f:X \to Y\) is Borel measurable if for all \(S \in \func{\scrB}{Y}\), \(\func{f^{-1}}{S} \in \func{\scrB}{B}\).
\end{definition}

\begin{theorem}
    Every continuous function is Borel measurable.
\end{theorem}
\section{The Lebesgue Integral}
The measurable function \(s:X \to \Reals\) is a \textbf{step} function if it takes on only finite number of values. If the distinct values are \(c_1, \dots, c_n\) and \(E_i = \func{s^{-1}}{c_i}\), then
\begin{equation*}
    s = \sum_{i = 1}^n c_i \indicator{E_i}
\end{equation*}

\begin{theorem}
    \(s\) is a measurable if and only if \(E_i \in \scrF\) for \(i = 1, \dots, n\).
\end{theorem}

\begin{example}
    Let \(E \in \scrF\) then 
    \begin{equation*}
        \func{\indicator{E}}{x} = \begin{cases}
            1 & x \in E\\
            0 & x \notin E
        \end{cases}
    \end{equation*}
    is a step function. Furthermore, let \(s\) be a simple function that takes on values \(c_1, c_2,\dots, c_n\) and let \(E_i = \func{s^{-1}}{c_i}\) for \(i = 1, \dots, n\). Then, 
    \begin{equation*}
        s = \sum_{i = 1}^n c_i \indicator{E_i}
    \end{equation*}
\end{example}

\begin{definition}
    Let \(s:X\to\Reals\) be a (non-negative) simple function and let \(c_1, \dots,c_n\) be distinct non-zero values of \(s\) with \(E_i = \func{s^{-1}}{c_i}\). Let \(E \in \scrF\) and define the \textit{integral of \(s\) over \(E\) with respect to \(\mu\)} as the sum
    \begin{equation*}
        \int_E s  = \sum_{i = 1}^n c_i \func{\mu}{E \cap E_i}
    \end{equation*}
    Note that, the integral might be \(+\infty\) since \(\func{\mu}{E \cap E_i}\) might be \(+\infty\).
\end{definition}

\begin{proposition}
    Let \(s\) and \(r\) be simple non-negative functions and \(E \in \scrF\).
    \begin{enumerate}
        \item \(\int_E s + r  = \int_E s   + \int_E r \).
        \item \(\int_E c s  = c\int_E s  \) for \(c \geq 0\).
        \item If \(s \leq r\), then \(\int_E s  \leq  \int_E r \).
    \end{enumerate}
\end{proposition}

\begin{definition}
    Let \(f:X \to \ExtReals\) be a non-negative measurable function and \(E \in \scrF\). The \textit{integral of \(f\) over \(E\) with respect to \(\mu\)} is defined as 
    \begin{equation*}
        \int_E f \diffOperator \mu = \sup \set<\int_E s >{s\leq f, s\ \mathrm{is\ simple}}
    \end{equation*}
\end{definition}

Let \(s\) be a simple function. We need to check that the new definiton of integral is equivalent to the old one in the case of simple functions. That is 
\begin{equation*}
    \int_E s \diffOperator \mu =  \sum_{i = 1}^n c_i \func{\mu}{E \cap E_i}
\end{equation*}

To justify why simple functions are used to approximate consider the following. 

\begin{theorem}
    Let \(f\) be a non-negative measurable function. Then, there exists a sequence of non-negative simple functions 
    \begin{equation*}
        0 \leq s_1 \leq s_2 \leq \dots \leq f
    \end{equation*}
    such that \(s_i \to f\) pointwise. Moreover, if \(f\) is bounded, \(s_i \rightrightarrows f\).
\end{theorem}

\begin{proof}
    Fix \(n\) and divide the interval \(\clop{0}{n}\) to \(n2^n\) subinterval of length \(2^{-n}\). 
    \begin{equation*}
        I_{n,i} = \set{\dfrac{i-1}{n2^n} \leq x < \dfrac{i}{n2^n}} \qquad i = 1, \dots n2^n
    \end{equation*}
    Then let \(E_{n,i} = \func{f^{-1}}{I_{n,i}}\) and \(F_n = \func{f^{-1}}{\clcl{n}{+\infty}}\). Note that \(E_{n,i}\) and \(F\) are mutually disjoint and cover \(X\).
    \begin{equation*}
        \func{s_n}{x} = \sum_{i = 1}^{n2^n} \bracket{\dfrac{i - 1}{2^n}} \indicator{E_{n,i}} + n \indicator{F_n}
    \end{equation*}
    then \(s_n \leq f\) and \(s_n \leq s_{n+1}\) for all \(n\).
\end{proof}

In contrast to Riemann integral we approximate by dividing the range of the function. Removing the conditions on \(x\)-axis. Gives good approximation without \(f\) having to be continuous.

\begin{proposition}
    Let \(f\) and \(g\) be non-negative measurable functions and \(E,F \in \scrF\). Then 
    \begin{enumerate}
        \item If \(f \leq g\), then \(\int_E f \diffOperator \mu \leq \int_E g \diffOperator \mu \).
        \item If \(E \subset F\), then \(\int_E f \diffOperator \mu \leq \int_F f \diffOperator \mu \).
        \item If \(\func{\mu}{E} = 0\), then \(\int_E f \diffOperator \mu = 0\).
    \end{enumerate}
\end{proposition}

\begin{theorem}[Chebyshev]
    Let \(f\) be a non-negative measurable function and let \(E \in \scrF\) and \(c > 0\). Define \(E_c = \set<x \in E>{\func{f}{x} \geq c}\), then 
    \begin{equation*}
        \func{\mu}{E_c} \leq \dfrac{1}{c} \int_{E} f \diffOperator \mu 
    \end{equation*}
\end{theorem}

\begin{corollary}
    Let \(f\) be a non-negative measurable function with \(\int_E f \diffOperator \mu < \infty\), then
    \begin{equation*} 
        \func{\mu}{\set<x\in E>{\func{f}{x} = + \infty}} = 0
    \end{equation*}
\end{corollary}

\begin{definition}
    If a property holds on a set \(E \in \scrF\) except for a subset of zero measure, we say that the property holds \textbf{almost everywhere} on \(E\).
\end{definition}

\begin{corollary}
    Let \(f\) be a non-negative function and \(E \in \scrF\). 
    \begin{equation*}
        \int_E f \diffOperator \mu = 0 \implies f \equiv 0 \mathrm{\ \alev \ on \ } E
    \end{equation*}
\end{corollary}

\begin{theorem}
    Let \(f\) be a non-negative function and \(A_1, A_2, \dots \) pairwise disjoint from \(\scrF\). 
    \begin{equation*}
        \int_{\bigcup_{i = 1}^{\infty} A_i} f \diffOperator \mu = \sum_{i= 1}^{\infty} \int_{A_i} f \diffOperator \mu 
    \end{equation*}
\end{theorem}

We can use integrals to define measurse. \textit{Gaussian measure} \(\mu_G\) is defined on measurable subsets of \(\Reals\)
\begin{equation*}
    \func{\mu_G}{A} = \dfrac{1}{\sqrt{2\pi}} \int_{A} e^{-x^2/2} \diffOperator\mu_L
\end{equation*}
Moreover, \(\mu_G\) is a probability measure. 

\begin{corollary}
    Let \(f\) and \(g\) be a non-negative functions and \(E \in \scrF\). If \(f = g\) \alev on \(E\)
    \begin{equation*}
        \int_E f \diffOperator \mu = \int_E g \diffOperator \mu 
    \end{equation*}
\end{corollary}

\section{Further properties of integrals}
Let \(\set{f_i}\) be a sequence of measurable functions with 
\begin{equation*}
    0 \leq f_1 \leq f_2 \leq \dots 
\end{equation*}
Then, \(f = \lim_{n \to \infty} f_n\) exists and is measurable. 
\begin{lemma}
    Let \(f\) be a non-negative measurable function on \(X\) and let \(E_1, E_2,\dots \) be a sequence of sets in \(\scrF\) with \(E_1 \subset E_2 \subset \dots\) and \(E = \cup_i E_i\). Then 
    \begin{equation*}
        \int_E f \diffOperator \mu = \lim_{i \to \infty} \int_{E_i} f \diffOperator \mu 
    \end{equation*}
\end{lemma}

\begin{theorem}[Monotone convergence]\label{thm:monotoneConvergence}
    Let \(f\) and \(\set{f_i}\) be described as above. Then for \(E \in \scrF\)
    \begin{equation*}
        \int_E f \diffOperator \mu = \lim_{n \to \infty} \int_E f_n \diffOperator \mu 
    \end{equation*}
\end{theorem}

\begin{remark}
    Let \(f\) be a non-negative measurable function and \(s_n\) be the step functions from the construction. By \ref{thm:monotoneConvergence} 
    \begin{equation*}
        \int_E s_n \diffOperator \mu \to \int_E f \diffOperator \mu 
    \end{equation*}
\end{remark}

\begin{theorem}
    Suppose \(f\) and \(g\) are two non-negative measurable function, \(c > 0\), and \(E \in \scrF\) 
    \begin{enumerate}
        \item \(\int_E f+ g \diffOperator \mu = \int_E f \diffOperator \mu + \int_E g \diffOperator \mu\).
        \item \(\int_E cf \diffOperator \mu = c \int_E f \diffOperator \mu\).
    \end{enumerate}
\end{theorem}

\begin{corollary}
    Let \(\set{f_i}\) be non-negative measurable functions. Then, \(\sum f_i\) is a non-negative measurable function and 
    \begin{equation*}
        \int_E \sum_{i = 1}^{\infty} f_n \diffOperator\mu = \sum_{i = 1}^n \int_E f_n \diffOperator \mu 
    \end{equation*}
\end{corollary}

\begin{lemma}
    The following two conditions are equivalent 
    \begin{enumerate}
        \item \(\int_E \abs{f} \diffOperator \mu < +\infty\).
        \item \(\int_E f^+ \diffOperator \mu < +\infty \) and \(\int_E f^- \diffOperator \mu < + \infty\).
    \end{enumerate}
\end{lemma}
\begin{proof}
    \(\abs{f} = f^+ + f^-\).
\end{proof}

\begin{definition}
    A measurable function \(f\) is \textbf{integrable} over \(E\) if either of the conditions hold. In this case, \(f \in \func{\scrL}{\mu,E}\). If \(E = X\), then \(f \in \func{\scrL}{\mu}\). For \(f \in \func{\scrL}{\mu,E}\) 
    \begin{equation*}
        \int_E f \diffOperator \mu  = \int_E f^+ \diffOperator \mu - \int_E f^- \diffOperator \mu 
    \end{equation*}
\end{definition}

\begin{theorem}
    Suppose \(f,g \in \func{\scrL}{\mu,E}\) and \(c \in \Reals\) 
    \begin{enumerate}
        \item \(cf \in \func{\scrL}{\mu,E}\) and \(\int_E cd \diffOperator \mu = c \int_E f \diffOperator \mu\). 
        \item \(f +g \in \func{\scrL}{\mu,E}\) and \(\int_E f + g \diffOperator \mu = \int_E f \diffOperator \mu + \int_E g \diffOperator \mu\).
        \item If \(f \leq g\), then \(\int_E f\diffOperator \mu = \int_E g \diffOperator \mu\).
    \end{enumerate}
\end{theorem}

\begin{corollary}
    Let \(f \in \func{\scrL}{\mu,E}\), then 
    \begin{equation*}
        \abs{\int_E f \diffOperator \mu} \leq \int_E \abs{f} \diffOperator \mu 
    \end{equation*}
\end{corollary}

\begin{lemma}[Fatou's lemma]
    Assume \(f_1, f_2, \dots\) are non-negative measurable function and let \(f = \liminf f_n\)
    \begin{equation*}
        \int_E f \diffOperator \mu \leq \liminf \int_E f_n \diffOperator \mu 
    \end{equation*}
\end{lemma}

\begin{theorem}[Lebesgue dominated convergence]
    Let \(f_1, f_2, \dots \) be a sequence of measurable functions and let \(E \in \scrF\). Suppose the following assumptions hold 
    \begin{enumerate}
        \item \(\lim_{n \to \infty} \func{f_n}{x}\) exists for all \(x \in E\).
        \item There is a non-negative measurabe function \(g \in \func{\scrL}{\mu,E}\) with \(g \geq \abs{f_n}\) on \(E\) for all \(n\).
    \end{enumerate}
    Then, \(\func{f}{x} = \lim_{n \to \infty} \func{f_n}{x}\) is integrable and 
    \begin{equation*}
        \int_E \lim_{n \to \infty} f_n \diffOperator \mu = \lim_{n \to \infty} \int_E f_n \diffOperator \mu
    \end{equation*}
\end{theorem}

\begin{corollary}
    Let \(\set{f_i}\) be a sequence of functions in \(\func{\scrL}{\mu,E}\) with 
    \begin{equation*}
        \sum_{i = 1}^{\infty} \int_E \abs{f_n} \diffOperator \mu < + \infty 
    \end{equation*}
    Then, 
    \begin{enumerate}
        \item \(\sum f_n\) converges absolutely \alev on \(E\) and is integrable on \(E\).
        \item  
        \begin{equation*}
            \int_E \sum_{n = 1}^{\infty} f_n \diffOperator \mu = \sum_{n = 1}^{\infty }\int_E f_n \diffOperator\mu 
        \end{equation*} 
    \end{enumerate}
\end{corollary}

\section{Lebesgue integral vs Riemann integral}
\begin{theorem}
    Let \(f\) be a bounded Riemann integrable function on \(\clcl{a}{b}\) with Riemann integral \(\int_a^b \func{f}{x} \diffOperator x\). Then, \(f \in \func{\scrL}{\mu_L,\clcl{a}{b}}\) and 
    \begin{equation*}
        \int_a^b f \diffOperator x = \int_{\clcl{a}{b}} f \diffOperator \mu_L
    \end{equation*}
\end{theorem}

\section{Radon-Nikodym theorem}
Suppose \((X,\scrF,\mu)\) is a measure space and \(f:X \to \ExtReals\) is integrable. Define the \textbf{indefinite integer}, \(f\dot \mu:\scrF \to \Reals\), as 
\begin{equation*}
    \func{f \dot \mu}{E} = \int_E f \diffOperator\mu 
\end{equation*}

\begin{proposition}
    \ 
    \begin{enumerate}
        \item \(f,g : X \to \ExtReals\) are integerable, then \((f + g)\dot \mu = f\dot \mu + g \dot \mu\).
        \item If \(c \in \Reals\), then \((cf) \dot \mu = c(f\dot \mu)\).
        \item \(f \geq 0\) if and only if \(f \dot \mu \geq 0\).
        \item \(f \dot \mu = f^+ \dot \mu - f^- \dot \mu\).
        \item If \(E \in \scrF\) and \(\func{\mu}{E} = 0\), then \(\func{f \dot \mu}{E} = 0\).
        \item \(f \dot \mu\) is countably additive.
    \end{enumerate}
    Therefore, \(f \dot \mu\) is a finite sign measure and can be decomposed into 
    \begin{equation*}
        f \dot \mu = f^{+}\dot \mu - f^{-} \dot \mu
    \end{equation*}
\end{proposition}

\begin{definition}
    A measure \(\nu\) is \textbf{absolutely continuous} relative to a measure \(\mu\), denoted by \(v \ll u\) if 
    \begin{equation*}
        E \in \scrF, \ \func{\mu}{E} = 0 \implies \func{\nu}{E} = 0
    \end{equation*}
\end{definition}

Hence, \(f \dot u\) is absolutely continuous relative to \(\mu\).
\begin{lemma}
    Suppose \(\nu\) is finite signed measure on \(\scrF\) and \(E \in \scrF\) such that \(\func{\nu}{E} > 0\). Then, there exists a measurable subset \(G \subset E\) such that \(\nu_G \geq 0\) and  \(\func{v}{G} > 0\).
\end{lemma}

\begin{lemma}
    Suppose \(\mu\) and \(\nu\) are two finite measure on \((X,\scrF)\) such that \(\nu \ll \mu\) and \(\nu \neq 0\). Then, there exists a non-negative integrable function \(f\) such that \(f \dot \mu \leq \nu\) and \(f \dot \mu\). 
\end{lemma}

\begin{theorem}[Radon-Nikodym theorem] 
    Suppose \(\nu\) is finite signed measure and \(\mu\) is a finite measure on \((X,\scrF)\) such that \(\nu \ll \mu\). Then, there exists an integrable function \(f:X \to \ExtReals\) such that \(\nu = f \dot \mu\) and \(f\) is unique a.e.. 
\end{theorem}

\section{Fubini theorem}
Let \((X,\scrM,\mu)\) and \((Y,\scrN,\nu)\) be two measure spaces. Let \(X \times Y\) demote the space 
\begin{equation*}
    X \times Y = \set<(x,y)>{x \in X,y \in Y}
\end{equation*}

\begin{definition}
    \(A \times B \subset X \times Y\) is a \textbf{product set} if \(A \in \scrM\) and \(B \in \scrM\). The smallest \(\sigma\)-field in \(X \times Y\) containing all product sets \(A \times B\) is denoted by \(\scrM \otimes \scrN\). 
\end{definition}

\begin{definition}
    For \(E \subset X \times Y\) and fix \(x \in X\), let \(E_x = \set<y \in Y>{(x,y) \in E}\). \(E_x\) is called the \(x\)-slice of \(E\).
\end{definition}

\begin{proposition}
    IF \(E \in \scrM \otimes \scrN\), then \(E_x \in \scrN\).
\end{proposition}

\begin{corollary}
    LEt \(f: X \times Y \to \ExtReals\) be meaurable with respect to \(\scrM \otimes \scrN\). For fixed \(x_0 \in X\), define \(f_{x_0} : Y \to \Reals\) given by \(\func{f_{x_0}}{y} = \func{f}{x_0,y}\). Then, for eah \(x_0 \in X\), \(f_{x_0}\) is a measurable function on \(Y\).
\end{corollary}

Suppose \(X\) and \(Y\) are \(\sigma\)-finite. We now make a measure on \(\scrM \otimes \scrN\) using \(\mu\) and \(\nu\).

\begin{definition}
    Let \(Z\) ba set and let \(\scrS\) be a collection of subsets of \(Z\). \(\scrS\) is called a \(\lambda\)-system if the following three properties hold. 
    \begin{enumerate}[label = \(\lambda\)\arabic*.]
        \item \(Z \in \scrS\).
        \item If \(E_1 \subset E_2 \subset \dots\) is an increasing sequence with each \(E_n \in \scrS\), then 
        \begin{equation*}
            \bigcup_{i = 1}^{\infty} E_n \in \scrS
        \end{equation*}
        \item If \(E,F \in \scrS\) and \(E \subset F\), then \(F - E \in \scrS\).
    \end{enumerate}
\end{definition}

\begin{definition}
    Let \(\scrP\) be a collection of subsets of \(Z\). \(\scrP\) is called a \(\pi\)-system if the following  property holds. 
    \begin{enumerate}[label = \(\pi\)\arabic*.]

        \item If \(A,B \in \scrP\), then \(A \cap B \in \scrP\).
    \end{enumerate}
\end{definition}

\begin{theorem}[Dynkin \(\pi\)-\(\lambda\) theorem]
    If \(\scrS\) is a \(\lambda\)-system and \(\scrP\) is a \(\pi\)-system with \(\scrP \subset \scrS\), then the smallest \(\sigma\)-field containing \(\scrP\), \(\func{\sigma}{\scrP}\) is contained in \(\scrS\).
\end{theorem}

\begin{proposition}
    If \(E \in \scrM \otimes \scrN\) and \(\phi_E : X \to \Reals\) is defined by \(\func{\phi_E}{x} = \func{\nu}{E_x}\), then \(\phi_E\) is measurable.
\end{proposition}

\begin{definition}
    Let \(E \in \scrM \otimes \scrN\) and define 
    \begin{equation*}
        \func{\pi'}{E} = \int_X \func{\phi_E}{x}\diffOperator \mu 
    \end{equation*}
    to be the product measure on \(E\).
\end{definition}

\begin{proposition}
    \(\pi'\) is a measure.
\end{proposition}

Suppose instead of \(x\)-slices we used \(y\)-slices and denoted the measure by \(\pi''\). 
\begin{theorem}[Fubini, version 1]
    \begin{equation*}
        \pi' = \pi''
    \end{equation*}
\end{theorem}

\begin{definition}
    The measure \(\pi' = \pi''\) is denoted by \(\mu \times \nu\) and is called the product measure on \(\scrM \otimes \scrN\).
\end{definition}

\begin{example}
    Let \(X = Y = \Reals\) and \(\scrM = \scrN = \func{\scrB}{\Reals}\). Also, let \(\mu = \nu =\mu_L\) the Lebesgue measure on \(\Reals\). We claim that, \(\scrM \otimes \scrN = \func{\scrB}{\Reals^2}\) and \(\mu \times \nu = \mu^2_L\), the Borel sets and Lebesgue measure in \(\Reals^2\).
\end{example}

\begin{example}
    We can show that 
    \begin{equation*}
        \func{\scrB}{\Reals^n} \otimes \func{\scrB}{\Reals^m} = \func{\scrB}{\Reals^{m+n}} 
    \end{equation*} 
    and \(\mu^{m}_L \times \mu^{n}_L = \mu^{m+n}_L\).
\end{example}

\begin{theorem}[Fubini, version 2]
    Let \(f:X \times Y \to \Reals\) be a non-negative measurable function. Then 
    \begin{enumerate}
        \item For each \(x_0 \in X\), \(\func{f}{x_0,y}\) is a measurable function of \(y\).
        \item For each \(y_0 \in X\), \(\func{f}{x,y_0}\) is a measurable function of \(x\).
        \item \(\int_Y \func{f}{x,y} \diffOperator \nu\) is a measurable function of \(x\).
        \item \(\int_X \func{f}{x,y} \diffOperator \mu\) is a measurable function of \(y\).
        \item 
        \begin{equation*}
            \int_{X \times Y} \func{f}{x,y} \diffOperator \mu \times \nu = \int_X \int_Y \func{f}{x,y} \diffOperator \nu \diffOperator \mu =  \int_Y \int_X \func{f}{x,y} \diffOperator \mu \diffOperator \nu
        \end{equation*}
    \end{enumerate}
\end{theorem}

\begin{theorem}[Fubini, version 2]
    Let \(f:X \times Y \to \Reals\) be an integrable function. Then 
    \begin{enumerate}
        \item For almost all \(x \in X\), \(\func{f}{x,y}\) is a integrable function of \(y\).
        \item For almost all \(y \in X\), \(\func{f}{x,y}\) is a integrable function of \(x\).
        \item \(\int_Y \func{f}{x,y} \diffOperator \nu\) is equal \alev to an integrable function on \(X\).
        \item \(\int_X \func{f}{x,y} \diffOperator \mu\) is equal \alev to an integrable function on \(Y\).
        \item 
        \begin{equation*}
            \int_{X \times Y} \func{f}{x,y} \diffOperator \mu \times \nu = \int_X \int_Y \func{f}{x,y} \diffOperator \nu \diffOperator \mu =  \int_Y \int_X \func{f}{x,y} \diffOperator \mu \diffOperator \nu
        \end{equation*}
    \end{enumerate} 
\end{theorem}
\section{Random variables, expectation values, and indepedence}