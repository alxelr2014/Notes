\chapter{Complex Analysis}
\section{Riemann view}
\begin{definition}[Holomorph] \label{eq:complexHolomorphism}
    A function \(f: U \overset{open}{\subset} \Complex \to \Complex\) is a holomorph function if it is complex differentiable in every point in \(U\). That is, 
    \begin{equation*}
        \lim_{h \to 0} \dfrac{\func{f}{z_0 + h} - \func{f}{z_0}}{h}
    \end{equation*}
    exists. Equivalently, there exists a \(c \in \Complex\) such that 
    \begin{equation} 
        \func{f}{z_0 + h} - \func{f}{z_0} - hc = \littleO{h} 
    \end{equation}
\end{definition}
Consdering \(f\) as a function of \(\Reals^2\) to \(\Reals^2\) 
\begin{equation*}
    \func{f}{x,y} =  \func{u}{x,y}+ i\func{v}{x,y} \sim \left(\func{u}{x,y}, \func{v}{x,y}\right)
\end{equation*}
implies that if \(f\) is differentiable then \(u\) and \(v\) are differentiable. Which means
\begin{equation*}
    \begin{cases}
        \func{u}{x + h_1 , y + h_2} - \func{u}{x,y} - \func{\PDiff{u}{x}}{x,y}h_1 - \func{\PDiff{u}{y}}{x,y}h_2 &= \littleO{h}\\
        \func{v}{x + h_1 , y + h_2} - \func{v}{x,y} - \func{\PDiff{v}{x}}{x,y}h_1 - \func{\PDiff{v}{y}}{x,y}h_2 &= \littleO{h}\\
    \end{cases}
\end{equation*}
therefore 
\begin{equation*}
    \func{f}{z + h} - \func{f}{z} - \begin{bmatrix*} 
        u_x & u_y\\
        v_x & v_y
    \end{bmatrix*} \begin{bmatrix*}
         h_1 \\
          h_2 
    \end{bmatrix*}= \littleO{h}\\
\end{equation*}
Comparing the equation above with \Cref{eq:complexHolomorphism} implies that 
\begin{equation*}
    u_x =v_y \quad \land \quad u_y = - v_x
\end{equation*}
Which are called \textbf{Cauchy-Riemann relations}. This means that, for a complex to be holomorphic, it must be differentiable in \(\Reals^2\) sense and its partials follow the Cauchy-Riemann relation. By Leboman-Menchove, we can replace differentiablity with the existence of partial derivatives. From a geometric prespective, a complex number can be written as 
\begin{equation*}
    \begin{bmatrix*}
        a& -b\\
        b & a
    \end{bmatrix*} = \sqrt{a^2 + b^2} \begin{bmatrix*}
        \cos \phi& - \sin \phi\\
        \sin \phi &  \cos \phi
    \end{bmatrix*}
\end{equation*}
for some angle \(\phi\). Then, the complex derivative of a function can be seen as a rotation and a scaling, which are both \textbf{conformal} (keeps the angle and direction) transformation. Hence, a holomorphic function is a function that is differentiable in \(\Reals^2\) sense and it is conformal at every point.
One can easily verify complex differentiation follows the same rule as real differentiation, for addition, multiplication, and division. Furthermore, the Chain rule holds even if the first function is a curve.
\section{Weirstrass view}
\begin{definition}[Analytical]
    A function \(f: U \overset{open}{\subset} \Complex \to \Complex \) is analytical if for each \(z_0 \in U\) there exists a \(r > 0\) such that the open ball with radius \(r\) ceneterd at \(z_0\) remains in \(U\) and the power series 
    \begin{equation*}
        \func{f}{z} = \sum_{n = 0}^\infty c_n (z - z_0)^n
    \end{equation*}
    for all \(z\) that \(\abs[z - z_0] < r\).
\end{definition}

\begin{lemma}
    For any power series there exists a \(0 \leq \rho \leq \infty\) such that if \(\abs{z - z_0} < \rho\) then the series is convergent and if \(\abs{z- z_0} > \rho\) it is divergent. Furthermore, if \(K\) is a compact subset of \(\set<z>{\abs{z - z_0} < \rho}\), then the series is uniformly and absolutely convergent in \(K\).
\end{lemma}


\begin{theorem}
    Suppose \(\func{f}{z} = \sum_n c_n (z - z_0)^n\) is power series with \(\rho > 0\) then, \(f\) is complex differentiable in the circle of convergence and 
        \begin{equation*}
            \func{f'}{z} = \sum_{n= 0}^\infty n c_n (z - z_0)^{n-1}
        \end{equation*}
        and the radius of convergence doesn't change. Therefore, \(f \in \CalC^\infty\). Furthermore, 
        \begin{equation*}
            c_n = \dfrac{\func{f^{(n)}}{z_0}}{n!}
        \end{equation*}
\end{theorem}

The above theorem shows that every analytical function is also holomorphic. 

Let \(\gamma : \clcl{a}{b} \to U \overset{open}{\subset} \Complex\) is a continuous piecewise continuously differentiable (at endpoints assume one-sided differentiablity) and \(f : U \to \Complex\)  is a continuous function. Then 
\begin{equation*}
    \int_{\gamma} \func{f}{z} \diffOperator z = \int_{\gamma} u \diffOperator x - v \diffOperator y + i \int_{\gamma} v \diffOperator x + u \diffOperator y
\end{equation*}

Now suppose \(f'\) is continuous. Then 
\begin{equation*}
    \int_{\gamma} \func{f'}{z} \diffOperator z = \func{f}{\func{\gamma}{b}} -  \func{f}{\func{\gamma}{a }}
\end{equation*}

\begin{example}
    Let \(\func{\gamma}{t} = R \cos t + i R \sin t\), for \(0 \leq t \leq 2\pi\) then 
    \begin{equation*}
        \int_{\gamma} z^n = \begin{cases}
            0 & n \neq -1\\
            2\pi i & n = -1
        \end{cases}
    \end{equation*}
\end{example}

\begin{definition}
An open connected set \(U \subset \Complex\) is \textbf{simply connected} if for any two points \(P,Q\) and two piecewise \(\CalC^1\) curves, \(\alpha,\beta : \clcl{0}{1} \to U\) with \(\func{\alpha}{0} = \func{\beta}{0} = P \) and \(\func{\alpha}{1} = \func{\beta}{1} = Q\) there exists a continuous function \(H: \clcl{0}{1} \times \clcl{0}{1} \to U\) such that 
\begin{equation*}
    \func{H}{0,t} = \func{\alpha}{t}, \quad \func{H}{1,t} = \func{\beta}{t}
\end{equation*}
and for each \(s\), \(\func{H}{s,t}\) a function of \(t\) is a piecewise \(\CalC^1\) curves that \(\func{H}{s,0} = P\) and \(\func{H}{s,1} = Q\). Equivalently, if \(\alpha\) is a closed curves, then there exist a continuous \(H : \clcl{0}{1} \times \clcl{0}{1} \to U\) such that for each \(s\), \(\func{H}{s,t}\) is a piecewise \(\CalC^1\) closed curve and \(\func{H}{1,t} = \func{\alpha}{0}\).
\end{definition}

\begin{theorem}[Greene's Theorem]
    If \(U\) is a simply connected region and \(D \subset U\) is such that the border of \(D\) is in \(U\) and it is a piecewise \(\CalC^1\) curve. Let \(P,Q\) be two continuously differentiable function on \(U\) then 
    \begin{equation*}
        \int_{\PDiffOperator D} P \diffOperator x + Q \diffOperator y = \iint_{D} \PDiff{Q}{x} - \PDiff{P}{y} \diffOperator x \diffOperator y
    \end{equation*}
\end{theorem}

\begin{theorem}[Cauchy integral]
    Suppose \(f: U \to \Complex\) is a holomorph function with continuous derivative, where \(U\) is simply connected region. Let \(\gamma : \clcl{a}{b} \to U\) be a simple closed piecewise \(\CalC^1\) then 
    \begin{equation*}
        \int_{\gamma} \func{f}{z} \diffOperator z = 0
    \end{equation*}
\end{theorem}

\begin{proof}
    \begin{align*}
        \int_{\gamma} \func{f}{z} \diffOperator z &=  \int_{\gamma} u \diffOperator x - v \diffOperator y + i \int_{\gamma} v \diffOperator x + u \diffOperator y \\
        &= \iint_{D} \left(-\PDiff{v}{x} - \PDiff{u}{y}\right) \diffOperator x \diffOperator y +  i\iint_{D} \left(\PDiff{u}{x} - \PDiff{v}{y}\right) \diffOperator x \diffOperator y \\
        &= 0
    \end{align*}
    By the Cauchy-Riemann relations.
\end{proof}

\begin{theorem}[Cauchy's integral formula]
    Let \(U\) be a connected simple region and \(\gamma\) is a simple closed piecewise \(\CalC^1\). Let \(z\) be a point in the region surrounded by \(\gamma\) then
    \begin{equation*}
        \func{f}{z} = \frac{1}{2 \pi} \int_{\gamma} \frac{\func{f}{\zeta}}{\zeta - z} \diffOperator \zeta
    \end{equation*}
    \(\gamma\) is counter clock wise.
\end{theorem}

Using this one can show that any holomorphic function is also analytical. Interestingly, for any \(R > 0\) that \(\abs[z - z_0] = R\) remains in \(U\), \(f\) has power series representation.