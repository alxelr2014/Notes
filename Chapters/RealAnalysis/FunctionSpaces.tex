\chapter{Function Spaces}
\thispagestyle{headings}

\begin{definition}[Point Convergence]
    for each point there is a \(\epsilon\)
\end{definition}
example \(x^n\), 1/2 lines, \(sqrt(x^2 + 1/n)\), rationals
\begin{definition}[Uniform Convergence]
    there is a \(\epsilon\) for all points
\end{definition}
\begin{corollary}
    Uniform convergence implies point convergence.
\end{corollary}
example x/n with restriction
\begin{theorem}
    \(f_n\) are uniformly convergent and continuous then \(f\) is continuous
\end{theorem}

\begin{definition}
    \(\mathcal{C}^0(X,\mathbb{R}):\) all continuous function from \(X \) to \(\mathbb{R}\). \(d(f,g) = \sup{\{|f(x) - g(x)| : x \in X\}}\)
\end{definition}
\begin{definition}
    \(\|f\| = \sup\{|f(x)| : x \in X\}\) therefore \(d(f,g) = \|f - g\|\)
\end{definition}

\begin{theorem}
    \(f_n:[a,b] \to \mathbb{R}\) uniformly convergent if \(f_n\) is Riemann integrable then \(f\) is Riemann integrable
    \[\int_{a}^{b} \underbrace{\lim{f_n}}_{f} = \lim{\int_{a}^{b}{f_n}}\]
\end{theorem}
\begin{lemma}
    metric spaces and \(f_n: X \to X'\) are uniformly convergent if \(f_n\) is bounded then \(f\) is bounded.
\end{lemma}

\begin{proposition}
    \((\mathcal{C}^0_b,d)\) is a complete metric space.
\end{proposition}
\begin{definition}
    \(\mathcal{C}^0_b(X,\mathbb{R})\) and 	\(\mathcal{C}_b(X,\mathbb{R})\) closed subset of \(\mathcal{B}(X,\mathbb{R})\)
\end{definition}
exampele: any compact metric space
\begin{theorem}
    \(f_n\) are differentiable functions
    \begin{enumerate}
        \item \(f_n'\) are uniformly convergent to \(g\)
        \item \(f_n\) are point convergent
    \end{enumerate}
\end{theorem}
\begin{proposition}
    \(f_n:[a,b] \to \mathbb{R}\) consider \(\sum{f_n}\)
    \begin{enumerate}
        \item \(f_n\) are riemann integrable and the series uniformly convergent then the \(\sum{f_n}\) is riemann integrable and
              \[ \int_{a}^{b}{\sum{f_n}} = \sum {\int_{a}^{b}{f_n}}\]
        \item
              Similarly for derivative
    \end{enumerate}
\end{proposition}
\begin{definition}[Wierstrass M test]
    This is super cool
\end{definition}
power series and convergence
radius of convergence of integral/derivative of power series is equal to the radius of convergence of the originalo series.
\begin{theorem}
    in the convergence cirle the power series is inifinitely integrable and differentiable, and coefficients are taylor coefficients
\end{theorem}
analytical definition
proof of analytical means analytical in interval using a pair sequence
