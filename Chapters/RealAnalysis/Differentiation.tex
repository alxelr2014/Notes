\chapter{Differentiation}
\thispagestyle{headings}

\begin{definition}
    Let \(I\) be an interval in \(\Reals\). If \(a\) is an interior point of \(I\), then we say that \(f: I \to \Reals\) is differentiable at \(a\) when the following limit exists:
    \begin{equation*}
        \lim_{x \to a} \dfrac{\func{f}{x} - \func{f}{a}}{x - a}
    \end{equation*}
    The limit, if exists, is denoted by \(\func{f'}{a}\).
    If \(a\) is an end point and the length of the interval is greater than zero, then the limit only exists from one direction.

    Equivalently, there exists a line \(l\), not parallel to \(y\)-axis, in form of \(l : \func{A}{x} = mx + b\), that is tangent to \(f\) at \(x = a\). In this case:
    \begin{align*}
        \lim_{x \to a}\dfrac{\func{f}{x} - \func{A}{x}}{x - a } & = 0 & \func{A}{a} & = \func{f}{a}
    \end{align*}
\end{definition}
In a general case, two functions \(f,g\) are tangent to each other at \(x = a\) if:
\begin{align}
    \lim_{x \to a} \dfrac{\func{f}{x} - \func{g}{x}}{x - a} & = 0 & \func{f}{a} & = \func{g}{a}
\end{align}
\begin{corollary}\
    \begin{enumerate}
        \item \(f\) is differentiable at \(a\) if it is continuous at \(a\).
        \item \label{extrma} If \(\func{f'}{a} > 0\), there exists \(\delta > 0\) such that for \(x \in \; \opop{a- \delta}{a} \; \cap I \implies \func{f}{x} < \func{f}{a} \) and for \(x \in \; \opop{a}{ a +  \delta}\; \cap I \implies \func{f}{x} < \func{f}{a}\). And if \(\func{f'}{a} < 0\) the inequality sign are reversed. Therefore, if \(f\) has a local extremum at \(a\), then in case \(\func{f'}{a}\) exists, \(\func{f'}{a} = 0\).
    \end{enumerate}
\end{corollary}
\begin{example}
    -- a function that its derivate is not continuous (with \(\sin\frac{1}{x}\)).
\end{example}
\begin{theorem}[Rolle's theorem] \label{Rolle}
    Let \(f: \clcl{a}{b}\to \Reals\) be a continuous and differentiable on the interval. If \(\func{f}{a} = 0, \func{f}{b} = 0\), then there exists \(c \in \; \opop{a}{b}\) such that:
    \begin{equation*}
        \func{f'}{c} = 0
    \end{equation*}
\end{theorem}
\begin{proof}
    If \(f \equiv 0\) on \(\clcl{a}{b}\) then its derivative \(\func{f'}{x} \equiv 0\) on \(\clcl{a}{b}\). If \(\func{f}{x} \neq 0\) for some \(x \in \opop{a}{b}\) then it must have a non-zero maximum or minimum at some \(c \in \opop{a}{b}\). Since \(\clcl{a}{b}\) is compact then by continuity of \(f\), \(\func{f}{\clcl{a}{b}}\) is also compact in \(\Reals\) and therefore \(f\) attains its maximum or minimum. We know that at least one of its extremities must lie in \(\opop{a}{b}\), say point \(c\), hence by \Cref{extrma} \(\func{f'}{c} = 0\).
\end{proof}
\begin{theorem}[Mean value theorem]\label{MVT}
    Let \(f: \clcl{a}{b} \to \Reals\) be a continuous and differentiable on the interval,then there exists \(c \in \; \opop{a}{b}\) such that:
    \begin{equation*}
        \func{f'}{c} = \dfrac{\func{f}{b} - \func{f}{a}}{b - a}
    \end{equation*}
\end{theorem}
\begin{proof}
    Define \(\func{g}{x} = \func{f}{x}  - \func{f}{a} - \dfrac{\func{f}{b} - \func{f}{a}}{b-a}(x-a)\). Then it is clear that \(\func{g}{a} = \func{g}{b} = 0\) and \(g\) is continous and differentiable on the interval. Then, by \Cref{Rolle} there exists \(c \in \opop{a}{b}\) such that \(\func{g'}{c} = 0\). Equivalently:
    \begin{align*}
         & \func{g'}{c} = \func{f'}{c} -  \dfrac{\func{f}{b} - \func{f}{a}}{b-a} = 0 \\
         & \implies \func{f'}{c} =  \dfrac{\func{f}{b} - \func{f}{a}}{b-a}
    \end{align*}
    which concludes the proof.
\end{proof}
\begin{corollary}[Growth Estimate]
    If \(\abs{\func{f'}{x}} \leq M \) in \(\opop{a}{b}\) then \(f\) satisfies the global lipschitz condition for all \(x,y \in \clcl{a}{b}\) \(\abs{\func{f}{x} - \func{f}{y} }\leq M \abs{x- y}  \).
\end{corollary}
\begin{corollary}
    Let \(f: \clcl{a}{b} \to \Reals\) is continuous and \(\func{f'}{x} < 0\) -- or \(\func{f'}{x} > 0\)-- for all \(x \in \opop{a}{b}\) then \(f\) is strictly increasing --or decreasing--  on \(\clcl{a}{b}\).
\end{corollary}
\begin{theorem}
    \(f: \clcl{a}{b} \to \Reals\) is continuous and differentiable on \(\opop{a}{b}\) then for \(\func{f'}{\opop{a}{b}}\) the intermediate value theorem holds and thus it is an interval.
\end{theorem}
\begin{proof}
    Let \(x_1, x_2 \in \opop{a}{b} \). WLOG assume \(\func{f'}{x_1} < \func{f'}{x_2}\), we wish to prove that for all \(y^* \in \opop{\func{f'}{x_1}}{ \func{f'}{x_2}}\) there is a \(x^* \in \opop{x_1}{x_2} \) such that \(\func{f}{x^{\ast}} = y^{\ast}\). Put \(\func{g}{x} = \func{f}{x} - y^{\ast}x\). By differentiability of \(f\) on \([a,b]\), \(g\) is differentiable on \(\clcl{a}{b}\). Then, \(\func{g'}{x_1} = \func{f'}{x_1} - y^{\ast} < 0\) and \(\func{g'}{x_2} =  \func{f'}{x_2} - y^{\ast} > 0\), therefore there are \(t_1,t_2 \in \opop{x_1}{x_2}\) such that \(\func{f}{t_1} < \func{f}{x_1}\) and \(\func{f}{t_2} < \func{f}{x_2}\). Since \(g\) is continuous on \(\clcl{x_1}{x_2}\) then it must attains its minimum at some \(x^{\ast} \in \clcl{x_1}{x_2}\). However \(x^* \) can't be \(x_1\) or \(x_2\) and hence \(x^{\ast} \in \opop{x_1}{x_2}\). It is then easy to see that \(\func{f'}{x^{\ast}} = y^{\ast}\).
\end{proof}
\begin{definition}[Darboux continous]
    A function \(f\) is Darboux continous if it posseses the intermediate value property.
\end{definition}
For example \(f'\) of differentiable function is Darboux continuous.
\begin{theorem}[Cauchy's mean value theorem]
    \(f,g : \clcl{a}{b} \to \Reals\) are continuous then there exists a \(c \in \; \opop{a}{b}\), such that:
    \begin{equation*}
        \func{f'}{c}\bracket{\func{g}{b} - \func{g}{a}} = \func{g'}{c}\bracket{\func{f}{b} - \func{f}{a}}
    \end{equation*}
\end{theorem}
\begin{proof}
    Define \(\func{h}{x} = \bracket{\func{f}{x} - \func{f}{a}}\bracket{\func{g}{b} - \func{g}{a}} -  \bracket{\func{g}{x} - \func{g}{a}}\bracket{\func{f}{b} - \func{f}{a}}\), then clearly \(\func{h}{a} =0, \func{h}{b} =0\) and \(\func{h}{x}\) is continous and differentiable on \(\clcl{a}{b}\). Hence by applying the \cref{Rolle} for some \(c \in \opop{a}{b}\) we have:
    \begin{align*}
         & \func{h'}{c} = 0                                                                                \\
         & \implies \func{f'}{c}(\func{g}{b} - \func{g}{a}) -  \func{g'}{c}(\func{f}{b} - \func{f}{a}) = 0 \\
         & \implies  \func{f'}{c}(\func{g}{b} - \func{g}{a}) = \func{g'}{c}(\func{f}{b} - \func{f}{a})
    \end{align*}
\end{proof}

\begin{theorem}[L'Hopital's rule]
    Suppose that \(\lim_{x \to a^+}{\func{f}{x}} = 0,\lim_{x \to a^+}{\func{g}{x}} = 0 \) where \(f,g\) are differentiable on a open interval \(I = \opop{a}{b}\) for some \(b\) such that \(\func{g'}{x} \neq 0\) in \(I\) except maybe at \(x = a\) and the limit
    \begin{equation*}
        \lim_{x \to a^+}{\dfrac{\func{f'}{x}}{\func{g'}{x}}} = L
    \end{equation*}
    exists, then:
    \begin{equation*}
        \lim_{x \to a^+}{\dfrac{\func{f}{x}}{\func{g}{x}}} = L
    \end{equation*}
\end{theorem}
\begin{proof}
    For a fixed \(\epsilon > 0\) there exists a \(\delta > 0\) such that:
    \begin{equation*}
        \abs{x - a} < \delta \implies \abs{\dfrac{\func{f'}{x}}{\func{g'}{x}} - L} < \dfrac{ \epsilon}{2}
    \end{equation*}
    then since \(\func{f}{t}, \func{g}{t} \to 0\) as \(t \to a^+\) from right side then there must be a \(t \in \opop{a}{x}\) such that
    \begin{equation*}
        \abs{\dfrac{\func{f}{x} - \func{f}{t}}{\func{g}{x} - \func{g}{t}} - \dfrac{\func{f}{x}}{\func{g}{x}}} < \dfrac{\epsilon}{2}
    \end{equation*}
    then simply:
    \begin{align}
        \abs{\dfrac{\func{f}{x}}{\func{g}{x}} - L} & \leq \abs{\dfrac{\func{f}{x}}{\func{g}{x}} - \dfrac{\func{f}{x} - \func{f}{t}}{\func{g}{x} - \func{g}{t}} } + \abs{\dfrac{\func{f}{x} - \func{f}{t}}{\func{g}{x} - \func{g}{t}} - L} \\
                                                   & < \dfrac{\epsilon}{2} + \abs{\dfrac{\func{f'}{\theta}}{\func{g'}{\theta}} - L}                                                                                                       \\
                                                   & < \epsilon
    \end{align}
    Note that \(\theta \in \; \opop{t}{x}\) and thus \(\abs{\theta - a} < \delta \)
\end{proof}
\begin{definition}[Higher order derivatives]
    \(f\) is said to be \(r_{\cardinalTH}\)-differentiable at \(x\) if it is differentiable \(r\) times. The \(r_{\cardinalTH}\) derivative of \(f\) is denoted as \(f^{(r)}\). If \(f^{(r)}\) exists for all \(r\) and \(x\) then \(f\) is said to be infinitely differentiable or smooth.
\end{definition}
\begin{definition}[Smoothness classes]
    The set of all \(f\) is continuosly \(r_{\cardinalTH}\)-differentiable is called class \(\calC^r\).
\end{definition}
\begin{definition}[Taylor polynomial]
    The \(r_{\cardinalTH}\)-order Taylor polynomial of an \(r_{\cardinalTH}\)-order differentiable function \(f\) at \(x\) is
    \begin{equation*}
        \func{P_r}{x,h} =\func{f}{x} + \func{f'}{x}h +  \dfrac{\func{f''}{x}}{2}h^2 + \dots +  \dfrac{\func{f^{(r)}}{x}}{r!} h^r = \sum_{n = 0}^{r}\dfrac{\func{f^{(n)}}{x}}{n!} h^n
    \end{equation*}
\end{definition}
\begin{theorem}[Taylor approximation theorem]
    Let \(f\) be a \(r\)-differentiable function at \(x\) then:
    \begin{enumerate}
        \item
              \begin{equation*}
                  \dfrac{\func{f}{x+h} - \func{P_r}{x,h}}{h^r} \to 0 \text{ as } h \to 0
              \end{equation*}
        \item
              and \(P_r\) is the only \(r_{\cardinalTH}\) degree polynomial that has such property.
        \item
              Furthermore, if \(f\) is \(r\)-differentiable on an interval \(I\) for every \(x,y \in I\), there exists \(\xi\) between \(x,y\) such that:
              \begin{equation*}
                  \func{f}{y} - \func{P_{r-1}}{x,y-x} = \dfrac{\func{f^{(r)}}{\xi }}{(r)!}(y-x)^{r}
              \end{equation*}
    \end{enumerate}
\end{theorem}
\begin{proof}
\begin{enumerate}
\item
For the base case \(r = 1\)
\begin{align*}
    \lim_{h \to 0}{\dfrac{\func{f}{x+h} - \func{f}{x} -\func{f'}{x}h}{h}} = \func{f'}{x} - \func{f'}{x} = 0
\end{align*}

and by induction we prove the case \(r = n \geq 2\)
          \begin{align*}
               & \lim_{h \to 0}{\dfrac{\func{f}{x+h} - \func{P_n}{x,h}}{h^n}} = 0                                                                      \\
               & \iff \forall \epsilon >0, \ \exists \delta > 0\ \text{such that } \ \abs{h} < \delta \implies \abs{\func{f}{x+h} - \func{P_n}{x,h}}< \epsilon \abs{h^n}
          \end{align*}

          Let \(\func{g}{h} = \func{f}{x+h} - \func{P_n}{x,h}\) then since both \(\func{f}{x+h}\) and \(\func{P_n}{x,h}\) are differentiable then we apply \Cref{MVT}
          \begin{align*}
              \func{g}{h} - \func{g}{0} &= \func{g'}{c}                                                   \\
                          & = \func{f'}{x+c} - \sum_{k = 1}^{n} \dfrac{\func{ f^{(k)}}{x}}{(k-1)!}c^{k-1} \\
                          & =  \func{f'}{x+c} - \sum_{k = 0}^{n-1} \dfrac{\func{f^{(k+1}}{x}}{k!}c^{k}   \\
                          & =  \func{f'}{x+c} - \sum_{k = 0}^{n-1} \dfrac{\func{ {f'}^{(k)}}{x}}{k!}c^{k}
          \end{align*}
          for some \(c \in \opop{0}{h}\). Note that \(f'\) is \((n-1)\)-differentiable at \(x\). Thus by induction for any \(\epsilon > 0\) there exists a \(\delta\) such that if \(c < \delta \) then:
          \begin{equation*}
            \abs{\func{f'}{x+c} - \sum_{k = 0}^{n-1} \dfrac{\func{ {f'}^{(k)}}{x}}{k!}c^{k}} < \epsilon \abs{c^{n-1}}
          \end{equation*}

          which means
          \begin{align*}
              \abs{\func{f}{x+h} - P_n(x,h)} & = \abs{\func{g}{h}} = \abs{h} \abs{\func{f'}{x+c} - \sum_{k = 0}^{n-1} \dfrac{\func{ {f'}^{(k)}}{x}}{k!}c^{k}}\\
                                  & < \abs{h} \epsilon \abs{c^{n-1}}< \epsilon \abs{h^n}
          \end{align*}

          Therefore, for any \(\epsilon\) if \(h < \delta\) then \(c < \delta\) and the result holds.
    \item
          Let \(\func{Q_r}{x,h}\) be another \(r_{\cardinalTH}\) degree polynomial such that
          \begin{equation*}
              \lim_{h \to 0}{\dfrac{\func{f}{x+h} -\func{Q_r}{x,h}}{h^r}} = 0
          \end{equation*}

          then
          \begin{equation*}
              \lim_{h \to 0}{\dfrac{\func{P_r}{x,h} -\func{Q_r}{x,h}}{h^r}} = 0
          \end{equation*}

          however this can only happen if \(\func{Q_r}{x,h} = \func{P_r}{x,h}\).
    \item
          Again for the base case \(r = 1\)
          \begin{equation*}
              \func{f}{y} - \func{f}{x} = \func{f'}{\xi}(y-x)
          \end{equation*}

          which is the \Cref{MVT}. for \(r = n\) we have that
          \begin{equation*}
              \func{g}{h} = \func{f}{x+h} - \func{P_{n-1}}{x,h} + Ch^n \implies \func{g}{0} = \func{g'}{0} = \dots = \func{g^{(n-1)}}{0} =0
          \end{equation*}
          Set \(C\) such that \(\func{g}{y-x} = 0\). Then by applying \Cref{Rolle} \((n-1)\) times

          \begin{align*}
              &\func{g}{0} = \func{g}{y-x} = 0 &\implies& \func{g'}{c_1} = 0 & c_1 &\in \opop{0}{y-x} \\
              &\func{g'}{0} = \func{g'}{c_1} = 0& \implies& \func{g'}{c_2} = 0 &c_2 &\in \opop{0}{c_1}\\
              && &\vdots &&\\
              & \func{g^{(n-2)}}{0} = \func{g^{(n-2)}}{c_{n-2}} = 0 &\implies& \func{g^{(n-1)}}{c_{n-1}} = 0 & c_{n-1} &\in \opop{0}{c_{n-2}}\\
              &\func{g^{(n-1)}}{0} = \func{g^{(n-1)}}{c_{n-1}} = 0 &\implies& \func{g^{(n)}}{\xi - x} = 0 & \xi - x &\in \opop{0}{c_{n-1}} 
          \end{align*}
          Expanding \(\func{g^{(n)}}{\xi - x}\) gives the following.
          \begin{align*}
              &\implies  \func{g^{(n)}}{\xi - x} =  \func{f^{(n)}}{\xi} + Cn! \\
              &\implies \func{f^{(n)}}{\xi} - \dfrac{n!}{(y-x)^n}(\func{f}{y} - \func{P_{n-1}}{x,y-x} ) = 0 \\
              &\implies \func{f}{y} - \func{P_{n-1}}{x,y-x}  =  \dfrac{\func{f^{(n)}}{\xi}}{n!}(y-x)^n & \xi &\in \opop{x}{y}
          \end{align*}
 \end{enumerate}
 which completes the proof.
\end{proof}

\begin{theorem}[Inverse function]
    Let \(I\) be an open set and \(f : I \to \Reals\) is continuous and differentiable such that its derivate is non-zero. Thus, \(f\) is either monotonic. Furthermore, it is one to one then it has a differentiable inverse \(f^{-1}\):
    \begin{equation*} 
        \func{f^{-1}}{x} = \dfrac{1}{\func{f'}{\func{f^{-1}}{x}}}
     \end{equation*}
\end{theorem}
\begin{proof}
    limit algebra
\end{proof}
