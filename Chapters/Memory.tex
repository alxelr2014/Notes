\chapter{Memory}
We want a fast, large, and a cheap memory. but ofc that is not possible.
Principle of locality gets us closer to our goal.
\section{Principle of Locality}
\subsection{Temporal locality}
if you accessed a part of memory, it is very likely you access the same part soon.
\subsection{Spatial locality}
if you accessed a part of memory, it is very likely you access the nearby parts soon.
To achieve our goal, we use memory hierarchy, from the big and slow to smaller and faster.
\begin{center}
    \begin{tabular}{c|c|c}
        Type                      & Speed(ns) & Size(byte) \\ \hline
        Registers                 & 1s        & 100s       \\\hline
        On-Chip Cache             & 10s       & Ks         \\\hline
        Second Level Cache (SRAM) & 10s       & Ks         \\\hline
        Main Memory (DRAM)        & 100s      & Ms         \\\hline
        Secondary Storage (Disk)  & 10s (ms)  & Gs         \\\hline
        Tertiary Storage (Tape)   & 10s (sec) & Ts
    \end{tabular}
\end{center}

