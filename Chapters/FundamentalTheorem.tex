\chapter{The Fundamental Theorem of Arithmetic}
induction, well-ordering principle, divisibility, gcd is commutative,associative, and distributive, relatively prime, primes, fundamental theorem of arithmetic.
\section{The series of reciprocals of the primes}
\begin{theorem}
    The infinite series \(\sum \frac{1}{p_n}\) diverges.
\end{theorem}
\begin{proof}
    Suppose the sum converges instead and let \(k\) be such that 
    \begin{align*}
        \sum_{n = k + 1}^{\infty} \dfrac{1}{p_n} \leq \dfrac{1}{2}
    \end{align*}
    Let \(Q = p_1 \dots p_k\), then for all \(r \geq 1\), 
    \begin{align*}
        \sum_{n = 1}^r \dfrac{1}{1 + nQ} &\leq \sum_{t =1}^{\infty} \bracket{\sum_{m = k + 1}^{\infty} \frac{1}{p_m}}^t\\
        & \leq \sum_{t =1}^{\infty} \bracket{\frac{1}{2}}^t\\
        & = 1\\
    \end{align*}
    By allowing \(r \to \infty\), we get 
    \begin{equation*}
        \sum_{n = 1}^{\infty} \dfrac{1}{1 + nQ} \leq 1
    \end{equation*}
    However, this is a constradiction as the sum diverges as 
    \begin{equation*}
    \sum_{n = 1}^{\infty} \dfrac{1}{1 + nQ} \leq \sum_{n = 1}^{\infty} \dfrac{1}{Q + nQ}  \leq \dfrac{1}{Q} \sum_{n = 2}^{\infty} \dfrac{1}{n} 
    \end{equation*}
    Therefore, \(\sum \frac{1}{p_n}\) must diverge.
\end{proof}
Euclidean algorithm, division algorithm, gcd algorithm.

\begin{exercise}
    \item If \((a,b) = 1\) and if \(c \mid a\) and \(d \mid b\), then \((c,d) = 1\).
    \begin{solution}
        Let \(e = (c,d)\), since \(e \mid c\), then \(e \mid a\) and similarly, \(e \mid b\). Therefore, \(e \mid (a,b)\) which means \(e = 1\).
    \end{solution} 
    \item If \((a,b) = (a,c) = 1\), then \((a,bc) = 1\).
    \begin{solution}
        Let \(d = (a,bc)\) and \(e = (b,d)\). Then, \(e \mid d\) and hence \(e \mid a\), as a result \(e \mid (a,b)\) which means \(e = 1\). Note that, \(d \mid bc\) but \((b,d) = 1\) thus, \(d \mid c\). Since \(d \mid a\), then \(d \mid (a,c)\) and hence \(d = 1\).
    \end{solution}
    \item If \((a,c) = 1\), then \((a,bc) = (a,b)\).
    \begin{solution}
        Let \(d = (a,bc)\) and \(e = (c,d)\). Then, \(e \mid d\) and hence \(e \mid a\), as a result \(e \mid (a,c)\) which means \(e = 1\). Note that, \(d \mid bc\) but \((c,d) = 1\) thus, \(d \mid b\). Since \(d \mid a\), then \(d \mid (a,b)\). Moreover, \((a,b) \mid d\) since \((a,b) \mid a\) and \((a,b) \mid bc\). Therefore, \(d = (a,b)\).
    \end{solution}
    \item If \(m \neq n\) compute the \(\func{\gcd}{a^{2^m} + 1, a^{2^n} + 1}\) in terms of \(a\). 
    \begin{solution}
        WLOG assume \(n < m\) and note that 
        \begin{equation*}
            a^{2^m} - 1 = a^{2^{m-n} \cdot 2^n} - 1 = \bracket{a^{2^n} - 1}  \bracket{a^{2^n} + 1} \bracket{a^{2 \cdot 2^n} + 1} \dots \bracket{a^{2^{m-n-1} \cdot 2^n} + 1} 
        \end{equation*}
        and hence 
        \begin{equation*}
            a^{2^n} + 1 \mid a^{2^m} - 1
        \end{equation*}
        Therfore, 
        \begin{equation*}
            \bracket{a^{2^n} + 1 , a^{2^m} + 1} = \bracket{2,a^{2^n} + 1} = \begin{cases}
                1 & a \text{ is even}\\
                2 & a \text{ is odd}
            \end{cases}
        \end{equation*}
    \end{solution}
    \item If \(a > 1\), then \((a^m -1 , a^n - 1) = a^{(m,n)} - 1\).
    \begin{solution}
        If \(m = n\), then the result hold obviously. Suppose \(n < m\) and note that 
        \begin{equation*}
            a^m - 1 = (a^{m-n})(a^n - 1) + (a^{m-n} - 1)
        \end{equation*}
        and therefore, \((a^m - 1, a^n - 1) = (a^{m - n} - 1 , a^n)\). By applying the Euclidean algorithm we arrive at the conclusion.
    \end{solution}
    \item Given \(n > 0\), let \(S\) be a set whose elements are positive integers \(\leq 2n\) such that if \(a\) and \(b\) are in \(S\) and \(a \neq b\), then \(a \nmid b\). What is the maximum number of integers that \(S\) can contain?
    \begin{solution}
        Note that \(S\) can not have more than \(n\) elements. To see this, consider the sets \(\set<m 2^k>{k \geq 0, \; m 2^k \leq 2n}\) for \(m = 1,3, \dots, 2n - 1\). There are \(n - 1\) such sets and they partition the set \(\set{1, 2 , \dots , 2n}\). No two elements of \(S\) can come from the same set, and as a result \(\abs{S} \leq n - 1\) by pigeonhole principle. However, note that \(S = \set{n+1 , n +2 , \dots , 2n}\) satisfies the conditions and has exactly \(n - 1\) elements. Therefore, the maximum of \(n- 1\) elements is attainable for all \(n > 0\). 
    \end{solution}
    \item If \(n > 1\) prove that the sum \(\sum_{k = 1}^n \frac{1}{k}\) is not an integer. Also show that for any signing of the sum \(\sum_{k = 1}^n (-1)^{a_k} \frac{1}{k}\) is not an integer.
    \begin{solution}
        Let \(p\) be the largest prime less than or equal to \(n\). Let \(r,s \in \Integers\) be such that \(s \neq 0\) and \((r,s) = 1\).
        \begin{equation*}
            \frac{r}{s} = \sum_{\substack{k = 1 \\ k \neq p}}^n (-1)^{a_k}\dfrac{1}{k}
        \end{equation*}
        We claims that \(p \nmid s\). For the sake of contradiction suppose there is an integer \(q\) such that \(s = pq\). Then, 
        \begin{align*}
            r &= s\bracket{\sum_{\substack{k = 1 \\ k \neq p}}^n (-1)^{a_k}\dfrac{1}{k}}\\
             &= \sum_{\substack{k = 1 \\ k \neq p}}^n (-1)^{a_k}\dfrac{pq}{k}
        \end{align*}
        Since \((p,k) = 1\) for all \(k \leq n\) and \(k \neq p\), then it must be the case that the sum
        \begin{equation*}
            \sum_{\substack{k = 1 \\ k \neq p}}^n (-1)^{a_k}\dfrac{q}{k}
        \end{equation*}
        is an integer. Therefore, we have shown that there is integer \(t\) such that \(r = pt\), which contradicts our assumption that \((r,s)  = 1\). Thus, \(p\) does not divide \(s\). To conclude, consider the sum 
        \begin{equation*}
            \dfrac{r}{s} + \dfrac{(-1)^{a_p}}{p} = \dfrac{pr + (-1)^{a_p}s}{ps}
        \end{equation*}
        which can not be integer as \(p \nmid s\).
    \end{solution}
\end{exercise}