\chapter{SQL}
\section{Creating table}
\begin{lstlisting}[language=sql]
CREATE TABLE stt(
    stid    CHAR(8) PRIMARY KEY,
    stname  VARCHAR(20) NOT NULL,
    stlev   CHAR(2) DEFAULT 'BS',
    stssn   CHAR(10) UNIQUE NOT NULL,
    CHECK (stlev IN ('BS', 'MS', 'PD'))
)
\end{lstlisting}
\begin{lstlisting}[language=sql]
CREATE TABLE stcot(
    stid    CHAR(8),
    coid  VARCHAR(20) NOT NULL,
    qtr   SMALLINT,
    yr   CHAR(5),
    grade DECIMAL(4,2),
    PRIMARY KEY (stid, coid),
    FOREIGN KEY (stid) REFERENCES stt(stid),
    FOREIGN KEY (coid) REFERENCES cot(coid)
)
\end{lstlisting}

\section{Remove table}
\begin{lstlisting}[language=sql]
CREATE TABLE stt [CASCADE | RESTRICT]
\end{lstlisting}

\section{Update table}
\begin{lstlisting}[language=sql]
ALTER TABLE stt 
    ADD COLUMN stdepid INT;

ALTER TABLE stt
    DROP COLUMN stssn [CASCADE | RESTRICT];

ALTER TABLE stt
    ALTER COLUMN stlev SET DEFAULT 'BS';

ALTER TABLE stt
    ADD CONSTRAINT con1 CHECK(stdepid  BETWEEN 11 AND 45);

ALTER TABLE stt
	DROP CONSTRAINT con1;
\end{lstlisting}
\begin{lstlisting}[language=sql]
INSERT INTO stt (stid,stname,stlev,stdepid)
    VALUES ('999','ali','BS',4);

INSERT INTO stt 
    VALUES ('888','mohsel','BS',5);

DELETE FROM stt WHERE
    stid = '999' AND stname = 'ali';

DELETE FROM stt WHERE TRUE;
-- is the same as
TRUNCATE TABLE stt;

\end{lstlisting}
In case we may want to delete a foreign key or what have you, we can determine the action in defintion 

\begin{lstlisting}[language=sql]
CREATE TABLE stcot(
    stid    CHAR(8),
    coid  VARCHAR(20) NOT NULL,
    qtr   SMALLINT,
    yr   CHAR(5),
    grade DECIMAL(4,2),
    PRIMARY KEY (stid, coid),
    FOREIGN KEY (stid) REFERENCES stt(stid) ON [DELETE| UPDATE] [RESTRICT | CASCADE | SET NULL | SET DEFAULT | NO ACTION],
    FOREIGN KEY (coid) REFERENCES cot(coid)
    -- on default is RESTRICT
)
\end{lstlisting}

\section{Retrieving information}
\begin{lstlisting}[language=sql]
SELECT * FROM stt;
SELECT stid FROM stt;
SELECT stid,stlev FROM stt WHERE stdepid = 11;
SELECT stid FROM stt WHERE stdepid IS [NOT] NULL;
SELECT stid FROM stt 
    WHERE stname [NOT] LIKE ['%N','M^%', '--A--'];
-- '%N' : ends with N
-- '^M%': does not start with M
-- '--[A-D]--' : exactly 5 characters and the third is A
SELECT * FROM stt ORDER BY stid [ASCENDING | DESCENDING];

-- We also have INTERSECT, UNION, UNION ALL, EXCEPT between two tables

SELECT stid FROM stt as T1 WHERE T1.stid = '98999';
-- Other keywords ANY, IN, NOT IN, ALL , EXISTS

SELECT [COUNT(*), MAX(grade), MIN(grade), SUM(grade * 2 ), SUM(grade)] FROM stcot
        WHERE qtr = 1
    GROUP BY (stid)
    HAVING COUNT(DISTINCT(COID)) > 10;

-- First WHERE is executed the GROUP By
-- When using GROUP BY, only the grouped entities and accumulative function are allowed to be queried

SELECT * FROM stt,stcot 
    WHERE stt.stid = stcot = stid;

SELECT * FROM stt [INNER | [LEFT | RIGHT | FULL] OUTER] JOIN stcot [ON stt.stid = stcot.stid] [USING(stid)]

-- LEFT OUTER if there is no row in stcot corresponding to the row in stt
-- RIGHT OUTER if there is no row in stt for some row in setcounter

WITH RECURSIVE rec_table(
    (SELECT ...)
    UNION [ALL]
    (SELECT ...)
)
\end{lstlisting}