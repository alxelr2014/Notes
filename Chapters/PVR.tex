\chapter{Present value relations}
\section{Cashflows and Assets}
Cashflow is the flow of cash :). Asset is a sequence of cashflows.
\begin{equation*}
    \text{Asset}_t = \set{\text{CF}_t , \text{CF}_{t+1} , \dots }
\end{equation*}
The value of an asset is a function of its cashflows.
\begin{equation*}
    \text{Value of asset}_t = \func{V_t}{\text{CF}_t , \text{CF}_{t+1} , \dots}
\end{equation*}

There are two distinct cases we valuating an assets
\begin{itemize}
    \item with no uncertainty; all the cashflows are known
    \item with uncertainty;
\end{itemize}
\subsection{No uncertainty}
A \textbf{numeraire} date should be picked, typically \(t = 0\), then cashflows are converted to \textbf{present value}
\begin{equation*}
    \func{V_0}{\text{CF}_1 , \text{CF}_{2} , \dots} = \left(\dfrac{\$_1}{\$_0} \right)\times \text{CF}_1  + \left(\dfrac{\$_2}{\$_0} \right)\times \text{CF}_2 + \dots
\end{equation*}
then the \textbf{net present value} is
\begin{equation*}
    \func{V_0}{\text{CF}_0 , \text{CF}_{1} , \dots} = \text{CF}_0 +  \left(\dfrac{\$_1}{\$_0} \right)\times \text{CF}_1  +  \dots
\end{equation*}
\begin{enumerate}
    \item when there is up front investment \(\text{CF}_0\) is negative.
    \item Note that any \(\text{CF}_t\) can be negative (future costs).
\end{enumerate}

\begin{align*}
    \$_0 & = (1 + r) \$_1   \\
    \$_0 & = (1 + r)^2 \$_2 \\
         & \vdots           \\
    \$_0 & = (1 + r)^T \$_T \\
\end{align*}

where \(r\) is \textbf{opportunity cost of capital} and \(\frac{\$_t}{\$_0}\) is called the \textbf{discount factor}.

\section{Perpetuity}
It is a paper that pays \(C\) cashflow annually till the end of time. The present value of a perpetuity is calculated as follow, assuming the interest rate is constant \(r\) and the perpetuity pays from the first year:
\begin{align*}
    \mathrm{PV} &= \dfrac{C}{r + 1} + \dots + \dfrac{C}{(r + 1)^n} + \dots \\
    &= \dfrac{C}{r + 1} \dfrac{1}{1 - \frac{1}{r + 1}} = \dfrac{C}{r + 1} \dfrac{r + 1}{r} \\
    &= \dfrac{C}{r}
\end{align*}

Now suppose \(C\) grows with grows rate \(g\), then the present value is 

\begin{align*}
    \mathrm{PV} &= \dfrac{C}{r + 1}  + \dfrac{C (1+g)}{(1+r)^2}+ \dots + \dfrac{C (1 + g)^{n-1}}{(r + 1)^n} + \dots \\
    &= \dfrac{C}{r + 1} \dfrac{1}{1 - \frac{1 + g}{r + 1}} = \dfrac{C}{r + 1} \dfrac{r + 1}{r - g} \\
    &= \dfrac{C}{r - g} , \quad r > g
\end{align*}

\section{Annuity}
It is a paper that pays a cashflow \(C\) for a period \(T\). Assuming the assumption made above, for the present value of annuity is 

\begin{align*}
    \mathrm{PV} &= \dfrac{C}{r + 1} + \dots + \dfrac{C}{(r + 1)^T}\\
    &= \dfrac{C}{r + 1} \dfrac{1 - \dfrac{1}{(r + 1)^T}}{1 - \frac{1}{r + 1}} \\
    &= \dfrac{C}{r} - \dfrac{C}{r (1 + r)^T}
\end{align*}

It is like holding out to perpetuity for \(T\) days and then selling. Equivalently, buy a perpetuity today and give a perpetuity at time \(T\). 

\section{Compound}
Let \(r\) be the \textbf{Annual Percentage Rate} amd \(n\) be the periods of compounding. Then, out of convention, \(\frac{r}{n}\) is the per-period rate for each period and thus the \textbf{Effective Annual Rate} is 
\begin{equation*}
    r_{\mathrm{EAR}} = \left( 1+ \dfrac{r}{n} \right)^n - 1
\end{equation*}
Also, continuous compounding happens when you let \(n \to \infty\) which means 
\begin{equation*}
    r_{\mathrm{EAR}} =e^r - 1
\end{equation*}

\section{Inflation}
 measures the purchasing value of money. Different from time-value of money which says that money will be worth less over time. Hence, inflation also depends on the price of goods. Suppose you have wealth \(W_0\) today and \(W_t\) at time \(t\) and the inflation changes at a constant rate of \(\pi\). Then 
 \begin{equation*}
    \left(1+ r_{\mathrm{nominal}} \right)^k = \dfrac{W_t}{W_0}
 \end{equation*}

 is the \textbf{nominal} return, which tells how much your money changed in this period as a number. But to figure out how much it actually changed, that is how much more/less you can consume, we have 
 \begin{align*}
    \left( 1+ r_{\mathrm{real}} \right)^k &= \dfrac{W_t}{W_0} \dfrac{1}{(1 + \pi)^k} \\
    &= \dfrac{(1+ r_{\mathrm{nominal}})^k}{(1 + \pi)^k} \\
    \implies  r_{\mathrm{real}} &= \dfrac{1+ r_{\mathrm{nominal}}}{1 + \pi} - 1 \\
    & \simeq r_{\mathrm{nominal}} - \pi
 \end{align*}
 
 which is the \textbf{real} return. 