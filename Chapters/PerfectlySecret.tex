\chapter{Perfectly Secret Encryption}
\section{perfectly secure encryption}
Let \(K\) and \(M\) be two random variables, where \(K\) is the result of \(\Gen\) and \(M\) is the message. We can assume that they are independent. Furthermore, \( C = \func{\Enc_K}{M}\) is also a random varible. By the Kerckhoff's principle, we assume that the distribution on \(M\) and \(\Enc\) is known and only \(K\) is unknown. 

\begin{definition}[Perfectly secure encrption]
    An encryption scheme is perfectly secure if for all \(c \in \Ciphers\) with \(\prob{C = c} > 0\):
    \begin{equation}
        \forall m \in \Messages, \quad \condProb{M = m}{C = c} = \prob{M = m}
    \end{equation}
 \end{definition} 

 \begin{proposition}
     An encryption scheme \(\Pi\) is perfectly secure if and only if 
     \begin{equation}
         \forall m,m' \in \Messages, \quad \prob{\func{\Enc_K}{m} = c} = \prob{\func{\Enc_K}{m'} = c}
     \end{equation}
 \end{proposition}

 \begin{proof}
     Suppose \(\Pi\) is perfectly secure then (assuming that \(\prob{M = m} > 0\))
     \begin{align*}
         \prob{\func{\Enc_K}{m} = c} &=  \condProb{C = c}{M = m} = \dfrac{\condProb{M = m}{C = c} \prob{C = c}}{\prob{M = m}}\\
         &= \dfrac{\prob{M = m} \prob{C= c}}{\prob{M= m}} = \prob{C = c}
     \end{align*}
     Now if the equation holds for \(\Pi\) then (again assuming that \(\prob{M = m} > 0 \))
     \begin{align*}
        \condProb{M = m}{C = c} &= \dfrac{\condProb{C = c}{M = m} \prob{M = m}}{\prob{C = c}}\\
        &= \dfrac{\func{\Enc_K}{m} \prob{M = m}}{\sum_{m^{\ast}} \condProb{C = c}{M = m^{\ast}} \prob{M = m^{\ast}}}\\
        &= \dfrac{\prob{M = m}}{\sum_{m^{\ast}} \prob{M = m^{\ast}}} = \prob{M = m}
     \end{align*}
 \end{proof}

 \section{Prefect adversarial indistinguishability}
 An encryption scheme is \textbf{perfectly indistinguishable} if no adversary \(\Adversary\) can succeed with probability better than \(\frac{1}{2}\). Formally, we run the following experiment \(\PrivK_{\Adversary, \Pi}^{eav}\)
 \begin{enumerate}
     \item \(\Adversary\) outputs a pair \(m_0,m_1 \in \Messages\).
     \item \(k = \Gen\) and \(b\) - chosen from \(\set{0,1}\) uniformly - then the \textbf{challenge ciphertext} \(c = \func{\Enc_k}{m_b}\) is given to \(\Adversary\).
     \item \(\Adversary\) tries to determine the which message was encrypted and then outputs \(b'\).
     \item 
     \begin{equation*}
        \PrivK_{\Adversary, \Pi}^{eav}
         \begin{cases}
            1 & b' = b \ \text{then}\  \Adversary \ \text{succeeds}\\
            0 & b' \neq b \  \text{then}\  \Adversary \ \text{fails}
         \end{cases}
     \end{equation*}
 \end{enumerate}

 Since \(\Adversary\) can guess randomly \(\prob{\PrivK_{\Adversary, \Pi}^{eav} = 1} \geq \frac{1}{2}\) and thus a scheme is perfectly indistinguishable if 
 \begin{equation*}
    \prob{\PrivK_{\Adversary, \Pi}^{eav} = 1}  = \frac{1}{2}, \quad \forall \Adversary
 \end{equation*}

 \begin{proposition}
     \(\Pi\) is perfectly secret if and only if it is perfectly indistinguishable.
 \end{proposition}

 \begin{proof}
    \begin{equation*}
    \prob{\PrivK_{\Adversary, \Pi}^{eav} = 1}  = \condProb{M = m }{C = c}
    \end{equation*}
 \end{proof}

 \section{One-time pad}
Let \(l \in \Naturals^{\ast}\) and \(\Messages = \Keys = \Ciphers = \set{0,1}^{l}\) then \textit{one-time pad} scheme is describe as follows 
\begin{itemize}
    \item \(\Gen\) is uniform.
    \item \(\func{\Enc_k}{m} = k \oplus m\).
    \item \(\func{\Dec_k}{c} = k \oplus c\).
\end{itemize}

\begin{theorem}
    One-time pad is perfectly secure.
\end{theorem}

\begin{proof}
    \begin{align*}
        \condProb{M = m }{C = c} &= \dfrac{\condProb{C = c}{M = m} \prob{M = m}}{\sum_{m^{\ast}} \condProb{C = c}{M = m^{\ast}} \prob{M = m^{\ast}}}\\
        &= \dfrac{\prob{K = c \oplus m}}{\sum_{m^{\ast}} \prob{K = c \oplus m^{\ast}} \prob{M = m^{\ast}}} \prob{M = m}\\
        &= \prob{M = m}
    \end{align*}
\end{proof}

\begin{proposition}
    If \(\Pi\) is perfectly secure then we must have \(\abs{\Keys} \geq \abs{\Messages}\).
\end{proposition}

\begin{proof}
    Suppose \(\abs{\Keys} < \abs{\Messages}\) and let \(c \in \Ciphers\) be a ciphertext and define \(\func{\Messages}{c}\) to the 
    \begin{equation*}
        \func{\Messages}{c} = \set<m>{m = \func{\Dec_k}{c} \ \text{for some}\ k \in \Keys }
    \end{equation*}
    Then \(\abs{\func{\Messages}{c}} \leq \abs{\Keys} < \abs{\Messages}\) and therefore there exists \(m \in \Messages\) such that \(m \notin \func{\Messages}{c}\) hence 
    \begin{equation*}
        \condProb{M = m}{C = c} = 0 \neq \prob{M = m}
    \end{equation*}
    Note that we assumed the distribution over \(\Messages\) is uniform.
\end{proof}

\begin{theorem}[Shannon's Theorem]
    \(\Pi\) with \(\abs{\Messages} = \abs{\Keys} = \abs{\Ciphers}\) is perfectly secure if and only if 
    \begin{enumerate}
        \item \(\Gen\) is uniform.
        \item \(\forall m \in \Messages\) and \(c \in \Ciphers\), \(\exists! k \in \Keys\) such that \(\func{\Enc_k}{m} = c\).
    \end{enumerate}
\end{theorem}

\begin{proof}
    
\end{proof}