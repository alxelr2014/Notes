\documentclass[12pt]{book}
\usepackage[a4paper,bindingoffset=0.2in,%
            left=0.75in,right=0.75in,top=1in,bottom=1in,%
            footskip=.25in]{geometry}
\usepackage{fancyhdr}
\setlength{\headheight}{15.2pt}
\usepackage{inputenc}
\pagestyle{fancy}

\renewcommand{\chaptermark}[1]{\markboth{\thechapter.\ #1}{}}
\renewcommand{\sectionmark}[1]{\markright{\thesection\ #1}}
\fancyhead[LE,RO]{\textbf{\thepage}}
\fancyhead[LO]{\textbf{\rightmark}}
\fancyhead[RE]{\textbf{\leftmark}}
\fancyfoot{}
\fancypagestyle{plain}
{
    \fancyhf{}
}


\usepackage{amsmath}
\usepackage{amssymb}
\usepackage{mathtools}
\usepackage{xcolor}
\usepackage{enumitem}
\usepackage{breqn}


\usepackage{common}
\usepackage{english-theorems}
\setcounter{tocdepth}{1}

\newcommand{\scdot}{\; \cdot \;}

\newcommand*{\textcal}[1]{%
  % family qzc: Font TeX Gyre Chorus (package tgchorus)
  % family pzc: Font Zapf Chancery (package chancery)
  \textit{\Large{{\fontfamily{pzc}\selectfont#1}}}%
}

\begin{document}
\tableofcontents
\clearpage
\ifodd\value{page}\else
\thispagestyle{empty}
\fi
\chapter{Real Numbers}
\thispagestyle{headings}
\section{Axiomatic Formulation of Real Numbers}
The building axioms of real numbers is devided into three groups based on the properties they are describing.
\begin{enumerate}
    \item Field axioms.
    \item Order axioms.
    \item Completeness axiom.
\end{enumerate}
\subsection{Field Axioms}
A field is a non-empty set \(\Field\) with two binary operations \textit{addition}, \(+\), and \textit{multiplication}, \(\cdot\). For all \(x,y,z \in  \Field\):
\begin{enumerate}[wide,start=1,label={Axiom \arabic*.}]
    \item Addition and multiplication are commutitive.
          \[x + y = y + x, \quad \ x \cdot y = y \cdot x\]
    \item Addition and multiplication are associative.
          \[x + (y + z) = (x + y) + z, \quad x\cdot (y\cdot z) = (x \cdot y) \cdot z\]
    \item  Multiplication distributes over addition.
          \[x \cdot (y + z) = x \cdot y + x \cdot z\]
    \item There exists a number \(0\) such that for every number \(x\):
          \[ x + 0  = 0 + x = x\]
    \item There exists a number \(1\) such that for every number \(x\):
          \[ x \cdot 1 = 1 \cdot x = x\]
    \item  For every number \(x\), there exists a number \(y\) such that:
          \[x + y = 0\]
          \(y\) is called the negative of \(x\) and is denoted by \(-x\).
    \item For every number \(x \neq 0\), there exists a number \(y\) such that:
          \[x \cdot y = 1\]
          \(y\) is called the reciprocal of \(x\) and is denoted by \(x^{-1}\) or \(\dfrac{1}{x}\).
\end{enumerate}
\subsection{Order Axioms}
The order axioms establishes an ordering on the numbers of \(\Field\) to determine which element is larger or smaller. To achieve an ordering, we define the set of positive real numbers \(\Field^+ \subset \Field\).
\begin{enumerate}[wide,resume,label={Axiom \arabic*.}]
    \item The \(\Field^+\) is closed under addition and multiplication.
          \[\forall x,y \in \Field^+, \quad (x + y) \in \Field^+ \text{ and } (x \cdot y) \in \Field^+\]
    \item \(0 \notin \Field^+\).
    \item For every number \(x \neq 0\), either \(x \in \Field^+\) or \(-x \in \Field^+\).
\end{enumerate}
We then define the binary operator \( > \) such that \(x > y \) whenever \((x - y) \in \Field^+\).
\subsection{Completeness Axiom}
Given that \((\Field, + ,\cdot , >)\) is an ordered field, we define the followings:
\begin{definition} [Upper bound]
    A set \(S \in \Field\) has an upper bound if for some \(a \in \Field\) is greater or equal to all element of \(S\). That is, \(\forall x \in S\: a \geq x\). We say that \(S\) is bounded from above.
\end{definition}
\begin{definition} [The least upper bound]
    \(a \in \Field\) is the least upper bound of a set \(S \subset \Field\) if it is smaller than every upper bound of \(S\). We say \(a\) is the supremum of \(S\), denoted by \(a = \sup{S}\).
\end{definition}
Note that, if the least upper bound exists, it must be unique.
\begin{enumerate}[wide,resume,label={Axiom \arabic*.}]
    \item If \(S\) is a non-empty set that bounded from above that it has supremum.
\end{enumerate}
\begin{theorem}
    There exists a unique set that satisifies all the axioms above. It is denoted by \(\Reals\), the set of real numbers.
\end{theorem}
\begin{proof}
    The existence of \(\Reals\) is proved in many ways. One way to construct real numbers uses \textit{Dedekind Cuts}. Let the pair of rational sets \((A,B)\) be a partition of \(\Rationals\) such that:
    \begin{enumerate}
        \item \(A \neq \emptyset\) and \(A \neq \Rationals\).
        \item \(\forall x,y \in \Rationals \; \mathrm{s.t.} \; x < y,\: y \in A \implies x \in A\).
        \item \(\nexists x \in A \; \mathrm{s.t.} \; \forall y \in A, x \geq y\).
    \end{enumerate}
    For convenience we let \(A\) represent the pair \((A,B)\) as \(A\) completely determines \(B\).
    We define \(+\), \(\cdot\), and \( > \) as follows:
    \begin{flalign*}
        A + B &= \set< a + b >{ a \in A, b\in B} &&\\
        \textcal{0} &= \set< a>{ a < 0}&& \\
        -A &= \set<a'>{\forall a \in A, a' < -a }&&
        \intertext{For \(\cdot\), we first take two set \(A\) and \(B\) that have some positive elements.}
        A \cdot B &= \set< a \cdot b >{a \in A \land a \leq 0, b\in B \land b \leq 0 } \: \cup \: \textcal{0} &&\\
        \intertext{If \(A\) or \(B\) did not have any positive elements, we first take the negative of the set, and then multiply the two sets and take the negative of the product. Similarly, we define the reciprocal of \(A\) if \(A\) has a positive element.}
        \textcal{1} &= \set<a >{ a < 1}&&\\
        A^{-1} &= \set< a' >{\forall a \in A, a > 0, a' < \dfrac{1}{a} } &&
        \intertext{Lastly:}
        A &> B \text{ if } A \supset B &&
    \end{flalign*}
    Also, if a non-empty set \(S\) of real numbers is bounded from above, then it has a supremum in \(\Reals\) equal to \( \bigcup S\). It is left to the reader that the \((\Reals, + , \cdot, >)\) satisifies the axioms above.

    The set of real numbers is unique in sense that if \((\Reals, + , \cdot, >)\) and \((\Reals',+',\cdot', >')\) both satisify the axioms, then there exists bijective mapping \(\alpha : \Reals \to \Reals'\) such that:
    \begin{flalign*}
        \func{\alpha}{x + y} &= \func{\alpha}{x} +' \func{\alpha}{y} &&\\
        \func{\alpha}{0} &= 0'&&\\
        \func{\alpha}{x\cdot y} &= \func{\alpha}{x}\cdot'\func{\alpha}{y}&&\\
        \func{\alpha}{1} &= 1' &&\\
        x < y &\iff \func{\alpha}{x} <' \func{\alpha}{y} &&
    \end{flalign*}
    Lastly, if \(S\) is a non-empty set in \(\Reals\) and \( \func{\alpha}{S} = \set<\func{\alpha}{x}>{ x \in S} \), then \(S\) is has an upper bound if and only if \(\func{\alpha}{S}\) has an upper bound. Furthermore \( \func{\alpha}{\sup{S}} = \sup{\func{\alpha}{S}}\).
\end{proof}
\begin{results}
    \leavevmode
    \begin{enumerate}
        \item The set of natural numbers \(\Naturals\) in \(\Reals\) is not bounded from above.
        \item Let \(x \in \Reals\) be such that for all \(n \in \Naturals\)
              \[ 0\leq x \leq \dfrac{1}{n}\]
              then \(x = 0\).
        \item (Archimedean Property) For all \(a,b > 0\) there exists \(n \in \Naturals\):
              \[ na > b\]
        \item Consider \(I_n = \clcl{a_n}{b_n} \;\forall n \in \Naturals \) such that \(I_1 \supset I_2 \supset \dots\;\). Then \(\cap{I_n}\) is not empty. Moreover, if for each \(e > 0\) there exists \( n\) such that \(b_n - a_n < e\), then \(\cap{I_n}\) is a single point.
        \item \(\sqrt{2} \in \Reals\). In addition, for all \(p > 0\), there is a positive real number \(q \) such that \(q^2 = p\)
    \end{enumerate}
\end{results}
\newpage
{\Large\textbf{Exercises}}
\begin{enumerate}
    \item Prove that the addition and multiplication identity elements are unique.
    \item Show that the Trichotomy law holds for \( (>) \). That is, exactly one of the following three is true.
          \[ x > y \qquad x = y \qquad y > x\]
    \item Show that \(1 \in \Field^+\).
    \item Show that if \(x > -1\) and \(n \in \Naturals\):
          \[(1 + x)^n \geq 1 + nx\]
          and equality only holds when \(n = 1\).
    \item Let \(F_p = \set{ 0, 1, \dots, p -1 }\) where \(p\) is a prime number. Define \(+\) and \(\cdot\) to be the modular addition and product modulus \(p\), respectively. Investigate whether if \(F_p\) can be ordered.
    \item Consider the set of all rational polynomials \(\Rationals[x]\):
          \[\Rationals[x] = \set< \frac{a_m x^m + \dots + a_1 x + a_0}{b_n x^n + \dots + b_1 x + b_0}>{ a_i, b_j \in \Rationals, b_n \neq 0}\]
          Show that \(\Rationals[x]\) under the normal addition and multiplication is a field. Furthermore, show that \(\Rationals^+[x] = \set< q \in \Rationals[x] >{ a_m \cdot b_n > 0 }\) constitutes an ordering on \(\Rationals[x]\).
\end{enumerate}



\chapter{Topology and Metric Spaces}
\thispagestyle{headings}
\section{Topology}

Let \(X\) be a set. A \textbf{topology} on \(X\) is a collection \(\scrT\)  of subsets of \(X\) called \textbf{open set} having the following properties
\begin{enumerate}
    \item If \(U_{\alpha} \in \scrT\) where \(\alpha \in A\) for any set \(A\) then, \(\cup_{\alpha \in A} U_{\alpha} \in \scrT\).
    \item If \(U_{\alpha} \in \scrT\) where \(\alpha \in A\) for any finite set \(A\) then, \(\cap_{\alpha \in A} U_{\alpha} \in \scrT\).
    \item \(X,\emptyset \in \scrT\).
\end{enumerate}
A topological space is a pair \((X,\scrT)\), where \(\scrT\) is a topology on \(X\).

\begin{example}
    On any set \(X\) we can define two topologies. The \textit{trivial topology} on \(X\) consists of \(\set{\emptyset, X}\). The \textit{discrete topology} on \(X\) is \(\powerSet{X}\).
\end{example}

If \(\scrT_1\) and \(\scrT_2\) are topologies on \(X\), we say that \(\scrT_1\) is \textit{weaker}/\textit{coarser} than \(\scrT_2\), or that \(\scrT_2\) is \textit{stronger}/\textit{finer} than \(\scrT_1\), if \(\scrT_1 \subset \scrT_2\).

\begin{definition}
    Let \(X\) be a topological space and \(x \in X\), \(N\) is called a \textbf{neighbourhood} of \(x\) if there exists an open set \(G\) such that \(x \in G \subset N\). We say that \(x\) is an interior point of \(N\) if \(N\) is neighbourhood of \(x\).
\end{definition}

\begin{proposition}
    A subset \(U\) is open if and only if \(U\) is a neighbourhood for all \(x \in U\).
\end{proposition}

\begin{proof}
    Let \(U\) be a neighbourhood for all of its points. That is, for every \(x \in U\) there is an open set \(G_x\) such that \(x \in G_x \subset U\). Then, \(\cup_x G_x \subset U\) however, \(U \subset \cup_x G_x\). Therefore, \(U = \cup_x G_x\) and by the first axiom \(U\) is an open set. If \(U\) is an open set, then \(U\) is a neighbourhood for each point \(x \in U\).
\end{proof}

\begin{definition}
    A subset \(F\) of \(X\) is called \textbf{closed} if \(F^c\) is open. \textbf{Closure} of a set \(E\) is the intersection of all closed sets that include \(E\) and it is denoted by \(\closure E\) or \(\bar{E}\). 
\end{definition}

\begin{proposition}
    Let \(X\) be a topological space and \(E\) a subset of \(X\). Then, \(\closure E\) is closed.
\end{proposition}

\begin{proof}
    We know that \(\closure E = \cap_{E \subset F} F\) where \(F\) are closed set. Then, \((\closure E)^c = \cup_{E \subset F} F^c\) which is an open set and hence \(\closure E\) is closed.
\end{proof}

\begin{proposition}
    If \(E\) is subset of a topological space \(X\) and \(x in X\), then \(x \in \closure E\) if and only if \(U \cap E = \emptyset\) for every open neighbourhood of \(U\) of \(x\).
\end{proposition}

\begin{proof}
    Suppose there exists an open neighbourhood of \(x\), \(U\) such that, \(U \cap E = \emptyset\). Then, \(F = \closure E \cap U^c\) is a closed that constains \(E\) hence, \(\closure E \subset F\) and \(x \notin \closure E\). If \(x \notin \closure E\), then \((\closure E)^c\) is an open neighbourhood of \(x\) which does not meet \(E\).
\end{proof}

\begin{definition}
    A \textbf{limit point} of \(E\) is a point \(x \in X\) such that \(E \cap U \backslash \set{x} \neq \emptyset\) for every neighbourhood \(U\) of \(x\). The set of all limit points of \(E\) is denoted by \(E'\) or \(\lim E\). If \(x \in E\) but \(x \notin \closure E\), then \(x\) is called an \textbf{isolated point} of \(E\).
\end{definition}

\begin{definition}
    A subset \(E\) of a topological space \(X\) is \textbf{perfect} if every point of \(E\) is a limit point.
\end{definition}

\begin{definition}
    Let \(E\) be subset of a topological space \(X\) then, the \textbf{interior} of \(E\) is the union of all open sets that are contained in \(E\), denoted by \(E^{\circ}\) or \(\interior E\).
\end{definition}

\begin{proposition}
    For any set \(E\) in topological space \(X\), interior of  \(E\) is the set of all interior points of \(E\) and \(\closure E =( \interior E)^c\).
\end{proposition}

\begin{definition}
    The \textbf{boundary} of \(E\) is defined as \(\closure E \backslash \interior E\) and it is denoted by \(\boundary E\) or \(\partial E\).
\end{definition}

\begin{proposition}
    Let \(E\) be subset of a topological space \(X\). \(E\) is closed if and only if \(E = \closure E\) and \(E\) is open if and only if \(E = \interior\). Furthermore, \(\boundary E = \emptyset\) if and only if \(E\) is both open and closed.
\end{proposition}

\begin{proof}
    Note that \(E \subset \closure E\) and when \(E\) is closed, \(\closure E \subset E\) therefore, \(E = \closure E\). 
\end{proof}

\begin{definition}
    A set \(D\) in a topological space \(X\) is called \textbf{dense} when, \(\closure D = X\). More generally, \(D\) is dense in subset \(E\), if \(E \subset \closure D\). A topological space in which there exists a countable dense set is called \textbf{separable}.
\end{definition}

\begin{proposition}
    \(D\) is dense in \(E\) if and only if for all \(x \in E\), \(D \cap U \neq \emptyset\) for any open neighbourhood \(U\) of \(x\).
\end{proposition}

\begin{proof}
    If there exists \(x \in E\) such that an open neighbourhood \(U\) of \(x\) does not intersect \(D\), then \(x \notin \closure D\) and hence \(D \not\supset E\). If for each \(x \in E\) every neighbourhood \(U\) of \(x\) intersects \(D\), then \(x \in \closure D\) hence \(E \subset \closure D\).
\end{proof}

Let \((X,\scrT)\) be a topological space and \(Y \subset X\). Then, \((Y,\scrT_y)\) is a \textbf{topological subspace} where \(\scrT_y = \set<U \cap Y>{U \in \scrT}\) is the \textbf{relative topology} of \(Y\).

\begin{definition}
    A \textbf{base} for a topolgy \(\scrT\) is a collection of open set \(\scrB \subset \scrT\) such that each \(U \in \scrT\) is a union of open sets in \(\scrB\). That is, \(U = \bigcup_{G \in \scrB} G\).
\end{definition}

\section{Metric spaces}
Let \(X\) be a non-empty set and \(x,y \in X\) then if there exists a non-negative real number \(\func{d}{x,y}\) with following three properties:
\begin{enumerate}
    \item \(\func{d}{x,y} = 0 \text{ if and only if } x = y\) (Positive definiteness).
    \item \(\func{d}{x,y} = \func{d}{y,x}\) (Symmetry).
    \item \(\func{d}{x,y} \leq \func{d}{x,z} + \func{d}{z,y}\) (Triangle inequality).
\end{enumerate}
the combination \(\metricSpace{X}{d}\) is called a \textbf{metric space} and \(\func{d}{x,y}\) is called the \textbf{metric}, or also \textbf{distance} function.
\begin{example}
    The Euclidean space \(\Reals^n = \{(x_1,x_2,\dots,x_n) : x_i \in \Reals\}\) with \raggedright \(\func{d}{x,y}=\sqrt{(x_1-y_1)^2 + \dots + (x_n - y_n)^2} \) makes a metric space. To prove this we must show the above properties work:
    \begin{enumerate}
        \item if \(\func{d}{x,y} = 0\) then:
              \[ \sqrt{(x_1-y_1)^2 + \dots + (x_n - y_n)^2} = 0 \]
              Therefore each of the terms must be zero:
              \begin{align*}
                  (x_i - y_i)^2 & = 0 \quad  \forall i \leq n \\
                  x_i - y_i     & = 0 \implies x_i = y_i
              \end{align*}
              Thus \(x = y\)
        \item  It is obvious that \((x_i - y_i)^2 = (y_i - x_i)^2\) and therefore \(\func{d}{x,y} = d(y,x)\)
        \item  The triangle inequality immediately follows from the Cauchy-Schwartz inequality.
    \end{enumerate}
\end{example}

We can expand the Euclidean norm by defining Minkowski \textit{\(p\)-norm} also called \textit{\(L^p\)-norm} for \(1 \leq p \leq \infty\) as follows:
\[\func{d_p}{x,y} = \left(\sum_{i}{|x_i - y_i|^p}\right)^{\frac{1}{p}}\]
and by taking the limit, \(p \to \infty\) we find out that:
\[\func{d_\infty}{x,y} = \max_{i} \{|x_i - y_i|\}\]

\begin{example}
    We can define \textbf{discrete distance} as follows:
    \[\func{d}{x,y} =
        \begin{cases}
            1 & x\neq y \\
            0 & x=y
        \end{cases}
    \]
    and it is pretty straightforward to show that the three properties hold.
\end{example}

\begin{definition}
    The \textbf{open ball} \(\func{B_r}{a}\) with radius \(r\) centered at \(a\) is the set of all points:
    \[ \func{B_r}{a} = \{ x \in X : \func{d}{x,a} < r\}\]
    and the \textbf{closed ball} \(\func{\overline{B}_r}{a}\) with radius \(r\) centered at \(a\) is the set of all points:
    \[\func{\overline{B}_r}{a} = \{ x \in X: \func{d}{x,a} \leq r\} \]
    The \textbf{sphere} \(\func{S_r}{a}\) with radius \(r\) centered at \(a\) is the set of all the points:
    \[ \func{S_r}{a} = \{ x \in X: \func{d}{x,a}  = r\} \]
\end{definition}

\begin{definition}
    Let \(\metricSpace{X}{d}\) be a metric space. A subset \(U \subset X\) is an open set if for all \(a \in U\) there exists \(\rho > 0\) such that \( \func{B_\rho}{a} \subset U\).
\end{definition}

Given this defintion of open sets, we can define a topolgy on metric space \(X\). 
\begin{enumerate}
    \item Firstly, we need to show that every union of open sets is open itself. Let \(U_{\alpha}\) be some open sets indexed by \(A\) and let \(x \in \cup_{\alpha} U_{\alpha}\). Then, there exists a \(\alpha \in A\) such that \(x \in U_{\alpha}\). Since, \(U_{\alpha}\) is open, there exists a ball \(\func{B_r}{x}\) which is contained in \(U_{\alpha}\). Clearly, \(\func{B_r}{x} \in \cup_{\alpha} U_{\alpha}\) and hence \(\cup_{\alpha} U_{\alpha}\) is open.
    \item Secondly, we show that intersection of finite collection of open sets in open. Let \(U_{\alpha}\) be open sets indexed by a finite set \(A\) and let \(x \in \cap_{\alpha} U_{\alpha}\). For each \(\alpha \in A\), there exists a ball \(\func{B_{r_{\alpha}}}{x}\) such that \(\func{B_{r_{\alpha}}}{x} \subset U_{\alpha}\). Let \(r = \min_{\alpha} r_{\alpha}\) and note that \(\func{B_r}{x} \subset \func{B_{r_{\alpha}}}{x} \subset U_{\alpha}\) for all \(\alpha \in A\). Thus, \(\func{B_{r}}{x} \subset \cap_{\alpha} U_{\alpha}\) hence, \(\cap_{\alpha} U_{\alpha}\) is open.
    \item Thirdly, we show that \(X\) and \(\emptyset\) are open. \(\emptyset\) is trivially open as it has no element. And \(\func{B_r}{x} \subset X\)  by defintion for all \(r\) hence, \(X\) is open as well.
\end{enumerate}

Consider the following re-defintions of concepts introduced in the previous section.

\begin{definition} [Internal Point]
    A point \(a \in X\) is called an internal point of \(U\) if \(\exists \rho > 0\) that the ball \(\func{B_\rho}{a}\) contained in \(U\).
\end{definition}

\begin{definition} [Interior]
    The interior of a set \(U\) denoted by \(U^\circ\) or \(\interior (U)\) is the set of all its interior points.
\end{definition}

\begin{definition} [Adherent Point]
    A point \(a \in X\) is called an adherent point of \(U\) if \(\forall \rho > 0\) the ball \(\func{B_\rho}{a}\) contains a point in \(U\).
\end{definition}

\begin{definition} [Limit Point]
    A point \(a \in X\) is called a limit point of \(U\) if \(\forall \rho > 0\) the set \(\func{B_\rho}{a} - \{ a\}\) contains a point in \(U\). The set of all limit points is denoted by \(S'\) or \(\lim S\).
\end{definition}

\begin{note}
    For any limit point \(a \in U\) every open ball \(\func{B_r}{a}\) contains infinitely many points in \(U\).
\end{note}

\begin{definition} [Closed Set]
    A subset \(C \subset X\) is closed set if it contains all of its adherent point.
\end{definition}

\begin{definition} [Closure]
    The closure of a set \(U\) denoted by \(\overline{U}\) or \(\closure U\) is set of all its adherent points.
\end{definition}

\begin{note}
    The closure of a set is a closed set.
\end{note}

We then, show that these re-definitions are equivalent to the topological defintions.

\begin{theorem}
    Subset \(C \subset X\) is closed if and only if \(X - C\) is open.
\end{theorem}

\begin{proof}
    Firstly we prove the necessity condition that is \(C\) is closed if \(X - C\) is open . We employ proof by contradiction. Let \(C\) be a closed subset of \(X\) such that its complement is not open. That is, for some \(a \in (X - C)\) there is no \(\rho > 0\) exists such that \(\func{B_\rho}{a} \subset (X-C)\). In other words, for all \(\rho,\: \exists p \in \func{B_\rho}{a} \;\mathrm{ s.t }\; p_\rho \in C\). Which implies that \(a\) is an adherent point of \(C\) but since \(C\) is closed then \(a \in C\) which is a contradiction. Similarly, one can show the sufficiency condition.
\end{proof}

\begin{corollary}
    \(X \text{ and } \emptyset\) are both closed and open.
\end{corollary}

\begin{remark} (Equivalent Definitions)
    \begin{enumerate}
        \item An open set is a union of open balls. Conversely, a union of open balls is an open set.
              \begin{prooflemma}
                  For every \(a \in U\) there is a ball \(\func{B_\rho}{a} \subset U\) thus \(\bigcup_{a \in U}{\func{B_\rho}{a}} \subset U\) and since \(a \in \func{B_\rho}{a}\) we must have \(\bigcup_{a \in U}{\func{B_\rho}{a}} \supset U\) hence \(U = \bigcup_{a \in U}{\func{B_\rho}{a}}\).

                  Now let \(U = \bigcup {\func{B_\rho}{a}}\) we need to show that \(U\) is open. Let \(b \in U\) then \(b\) must be a point in at least one of those balls. Let \(b \in B_r(c)\) and \(\rho = r - d(b,c)\). We will show that \(B_\rho(b) \subset B_r(c) \subset U\), for any \(x \in B_\rho(b)\) by triangle inequality we have \(d(x,c) \leq d(x,b) + d(b,c) < \rho + d(b,c) = r\) which means \(x \in B_r(c)\).
              \end{prooflemma}
        \item A set is open if and only if all of its members are interior points. Therefore, \(U = \interior U\).
        \item Let \(I = \{ S \subset U : S \text{ is open}\}\) then \(\interior U = \displaystyle{ \bigcup_{S \in I} S}\).
        \item Let \(I = \{ U \subset S : S \text{ is closed}\}\) then \(\closure U = \displaystyle{ \bigcap_{S \in I} S}\).
    \end{enumerate}
\end{remark}

Let \(\metricSpace{X}{d}\) be a metric space and \(Y \subset X\) then \(Y\) may inherit its metric from \(X\) and \(\metricSpace{Y}{d}\) would also be a metric space and is called a \textbf{metric subspace} of \(X\). We will investigate the nature of open and closed sets in subspaces. Let \(\func{B_\rho^Y}{y} = \{ p \in Y : d(y,p) < \rho\}\) Then, it is easy to see that:
\[B_\rho^Y(y) = B_\rho(y) \bigcap Y \]

\begin{corollary} \label{InheritancePrinciple}
    Let \(\metricSpace{X}{d}\) be a metric space and \(Y \subset X\) is a metric subspace of \(X\) then \(U \subset Y\) is an open subset of \(Y\) if and only if there is a open set \(V \subset X\) such that \(U = V \cap Y\). Similarly, for any closed set \(C \subset Y\) there is a closed set \(D \subset X\) such that \(C = D \cap Y\).
\end{corollary}

\begin{proof}
    Ofcourse, if \(U \subset Y\) is open in \(Y\) then by definition it can be represent as a union of open ball \(\func{B_r^Y}{a} \). Each of these balls is the intersection of a \(\func{B_r^X}{a} \cap Y\). Therefore
    \begin{equation*}
        U = \bigcup \func{B_r^Y}{a} = \bigcup \left(\func{B_r^X}{a} \cap Y\right) = \left( \bigcup \func{B_r^X}{a} \right) \cap Y = V \cap Y
    \end{equation*}
    Furthermore, if \(a \in V \cap Y\) then there exists a ball \(\func{B_r^X}{a} \subset V\). Therefore
    \begin{equation*}
        \func{B_r^Y}{a} = \func{B_r^X}{a} \cap Y \subset V \cap Y = U
    \end{equation*}
    The case for closed subsets can be proved using the complements.
\end{proof}

{\Large\textbf{Exercises}}
\begin{enumerate}
    \item Show that \(\closure S = S \cup \lim S\)
\end{enumerate}
\newpage

\section{Convergence}
Let \(X\) be a topological space. A \textbf{sequence} is a function in form of \( a : \set{k, k+1, k+2, \dots } \to X\) where \(k \in \Integers \). Conventionally, instead of \(\func{a}{n}\), \(a_n\) is used. The sequence \(\{ a_n\}\) is \textbf{convergent} to \(a \in X\) if for all neighbourhood \(U\) of \(a\) there exists \(N\) such that:
\begin{equation*}
    n \geq N \implies a_n \in U
\end{equation*}
and it is denoted by \(a_n \to a\) or \(\displaystyle{a = \lim_{n \to \infty}{a_n}}\). Given a topolgy, convergence is not necessarily well-behaved. For example, in the trivial topolgy, if \(a_n \to a\), then \(a_n \to b\) for any \(b \in X\). To do away with this we consider \textbf{Hausdorff spaces}.

\begin{definition}
    Let \(X\) be a topological space. \(X\) is a Hausdorff space if for any two \(x,y \in X\) where \(x \neq y\), there exists disjoint open sets \(U\) and \(V\) such that \(x \in U\) and \(y \in V\).
\end{definition}

\begin{proposition}
    Let \(X\) be a Hausdorff space and \(x_n\) is a sequence in \(X\). If \(x_n \to x\) and \(x_n \to y\) as \(n \to \infty\), then \(x = y\).
\end{proposition}

\begin{proof}
    If \(x \neq y\), then there are disjoint open set \(U\) and \(V\) with \(x \in U\) and \(y \in V\). If \(x_n \to x\), then there exists \(N\) such that for \(n \geq N\), \(x_n \in U\). But this implies that for \(n \geq N\), \(x_n \notin V\). Meaning, there exists no \(N'\) that for \(n \geq N'\), \(x_n \in V\) hence, \(x_n\) does not converge to \(y\).
\end{proof}

In the case of a metric space, the set \(\set{ a_k, a_{k + 1} , \dots }\) is bounded in \(X\), that is, there exist \(K > 0\) and a point \(b \in X\) such that \(\forall n,\; a_n \in \func{B_K}{b}\).

Another problem with the defintion of convergence is its dependence on a convergence point. So naturally the following question comes up. Is there a way to show the convergence of sequence based on itself?
For that, we need to define \textbf{Cauchy sequence}. A sequence \(\set{a_n}\) in a metric space \(X\) is a Cauchy sequence if:
\[\forall \epsilon > 0,\: \exists N \quad \suchThat \quad n, m \geq N \implies \func{d}{a_n,a_m} < \epsilon \]

\begin{lemma}
    In a metric space \(X\), convergence of \(a_n\) to \(a\) is equivalent to 
    \begin{equation*}
        \forall \epsilon > 0,\: \exists N \quad \suchThat \quad n \geq N \implies \func{d}{a_n , a} < \epsilon
    \end{equation*}
\end{lemma}
\begin{proof}
    If \(a_n \to a\), then for \(\func{B_{\epsilon}}{a}\) there exists \(N\) such that \(n \geq N \implies a_n \in \func{B_{\epsilon}}{a}\). On the other hand, for any neighbourhood \(U\) we can find \(\epsilon > 0\) such that \(\func{B_{\epsilon}}{a} \subset U\). Hence, if the metric convergence condition holds, then topological convergence holds, as well.
\end{proof}

\begin{theorem} \label{ConvergenceCauchy}
    Every convergent sequence is a Cauchy sequence.
\end{theorem}

\begin{proof}
    For a given \(\epsilon > 0 \) we know there exist \(N\) such that:
    \[ n \geq N \implies \func{d}{a_n,a} < \frac{\epsilon}{2}\]
    and equivalently:
    \[ m \geq N \implies \func{d}{a_m,a} < \frac{\epsilon}{2}\]
    and since by the triangle inequality we have:
    \[  \func{d}{a_m,a_n} \leq  \func{d}{a_m,a} +  \func{d}{a_n,a} \leq \frac{\epsilon}{2} + \frac{\epsilon}{2} = \epsilon\]
\end{proof}

\begin{definition} [Subsequence]
    We call \(\set{ b_n }\) a \textbf{subsequence} of \(\set{ a_n }\) if there is a sequence of positive integers \(n_1 < n_2 < n_3 < \dots \) such that for each \(k\), \(b_k = a_{n_k}\).
\end{definition}

{\Large\textbf{Exercises}}
\begin{enumerate}
    \item Show that if a sequence \(\set{a_n }\) is convergent, then its limit is unique. That is, if \(a_n \to a \) and \(a_n \to b\) as \(n \to \infty\) then \(a = b\).
    \item Prove that every subsequence of a convergent sequence converges and it converges to the same limit.
\end{enumerate}
\newpage

\section{Completeness}
A metric space \(\metricSpace{X}{d}\) is \textbf{complete} if every Cauchy sequence converges.

\begin{proposition}
    \(\Reals\) with the normal Euclidean norm is a complete metric space.
\end{proposition}

To prove it, we need the following lemmas.

\begin{lemma} \label{Bounded}
    If \(\set{ a_n }\) is a Cauchy sequence in a metric space \(\metricSpace{X}{d}\) then the set \(S = \set{ a_k , a_{k + 1}, \dots }\) is bounded.
\end{lemma}

\begin{prooflemma}
    For a fixed \(\epsilon > 0\) we know there exists \(N\) such that:
    \begin{equation*}
        m,n \geq N \implies \func{d}{a_n,a_m} < \epsilon
    \end{equation*}
    especially:
    \begin{equation*}
        n \geq N \implies \func{d}{a_n,a_N} < \epsilon
    \end{equation*}
    Since there is only finitely many indices less than \(N\) then we can determine the largest \(\func{d}{a_N, a_m}\)for all \(m\) less than \(N\) lets denote it by \(A\). Finally, let \(K = \max \set{ \epsilon, A}\) then, \(\func{B_K}{a_N} \) contains all the elements of sequence.
\end{prooflemma}

\begin{lemma} \label{convergenceSubsequence}
    If one of the subsequences of Cauchy sequence is convergent, then the Cauchy sequence is convergent to the same element.
\end{lemma}

\begin{prooflemma}
    Let \(a_{n_k} \to a \) when \(k \to \infty\) That is, for a given \(\epsilon > 0,\: \exists N_1\) such that:
    \begin{equation*}
        k \geq N_1 \implies \func{d}{a_{n_k},a} < \frac{\epsilon}{2}
    \end{equation*}
    and since \(\set{a_n} \) is a Cauchy sequence then we also know that there exists \(N_2\) such that:
    \begin{equation*}
        q,m \geq N_2 \implies \func{d}{a_m,a_q} < \frac{\epsilon}{2} 
    \end{equation*}
    Let \(N = \max \set{ N_1, N_2} \) and \(n_q \geq N\) consequently:
    \begin{equation*}
        n_q,m \geq N \implies \func{d}{a_m,a_{n_q}} < \frac{\epsilon}{2}
    \end{equation*}

    and by the triangle inequality we have:
    \begin{equation*}
        \func{d}{a_m,a} \leq \func{d}{a_m, a_{n_q}} + \func{d}{a_{n_q},a} \leq \frac{\epsilon}{2} + \frac{\epsilon}{2} = \epsilon
    \end{equation*}
    which proves the convergence of \(a_n \to a\).
\end{prooflemma}

We present a proof for the completeness of \(\Reals\) under the Euclidean norm.

\begin{proof} \label{RealComplete}
    Let \(\set{ a_n }\) be a Cauchy sequence. Then by \Cref{Bounded}, the sequence is bounded and there is a closed interval \(I_0 = \clcl{a}{b} \) in which all \(a_n\) lie. Consider the closed intervals \(\clcl*{a}{\dfrac{a + b}{2}} \) and \(\clcl*{\dfrac{a + b}{2}}{b}\). Since the sequence has infinitely many terms then there are infinitely many terms in at least one of the two intervals. Let that interval be \(I_1\) and choose \(x_1 \in I_1\) where \(x_1 = a_{n_1}\) for some \(n_1\).
    Repeat the process for \(I_1\) to get \(I_2 \) and \(x_2 = a_{n_2}\) where \(n_2 > n_1\). Since there are infinitely many terms in \(I_2\) we can find such \(n_2\). By continuing this process we have a subsequence \(\{ x_k \}\) and a sequence of nested closed sets \(\set{ I_k = \clcl{a_k}{b_k}}\). Since for all \(\epsilon > 0\) there exists \(K\) such that \(b_K - a_K < \epsilon\) then the intersection of \(\set{I_k}\) is a point, say \(y\). We claim that that \(x_k \to y\), that is:
    \begin{equation*}
        \forall \epsilon > 0 \; \exists N \in \Naturals \quad \suchThat \quad n \geq N \implies \abs{x_n - y} < \epsilon  
    \end{equation*}
    
    Since \(y = \bigcap I_k\) then \(y \in I_k\) for all \(k\), especially \(y \in I_n\). Therefore, \(\abs{x_n - y}\) is smaller than or equal to the length of \(I_n\) which is \(\dfrac{b - a}{2^n} \leq \dfrac{b-a}{2^N}\). By setting \(N > \log_2{\dfrac{b-a}{\epsilon}}\) we have:
    \begin{equation*}
        \abs{x_n - y} \leq \dfrac{b - a}{2^n} \leq \dfrac{b-a}{2^N} < \epsilon
    \end{equation*}
    Therefore \(\Reals\) is a complete metric under Euclidean norm.
\end{proof}


Let \(\metricSpace{X}{d}\) and \(\metricSpace{X'}{d'}\)be two metric spaces. Define the following norms on the Cartesian product \(X \times X'\):
\begin{enumerate}
    \item \(\func{D_1}{(x,x'),(y,y')} = \func{d}{x,y} + \func{d'}{x,y}\)
    \item \(\func{D_2}{(x,x'),(y,y')} = \sqrt{\func{d}{x,y}^2 +\func{d'}{x',y'}^2}\)
    \item \(\func{D_3}{(x,x'),(y,y')} = \max \{ \func{d}{x,y} , \func{d'}{x',y'}\}\)
\end{enumerate}

Let \(p_1 = (x,x')\) and \(p_2 = (y,y')\):
\begin{equation*}
    \func{D_3}{p_1,p_2} \leq \func{D_2}{p_1,p_2} \leq \func{D_1}{p_1,p_2} \leq 2\func{D_3}{p_1,p_2} 
\end{equation*}

Then, it is easy to see that if a sequence \(\{ a_n \}\) is convergent under one of these norms, it is convergent to the same value under the other two.
The same is true if the sequence is a Cauchy sequence.

By induction we can generalize it to \(X_1 \times X_2 \times \dots \times X_n\). For example, \(\Reals^n\) is complete metric under all the three norms introduced above. That is, every Cauchy sequence in \(\Reals^n\) is convergent. To show this assume the sequence \(\set{ x_i }\) is a Cauchy sequence under, WLOG, \(D_1\):
\begin{equation*}
    \forall \epsilon > 0, \, \exists N \quad \suchThat \quad i,j \geq N \implies \func{D_1}{x_i,x_j} < \epsilon 
\end{equation*}

Then for the \(k\)-th coordinate:
\begin{equation*}
    \abs{x_{i_k} - x_{j_k}}  < \func{D_1}{x_i,x_j}  < \epsilon 
\end{equation*}

Therefore, for every coordinate, the image of the sequence on that coordinate is a Cauchy sequence. Since \(\Reals\) is complete then \(\set{ x_{i_k} } _i\) is convergent to some \(x_k\) for all \(k\). We claim that \(x_i \to x = (x_1, \dots, x_n)\) as \(i \to \infty\):
\begin{equation*}
    \func{D_1}{x,x_i} = \abs{x_{i_1} - x_1} + \abs{x_{i_2} - x_2}  + \dots + \abs{x_{i_n} - x_n} 
\end{equation*}

We have shown that \(\set{ x_{i_k}} _i\) is convergent to \(x_k\) then there must be \(N_1,N_2, \dots N_n\) such that for all \(k\):
\begin{equation*}
    \forall \epsilon,\quad i \geq N_k \implies \abs{x_{i_k} - x_k} < \frac{\epsilon}{n}
\end{equation*}
Setting \( N = \underset{1 \leq k \leq n}{\max}{N_k} \):
\begin{equation*}
    \func{D_1}{x,x_i} < n \cdot \frac{\epsilon}{n} = \epsilon
\end{equation*}

\begin{theorem}
    Let \(\metricSpace{X}{d}\) be a complete metric space and \(Y \subset X\) is a complete metric space if and only \(Y\) is a closed subset of \(X\).
\end{theorem}

\begin{proof}
    It is clear that \(Y\) being closed is necessary for \(Y\) being a complete metric subspace. To show that is also sufficient, we need ot show that if \(Y\) is a complete metric subspace then it is closed. Assume the contrary, that is there exists an adherent point of \(Y\), \(a \notin Y\). Since \(a\) is an adherent point of \(Y\) then for all \(\rho > 0\) there exists a point \(x \in \func{B_\rho}{a}\) such that \(x \in Y\). For each \(n\) let \(\rho = \frac{1}{n}\) and choose a point \(x_n \in Y\)
    It is clear that \(\set{ x_n }\) is convergent to \(a\). From \Cref{ConvergenceCauchy} \(\set{ x_n }\) is a Cauchy sequence. Since \(Y\) is complete then \(a\) must be in \(Y\) which is a contradiction.
\end{proof}

\begin{theorem}[Baire Category Theorem]
    Let \(X\) be a complete metric space and \(G_n\) be a sequence of dense open sets in \(X\). Then, \(\cap_{n = 1}^{\infty} G_n\) is dense in \(X\).
\end{theorem}

\begin{theorem}
    Let \((X,d)\) be a complete metric space, and suppose that \(f:X \to X\) has the property that there exists \(\alpha < 1 \) such that 
    \begin{equation*}
        \func{d}{\func{f}{x}, \func{f}{y}} < \alpha \func{d}{x,y}
    \end{equation*}
    for all \(x,y \in X\). Then, there exists a unique point \(x \in X\) such that \(\func{f}{x} x\). Moreover, if \(x_0 \in X\), and \(x_{n+1} = \func{f}{x_n}\), then \(x = \lim_{n \to \infty} x_n\).
\end{theorem}
{\Large\textbf{Exercises}}
\begin{enumerate}
    \item Show that if a sequence \(\set{ a_n }\) is convergent, then its limit is unique. That is, if \(a_n \to a \) and \(a_n \to b\) as \(n \to \infty\) then \(a = b\).
    \item Prove that every subsequence of a convergent sequence converges and it converges to the same limit.
\end{enumerate}
\newpage

\section{Continuity}
\begin{definition} [Continuity]
    Let \((X,\scrT_X)\) and \((Y,\scrT_Y)\) be two topological spaces and \(f : X \to Y\) be a function. We say \(f\) is continuous if for every open subset \(V\) of \(Y\) the pre-image of it is an open set in \(X\).
    \[\func{f^{\mathrm{pre}}}{V}= \func{f^{-1}}{V} = \{ x \in X : \func{f}{x} \in V\} \]

    Furthermore, \(f\) is continuous at a point \(x \in X\) when for all subset \(W\) of  \(Y\) that \( \func{f}{x} \) is an internal point of, there is an open set \(U\) containing \(x\) such that \(\set{ f(y) : y \in U } \subset W\). In other words \(x\) is an internal point of \(\func{f^{\mathrm{pre}}}{W}\).
\end{definition}

\begin{proposition}
    \(f\) is continuous if and only if \(f\) is continuous at every point \(x \in X\).
\end{proposition}

\begin{proof}
    Firstly, if \(f\) is continuous we show that \(f\) is continuous at every point \(x \in X\). Let \(V\) be an open set around \( \func{f}{x}\) then \(x \in \func{f^{\mathrm{pre}}}{V}\) must be an internal point since \(\func{f^{\mathrm{pre}}}{V}\) is open.
    Secondly, if \(f\) is continuous at every point \(x \in X\) then \(f\) is continuous. Let \(V = \set{\func{f}{x} : x \in U}\) be an open set in \(Y\). For any \(x \in U\), \(\func{f}{x}\) is an internal point of \(V\) and since \(f\) is continuous at \(x\), \(x\) is an internal point of \(U\) which means every point \(x \in U\) is an internal point of \(U\) and thus \(U = \func{f^{\mathrm{pre}}}{V}\) is open.
\end{proof}

\begin{theorem} [\(\epsilon-\delta\) condition]
    Let \((X,d)\) and \((Y,d')\) be two metric space. Continuity at a point \(x\) is equivalent to the existence a \(\delta > 0\) for all \(\epsilon > 0\) such that:
    \[\func{d}{x,y} < \delta \implies \func{d'}{\func{f}{x},\func{f}{y}} < \epsilon \]
\end{theorem}

\begin{proof}
    Let \(V = \set{ \func{f}{y} :\func{d'}{\func{f}{x},\func{f}{y}} < \epsilon}\) then \(V\) is open and hence \(f(x)\) is an internal point of \(V\). By continuity at point \(x\), \(x\) must be an internal point of \(\func{f^{\mathrm{pre}}}{V}\). In other words, there exists a \(\delta > 0\) such that \(U = \set{ y : \func{d}{x,y} < \delta } \subset f^{\mathrm{pre}}(V)\).
    Take an open set \(U \subset Y\), then assuming the \(\epsilon-\delta\) condition, we will show that \(\func{f^{\mathrm{pre}}}{U}\) is open. Let \(y \in U\) then there is \(x \in \func{f^{\mathrm{pre}}}{U}\) such that \(\func{f}{x} = y\). From openness of \(U\), there is a \( \epsilon > 0\) such that \(\func{B_\epsilon}{y} \subset U\), also by continuity condition, there exists a \(\delta > 0 \) such that:
    \begin{equation*}
        \func{d}{x,z} < \delta \implies \func{d'}{\func{f}{x},\func{f}{z}} < \epsilon
    \end{equation*}
    The openness of \(\func{f^{\mathrm{pre}}}{U}\) is equivalent to \(\func{B_\delta}{x} \subset \func{f^{\mathrm{pre}}}{U}\), which clearly holds, since for any \(z \in \func{B_\delta}{x} \implies \func{f}{z} \in \func{B_\epsilon}{y} \subset U\).
\end{proof}

\begin{example}
    Let \(\metricSpace{X}{d}\) be a metric space with \(\func{d}{x,y}\) being the discrete metric, \(f : X \to X'\) where \(\metricSpace{X'}{d'}\) is an arbitary metric space. Then \(f\) is always continuous. Since for every point \(a\) the open ball \(\func{B_{\frac{1}{2}}}{a} = \{ a \}\), and union of open sets is an open set itself, then every subset of \(X\) is open.
\end{example}
Equivalently, \(f\) is continuous at \(a\) if for all \(\epsilon > 0\), \(a\) is an internal point of \(\func{f^{\mathrm{pre}}}{\func{B_\epsilon}{\func{f}{a}}}\). That is there exists \(\delta > 0\) such that, \(\func{B_\delta}{a}  \subset \func{f^{\mathrm{pre}}}{\func{B_\epsilon}{\func{f}{a}}}\). More generally, if \(X\) has the discrete topolgy or \(X'\) has the trivial topolgy, then \(f:X\to X'\) is always continous.

\begin{proposition}
    Let \(f: (X,\scrT_1) \to (X,\scrT_2)\) be the identity function. \(f\) is continous if and only if \(\scrT_1\) is stronger that \(\scrT_2\).
\end{proposition}

\begin{proof}
    If \(f\) is continous, then every open set \(V \in \scrT_2\) is an open set in \(\scrT_1\). Hence, \(\scrT_2 \subset \scrT_1\). If \(\scrT_2 \subset \scrT_1\) and \(V\) is an open set in \(\scrT_2\), then \(V = \func{f^{\mathrm{pre}}}{V} \in \scrT_1\) is open and thus \(f\) is continous.
\end{proof}

\begin{theorem}
    Let \(\metricSpace{X}{d}\) and \(\metricSpace{X'}{d'}\) be two metric spaces and \( f: X \to X' \). \(f\) is continuous at \( a \in X\) if and only if for every sequence \(\set{ a_n}\) in \(X\) with \(a_n \to a\) we have \(\func{f}{a_n} \to \func{f}{a}\).
\end{theorem}

\begin{proof}
    Let \(f\) be continuous at \(a\) and \(a_n \to a\). From continuity of \(f\), for each given \(\epsilon\), there is a \(\delta\) such that:
    \begin{equation*}
        \func{d}{x,a} < \delta \implies \func{d'}{\func{f}{x},\func{f}{a}} < \epsilon
    \end{equation*}
    From the convergence of \(\set{a_n}\), for each given \(\delta\), there is a \(N\) such that:
    \begin{equation*}
        \forall n \geq N \implies \func{d}{a_n,a} < \delta
    \end{equation*}
    By merging these two equations we will get:
    \begin{equation*}
        \forall n \geq N \implies  \func{d}{a_n,a} < \delta \implies \func{d'}{\func{f}{a_n},\func{f}{a}} < \epsilon
    \end{equation*}
    which was what was wanted.

    If \(f\) is not continuous, there must be an \(\epsilon > 0\) that for all \(\delta > 0 \), for some  \(x \in \func{B_\delta}{a}\), \( \func{d'}{\func{f}{x},\func{f}{a}} \geq \epsilon\). Especially, for each \(n \in \Naturals\), let \(\delta = \frac{1}{n}\) and \(x_n\) have the described property. Since \(x_n \to a\) by our assumption \(\func{f}{x_n} \to \func{f}{a}\), which is a contradiction and thus \(f\) is continuous.
\end{proof}

\begin{definition}
    Let \(X\) and \(Y\) be two topological spaces and \(f:X \to Y\). \(f\) is a \textbf{homeomorphism} of \(X\) to \(Y\) if \(f\) is bijective and, \(f\) and \(f^{-1}\) are continous. Furthermore, \(X\) and \(Y\) are \textbf{homeomorphic} if there exists a homeomorphism between them.
\end{definition}

{\Large\textbf{Exercises}}
\begin{enumerate}
    \item Let \(\metricSpace{X}{d}, (X',d'), \text{ and } (X'',d'')\) be metric spaces and \(f : X \to X'\), \(g : X' \to X''\) be two functions. If \(f\) is continous at \(a\) and \(g\) is continous at \(\func{f}{a}\), then \(g \circ f\) is continuous at \(a\).
    \item Let \((X_i,d_i), \; i = 1, \dots k\) be metric spaces. Define \(D\) to be any of the three discussed metric over \( X = X_1 \times X_2 \times \dots \times X_k\). Then, the projection function, \(\func{\pi_j}{x} : X \to X_j\) is continuous for all \(j\).
    \begin{equation*}
        \func{\pi_j}{x_1,x_2, \dots, x_n} = x_j
    \end{equation*}

    \item Let \((X,D)\) be defined as above and let \((X',d')\) be a matic space, and \(f : X' \to X\). \(f\) is continous at \(a' \in X'\) if and only if \(\pi_j \circ f\) is continuous for all \(j = 1, \dots, k\).
    \item The four algebraic operations are continuous on their domain.
          \begin{flalign*}
              \text{\large{$+$}} &: \Reals \times \Reals \to \Reals, \quad \text{\large{$+$}}(x,y) = x + y &&\\
              \text{\large{$-$}} &: \Reals \times \Reals \to \Reals, \quad \text{\large{$-$}}(x,y) = x - y &&\\
              \text{\large{\(\times\)}} &: \Reals \times \Reals \to \Reals, \quad \text{\large{\(\times\)}}(x,y) = x \times y &&\\
              \text{\large{\(\div\)}} &: \Reals \times \left(\Reals-\{0\}\right) \to \Reals, \quad \text{\large{\(\div\)}}(x,y) = x \div y &&
          \end{flalign*}
          Where the metric of \(\Reals\) on the right hand side is the common Euclidean metric, and on the left hand side is any of the three metric.
\end{enumerate}

\newpage
\section{Covering compactness}
\begin{definition}
    Let \(X\) be a topological space. A \textbf{covering} for s set \(E \subset X\) is a collection of \(U_\alpha\) of open subsets of \(X\) such that \(E \subset \bigcup U_\alpha \).
\end{definition}

\begin{definition}
    A  \textbf{subcovering} of \(\scrU = \set<U_{\alpha}>{\alpha \in A}\) is a collection \(\scrV = \set<U_{\alpha}>{\alpha \in B}\) where \(B \subset A\) and \(\scrV\) covers the same set.
\end{definition}

\begin{definition}
    A subset \(K\) of \(X\) is called \textbf{compact} if every open cover of \(K\) has a finite subcover.
\end{definition}

\begin{example}
    \(\scrU = \set<\opop{x-1}{x+1}>{x \in \Reals}\) is covering for \(\Reals\). It Obviously has a countable subcovering, however, it does not have finite subcovering.
\end{example}

\begin{theorem}
    Closed and bounded intervals in \(\Reals\) are compact.
\end{theorem}

\begin{proposition}
    If \(X\) is a compact space and \(\set{K_n}\) is a sequence of non-empty closed subsets of \(X\), with \(K_{n+1} \subset K_n\) for all \(n\), then \(\bigcap_{n = 1}^{\infty} K_n\) is non-empty.
\end{proposition}

\begin{proposition}
    If \(E\) is an infinite subset of a compact set, then \(E\) has a limit point in \(K\).
\end{proposition}

\begin{proposition}
    A subset of a topological space is compact if and only if it is compact in iteself with the relative topology.
\end{proposition}

\begin{proposition}
    If \(X\) is a Hausdorff space and \(K \subset X\) is compact, then \(K\) is closed.
\end{proposition}

\begin{definition}
    Let \(X\) be a metric space and \(E\) a subset of \(X\). The \textbf{diameter} of \(E\) is defined to be \(\diam E = \sup \set<\func{d}{x,y}>{x,y \in E}\). \(E\) is said to be bounded if its diameter is finite.
\end{definition}

\begin{definition}
    Let \(E\) be subset of a metric space \(X\). \(E\) is \textbf{totally bounded} if for all \(\epsilon > 0\),  there exists a finite subset \(\set{x_1, \dots , x_n}\) of \(X\) such that \(E \subset \bigcup_{k= 1}^n \func{B_{\epsilon}}{x_k}\). A set \(F\) such that \(E \subset  \bigcup_{x \in F} \func{B_{\epsilon}}{x}\) is called an  \(\epsilon\)-\textbf{net} for \(E\).
\end{definition}

\begin{proposition}
    If \(E\) is totally bounded, then:
    \begin{enumerate}
        \item \(E\) is bounded.
        \item \(\closure E\) is totally bounded.
        \item Any subset of \(E\) is totally bounded.
    \end{enumerate}
\end{proposition}

\begin{theorem}
    A metric space \(X\) is complete if and only if \(X\) is complete and totally bounded.
\end{theorem}

\begin{definition}
    A subset \(E\) of topological space is \textbf{relatively compact} or \textbf{precompact} if \(\closure E\) is compact.
\end{definition}

\begin{corollary}
    A subset of a complete metric space is relatively compact if and only if it is totally bounded.
\end{corollary}

\begin{theorem}
    A set \(E\) in \(\Reals^k\) is closed and bounded  if and only if it is compact.
\end{theorem}

\begin{note}
    This is generally not true for other metric spaces.
\end{note}

\begin{theorem}[Weierstrass]
    Every bounded infinite subset of \(\Reals^k\) has a limit point in \(\Reals^k\).
\end{theorem}

{\Large\textbf{Exercises}}
\begin{enumerate}
    \item Show that \(\mathbb{Q} \cap [0,1]\) is not covering compact, directly from the definition.
\end{enumerate}


\newpage
\section{Sequential compactness}

A subset \(K \subset X\) is \textbf{compact} if it has the Boltzano-Weierstrass property, if for all sequences \(\set{a_n}\) in \(K\) there exists a subsequence of \(\set{a_n}\) that converges to a point\(a \in K\).

\begin{theorem}
    Compactness is equivalent to the existence a finite subcovering for every covering.
\end{theorem}
\begin{proof}
    To prove the theorem, let us define:

    We will show for metric spaces compactness is equivalent to covering compact.
    lebegue number
\end{proof}

% \begin{corollary}
%     If \(K\) is compact then \(K\) must be closed and bounded.
% \end{corollary}

% \begin{proof}
%     Obviously if \(K\) is not closed then there must be a limit point \(a \notin K\) such that the sequence \(\{a_n\}\) converges to \(a\). We have shown every subsequence of a convergent sequence converges to the same value, therefore \(K\) is not compact.
%     If \(K\) is unbounded then for each point \(a \in K\) for all \(n \in \Naturals\), the ball \(\func{B_n}{a}\) has a point other than \(a\) in \(K\). Then we can select \(a_n\) to be a point. Cleary no subsequence of \(\{a_n\}\) can be convergent.
% \end{proof}

% \begin{theorem}
%     If \(K \subset X\) is compact and \(C\) is a closed subset of \(X\) such that \(C \subset K\), then \(C\) is compact.
% \end{theorem}

% \begin{proof}
%     Take a sequence \(\{a_n\} \in C\). Since \(\{a_n\} \in K\) then it has a convergent subsequence \(b_k = a_{n_k}\). Let \(b \in K\) be the point of convergence of \(\{b_k\}\). Since \(\{b_k\} \in C\) and \(C\) is closed then \(b \in C\) and therefore \(C\) is compact.
% \end{proof}

% \begin{proposition}
%     A subset in \(\Reals^n\) is compact if and only if it is closed and bounded.
% \end{proposition}

% \begin{proof}
%     Using the same idea as \Cref{RealComplete} one can show the case for \(n = 1\). Assume the propsition is true for \(n = k - 1\) and let \(K \in \Reals^k\) be a closed and bounded set and \(\{a_n\} \in K\). Furthermore, let \(\{b_n\}\) be the projection of \(\{a_n\}\) onto \(\Reals^{k-1}\) and \(\{c_n\}\) be the projection of \(\{a_n\}\) on to \(k_{\cardinalTH}\) dimension. By induction, there exists a convergent subsequence \(\{b_{n_m}\}\). For \(\{c_{n_m}\}\) there exists a convergent subsequence \(\{c_{n_m}^i\}_i\) as well. It is easy to see that \(\{ a_{n_m}^i\}_i \) is a convergent subsequence of \(\{a_n\}\).
% \end{proof}

% \begin{corollary}
%     \([a,b]\) is compact in \(\Reals\).
% \end{corollary}

Let \(\set{a_n}\) be a sequence in \(\Reals\). We define:
\begin{align*}
    \limsup a_n &= \overline{\lim} \; a_n = \lim_{n \to \infty} \bracket[[\Big]]{\sup{\{a_k : k\geq n\}}}\\
    \liminf a_n &= \underline{\lim} \; a_n = \lim_{n \to \infty} \bracket[[\Big]]{\inf{\{a_k : k\geq n\}}}
\end{align*}

\begin{note}
    The limits, \(\limsup a_n \) and \(\liminf a_n\), always exists. Albeit they might be infinite.
\end{note}

Let \(\set{a_n}\) be a bounded sequence in \(\Reals\), and \(A^*\) is the set of all limit points of all subsequence of  \(\set{a_n}\). We know that \(A^*\) is not empty and since  \(\set{a_n}\) is bounded and then \(A^*\) must be bounded as well. Thus, by completeness axiom, \(A^*\) has infimum and supremum. Moreover, \(\sup A^*, \inf A^* \in A^*\).

\begin{proposition}
    A bounded sequence \(\set{a_n}\) is convergent if and only if \(\;\limsup a_n = \liminf a_n\).
\end{proposition}

\begin{corollary}
    If \(K\) is a compact subset of \(\Reals\) then \(K\) has minimum and maximum. That is, there are \(M,m \in K\) such that \(\forall x \in K,\; m \leq x \leq M\).
\end{corollary}

\begin{proof}
    Since \(K\) is bounded then it has supremum and infimum in \(\Reals\). Obviously, there are convergent sequences \(\set{a_n}\) and \(\set{b_n}\) such that \(a_n \to m = \inf K\) and \(b_n \to M = \sup K\). By compactness of \(K\), \(M,m \in K\).
\end{proof}

\begin{theorem}
    \(\metricSpace{X}{d}\) and \(\metricSpace{X'}{d'}\) are metric spaces and \(K \subset X\) is compact. If \(f : X \to X'\) is continuous, then \(\func{f}{K}\) is a compact subset of \(X'\).
\end{theorem}

\begin{proof}
    Let \(\set{y_n} \in \func{f}{K}\) and \(\set{x_n} \in K\) are such that \(\func{f}{x_n} = y_n\). Since \(K\) is compact there is a convergent subsequence \(\set{x_{n_k}}\) and since \(f\) is continous \( \set{ y_{n_k} = \func{f}{x_{n_k}}}\) is also convergent. Hence \(\func{f}{K}\) is compact.
\end{proof}

\begin{corollary}
    Let \(\metricSpace{X}{d}\) be a metric space and \(f : X \to \Reals\) is continuous. If \(K\) is a compact subset of \(X\). Then \(f\) attains maximum and minimun in \(\Reals\).
\end{corollary}

\begin{note}
    For a continuous function \(f : X \to X'\) it is not necessary that the image of an open/closed set to be open/closed.
\end{note}

\begin{definition} [Uniform continuity]
    Let \(\metricSpace{X}{d}\) and \(\metricSpace{X'}{d'}\) be metric spaces.  \(f :X \to X'\) is uniformly continuous if:
    \begin{equation*}
        \forall \epsilon > 0\; \exists \delta > 0, \; x,y \in X, \; \func{d}{x,y} < \delta \implies \func{d'}{\func{f}{x},\func{f}{y}} < \epsilon
    \end{equation*}
\end{definition}

\begin{proposition}
    \(f: X \to X'\) is uniformly continous if and only if for every pair sequence \(\set{\pair{x_n}{y_n}}\) in \(X\) satisfying \(\func{d}{x_n,y_n} \to 0\) we have \(\func{d'}{\func{f}{x_n}, \func{f}{y_n}} \to 0\).
\end{proposition}

\begin{proof}
    Necessity: We have
    \begin{align*}
         & \forall \epsilon \; \exists \delta \  \suchThat \ \forall x,y \in X, \func{d}{x,y} < \delta \implies \func{d'}{\func{f}{x}, \func{f}{y}} < \epsilon \\
         & \forall \delta \; \exists N \ \suchThat \ n \geq N \implies \func{d}{x,y} < \delta
    \end{align*}
    combining the two brings us at the conclusion.

    Sufficiency: Suppose for the sake of contradtion that:
    \begin{equation*}
        \exists \epsilon \; \forall \delta \; \exists x,y \in X  \ \suchThat \ \func{d}{x,y} < \delta \land \func{d'}{\func{f}{x}, \func{f}{y}} \geq \epsilon
    \end{equation*}
    then let \(\delta = \frac{1}{n}\) and make the sequence pair \(\set{\pair{x_n}{y_n}}\). Clearly, \(\func{d}{x_n,y_n} \to 0\) therefore, \(\func{d'}{\func{f}{x}, \func{f}{y}} \to 0\). Which is a contradition since \(\func{d'}{\func{f}{x}, \func{f}{y}} \geq \epsilon \).
\end{proof}

\begin{proposition}
    \(\metricSpace{X}{d}\) and \(\metricSpace{X'}{d'}\) are matric spaces and \(X\) is compact. If \(f: X \to X'\) is continuous then it is uniformly continuous.
\end{proposition}

\begin{proof}
    Similarly, for the sake of contradicition suppose
    \begin{equation*}
        \exists \epsilon \; \forall \delta \; \exists x,y \in X \ \suchThat \ \func{d}{x,y} < \delta \land \func{d'}{\func{f}{x}, \func{f}{y}} \geq \epsilon
    \end{equation*}
    and let \(\delta = \frac{1}{n}\) and make the sequence pair \(\pair{x_n}{y_n}\). By compactness of \(X\), there are two convergent subsequence \(\set{x_{n_k}}\) and \(\set{y_{n_k}}\). Since \(\func{d}{x_n,y_n} \to 0\) then if \(x_{n_k} \to x\), \(y_{n_k} \to x\) as well. By continuity of \(f\), \(\func{f}{x_{n_k}} \to \func{f}{x}\) and \(\func{f}{y_{n_k}} \to \func{f}{x}\) and thus \(\func{d'}{\func{f}{x_{n_k}}, \func{f}{y_{n_k}}} \to 0 \). Which is a contradicition as for sufficiently large \(K\), \(k \geq K \implies \func{d'}{\func{f}{x}, \func{f}{y}} \geq \epsilon\)
\end{proof}

% Define the \textbf{diamter} of a set \(S\) to be:
% \[\diam{S} = \sup{\{d(s,s') : s,s' \in S\}}\]
% the cleary for bounded sets we have:
% \[\diam{S} < +\infty\]

% \begin{proposition}
%     Let \(\metricSpace{X}{d}\) be a metric space and \(\{K_n\}\) is a sequence of compact subset of \(X\) with \(K_1 \supset K_2 \supset \dots\).
%     \begin{enumerate}
%         \item \(\bigcap K_n\) is not empty.
%         \item If \(\diam{K_n} \to 0\) then \(\bigcap K_n\) is a singular point.
%     \end{enumerate}
% \end{proposition}

% \begin{proof} \leavevmode
%     \begin{enumerate}
%         \item Consider the sequence \(\{a_n\}\) such that \(a_n \in K_n\). Since \(a_n \in K_1\) for all \(n\), then there is a convergent subsequence \(\{a_{n_k}\}\) with \(a_{n_k} \to a\). \(a \in K_1\), however, \(a \in K_2\) and so on, as well. Therefore \(a \in \bigcap K_n\).
%         \item Let \(a,b \in \bigcap K_n\). Then, \(a,b \in K_n\) for all \(n\) and we must have that \(\func{d}{a,b} \leq \diam K_n\). Therefore, \(a = b \).
%     \end{enumerate}
% \end{proof}

{\Large\textbf{Exercises}}
\begin{enumerate}
    \item Prove that \(\sqrt{\mid x \mid} :\Reals \to \Reals\) is uniformly continuous.
\end{enumerate}

\newpage
\section{Connectedness}
\begin{definition}
    Let \(X\) be a topological space. \(X\) is \textbf{disconnected} if there are non-empty open sets \(A\) and \( B\) are found such that
    \begin{equation*}
        A \bigcap B = \emptyset, \quad A \bigcup B = X
    \end{equation*}
    \(X\) is said to be \textbf{connected} if it is not disconnected. A subset \(S\) of \(X\) is connected if it is connected when considering its relative topolgy.
\end{definition}

\begin{definition}
    
\end{definition}
\begin{example}
    The following subsets of \(\Reals\) are disconnected:
    \begin{enumerate}
        \item \(S =\clop{-1}{0} \;\cup\; \opcl{0}{1}\).
        \item \(\Rationals\).
        \item \(S = \clcl{1}{0} \cup \clcl{1}{2}\).
    \end{enumerate}
\end{example}

\begin{definition}
    Two sets \(A\) and \(B\) in a topological space \(X\) are said to be \textbf{separable} if \(A \cap \closure B = \closure A \cap B = \emptyset\).
\end{definition}

\begin{proposition}
    \(X\) is connected if and only if it can not be written in form of two non-empty separable sets.
\end{proposition}

\begin{definition}
    \(S \subset \Reals\) is an \textbf{intervals} if when \(a , c \in S\) and \(a < b < c\) then \(b \in S\).
\end{definition}
\begin{example}
    \(\Reals\) and its intervals are connected. In fact the only connected subsets of \(\Reals\) are its intervals.
\end{example}
\begin{theorem}
    Let \(X\) and \(X'\) be two topological spaces. Let \(f : X \to X'\) be continuous and \(S\) be a connected subset of \(X\). Then, \(\func{f}{S}\) is connected in \(X'\).
\end{theorem}

\begin{corollary} [Mean value theorem]
    If \(f : \clcl{a}{b} \to \Reals\) is a continous function and \(\func{f}{a} = A, \func{f}{b} = B\) then for every \(C\) between \(A\) and \(B\) there exists a \(c \in \clcl{a}{b}\) such that \(\func{f}{c} = C\).
\end{corollary}
\begin{proposition}
    If \(S \subset X\) is a connected set then every \(S \subset T \subset \bar{S}\) is connected.
\end{proposition}

\begin{proposition}
    Let \(\set{E_{\alpha}}\) be collection of connected sets with \(\cap_{\alpha} E_{\alpha} \neq \emptyset\). Then, \(\cap_{\alpha} E_{\alpha}\) is connected.
\end{proposition}

\begin{definition}
    Let \(x\) be a point in a topological space \(X\). The \textbf{connected component} of \(x\), \(C_x\), is the union of the all the connected set including \(x\).
\end{definition}

\begin{proposition}
    For any \(x,y \in X\)
    \begin{enumerate}
        \item \(C_x\) is connected and closed.
        \item \(C_x\) and \(C_y\) are either disjoint or equal to each other.
    \end{enumerate}
\end{proposition}

\begin{definition}
    A topological space \(X\) is said to be \textbf{totally disconceted} if \(C_x = \set{x}\) for all \(x \in X\). A subset \(S\) of \(X\) is totally disconceted if it is totally disconceted when considering its relative topology.
\end{definition}

\begin{definition}
    The \textit{graph} of a function \(f:M \to N\), \(G_f\), given by \(G_f = \set<(x,\func{f}{x})>{ x \in M}\).
\end{definition}

\begin{theorem}
    The graph of a continous function over a connected set is connected.
\end{theorem}
\begin{example}
    Topological curve is connected and also its closure is connected.
\end{example}


\begin{definition}[Path connected]
    A set \(S\) is path connected if for every pair of points \(p,q \in S\) there exists a continous function \(\gamma: \clcl{a}{b} \to S\) such that \(\func{\gamma}{a} = p\) and \(\func{\gamma}{b} = q\).
\end{definition}
\begin{theorem}
    If a set \(S\) is path connected, then it is connected but the inverse is not true.
\end{theorem}
\begin{example}
    Infinite broom is path connected but toplogical sine curve is not.
\end{example}
\begin{proposition}
    If \(f\) is continuous function and \(S\) is a path connected set, then the image of \(S\) under \(f\) is path connected.
\end{proposition}
\begin{proposition}
    Every open set of \(\Reals\) is the union of countably many disjoint open intervals.
\end{proposition}
{\Large\textbf{Exercises}}
\begin{enumerate}
    \item
\end{enumerate}


\newpage


\section{Cantor set}
\begin{definition}
    define cantor set
\end{definition}

\begin{proposition}
    Cantor set is a perfect set.
\end{proposition}

\begin{proposition}
    Cantor set is totally disconnected.
\end{proposition}
\begin{theorem}
    Let \(K\) be a complete, totally disconnected, and compact metric space. Then, \(K\) is homeomorphic to Cantor set.
\end{theorem}

\begin{theorem}
    Let \(P\) be a non-empty perfect set in \(\Reals^k\). Then, \(P\) is uncountable.
\end{theorem}

{\Large\textbf{Exercises}}
\begin{enumerate}
    \item Show that \(\Rationals \cap \clcl{0}{1}\) is not covering compact, directly from the definition.
\end{enumerate}
%\chapter{Differentiation}
\thispagestyle{headings}

\begin{definition}
    Let \(I\) be an interval in \(\mathbb{R}\). If \(a\) is an interior point of \(I\), then we say that \(f: I \to \mathbb{R}\) is differentiable at \(a\) when the following limit exists:
    \[\lim_{x \to a} \dfrac{f(x) - f(a)}{x - a} \]
    The limit, if exists, is denoted by \(f'(a)\).
    If \(a\) is an end point and the length of the interval is greater than zero, then the limit only exists from one direction.

    Equivalently, there exists a line \(l\), not parallel to \(y\)-axis, in form of \(l : A(x) = mx + b\), that is tangent to \(f\) at \(x = a\). In this case:
    \[\lim_{x \to a}\dfrac{f(x) - [mx + b]}{x - a } = 0 \qquad A(a) = f(a) \]
\end{definition}
In a general case, two functions \(f,g\) are tangent to each other at \(x = a\) if:
\begin{equation}
    f(a) = g(a) \qquad \lim_{x \to a}{\dfrac{f(x) - g(x)}{x - a} = 0}
\end{equation}
\begin{corollary}\leavevmode
    \begin{enumerate}
        \item \(f\) is differentiable at \(a\) if it is continuous at \(a\).
        \item \label{extrma} If \(f'(a) > 0\), there exists \(\delta > 0\) such that for \(x \in \; ]a- \delta,a[\; \cap I \implies f(x) < f(a) \) and for \(x \in \; ]a, a +  \delta[\; \cap I \implies f(x) < f(a)\). And if \(f'(a) < 0\) the inequality sign are reversed. Therefore, if \(f\) has a local extremum at \(a\), then in case \(f'(a)\) exists, \(f'(a) = 0\).
    \end{enumerate}
\end{corollary}
\begin{example}
    a function that its derivate is not continuous (with \(\sin\frac{1}{x}\)).
\end{example}
\begin{theorem}[Rolle's theorem] \label{Rolle}
    Let \(f: [a,b] \to \mathbb{R}\) be a continuous and differentiable on the interval. If \(f(a) = 0, f(b) = 0\), then there exists \(c \in \; ]a,b[\) such that:
    \[f'(c) = 0\]
\end{theorem}
\begin{proof}
    If \(f \equiv 0\) on \([a,b]\) then its derivative \(f'(x) \equiv 0\) on \([a,b]\). If \(f(x) \neq 0\) for some \(x \in ]a,b[\) then it must have a non-zero maximum or minimum at some \(c \in ]a,b[\). Since \([a,b]\) is compact then by continuity of \(f\), \(f([a,b])\) is also compact in \(\mathbb{R}\) and therefore \(f\) attains its maximum or minimum. We know that at least one of its extremities must lie in \(]a,b[\), say point \(c\), hence by \Cref{extrma} \(f'(c) = 0\).
\end{proof}
\begin{theorem}[Mean value theorem]\label{MVT}
    Let \(f: [a,b] \to \mathbb{R}\) be a continuous and differentiable on the interval,then there exists \(c \in \; ]a,b[\) such that:
    \[f'(c) = \dfrac{f(b) - f(a)}{b - a}\]
\end{theorem}
\begin{proof}
    Define \(g(x) = f(x)  - f(a) - \dfrac{f(b) - f(a)}{b-a}(x-a)\). Then it is clear that \(g(a) = g(b) = 0\) and \(g\) is continous and differentiable on the interval. Then by \Cref{Rolle} there exists \(c \in ]a,b[\) such that \(g'(c) = 0\). Equivalently:
    \begin{align*}
         & g'(c) = f'(c) -  \dfrac{f(b) - f(a)}{b-a} = 0 \\
         & \implies f'(c) =  \dfrac{f(b) - f(a)}{b-a}
    \end{align*}
    which concludes the proof.
\end{proof}
\begin{corollary}[Growth Estimate]
    If \(|f'(x)| \leq M \) in \(]a,b[\) then \(f\) satisfies the global lipschitz condition for all \(x,y \in [a,b]\) \(|f(x) - f(y) | \leq M |x-y|  \).
\end{corollary}
\begin{corollary}
    Let \(f: [a,b] \to \mathbb{R}\) is continuous and \(f'(x) < 0\) (or \(f'(x) > 0\)) for all \(x \in ]a,b[\) then \(f\) is strictly increasing (or decreasing) on \([a,b]\).
\end{corollary}
\begin{theorem}
    \(f: [a,b] \to \mathbb{R}\) is continuous and differentiable on \(]a,b[\) then for \(f'(]a,b[)\) the intermediate value theorem holds and thus it is an interval.
\end{theorem}
\begin{proof}
    Let \(x_1, x_2 \in ]a,b[\ \). WLOG assume \(f'(x_1) < f'(x_2)\), we wish to prove that for all \(y^* \in ]f'(x_1), f'(x_2)[\) there is a \(x^* \in ]x_1,x_2[ \) such that \(f(x^*) = y^*\). Put \(g(x) = f(x) - y^*x\). By differentiability of \(f\) on \([a,b]\), \(g\) is differentiable on \([a,b]\). Then, \(g'(x_1) = f'(x_1) - y^* < 0\) and \(g'(x_2) =  f'(x_2) - y^* > 0\), therefore there are \(t_1,t_2 \in ]x_1,x_2[\) such that \(g(t_1) < g(x_1)\) and \(g(t_2) < g(x_2)\). Since \(g\) is continuous on \([x_1,x_2]\) then it must attains its minimum at some \(x^* \in [x_1,x_2]\). However \(x^* \) can't be \(x_1\) or \(x_2\) and hence \(x^* \in ]x_1,x_2[\). It is then easy to see that \(f'(x^*) = y^*\).
\end{proof}
\begin{definition}[Darboux continous]
    A function \(f\) is Darboux continous if it posseses the intermediate value property.
\end{definition}
For example \(f'\) of differentiable function is Darboux continuous.
\begin{theorem}[Cauchy's mean value theorem]
    \(f,g : [a,b] \to \mathbb{R}\) are continuous then there exists a \(c \in \; ]a,b[\), such that:
    \begin{equation*}
        f'(c)(g(b) - g(a)) = g'(c)(f(b) - f(a))
    \end{equation*}
\end{theorem}
\begin{proof}
    Define \(h(x) = (f(x) - f(a))(g(b) - g(a)) -  (g(x) - g(a))(f(b) - f(a))\), then clearly \(h(a) =0, h(b) =0\) and \(h(x)\) is continous and differentiable on \([a,b]\). Hence by applying the \cref{Rolle} for some \(c \in ]a,b[\) we have:
    \begin{align*}
         & h'(c)  = 0                                            \\
         & \implies f'(c)(g(b) - g(a)) -  g'(c)(f(b) - f(a)) = 0 \\
         & \implies  f'(c)(g(b) - g(a)) =  g'(c)(f(b) - f(a))
    \end{align*}
\end{proof}

\begin{theorem}[L'Hopital's rule]
    Suppose that \(\lim_{x \to a^+}{f(x)} = 0,\lim_{x \to a^+}{g(x)} = 0 \) where \(f,g\) are differentiable on a open interval \(I = ]a,b[\) for some \(b\) such that \(g'(x) \neq 0\) in \(I\) except maybe at \(x = a\) and the limit
    \begin{equation*}
        \lim_{x \to a^+}{\dfrac{f'(x)}{g'(x)}} = L
    \end{equation*}
    exists, then:
    \begin{equation*}
        \lim_{x \to a^+}{\dfrac{f(x)}{g(x)}} = L
    \end{equation*}
\end{theorem}
\begin{proof}
    For a fixed \(\epsilon > 0\) there exists a \(\delta > 0\) such that:
    \begin{equation*}
        \abs{x - a} < \delta \implies \abs{\dfrac{f'(x)}{g'(x)} - L} < \dfrac{ \epsilon}{2}
    \end{equation*}
    then since \(f(t), g(t) \to 0\) as \(t \to a\) from right side then there must be a \(t \in ]a,x[\) such that
    \begin{equation*}
        \abs{\dfrac{f(x) - f(t)}{g(x) - g(t)} - \dfrac{f(x)}{g(x)}} < \dfrac{\epsilon}{2}
    \end{equation*}
    then simply:
    \begin{align}
        \abs{\dfrac{f(x)}{g(x)} - L} & \leq \abs{\dfrac{f(x)}{g(x)} - \dfrac{f(x) - f(t)}{g(x) - g(t)} } + \abs{\dfrac{f(x) - f(t)}{g(x) - g(t)} - L} \\
                                     & < \dfrac{\epsilon}{2} + \abs{\dfrac{f'(\theta)}{g'(\theta)} - L}                                               \\
                                     & < \epsilon
    \end{align}
    Note that \(\theta \in \; ]t,x[\) and thus \(\abs{\theta - a} < \delta \)
\end{proof}
\begin{definition}[Higher order derivatives]
    \(f\) is said to be \(r_{\text{th}}\)-differentiable at \(x\) if it is differentiable \(r\) times. The \(r_{\text{th}}\) derivative of \(f\) is denoted as \(f^{(r)}\). If \(f^{(r)}\) exists for all \(r\) and \(x\) then \(f\) is said to be infinitely differentiable or smooth.
\end{definition}
\begin{definition}[Smoothness classes]
    The set of all \(f\) is continuosly \(r_{\text{th}}\)-differentiable is called class \(\mathcal{C}^r\).
\end{definition}
\begin{definition}[Taylor polynomial]
    The \(r_{\text{th}}\)-order Taylor polynomial of an \(r_{\text{th}}\)-order differentiable function \(f\) at \(x\) is
    \begin{equation*}
        P_r(x,h) =f(x) + f'(x)h +  \dfrac{f''(x)}{2}h^2 + \dots +  \dfrac{f^{(r)}(x)}{r!} h^r = \sum_{n = 0}^{r}\dfrac{f^{(n)}(x)}{n!} h^n
    \end{equation*}
\end{definition}
\begin{theorem}[Taylor approximation theorem]
    Let \(f\) be a \(r\)-differentiable function at \(x\) then:
    \begin{enumerate}
        \item
              \begin{equation*}
                  \dfrac{f(x+h) - P_r(x,h)}{h^r} \to 0 \text{ as } h \to 0
              \end{equation*}
        \item
              and \(P_r\) is the only \(r_{\text{th}}\) degree polynomial that has such property.
        \item
              Furthermore, if \(f\) is \(r\)-differentiable on an interval \(I\) for every \(x,y \in I\), there exists \(\xi\) between \(x,y\) such that:
              \begin{equation*}
                  f(y) - P_{r-1}(x,y-x) = \dfrac{f^{(r)}(\xi )}{(r)!}(y-x)^{r}
              \end{equation*}
    \end{enumerate}
\end{theorem}
\begin{proof} \leavevmode
    \begin{enumerate}
        \item
              For the base case \(r = 1\)
              \begin{align*}
                  \lim\limits_{h \to 0}{\dfrac{f(x+h) - f(x) -f'(x)h}{h}} = f'(x) - f'(x) = 0
              \end{align*}

              and by induction we prove the case \(r = n \geq 2\)
              \begin{align*}
                   & \lim\limits_{h \to 0}{\dfrac{f(x+h) - P_n(x,h)}{h^n}} = 0                                                                      \\
                   & \iff \forall \epsilon >0, \ \exists \delta > 0\ \text{such that } \ |h| < \delta \implies |f(x+h) - P_n(x,h)| < \epsilon |h^n|
              \end{align*}

              Let \(g(h) = f(x+h) - P_n(x,h)\) then since both \(f(x+h)\) and \(P_n(x,h)\) are differentiable then we apply \Cref{MVT}
              \begin{align*}
                  g(h) - g(0) & = g(h) = h(g'(c))                                                   \\
                              & = h(f'(x+c) - \sum_{k = 1}^{n}{\dfrac{f^{(k)}(x)}{(k-1)!}c^{k-1}})  \\
                              & =  h(f'(x+c) - \sum_{k = 0}^{n-1}{\dfrac{f^{(k+1)}(x)}{k!}c^{k}})   \\
                              & =  h(f'(x+c) - \sum_{k = 0}^{n-1}{\dfrac{f^{'^{(k)}}(x)}{k!}c^{k}})
              \end{align*}
              for some \(c \in ]0,h[\). Note that \(f'\) is \((n-1)\)-differentiable at \(x\) thus by induction for any \(\epsilon > 0\) there exists a \(\delta\) such that if \(c < \delta \) then:
              \begin{equation*}
                  |f'(x+c) - \sum_{k = 0}^{n-1}{\dfrac{f^{'^{(k)}}(x)}{k!}c^{k}}| < \epsilon |c^{n-1}|
              \end{equation*}

              which means
              \begin{align*}
                  |f(x+h) - P_n(x,h)| & = |g(h)| = |h| |f'(x+c) - \sum_{k = 0}^{n-1}{\dfrac{f^{'^{(k)}}(x)}{k!}c^{k}}| \\
                                      & < |h| \epsilon |c^{n-1}|< \epsilon |h^n|
              \end{align*}

              Therefore for any \(\epsilon\) if \(h < \delta\) then \(c < \delta\) and the result holds.
        \item
              Let \(Q_r(x,h)\) be another \(r_{\text{th}}\) degree polynomial such that
              \begin{equation*}
                  \lim\limits_{h \to 0}{\dfrac{f(x+h) -Q_r(x,h)}{h^r}} = 0
              \end{equation*}

              then
              \begin{equation*}
                  \lim\limits_{h \to 0}{\dfrac{P_r(x,h) -Q_r(x,h)}{h^r}} = 0
              \end{equation*}

              however this can only happen if \(Q_r(x,h) = P_r(x,h)\).
        \item
              Again for the base case \(r = 1\)
              \begin{equation*}
                  f(y) - f(x) = f'(\xi)(y-x)
              \end{equation*}

              which is the \Cref{MVT}. for \(r = n\) we have that
              \begin{equation*}
                  g(h) = f(x+h) - P_{n-1}(x,h) + Ch^n \implies g(0) = g'(0) = \dots = g^{(n-1)} =0
              \end{equation*}
              Set \(C\) such that \(g(y-x) = 0\). Then by applying \Cref{Rolle} \((n-1)\) times

              \begin{flalign*}
                  &&g(0) = g(y-x) = 0 &\implies g'(c_1) = 0 \quad c_1\in ]0,y-x[ &&\\
                  &&g'(0) = g'(c_1) = 0& \implies g'(c_2) = 0 \quad c_2 \in ]0,c_0[ &&\\
                  && &\vdots &&\\
                  &&g^{(n-2)}(0) = g^{(n-2)}(c_{n-2}) = 0 &\implies g^{(n-1)}(c_{n-1}) = 0 \quad c_{n-1} \in ]0,c_{n-2}[&&\\
                  &&g^{(n-1)}(0) = g^{(n-1)}(c_{n-1}) = 0 &\implies g^{(n)}(\xi - x) = 0 \quad \xi - x \in ]0,c_{n-1}[ \;\subset\; ]0,y-x[&&\\
                  &&\implies  g^{(n)}(\xi - x) =  f^{(n)}(\xi) + Cn! &= f^{(n)}(\xi) - \dfrac{n!}{(y-x)^n}(f(y) - P_{n-1}(x,y-x) ) = 0 &&\\
                  &&\implies f(y) - P_{n-1}(x,y-x)  &=  \dfrac{f^{(n)}(\xi)}{n!}(y-x)^n \qquad \xi \in ]x,y[&&
              \end{flalign*}
    \end{enumerate}
    which completes the proof.
\end{proof}

\begin{theorem}[Inverse function]
    Let \(I\) be an open set and \(f : I \to \mathbb{R}\) is continuous and differentiable such that its derivate is non-zero. Thus, \(f\) is either monotonic. Furthermore, it is one to one then it has a differentiable inverse \(f^{-1}\):
    \[f^{-1}(x) = \dfrac{1}{f'(f^{-1}(x))}\]
\end{theorem}
\begin{proof}
    limit algebra
\end{proof}

%\chapter{Integration}
\thispagestyle{headings}

\begin{definition}[Partition]
    \(I = [a,b] \in \mathbb{R}\). A partition of I is a finite ordered sequence of points in \(I\).
    \begin{equation*}
        P = \{x_0, x_1, \dots, x_n | a = x_0 \leq x_1 \leq \dots \leq x_n = b\}
    \end{equation*}
    A partition pair \((P,T)\) is set
    \begin{equation*}
        (P,T) = \{x_0, t_1,x_1, t_2,x_2 \dots x_{n-1}, t_n, x_n | a = x_0 \leq t_1 \leq x_1 \leq \dots \leq t_n \leq x_n = b\}
    \end{equation*}
    Moreover, define \(\|P\| = \max{(x_i - x_{i-1})}\).
\end{definition}
The Riemann sum of a function \(f\) on the interval \([a,b]\) with respect to the pair partition \((P,T)\) is:
\begin{equation*}
    R(f,P,T)= \sum_{i = 1}^{n} {f(t_i)(x_i - x_{i-1})}
\end{equation*}
\begin{definition}[Riemann Integrability]
    If there exist a number \(S\) that for all \(\epsilon > 0\) there exist a \(\delta > 0\) such that for all partition \((P,T)\) that if \(\|P\| < \delta\) implies \(|S - R(f,P,T)| < \epsilon\), then \(f\) is Riemann integrable and \(S\) is the integral of \(f\) on \(I\) denoted by
    \begin{equation*}
        \int_{a}^{b}{f}
    \end{equation*}
    . Furthermore, if \(S\) exists then it is unique.
    Denote the set of all Riemann integrable function on an interval \(I\) as \(\mathcal{R}_I\).
\end{definition}
\begin{theorem}
    Suppose \(f \in \mathcal{R}_I\) then \(f\) is bounded.
\end{theorem}
\begin{proof}
    By Riemann integrability of \(f\), for a \(\epsilon > 0\) there is \(\delta > 0\) such that for any partition pair \((P,T)\) that \(\|P\| < \delta\) then \(|S - R(f,P,T)| < \epsilon\). Consider twp partition pair \((P,T)\) and \((P,T')\) on \(I\) with  \(\|P\| < \delta\) and \(t_i = t'_i\) for all \(i\) except \(j\). Then by triangle inequality:
    \begin{align*}
         & |R(f,P,T) - R(f,P,T')| \leq |S - R(f,P,T)| + |S - R(f,P,T')| \leq 2 \epsilon       \\
         & \implies |R(f,P,T) - R(f,P,T')| = (x_j - x_{j-1})|f(t_j) - f(t'_j)| \leq 2\epsilon
    \end{align*}
\end{proof}
\begin{corollary}  \leavevmode
    \begin{enumerate}
        \item 	If \(f,g \in \mathcal{R}_I\) and \(c \in \mathbb{R}\) then \(f + cg \in \mathcal{R}_I\) and
              \begin{equation*}
                  \int_a^b{f +cg } =  \int_a^b{f} + c\int_a^b{g}
              \end{equation*}
        \item For a constant function \(f(x) = c \) its integral is \(c(b-a)\)
        \item If \(f(x) \geq 0\) then \(\int_{a}^{b}{f} \geq 0\)
    \end{enumerate}
\end{corollary}
\begin{definition}[Commen refinement]
    Let \(P_1,P_2\) be two partitions on an interval \(I\). Their common refinement \(P^* = P_1 \lor P_2\) is defined as
    \begin{equation*}
        P^* = \{z_0 \leq z_1 \leq \dots \leq z_{m} | z_i \in P_1 \cup P_2\}
    \end{equation*}
\end{definition}
\begin{definition}[Darboux Integral]
    Suppose \(f : [a,b] \to \mathbb{R}\) is a bounded function. Define the upper Darboux and lower Darboux sums with respect to a partition \(P\) as follow
    \begin{align*}
        \text{U}(f,P) & = \sum_{i = 1}^n{M_i(x_i - x_{i-1})} & M_i = \sup\limits_{x \in [x_{i-1},x_i]}{\{f(x)\}} \\
        \text{L}(f,P) & = \sum_{i = 1}^n{m_i(x_i - x_{i-1})} & m_i = \inf\limits_{x \in [x_{i-1},x_i]}{\{f(x)\}}
    \end{align*}
    Consider \(P'\) a refinement of \(P\), then the following inequalities hold:
    \begin{equation*}
        \text{L}(f,P) \leq \text{L}(f,P') \leq 	\text{U}(f,P') \leq \text{U}(f,P)
    \end{equation*}
    Therefore as the partition gets refined the upper sum decrease and the lower sum increase. Since both of these sums are bounded then by the completeness axiom the upper and lower integral
    \begin{align*}
        \overline{\int_{a}^{b}}f  & = \inf{\{\text{U}(f,P)\}} \\
        \underline{\int_{a}^{b}}f & = \sup{\{\text{L}(f,P)\}}
    \end{align*}
    exist. In case they are equal, \(f\) is said to be Darboux integrable.
\end{definition}
\begin{theorem}
    Darboux integrability is equivalent to Riemann integrability and the value of integrals are equal.
\end{theorem}

\begin{proof}
    Firstly, assume \(f\) is bounded and Darboux integrable. Equivalently, for any \(\epsilon_1 > 0\) there exists a partition \(P\) such that

    \begin{equation*}
        \text{U}(f,P) - \text{L}(f,P) <\epsilon_1
    \end{equation*}

    Let \(\|P\| = \delta_1 \) and \(0< \delta < \delta_1\) such that if a partition \(Q\) has \(\|Q\| < \delta\) then for all partition pairing \(|I - R(f,Q,T) | < \epsilon\). Consider \(P^* = Q \lor P\). It is clear that

    \begin{equation*}
        \text{U}(f,P^*) - \text{L}(f,P^*) <\epsilon_1 \quad \text{and} \quad \|P^*\| < \delta
    \end{equation*}

    To estimate \(\text{U}(f,Q)\) and \(\text{L}(f,Q)\) consider their difference with \(\text{U}(f,P^*)\) and \(\text{L}(f,P^*)\), respectively.

    \begin{equation*}
        \text{U}(f,Q) - \text{U}(f,P^*) = \sum_{i = 1}^{n}{M^Q_i (x^Q_i - x^Q_{i-1})} - \sum_{i = 1}^{n^*}{M^*_i (x^*_i - x^*_{i-1})}
    \end{equation*}

    The sums are different only in \(x^*_i \in P\). Therefore, their difference is in the intervals that are have an endpoint in \(P\) and for each of these interval the difference is \((M_j^Q - M_i^*)(x^*_i - x^*_{i-1})\), note that \(j\) is dependent on \(i\), hence

    \begin{equation*}
        \text{U}(f,Q) - \text{U}(f,P^*) = \sum_{i = 1}^{n^P}{((M_j^Q - M_i^*)(x^*_i - x^*_{i-1})} < 2Mn^P\delta
    \end{equation*}

    where \(M\) is the bound of \(f\). Similary for the lower bounds we get:

    \begin{equation*}
        \text{L}(f,P^*) - \text{L}(f,Q) = \sum_{i = 1}^{n^P}{((m_i^*- m_j^Q )(x^*_i - x^*_{i-1})} < 2Mn^P\delta
    \end{equation*}

    As a result if set \(\delta\) such that \(\text{U}(f,Q) - \text{L}(f,Q) < \epsilon\) we will be done, since for any partition \(T\) \(R(f,Q,T),I \in [ \text{L}(f,Q),\text{U}(f,Q)]\) hence \(| I - R(f,Q,T) < \epsilon\). To do so notice

    \begin{equation*}
        \text{U}(f,Q) - \text{L}(f,Q) = \text{U}(f,Q) - \text{U}(f,P^*) + \text{U}(f,P^*) - \text{L}(f,P^*)  + \text{L}(f,P^*) -  \text{L}(f,Q) < 4Mn^P\delta + \epsilon_1
    \end{equation*}

    which will be less \(\epsilon\) if

    \begin{equation*}
        \delta = \min{(\dfrac{\epsilon}{6Mn^P}, \delta_1)} , \qquad \epsilon_1 = \dfrac{\epsilon}{3}
    \end{equation*}

    Secondly, assum \(f\) is Riemann integrable. Then for any fixed \(\epsilon > 0\) then for any two pair partition \((P,T), (P,T')\) such that \(\|P\| < \delta\) then
    \begin{equation*}
        R(f,P,T) - R(f,P,T') < \dfrac{\epsilon}{3}
    \end{equation*}

    Then choose \(T\) such that
    \begin{equation*}
        \text{U}(f,P) - R(f,P,T) < \dfrac{\epsilon}{3}
    \end{equation*}
    that is, choose \(t_i\) such that
    \begin{align*}
         & M_i - f(t_i) < \dfrac{\epsilon}{3(b-a)}                                                                                                \\
         & \implies \sum_{i=1}^{n}{(M_i - f(t_i))(x_i - x_{i-1})} <\dfrac{\epsilon}{3(b-a)} \sum_{i = 1}^{n}{x_i - x_{i-1}} < \dfrac{\epsilon}{3}
    \end{align*}
    Similarly one can choose \(T'\) so that
    \begin{equation*}
        R(f,P,T')- \text{L}(f,P)  < \dfrac{\epsilon}{3}
    \end{equation*}
    Therefore:
    \begin{equation*}
        \text{U}(f,P) -  \text{L}(f,P) = \text{U}(f,P) - R(f,P,T) + R(f,P,T) - R(f,P,T') +  R(f,P,T')- \text{L}(f,P) < \epsilon
    \end{equation*}
\end{proof}
example: f and g differ in only one point.
\begin{definition}[Zero set]
    A set \(A \subset \mathbb{R}\) is a zero set if for each \(\epsilon > 0\) there is a countable covering of \(A\) of open intervals \(]a_i, b_i[\) such that:
    \begin{equation}
        \sum_{i = 1}^{\infty}{b_i - a_i} \leq \epsilon
    \end{equation}
    If a property holds for all points except those in a zero set then one says that the property holds almost everywhere.
\end{definition}

\begin{proposition}
    The following properties hold for zero sets:
    \begin{enumerate}
        \item
              Covering of \(A\) with open intervals is equivalent to covering with closed interval.
        \item
              A finit set is a zero set.
        \item
              A countable union of zero set is a zero set.
    \end{enumerate}
\end{proposition}
\begin{definition}[Oscillation]
    Suppose \(f : I \to \mathbb{R}\) where \(I\) is an interval and \(x \in I\) then the oscillation of \(f\) at \(x\) is
    \begin{align*}
        \text{Osc}(f,x) & =\limsup\limits_{t \to x} f(t) - \liminf\limits_{t \to x} f(t) \\
                        & = \lim\limits_{h \to 0} \diam{f([x-h,x+h])}
    \end{align*}
\end{definition}
\begin{proposition}
    \(f\) is continuous at \(x\) if and only if \(\text{Osc}(f,x) = 0\).
\end{proposition}
\begin{theorem}[Riemann-Lebesgue theorem]
    The function \(f\) is Riemann integrable if and only if it is bounded and the set of its discountinuities is zero set.
\end{theorem}
\begin{proof}
    First assume \(f\) is Riemann integrable. Let \(\mathcal{D}\) be the set of all its discontinuities. Moreover, \(\mathcal{D}_n = \{x | \text{Osc}(f,x) \geq \frac{1}{n}\}\). Thus it is clear that \(\mathcal{D} = \bigcup \mathcal{D}_n \). We will show that each \(\mathcal{D}_n\) is a zero set. By Riemann integrability of \(f\), for any \(\epsilon >0\) we have a partition \(P\) such that
    \begin{equation*}
        \text{U}(f,P) - \text{L}(f,P) < \dfrac{\epsilon}{n}
    \end{equation*}
    We call \([x_{i-1},x_i] \in P\) a bad interval if there exist \(x \in [x_{i-1},x_i]\) an interior point, such that \(x \in \mathcal{D}_n\).
    \begin{align*}
         & \sum_{\text{bad}}{(M_i - m_i)(x_i - x_{i-1})} < 	\text{U}(f,P) - \text{L}(f,P) < \dfrac{\epsilon}{n}                      \\
         & \dfrac{1}{n} \sum_{\text{bad}}{(x_i - x_{i-1}) } < \sum_{\text{bad}}{(M_i - m_i)( x_i - x_{i-1}) } < \dfrac{\epsilon}{n} \\
         & \qquad \implies \sum_{\text{bad}}{(x_i - x_{i-1}) } < \epsilon
    \end{align*}
    and since the endpoints are finite then \(\mathcal{D}_n\) is a zero set and therefore, \(\mathcal{D}\) is zero set.
    Seconde assume that \(f:[a,b] \to \mathbb{R}\) is bounded and \(\mathcal{D}\) is a zero set. Choose \(n\) such that:
    \begin{equation*}
        \dfrac{1}{n} < \epsilon_1
    \end{equation*}
    for \(\epsilon_1\) that is to be determined. Since \(\mathcal{D}_n \subset \mathcal{D} \) then it is a zero set as well. In other words for any \(\epsilon_2\) there is covering of \(\mathcal{D}_n\), \(I_1, I_2, \dots\) such that
    \begin{equation*}
        \sum{\diam{I_i}} < \epsilon_2
    \end{equation*}
    For any \(x \notin \mathcal{D}_n\) we know that is an open interval \(J_x\) such that \(M_{J_x} - m_{J_x} < \dfrac{1}{n}\). Let \(I = \bigcup I_i\) and  \(J = \bigcup J_x\).
    It is clear than \(I \cup J\) is a covering of \([a,b]\). Since \([a,b]\) is compact then the open covering has a Lebesgue number\(\lambda\). Let \(P\) be a partition such that \(\|P\| < \lambda\) then an interval \([x_{i-1}, x_i]\) is bad if it is wholly within a \(I_i\) and it is good if it is not.
    \begin{align*}
        \text{U}(f,P) - \text{L}(f,P) & = \sum_{\text{good}}{(M_i - m_i)(x_i - x_{i-1})}+ \sum_{\text{bad}}{(M_i - m_i)(x_i - x_{i-1})} \\
                                      & < \dfrac{1}{n} \sum_{\text{good}}{(x_i - x_{i-1})} + 2M \sum_{\text{bad}}{(x_i - x_{i-1})}      \\
                                      & < \dfrac{b-a}{n} + 2M\epsilon_2 < (b-a)\epsilon_1 +  2M\epsilon_2 = \epsilon
    \end{align*}
    by setting \(\epsilon_1 = \dfrac{\epsilon}{2(b-a)}\) and \(\epsilon_2 = \dfrac{\epsilon}{4M}\).
\end{proof}
\begin{corollary}
    \leavevmode
    \begin{enumerate}
        \item
              Any continuous function \(f:[a,b] \to \mathbb{R}\) is integrable.
              \begin{prooflemma}
                  Since there is no point of discontinuity then it is a zero set.
              \end{prooflemma}
        \item
              Any monotonic function \(f:[a,b] \to \mathbb{R}\) is integrable.
        \item
              Product of two integrable function is integrable.
    \end{enumerate}
\end{corollary}
\begin{theorem}[Fundamental theorem of calculus]
    If \(f\) is an integrable function then its indefinite integral
    \begin{equation*}
        F(x) = \int_{a}^{x}{f(t)dt}
    \end{equation*}
    is continuous at \(x\). Furthermore, its derivative is equal to \(f(x)\) at every point \(x\) that \(f\) is continuous.
\end{theorem}
\begin{definition}
    \(F(x)\) is anti-derivate of \(f(x): [a,b] \to \mathbb{R}\) if
    \begin{equation*}
        F'(x) = f(x)
    \end{equation*}
    for all \(x \in [a,b]\).
\end{definition}
\begin{corollary}
    Every continuous function has an anti-derivative.
\end{corollary}
\begin{theorem}
    Anti-derivate of a Riemann integrable function if exists differs from its indefinite integral by a constant.
\end{theorem}

%\chapter{Series}
\thispagestyle{headings}
\begin{theorem}
    A series \(s_n\) is convergent if and only if for each \(\epsilon > 0\) there exists a \(N \in \mathbb{N}\) such that:
    \begin{equation*}
        m,n \geq N \implies \abs{\sum_{i = n}{m}{a_m}} \leq \epsilon
    \end{equation*}
\end{theorem}
\begin{proof}
    Obviously any convergent sequence is Cauchy. Furthermore, due to completeness of \(\mathbb{R}\) every Cauchy sequence is convergent.
\end{proof}
\begin{corollary}
    The series \(s_n\) is convergent if \(a_n \to 0\).
\end{corollary}
\begin{theorem}	\leavevmode
    \begin{enumerate}
        \item
              If \(|a_n| < b_n\) for all \(n > N\) for a sufficiently large \(N\) then convergence of \(\sum{b_n}\) implies the convergence \(\sum{a_n}\).
        \item
              If \(0 < b_n < a_n\) for all \(n > N\) for a sufficiently large \(N\) then divergence of \(\sum{b_n}\) implies the divergence \(\sum{a_n}\).
    \end{enumerate}
\end{theorem}
\begin{corollary}
    Absolute convergence implies convergence.
\end{corollary}
\begin{theorem}[Integral Test]
    Consider the improper integral \(\int_{0}^{\infty}{f}\) and the series \(\sum_{i = 1}^{\infty}{a_k}\)
    \begin{enumerate}
        \item \(0 \leq a_k \leq f(x)\) for sufficiently large \(k\) and each \(x \in \; ]k-1, k]\), then the convergence of integral implies the convergence of the series.
        \item Similarly if \(0 \leq f(x) \leq a_k\) for sufficiently large \(k\) and each \(x \in \; [k,k+1[\), then the divergence of integral implies the divergence of the series.
    \end{enumerate}
\end{theorem}
\begin{definition}
    The exponential growth rate of the series \(\sum{a_n}\)
    \begin{equation*}
        \alpha = \limsup_{k \to \infty}{\sqrt[k]{a_k}}
    \end{equation*}
\end{definition}
\begin{theorem}[Root test]
    If \(\alpha < 1\) the series is convergent and if \(\alpha > 1\) it is divergent. If \(\alpha = 1\) the test is inconclusive.
\end{theorem}
\begin{theorem}[Ratio test]
    Let the ratio between successive terms of the series \(a_k\) be \({r_k = |\frac{a_{k+1}}{a_k}|}\)
    \begin{equation*}
        \rho = \limsup {r_k} \qquad \lambda  = \liminf{r_k}
    \end{equation*}
    If \(\rho < 1\) then the series converges, if \(\lambda > 1\)  then the series diverges, and otherwise the ratio test is inconclusive.
\end{theorem}
\begin{theorem}
    Let \(a_1 \geq a_2 \geq \dots \geq 0\) be a decreasing non-negative sequence then the alternating series
    \begin{equation}
        \sum_{n = 1}^{\infty}{a_n (-1)^n}
    \end{equation}
    is convergent.
\end{theorem}
\begin{theorem}
    Suppose \(\sum{c_k x^k}\) is a power series. Its radius of convergence \(R\) is unique and is such that for \(|x| < R\) the power series converges and for \(|x| > R\) diverges.
    \begin{equation*}
        R = \dfrac{1}{\limsup {\sqrt[k]{c_k}}}
    \end{equation*}
\end{theorem}

%\chapter{Function Spaces}
\thispagestyle{headings}

\begin{definition}[Point Convergence]
    for each point there is a \(\epsilon\)
\end{definition}
example \(x^n\), 1/2 lines, \(sqrt(x^2 + 1/n)\), rationals
\begin{definition}[Uniform Convergence]
    there is a \(\epsilon\) for all points
\end{definition}
\begin{corollary}
    Uniform convergence implies point convergence.
\end{corollary}
example x/n with restriction
\begin{theorem}
    \(f_n\) are uniformly convergent and continuous then \(f\) is continuous
\end{theorem}

\begin{definition}
    \(\mathcal{C}^0(X,\mathbb{R}):\) all continuous function from \(X \) to \(\mathbb{R}\). \(d(f,g) = \sup{\{|f(x) - g(x)| : x \in X\}}\)
\end{definition}
\begin{definition}
    \(\|f\| = \sup\{|f(x)| : x \in X\}\) therefore \(d(f,g) = \|f - g\|\)
\end{definition}

\begin{theorem}
    \(f_n:[a,b] \to \mathbb{R}\) uniformly convergent if \(f_n\) is Riemann integrable then \(f\) is Riemann integrable
    \[\int_{a}^{b} \underbrace{\lim{f_n}}_{f} = \lim{\int_{a}^{b}{f_n}}\]
\end{theorem}
\begin{lemma}
    metric spaces and \(f_n: X \to X'\) are uniformly convergent if \(f_n\) is bounded then \(f\) is bounded.
\end{lemma}

\begin{proposition}
    \((\mathcal{C}^0_b,d)\) is a complete metric space.
\end{proposition}
\begin{definition}
    \(\mathcal{C}^0_b(X,\mathbb{R})\) and 	\(\mathcal{C}_b(X,\mathbb{R})\) closed subset of \(\mathcal{B}(X,\mathbb{R})\)
\end{definition}
exampele: any compact metric space
\begin{theorem}
    \(f_n\) are differentiable functions
    \begin{enumerate}
        \item \(f_n'\) are uniformly convergent to \(g\)
        \item \(f_n\) are point convergent
    \end{enumerate}
\end{theorem}
\begin{proposition}
    \(f_n:[a,b] \to \mathbb{R}\) consider \(\sum{f_n}\)
    \begin{enumerate}
        \item \(f_n\) are riemann integrable and the series uniformly convergent then the \(\sum{f_n}\) is riemann integrable and
              \[ \int_{a}^{b}{\sum{f_n}} = \sum {\int_{a}^{b}{f_n}}\]
        \item
              Similarly for derivative
    \end{enumerate}
\end{proposition}
\begin{definition}[Wierstrass M test]
    This is super cool
\end{definition}
power series and convergence
radius of convergence of integral/derivative of power series is equal to the radius of convergence of the originalo series.
\begin{theorem}
    in the convergence cirle the power series is inifinitely integrable and differentiable, and coefficients are taylor coefficients
\end{theorem}
analytical definition
proof of analytical means analytical in interval using a pair sequence


% \chapter{Multivariable Calculus}
\thispagestyle{headings}
%\section{Linear Algebra}
\subsection{Vector Spaces}

\begin{definition}[Normed vector space]
    Let \(V\) be a vector space. A \textbf{norm} is a real valued function \(\norm{ \scdot }: V \to \Reals\) which has the following properties
    \begin{enumerate}
        \item \(\forall x \in V, \; \norm{x} > 0\).
        \item \(\norm{x} = 0 \implies x = 0\).
        \item \(\forall x \in V \; \forall \alpha \in \Field, \; \norm{\alpha x} = \abs{\alpha} \norm{x}\).
        \item \(\forall x,y \in V \; \norm{x+y} \leq \norm{x} + \norm{y}\).
    \end{enumerate}
\end{definition}

Each normed vector space induces a metric space \(\metricSpace{V}{d}\) where \(\func{d}{x,y} = \norm{y - x}\).

\begin{theorem}
    In every normed space \(\normedSpace{V}{\norm{\scdot}}\) we have
    \begin{equation*}
        \abs{ {\norm{v} - \norm{w}}} \leq \norm{v - w}
    \end{equation*}
    Hence the norm is Lipschitz continuous.
\end{theorem}


\begin{definition}
    Assume \(V\) is a vector space and let \(\norm{\scdot}_1, \; \norm{\scdot}_2\) be two norms for \(V\). They are said to be equivalent when
    \begin{equation*}
        \exists c_1,c_2 > 0 \; \forall x : \qquad c_1 \norm{x}_1 \leq \norm{x}_2 \leq c_2 \norm{x}
    \end{equation*}
\end{definition}

To check if the above definition is indeed an equivalence relation, we must show that following:
\begin{description}
    \item [Reflexive] \(\norm{\scdot}_1 \sim \norm{\scdot}_1\).
    \item [Symmetric] \(\norm{\scdot}_1 \sim \norm{\scdot}_2 \implies \norm{\scdot}_2 \sim \norm{\scdot}_1\).
    \item [Transitive] \( \norm{\scdot}_1 \sim \norm{\scdot}_2 , \; \norm{\scdot}_2 \sim \norm{\scdot}_3 \implies \norm{\scdot}_1 \sim \norm{\scdot}_3\).
\end{description}

\begin{remark}
    Equivalent norms induce equivalent metrics, hence they induce the same topology.
\end{remark}

\begin{theorem} \label{th:normsEquivalent}
    All norms defined on a finite dimensional vector space \(V\) are equivalent.
\end{theorem}

\begin{proof}
    Let \(\norm\) be an arbitrary norm on \(V\) and \(\{e_1, e_2, \dots , e_n\} \) be a basis of \(V\). Let \(\norm_2\) be \(L_2\)-norm (Euclidean norm). It will suffice to show \(\norm \sim \norm_2\). Let
    \begin{equation*}
        M = \max \! \left( \norm{e_1}, \dots , \norm{e_n} \right)
    \end{equation*}
    Take \(x \in V\), writing \(x = \sum_{i = 1}^n {\xi_i e_i}\) we have:
    \begin{equation*}
        \norm{x} = \norm{\sum_{i = i}^n {\xi_i e_i}} \leq \sum_{i = 1}^n \abs{\xi_i} \norm{e_i} \leq M \sqrt{n} \norm{x}_2
    \end{equation*}
    Taking \(c_2 = M \sqrt{n}\) proves the right inequality. For the left inequality we need the following lemma
    \begin{lemma} \label{lm:ContinuityOfNorm}
        If \(V\) is a normed vector space with \(\norm_2\), as defined above, is viewed as metric space \(\metricSpace{V}{\norm_2}\) then \(\norm : V \to \Reals\) is continuous.
    \end{lemma}

    \begin{prooflemma}
        Let \(x_0 \in V\) and \(M\) be defined as above. For any \(\epsilon > 0\) consider \(\delta = \frac{\epsilon}{M \sqrt{n}}\) then if \(\norm{x - x_0}_2 < \delta\)
        \begin{equation*}
            \abs{{\norm{x} - \norm{x_0}}} \leq \norm{x - x_0} \leq M \sqrt{n} \norm{x - x_0} \leq \epsilon
        \end{equation*}
    \end{prooflemma}

    Now consider the sphere of radius \(r = 1\) centered at \(0\), \(\func{S_1}{0} = S_1 = \{x \in V : \norm{x}_2 = 1\}\). One can show that \(S\) is compact (\Cref{th:CompactnessOfFiniteDimensional}). Therefore, \(\norm{x}\) assumes its minimum on \(S\). Let \( a = \norm{x_0}\) be the minimum. Since \(0 \notin S\) then \(a > 0\). By letting \(y = x / \norm{x}_2 \), we have \(y \in S\) and thus \(a \leq \norm{y}\) which is
    \begin{equation*}
        a \norm{x}_2 \leq \norm{x}
    \end{equation*}
    Taking \(c_1 = a\) proves the theorem.
\end{proof}

\begin{theorem}\label{th:CompactnessOfFiniteDimensional}
    Let \(\normedSpace{V}{\norm}\) be a normed space over a normed complete field \(\Field\). The following are equivalent
    \begin{enumerate}
        \item \(V\) is finite dimensional. \label{it:COFD_1}
        \item every bounded closed set in \(V\) is compact. \label{it:COFD_2}
        \item the closed unit ball in \(V\) is compact. \label{it:COFD_3}
    \end{enumerate}
\end{theorem}
\begin{proof}
    \Cref{it:COFD_1} \(\implies\) \Cref{it:COFD_2}: It is similar to proving a closed set \(\Reals^n\) is compact using the fact a closed interval is compact in \(\Reals\).

    \Cref{it:COFD_2} \(\implies\) \Cref{it:COFD_3}: Trivial.

    \Cref{it:COFD_3} \(\implies\) \Cref{it:COFD_1}: Requires the following lemma:
    \begin{lemma} [Riesz's lemma} \label{lm:RieszsLemma}
        If \(V\) is a normed vector space and \(W\) is a closed proper subspace of \(V\) and \(\alpha \in \Reals\) with \(0 < \alpha < 1\), then there exists an \(v \in V\) with \(\norm{v} = 1\) such that \(\norm{v- w} \geq \alpha \) for all \(w \in W\)
    \end{lemma}
    Now suppose \(V\) were to be an infinite dimensional vector space. Then by the \Cref{lm:RieszsLemma} there is sequence of unit vectors \({x_n}\) such that \(\forall m,n \in \Naturals, \; \norm{x_n - x_m} > \alpha\) for some \(0 <\alpha < 1\). Which implies that no subsequence of \(\set{x_n}\) is convergent and hence the closed unit ball can not be compact.
\end{proof}

\begin{example}
    The closed unit ball in the infinite dimensional vector space \(\func{C}{\clcl{0}{1},\Reals}\) with \(\norm{f} = \max \func{f}{x}\) is not compact.  Take \(\func{f_n}{x} = x^n\). Obviously \(\norm{f_n} = 1\), however \(f_n\) doesn't uniformly converge and hence \(f_n\) doesn't have a limit in \(\func{C}{\clcl{0}{1},\Reals}\) with the \(\max\) norm. Consider the following norm
    \begin{equation*}
        \norm{f}_I = \int_0^1 \abs{\func{f}{x}} \diffOperator x
    \end{equation*}
    Note that \(\norm_I\) and \(\norm_{\max} \) are not equivalent. Let \(\func{g}{x} = 0\) for all \(x \in \clcl{0}{1}\). Then
    \begin{equation*}
        \norm{f_n - g}_I = \dfrac{1}{n+1} \to 0 \quad \text{as} n \to \infty.
    \end{equation*}
\end{example}

\begin{definition}[Banach space}
    A normed vector space \(V\) that is complete is a \textbf{Banach space}. A \textbf{Hilbert Space} is a Banach space whose norm is induced by an inner product.
\end{definition}


\begin{proposition} \label{pr:FiniteIsBanach}
    A normed finite dimensional vector space \(V\) over a normed complete field \(\Field\), is Banach space.
\end{proposition}

\begin{proof}
    Let \(\set{v_i} \in V\) be a Cauchy sequence, and \(\set{e_1, \dots ,e_n}\) be a basis for \(V\) with the norm \(L^1\), that is if \(v = (\xi^1 , \dots ,\xi^n)\) then \(\norm{v} = \sum_{m = 1}^n \abs{\xi^m}\). Then if \(v_i = (\xi^1_i , \dots , \xi^n_i)\)
    \begin{equation*}
        \abs{\xi^m_i - \xi^m_j} \leq \sum_{m=1}^n \abs{\xi^m_i - \xi^m_j} \leq \norm{v_i - v_j} < \epsilon
    \end{equation*}
    then \(\set{\xi^m_i}_i\) are a Cauchy sequence in \(\Field\) and hence they converge \(\xi^m_i \to \xi^m\). Then, clearly \(v_i \to v = (\xi^1 , \dots , \xi^n)\) as each component converges.
\end{proof}

\begin{example}
    \(\Rationals\) form a vector space itself over itself. It is finite dimensional as \(\set{\DSOne_\Rationals}\) is the basis, however the sequence
    \begin{equation*}
        1 ,\; 1.4 ,\; 1.41 ,\; \dots
    \end{equation*}
    does not converge even though it is Cauchy.
\end{example}

\subsection{Linear Maps}
Let \(V\) and \(W\) be a vector spaces over \(\Field\). A map \(T: V \to W\) is \textbf{linear} if
\begin{equation*}
    \func{T}{x + \lambda y} = \func{T}{x} + \lambda \func{T}{y}
\end{equation*}
for all \(x,y \in V\) and \(\lambda \in \Field\).

\begin{definition}
    Let \(\normedSpace{V}{\norm_V}\) and \(\normedSpace{W}{\norm_W}\) be normed spaces then, a linear transformation \(T : V \to W\) is \textbf{bounded} if there exists a constant \(C > 0\) such that
    \begin{equation*}
        \norm{Tv}_W \leq C \norm{v}_V
    \end{equation*}
    for all \(v \in V\). We denote the set of all linear map from \(V \to W\) as \(\func{\CalL}{V,W}\) and the set of all bounded linear maps as \(\func{\CalB}{V,W}\). If \(T \in \func{\CalL}{V,W}\) is bijective such that \(T^{-1} \in \func{\CalL}{V,W}\), then \(T\) is called an \textbf{isomorphism} and \(V,W\) are \textbf{isomorphic}. An operator \(T \in \func{\CalL}{V,W}\) is called \textbf{isometric} if \(\norm{Tv}_W = \norm{v}_V\) for all \(v \in V\).
\end{definition}

\begin{definition}
    If \(\normedSpace{V}{\norm_V},\normedSpace{W}{\norm_W}\) are normed spaces then the \textbf{operator norm} of a linear transformation \(T : V \to W\) is
    \begin{equation*}
        \norm{T} = \sup \left\{\dfrac{\norm{Tv}_W}{\norm{v}_V} \middle| v \neq 0 \right\}
    \end{equation*}
\end{definition}

\begin{proposition}
    Let \(T : U \to V\) and \(T' : V \to W\) be two linear transformations.
    \begin{equation*}
        \norm{T' \circ T} \leq \norm{T} \norm{T'}
    \end{equation*}
\end{proposition}

\begin{proof}
    for an arbitrary non-zero \(x \in U\)
    \begin{equation*}
        \norm{\func{T' \circ T}{x}}_W \leq \norm{T'} \norm{Tx}_V \leq \norm{T'} \norm{T} \norm{x}_U
    \end{equation*}
    which implies
    \begin{equation*}
        \norm{T' \circ T} \leq \norm{T} \norm{T'}
    \end{equation*}
\end{proof}

\begin{theorem} \label{th:linearTransformation}
    Let \(\normedSpace{V}{\norm_V}\) and \(\normedSpace{W}{\norm_W}\) be normed spaces and \(T: V \to W\) be a linear transformation. The following are equivalent
    \begin{enumerate}
        \item \(\norm{T}\) is finite. \label{it:LT_1}
        \item \(T\) is bounded. \label{it:LT_2}
        \item \(T\) is Lipschitz continuous. \label{it:LT_3}
        \item \(T\) is continuous at a point. \label{it:LT_4}
        \item \(\sup_{\norm{v}_V = 1} \norm{Tv}_W < \infty\). \label{it:LT_5}
    \end{enumerate}
\end{theorem}

\begin{proof}
    \cref{it:LT_1} \(\Rightarrow\) \cref{it:LT_2}: Obviously
    \begin{align*}
        \dfrac{\norm{Tv}_W}{\norm{v}_V} & \leq \norm{T}            \\
        \implies \norm{Tv}_W            & \leq \norm{T} \norm{v}_V
    \end{align*}
    note that if \(v = 0\) then \(Tv = 0\) as well and thus the last inequality holds for all \(v \in V\).

    \cref{it:LT_2} \(\Rightarrow\) \cref{it:LT_3}:
    \begin{equation*}
        \norm{Tv - Tu}_W = \norm{T(u - v)}_W \leq C \norm{u - v}_V
    \end{equation*}

    \cref{it:LT_3} \(\Rightarrow\) \cref{it:LT_4}: Trivial.

    \cref{it:LT_4} \(\Rightarrow\) \cref{it:LT_5}: Let \(T\) be continuous at \(u \in V\). Then there is  a \(\delta > 0 \) such that
    \begin{equation*}
        \norm{v-u} < \delta \implies \norm{Tv - Tu}_W = \norm{T(v-u)}_W < 1
    \end{equation*}
    Now for an arbitrary non-zero \(v\) we have
    \begin{equation*}
        \norm{\left( \dfrac{\delta v}{2\norm{v}_V} + u \right) - u}_V < \delta
    \end{equation*}
    Therefore
    \begin{align*}
         & \norm{\func{T}{\dfrac{\delta v}{2\norm{v}_V}}}_W  < 1         \\
         & \norm{\func{T}{\dfrac{v}{\norm{v}_V}}}_W  < \dfrac{2}{\delta}
    \end{align*}

    \cref{it:LT_5} \(\Rightarrow\) \cref{it:LT_1}: Let \(v \in V\) be an arbitrary vector. Then
    \begin{align*}
                 & \sup \norm{\func{T}{\dfrac{v}{\norm{v}_V}}}_W < \infty \\
        \implies & \sup \dfrac{\norm{Tv}_W}{\norm{v}_W} < \infty
    \end{align*}

\end{proof}




\begin{theorem} \label{th:finiteDimensionTransformationContinuous}
    If \(V\) is a finite dimensional normed vector space then any linear transformation \(T : V \to W\) is continuous.
\end{theorem}

\begin{proof}
    Since \(V\) is finite dimensional, according to \Cref{th:normsEquivalent}, any two norms are equivalent. Hence, take \(\norm_2\) to be Euclidean norm over a basis \(\{e_1, \dots , e_n\}\). Let \(x\) be such that \(\norm{x}_2 < \delta\) for some \(\delta > 0\). Therefore, \(\abs{\xi_i} < \delta^2\)
    \begin{equation*}
        \norm{Tx}_W = \norm{\sum_{i = 1}^n \xi_i \func{T}{e_i}}_W \leq \sum_{i = 1}^n \abs{\xi_i} \norm{\func{T}{e_i}}_W \leq \delta^2 K
    \end{equation*}
    where \(K = \max \norm{\func{T}{e_i}}_W \). By letting \(\delta = \sqrt{\frac{\epsilon}{K}}\) we proved continuity at \(0\) and hence the continuity by \Cref{th:linearTransformation}.
\end{proof}

Another proof of \Cref{pr:FiniteIsBanach}

\begin{proof}
    Let \(\{e_1, \dots , e_n\}\) be a basis for \(V\) and \(\phi : V \to \Field^n\) be the representation map for the basis. Since \(\phi\) is a linear map and a bijection then \(\phi\) is homeomorphism. Consider a Cauchy sequence \(\set{v_k} \in V\) and let \(x_k = \func{\phi}{v_k}\) then by continuity of \(\phi\) and \(\phi^{-1}\) we have
    \begin{equation*}
        \abs{x_i - x_j} = \abs{\func{\phi}{v_i} - \func{\phi}{v_j}} \leq \norm{\phi} \norm{v_i - v_j} \leq \norm{\phi} \norm{\func{\phi^{-1}}{x_i} - \func{\phi^{-1}}{x_j}} \leq \norm{\phi} \norm{\phi^{-1}} \abs{x_i - x_j}
    \end{equation*}
    hence \(\set{x_k}\) are Cauchy in \(\Field^n\) which by completeness of \(\Field\) implies that they are convergent, \(x_k \to x\). Let \(v = \func{\phi^{-1}}{x}\) then by the right side of the inequality \(v_k \to v\).
\end{proof}

\begin{remark}
    As seen in the last proof, for a bijective linear transformation \(T\)
    \begin{equation*}
        1 \leq \norm{T} \norm{T^{-1}}
    \end{equation*}
\end{remark}

\begin{definition}[Dual space}
    Let \(V\) be a normed space over the normed field \(\Field\), then the \textbf{topological/continuous dual space} of the normed space \(V\) is
    \begin{equation*}
        V^\ast  = \func{\CalL}{V,\Field}
    \end{equation*}
    elements of \(V^\ast\) are called \textbf{bounded functionals} on \(V\).
\end{definition}

\begin{remark}
    Dual space is defined for all vector spaces, however, in analysis we study the topological dual space which only in the finite dimensional case coincide with the algebraic dual space.
\end{remark}

\begin{proposition}
    For a finite dimensional normed vector space \(V\), \(\dim V^\ast = \dim V\).
\end{proposition}

\begin{proof}
    Let \(\set{e_1, \dots , e_n}\) be a basis for \(V\) then, consider the following linear functions
    \begin{equation*}
        e^\ast_1, \dots , e^\ast_n \in V^\ast
    \end{equation*}
    where
    \begin{equation*}
        \func{e^\ast_i}{e_j} = \begin{cases}
            1 & i = j    \\
            0 & i \neq j
        \end{cases}
    \end{equation*}
    we claim that \(\set{e^\ast_1, \dots , e^\ast_n}\) is a basis for \(V^\ast\). It is easy to see that they are as for each \(j\)
    \begin{equation*}
        \left[\sum_{i = 1}^n c_i e^\ast_i\right} e_j = c_j
    \end{equation*}
    and for each \(\phi \in \func{\CalL}{V,\Field}\) we have
    \begin{equation*}
        \func{\phi}{e_j} = \alpha_i =  \sum_{i = 1}^n \alpha_i \func{e^\ast_o}{e_i}\\
    \end{equation*}
    hence \(\dim V^\ast = n = \dim V\).
\end{proof}

\begin{theorem}
    For two normed vector spaces \(V,W\), \(\normedSpace{\func{\CalB}{V,W}}{\norm{T}}\) is a normed vector space. Moreover, it is a Banach space when \(W\) is a Banach space.
\end{theorem}


\begin{proof}
    Clearly \(\func{\CalB}{V,W}\) is a vector space. For its norm \(\norm{T}\) we have
    \begin{enumerate}
        \item \(\norm{T} \geq 0\) by definition.
        \item if \(\alpha \in \Field_W\) then
              \begin{equation*}
                  \norm{\alpha T} = \sup \left\{ \dfrac{\norm{(\alpha T)v}_W}{\norm{v}_V} \middle| v \neq 0 \right\} = \abs{\alpha} \sup \left\{ \dfrac{\norm{Tv}_W}{\norm{v}_V} \middle| v \neq 0 \right\} = \abs{\alpha} \norm{T}
              \end{equation*}
        \item for the triangle inequality
              \begin{align*}
                  \norm{T_1 + T_2} & = \sup \left\{ \dfrac{\norm{(T_1 + T_2)v}_W}{\norm{v}_V} \right\}                                                     \\
                                   & \leq \sup \left\{ \dfrac{\norm{T_1v}_W + \norm{T_2v}_W}{\norm{v}_V} \right\}                                          \\
                                   & = \sup \left\{ \dfrac{\norm{T_1v}_W}{\norm{v}_V} \right\} + \sup \left\{ \dfrac{\norm{ T_2 v}_W}{\norm{v}_V} \right\} \\
                                   & = \norm{T_1} + \norm{T_2}
              \end{align*}
    \end{enumerate}
    Suppose \(W\) is a Banach space and \(\{T_i\} \in \func{\CalB}{V,W}\) is a Cauchy sequence. Then for all \(v \in V\)
    \begin{equation*}
        \forall \epsilon \, \exists N \; \suchThat \; m,n > N \implies \norm{T_m v - T_n v}_W \leq \norm{T_m - T_n}\norm{v}_V < \epsilon
    \end{equation*}
    \(\{T_i v\}\) is a Cauchy sequence. Since \(W\) is complete then \(T_i v \to Tv\) for some function \(T\). We claim that \(T\) is a bounded linear map and is the limit of \(T_i \to T\).
    \begin{align*}
        \func{T}{v + cu} & = \lim_{i \to \infty} \func{T_i}{v + cu} = \lim_{i \to \infty} T_i v + c T_i u \\
                         & = Tv + c Tu
    \end{align*}
    Note that  \( \abs{{\norm{T_m} - \norm{T_n}}} \leq \norm{T_m - T_n}\) and hence \(\norm{T_i}\) is a Cauchy in sequence in \(\Reals\) that has a limit \(t\). There exists a \(N\) such that \(\abs{{\norm{T_n} - t}} < 1\) for all \(n \geq N\).
    \begin{equation*}
        \dfrac{\norm{Tv}_W}{\norm{v}_V} = \lim_{i \to \infty}  \dfrac{\norm{T_i v}_W}{\norm{v}_V} < t + 1
    \end{equation*}
    hence \(T\) is bounded and \(T \in \func{\CalB}{V,W}\). Finally, we show that \(T_i \to T\). For an arbitrary \(v \neq 0\) and \(\epsilon > 0\) there exist \(N\) such that
    \begin{align*}
        n \geq N \implies \norm{T_i v - Tv}_W < \epsilon \norm{v}_V
    \end{align*}
    which means that
    \begin{equation*}
        \norm{T_i - T} = \sup \dfrac{\norm{T_i v - Tv}_W}{\norm{v}_V} < \epsilon
    \end{equation*}
    Therefore \(T_i \to T\) as desired.
\end{proof}

\begin{theorem}
    Let \(\normedSpace{V}{\norm}\) be a normed space. Then any linear transformation \(T: \Reals^n \to V\) is continuous. Furthermore, if \(T\) is a bijection, it is a homeomorphism.
\end{theorem}

\begin{proof}
    Since \(\Reals^n\) is finite then by \Cref{th:finiteDimensionTransformationContinuous}, \(T\) is continuous. Assuming \(T\) is bijective, we must show that its inverse \(T^{-1}\) is continuous as well. Since \(T\) is a bijection then \(T\) is a linear isomorphism and \(\dim V = \dim \Reals^n = n\) hence \(T^{-1}: V \to \Reals^n\) is a continuous map.
\end{proof}


\begin{theorem} \label{th:LinearInvertibility}
    Let \(V,W\) be two finite dimensional normed vector spaces. \(T : V \to W\) linear transformation is invertible if and only if there exists a \(c\) such that:
    \begin{equation*}
        c \norm{v}_V \leq \norm{Tv}_W
    \end{equation*}
\end{theorem}

\begin{proof}
    If \(T\) is invertible then \(T^{-1} : W \to V \) is bounded and thus
    \begin{equation*}
        \norm{T^{-1}w}_V \leq c \norm{w}_W
    \end{equation*}
    and since \(T\) is bijective then there exists \(v\) such that \(w = Tv\) which implies
    \begin{equation*}
        \norm{y}_V \leq c\norm{Ty}_W
    \end{equation*}
    If there exists such \(c\) then \(\norm{Tx} > 0\) for all non-zero \(x\) and hence \(\ker T  = 0\) which implies that \(T\) is a bijection and is invertible.
\end{proof}

\begin{remark}
    the supremum of such \(c\) is \(\norm{T^{-1}}^{-1}\) which is called the \textbf{conorm} of \(T\).
\end{remark}

\begin{definition}[General linear group}
    The \textbf{general linear group} of a vector space, written \(\func{\GL}{V}\) is the set of all bijective linear transformation.
\end{definition}

\begin{proposition}
    If \(V\) is a finite (also works for infinite) vector space then \(\func{\GL}{V}\) is open in \(\func{\CalL}{V,V}\), in fact, if \(f \in \func{\GL}{V}\) then the open ball centered at \(f\) with radius \(\norm{f^{-1}}^{-1}\) remains in \(\func{\GL}{V}\). Furthermore, the inverse operator \(i : \func{\GL}{V} \to \func{\GL}{V}\), \(\func{i}{T} = T^{-1}\) is continuous.
\end{proposition}

\begin{proof}
    First assume \(f = \DSOne_V\) then we prove that any linear \(g\) that \(\norm{\DSOne_V - g} < 1\) is invertible which then implies bijectivity (true for linear maps). Let \(\norm{v} = 1\) then
    \begin{equation*}
        \abs{\norm{v} - \norm{gv}} \leq \norm{v - gv} \leq \norm{\DSOne_V - g} \norm{v} < 1
    \end{equation*}
    Therefore
    \begin{equation*}
        0 < \norm{gv} < 2
    \end{equation*}
    which means \(\ker g = \set{0}\) and since \(V\) is finite then then \(g\) is invertible. For a general \(f\), we have that
    \begin{equation*}
        \norm{1 - f^{-1} \circ g} \leq \norm{f^{-1}}\norm{f -g} < 1
    \end{equation*}
    therefore \(f^{-1} \circ g\) is invertible and as a consequence \(g = f \circ f^{-1} \circ g\) is invertible. To prove inverse operator is continuous, fix \(\epsilon > 0\) then for a \(\delta > 0\) if \(\norm{T-S} < \delta\) then
    \begin{align*}
        \norm{\DSOne_V- T^{-1} \circ S}= \norm{T^{-1} \circ T  - T^{-1} \circ S}           & \leq \norm{T^{-1}} \norm{T-S} < \delta \norm{T^{-1}} \\
        \implies  \norm{T^{-1} - S^{-1}} \leq \norm{T^{-1}\circ S - \DSOne_V}\norm{S^{-1}} & < \delta \norm{T^{-1}} \norm{S^{-1}}
    \end{align*}
    note that by letting \(\delta = \norm{T^{-1}}^{-1}/2\) then
    \begin{equation*}
        \norm{S} > -\dfrac{\norm{T^{-1}}^{-1}}{2} + \norm{T} > \dfrac{\norm{T^{-1}}^{-1}}{2}
    \end{equation*}
    also if for any invertible linear map \(R\)
    \begin{equation*}
        \norm{R} > a \implies \norm{Rx} > a\norm{x} \implies \dfrac{\norm{y}}{a} = \dfrac{\norm{R\circ \func{R^{-1}}{y}}}{a} > \norm{R^{-1}y}
    \end{equation*}
    which means that \(\norm{S^{-1}} < 2 \norm{T^{-1}}\), hence by letting
    \begin{equation*}
        \delta = \min \set{\dfrac{\epsilon \norm{T^{-1}}^2}{2} , \dfrac{\norm{T^{-1}}^{-1}}{2}}
    \end{equation*}
    we proved the continuity.
\end{proof}


\begin{definition}
    Let \(V_1, V_2 ,\dots , V_n\) be  normed vector spaces. Then \(\phi : V_1 \times \dots \times V_n \to W\) is \textbf{\(n\)-linear} if by fixing any \(n-1\) component, \(\phi\) is linear relative to the remaining component.
\end{definition}

\begin{proposition}
    If \(V_1, V_2, \dots , V_n\) are  normed vector spaces and \(\ \phi : V_1 \times \dots \times V_n \to W \) is a \(n\)-linear then the followings are equivalent
    \begin{enumerate}
        \item \(\phi\) is continuous. \label{it:nLP_1}
        \item \(\phi\) is continuous at 0. \label{it:nLP_2}
        \item \(\phi\) is bounded, that is there exists a constant \(C > 0\) such that \label{it:nLP_3}
              \begin{equation*}
                  \norm{\func{\phi}{v_1, \dots ,v_n}}_W \leq C \norm{v_1}_{V_1} \dots \norm{v_n}_{V_n}
              \end{equation*}
    \end{enumerate}
\end{proposition}

\begin{remark}
    As oppose to linear transformation, \(n\)-linear function's continuity does not imply uniform continuity.
\end{remark}

\begin{proof}
    \Cref{it:nLP_1} \(\implies\) \Cref{it:nLP_2}: Trivial.

    \Cref{it:nLP_2} \(\implies\) \Cref{it:nLP_3}: For the sake of contradiction, suppose \Cref{it:nLP_3} is false. That is, for every \(k \in \Naturals\) there exists a point \(v_k = (v^1_k, \dots , v^n_k)\) such that
    \begin{equation*}
        \norm{\func{\phi}{v^1_k, \dots , v^n_k}}_W > n^n \norm{v^1_k}_{V_1} \dots \norm{v^n_k}_{V_k}
    \end{equation*}
    Note that \(v^m_k\) can not be zero for any \(k\) and \(m\), otherwise \(\func{\phi}{v_k} = 0 \). Define
    \begin{equation*}
        w^m_k = \dfrac{v^m_k}{n\norm{v^m_k}_{V_k}} \to 0
    \end{equation*}
    which from the continuity at 0 implies that \(w_k = (w^1_k, \dots , w^n_k) \to 0\). However,
    \begin{equation*}
        \norm{ \func{\phi}{w_k} - \func{\phi}{0}}_W > n^n \frac{1}{n} \dots \frac{1}{n} = 1
    \end{equation*}
    which is a contradiction.

    \Cref{it:nLP_3} \(\implies\) \Cref{it:nLP_1}. Let \(v_n \to v\) and define the points
    \begin{equation*}
        \bar{v}^m_k = (v^1 , \dots , v^m, v^{m+1}_k , \dots , v^n_k) , \qquad \bar{v}^0_k = v_k
    \end{equation*}
    and \(\bar{v}^n_k = v\). Note that \(v^m_k\) are bounded for sufficiently large \(k \geq N_1\), therefore there exists \(M\) such that \(\forall m, \; \norm{v^m_k}_{V_m} \leq M\). Also, pick \(M\) such that \(\forall m, \; \norm{v^m}_{V_m} \leq M\) as well. Then
    \begin{align*}
        \norm{\func{\phi}{v_k} - \func{\phi}{v}}_W & \leq \sum_{i = 1}^n \norm{\func{\phi}{\bar{v}^{i-1}_k  }- \func{\phi}{\bar{v}^i_k}}_W                                                              \\
                                                   & = \sum_{i = 1}^n \norm{\func{\phi}{\bar{v}^{i - 1}_k - \bar{v}^{i}_k }}_W                                                                          \\
                                                   & \leq \sum_{i = 1}^n C \norm{v^1}_{V_1} \dots \norm{v^{i-1}}_{V_{i-1}} \norm{v^i_k - v^i}_{V_i} \norm{v^{i+1}_k}_{V_{i+1}} \dots \norm{v^n_k}_{V_n} \\
                                                   & \leq CM^{n-1} \sum_{i = 1}^n \norm{v^i_k - v^i}_{V_i}
        \intertext{pick \(N_2\) such that for all \(k \geq N_2\), for each \(i, \; \norm{v^i_k - v^i}_{V_i} < \frac{\epsilon}{nCM^{n-1}}\) then}
        \norm{\func{\phi}{v_k} - \func{\phi}{v}}_W & < CM^{n-1}  \sum_{i = 1}^n \frac{\epsilon}{nCM^{n-1}} = \epsilon
    \end{align*}
\end{proof}

We denote  the set of all \(n\)-linear functions from \(V_1 \times \dots \times V_n \to W\) by \(\func{\CalL^n}{V_1 \times \dots \times V_n, W}\).
\begin{proposition} \label{pr:nLinearIsmorphicLinear}
    Let \(V_1, \dots , V_n, W\) be normed vector spaces. Then \(\func{\CalL^n}{V_1 \times \dots \times V_n, W}\) and \(\func{\CalL}{V_1 , \func{\CalL}{V_2, \dots ,\func{\CalL}{V_n,W}}}\) are isomorphic.
\end{proposition}

\begin{proof}
    We want to prove
    \begin{equation*}
        \func{\CalL^n}{V_1 \times \dots \times V_n, W} \cong \func{\CalL}{V_1 , \func{\CalL}{V_2, \dots ,\func{\CalL}{V_n,W}}}
    \end{equation*}
    consider the mapping \(T : \func{\CalL}{V_1 ,\func{\CalL}{V_2, \dots ,\func{\CalL}{V_n,W}}} \to \func{\CalL^n}{V_1 \times \dots \times V_n, W}\), such that for any \(v_1 \in V_1,\; \dots, \; v_n \in V_n\)
    \begin{equation*}
        \func{\alpha}{(v_1)(v_2) \dots (v_n)} = \func{\func{T}{\alpha}}{v_1,v_2, \dots, v_n}
    \end{equation*}
    First note that \(T\) is linear. Then if \(\func{T}{\alpha} = 0\) implies \(\alpha  = 0\), thus \(T\) is injective and hence bijective.
\end{proof}

\begin{definition}[Positive definite}
    Matrix \(A \in \SquareMatrices[\Reals}{n}\) is \textbf{positive definite} whenever \(A\) is symmetric and 
    \begin{equation*}
        \forall x \in \Reals^n \backslash \set{0}, \ x^T A x > 0
    \end{equation*}
\end{definition}

\begin{theorem}
     Every positive definite matrix \(A\) is diagonizable. In face, there exists an orthognal matrix \(P\) such that 
     \begin{equation*}
         PAP^T = \begin{bmatrix}
             \lambda_1 & 0&\dots & 0 \\
             0 & \lambda_2 & \dots & 0\\
             \vdots & \ddots & & \vdots\\
             0 & \dots & 0 & \lambda_n
         \end{bmatrix}
     \end{equation*}
     where \(\lambda_i > 0\) for each \(i\).
\end{theorem}

{\Large\textbf{Exercises}}
\begin{enumerate}
    \item Show that for a linear transformation \(T\), \(\norm{T} = \sup_{\norm{v}_V \leq 1} \norm{Tv}_W\).
    \item Prove or disprove that if \(x^T A x = x^T A^T x\) for all \(x \in \Reals^n \backslash \set{0}\) then \(A = A^T\).
\end{enumerate}
\newpage


%\chapter{Differentiation}
Let \(V,W\) be finite dimensional vector spaces and \(f: U \subset V \to W\) where \(U\) is open. Then \(f\) is differentiable at \(x_0\) when a linear transformation \(T : V \to W\) such that
\begin{equation*}
    \lim_{\norm{h} \to 0} \dfrac{\norm{\func{f}{x_0 + h} - \func{f}{x_0} - \func{T}{h}}}{\norm{h}}= 0
\end{equation*}
Or equivalently there exists a sublinear function \(\func{R}{h}\) such that
\begin{equation*}
    \func{f}{x_0 + h} - \func{f}{x_0} - Th = \func{R}{h} \qquad \frac{\func{R}{h}}{\norm{h}} \to 0
\end{equation*}
\(T\) if it exists is unique, represented by \(\func{f'}{x_0}\), \(\DiffOperator f\), or \(\func{\diffOperator f}{x}\) and called the \textbf{total derivative} or \textbf{Fr\'{e}chet derivative}.

\begin{example}
    Any linear function \(f : V \to W\) with \(\func{f}{v} = Tv + b\) where \(b \in W\) is differentiable and \(\func{\DiffOperator f}{v} = T\). Since
    \begin{equation*}
        \norm{h}_V < \delta \implies \norm{\func{f}{v + h} - \func{f}{v} - \operatorFunc{\DiffOperator \func{f}{v}}{h}}_W = \norm{T(v+h) - Tv - Th}_W = 0 < \epsilon \norm{h}_V
    \end{equation*}
    Hence, the derivative of any linear function is constant.
    Consider \(S : V \times V \to V\) with \(\func{S}{v,u} = v + u\). \(S\) is differentiable because \(S\) is linear (why?). We claim that \(\DiffOperator S = S\) as
    \begin{equation*}
        \norm{\func{S}{(v + h),(u + k)} - \func{S}{v,u} - \func{S}{h,k}} = 0
    \end{equation*}
\end{example}

\begin{example}
    Let \(\mu : \Reals \times V \to V\) with \(\func{\mu}{r,x} = rx\). Then \(\mu\) is differentiable and \(\operatorFunc{\func{\DiffOperator \mu}{r,x}}{t,h} = rh + tx\) as
    \begin{align*}
        \norm{\func{\mu}{(r + t),(x + h)} - \func{\mu}{r,x} - \operatorFunc{\func{\DiffOperator \mu}{r,x}}{t,h}} & = \norm{rx + rh + tx + th - rx - rh - tx}     \\
                                                                                                                 & = \abs{t} \norm{h} \leq \epsilon \norm{(t,h)}
    \end{align*}
    by letting \(\norm{(t,h)} = \sqrt{t^2 + \norm{h}^2}\) and \(\delta = \epsilon\).
\end{example}

\begin{proposition}
    Differentiability of \(f\) at \(x\) implies continuity at \(x\).
\end{proposition}

\begin{proof}
    \begin{equation*}
        \norm{\func{f}{x + h} - \func{f}{x}} = \norm{\operatorFunc{\DiffOperator \func{f}{x}}{h} + \func{R}{h}} \leq \norm{\DiffOperator \func{f}{x}}\norm{v} + \norm{\func{R}{v}} \to 0
    \end{equation*}
    as \(v \to 0\).
\end{proof}

\begin{proposition} \label{eq:partialDerivative}
    Assume \(f: U \subset V \to W\) is differentiable at \(x_0\) and let \(u \in V\) be a non-zero vector then
    \begin{equation*}
        \func{f'}{x_0} (u) = \lim_{t \to 0} \dfrac{\func{f}{x_0 + tu} - \func{f}{x_0}}{t}
    \end{equation*}
\end{proposition}

\begin{proof}
    Let \(h = tu\) then
    \begin{align*}
        \func{R}{tu}                     & = \func{f}{x_0 + tu} - \func{f}{x_0} - \func{T}{tu}            \\
                                         & = \func{f}{x_0 + tu} - \func{f}{x_0} - t\func{T}{u}            \\
        \implies \dfrac{\func{R}{tu}}{t} & = \dfrac{ \func{f}{x_0 + tu} - \func{f}{x_0}}{t} - \func{T}{u} \\
        \implies \lim_{t \to 0}          & \dfrac{ \func{f}{x_0 + tu} - \func{f}{x_0}}{t} = \func{T}{u}
    \end{align*}
\end{proof}

\begin{definition}[Directional derivative]
    If we let \(\norm{u} = 1\) then the limit in \Cref{eq:partialDerivative} becomes the \textbf{directional derivative} of \(f\) in the direction of \(u\) and is denoted by \(\DiffOperator_u f\).
\end{definition}

\begin{remark}
    The existence of all directional derivatives of \(f\) doesnt imply its differentiability or even its continuity.
\end{remark}

\begin{remark}
    If \(\DiffOperator f: U \to \func{\calL}{V,W}\) is continuous then each \(\PDiff{f_i}{x_j}\) is continuous. Since
    \begin{equation*}
        \DiffOperator \func{f}{x} = \begin{bmatrix}
            \func{\PDiff{f_1}{x_1}}{x} & \dots  & \func{\PDiff{f_1}{x_n}}{x} \\
            \vdots                     & \ddots &                            \\
            \func{\PDiff{f_m}{x_1}}{x} & \dots  & \func{\PDiff{f_m}{x_n}}{x}
        \end{bmatrix}
    \end{equation*}
    and the reverse is true as well.
\end{remark}

\begin{theorem} \label{th:DifferentiabilityCriteria}
    \(f : V \to W\) has all of its partial derivative in a neighbourhood of \(u \in U\) and they're continuous at \(u\) then \(f\) is differentiable at \(u\). Especially, if \(\PDiff{f_i}{x_j}\) exist and are continuous at every point of \(U\) then \(f \in \calC^1\).
\end{theorem}

\begin{proof}
    We prove that each \(f_i\) is differentiable. Let \(\set{e_1, \dots , e_n}\) be a basis for \(V\) and take \(\norm{x} = \sum \abs{\xi_j}\). Consider a convex neighbourhood \(E\) of \(a\). Then, for a given \(\epsilon > 0\) we will show there exists a \(\delta > 0\) such that
    \begin{equation*}
        \norm{h} < \delta \implies \norm{\func{f_i}{a + h} - \func{f_i}{a} - \sum_{j = 1}^n \operatorFunc{\DiffOperator_{e_j} \func{f_i}{a}}{h_j}} \leq \epsilon \norm{h}
    \end{equation*}
    Cosider the point sequence \(a^k =\sum_{j < k} a_j e_j + \sum_{j \geq k} (a_j + h_j)e_j \) where \(a^1 = a + h\) and \(a^{n + 1} = a\) then
    \begin{equation*}
        \norm{\func{f_i}{a + h} - \func{f_i}{a} - \sum_{j = 1}^n \operatorFunc{\DiffOperator_{e_j} \func{f_i}{a}}{h_j}}  \leq \sum_{k = 1}^{n} \norm{\func{f_i}{a^k} - \func{f_i}{a^{k+1}} - \operatorFunc{\DiffOperator_{e_k} \func{f_i}{a}}{h_k}}
    \end{equation*}
    hence we are done if
    \begin{equation*}
        \norm{\func{f_i}{a^k} - \func{f_i}{a^{k+1}} - \operatorFunc{\DiffOperator_{e_k} \func{f_i}{a}}{h_k}} \leq \epsilon \abs{h_k}
    \end{equation*}
    for \(k = n\)
    \begin{equation*}
        \norm{\func{f_i}{a^n} - \func{f_i}{a} - \operatorFunc{\DiffOperator_{e_n} \func{f_i}{a}}{h_n}}
    \end{equation*}
    which equivalent to the existence \(n_\cardinalTH\) partial derivative of \(a\). and for \(k < n\)
    \begin{align*}
         & \norm{\func{f_i}{a^k} - \func{f_i}{a^{k+1}} - \operatorFunc{\DiffOperator_{e_k} \func{f_i}{a}}{h_k}}                                                                                                                                  \\
         & \leq \norm{\func{f_i}{a^k} - \func{f_i}{a^{k+1}} - \operatorFunc{\DiffOperator_{e_k} \func{f_i}{a^k}}{h_k}}  + \norm{\operatorFunc{\DiffOperator_{e_k} \func{f_i}{a^k}}{h_k} - \operatorFunc{\DiffOperator_{e_k} \func{f_i}{a}}{h_k}}
    \end{align*}
    which uses the existence of partial derivatives in neighbourhood and its continuity.
\end{proof}

\begin{proposition}
    Let \(f,g : V \to W\) be differentiable at \(x\) and \(h : W \to U\) be differentiable at \(y = \func{f}{x}\). Furthermore, let \(c\) be an scalar then
    \begin{enumerate}
        \item \(\func{\DiffOperator \,}{ f + cg} = \DiffOperator f + c \DiffOperator g\).
        \item  \(h \circ f\) is differentiable at \(x\) and
              \begin{equation*}
                  \func{\DiffOperator \,}{ h \circ f} =  \left( (\DiffOperator h) \circ f \right) \circ \DiffOperator f
              \end{equation*}
    \end{enumerate}
\end{proposition}

\begin{proof} \leavevmode
    \begin{enumerate}
        \item we have
              \begin{align*}
                   & \norm{ \operatorFunc{f + cg}{x + k} - \operatorFunc{f + cg}{x} - \operatorFunc{\DiffOperator \func{f}{x} + c \DiffOperator \func{g}{x}}{k}}                                          \\
                   & \leq \norm{ \func{f}{x + k} - \func{f}{x} - \operatorFunc{\DiffOperator \func{f}{x}}{h}} + \abs{c}\norm{\func{g}{x + k} - \func{g}{x} - \operatorFunc{\DiffOperator \func{g}{x}}{h}}
              \end{align*}
        \item we know that
              \begin{equation*}
                  \begin{cases}
                      \func{f}{x + k} - \func{f}{x} - \operatorFunc{\DiffOperator \func{f}{x}}{k}  = \func{R}{k} \\
                      \func{h}{y + l} - \func{h}{y} - \operatorFunc{\DiffOperator \func{h}{y}}{l}  = \func{S}{l}
                  \end{cases}
              \end{equation*}
              and we wish to prove that
              \begin{equation*}
                  \func{h \circ f}{x + k} - \func{h \circ f}{x} - \operatorFunc{\DiffOperator \func{h}{\func{f}{x}} \circ \DiffOperator \func{f}{x}}{k} = \func{T}{k}
              \end{equation*}
              where \(\norm{\func{T}{k}} \leq \epsilon \norm{k}\) whenever \(\norm{k} < \delta\). Let \(l = \func{f}{x + k} - \func{f}{x}\) and substituting into the second equation
              \begin{align*}
                  \func{h}{\func{f}{x + k}} & - \func{h}{\func{f}{x}} - \operatorFunc{\DiffOperator \func{h}{y}}{\func{f}{x + k} - \func{f}{x}}                                                                                         \\
                                            & =  \func{h}{\func{f}{x + k}} - \func{h}{\func{f}{x}} - \operatorFunc{\DiffOperator \func{h}{y}}{\operatorFunc{\DiffOperator \func{f}{x}}{k}  + \func{R}{k}}                               \\
                                            & = \func{h}{\func{f}{x + k}} - \func{h}{\func{f}{x}} - \operatorFunc{\DiffOperator \func{h}{y} \circ \DiffOperator \func{f}{x}}{k} - \operatorFunc{\DiffOperator \func{h}{y}}{\func{R}{k}} \\
                                            & = \func{T}{k} - \operatorFunc{\DiffOperator \func{h}{y}}{\func{R}{k}} = \func{S}{l}                                                                                                       \\
                  \implies \func{T}{k}      & = \func{S}{l} + \operatorFunc{\DiffOperator \func{h}{y}}{\func{R}{k}}
              \end{align*}
    \end{enumerate}
\end{proof}


\begin{proposition}
    \(f : U \subset V \to W_1 \times \dots \times W_n\) is differentiable at \(x_0\) if and only if all its component is differentiable at \(x_0\). Furthermore, \(\DiffOperator f = (\DiffOperator f_1, \dots , \DiffOperator f_n)\).
\end{proposition}

\begin{proof}
    Define the following norm on \(W_1 \times \dots \times W_n\)
    \begin{equation}
        \norm{(w_1, \dots w_n)} = \sum_{i = 1}^n \norm{w_i}_{W_i}
    \end{equation}
    then
    \begin{equation*}
        \norm{ \func{f}{x_0 + h} - \func{f}{x_0} - \operatorFunc{\DiffOperator \func{f}{a}}{h}} = \sum_{i = 1}^n \norm{ \func{f_i}{x_0 + h} - \func{f_i}{x_0} - \operatorFunc{\DiffOperator \func{f_i}{a}}{h}}
    \end{equation*}
    and since every other norm is equivalent to the norm defined above, we are done.
\end{proof}

\begin{theorem}[Leibnitz rule]
    Let \(V_1, V_2, \dots , V_n\) be finite dimensional vector spaces and \(f: V_1 \times \dots \times V_n \to W\) is a \(n\)-linear function. \(f\) is differentiable at \(a = (a_1, \dots , a_n)\) and
    \begin{equation*}
        \operatorFunc{\func{\DiffOperator f}{a}}{h_1, \dots h_n} = \func{f}{h_1, a_2, \dots, a_n} + \func{f}{a_1, h_2, \dots, a_n} + \dots + \func{f}{a_1, a_2, \dots, h_n}
    \end{equation*}
\end{theorem}

\begin{proof}
    we have that
    \begin{equation*}
        \func{f}{a + h} = \sum_{\xi_i \in \set{a_i,h_i}} \func{f}{\xi_1, \dots , \xi_n}
    \end{equation*}
    therefore
    \begin{equation*}
        \func{f}{a + h} - \func{f}{a} - \sum_{i = 1}^{n} \func{f}{a_1, \dots, a_{i-1}, h_i, a_{i+1}, \dots , a_n} = \sum_{\substack{\xi_i \in \set{a_i,h_i} \\ \text{at least two \(h_i\)}}} \func{f}{\xi_1, \dots , \xi_n}
    \end{equation*}
    Let \(\delta = 1\) then  \(\norm{h} = \sum \norm{h_i} < 1\) also \(i,j, \; \norm{h_i}\norm{h_j} \leq \norm{h}^2\). Hence if we define
    \begin{equation*}
        A = \max \set[\prod_{i \in I} {\norm{a_i}}]{I \subset \Naturals_n}
    \end{equation*}
    then
    \begin{equation*}
        \sum_{\substack{\xi_i \in \set{a_i,h_i} \\ \text{at least two \(h_i\)}}} \func{f}{\xi_1, \dots , \xi_n} \leq (2^n - n - 1)A \norm{h}^2
    \end{equation*}
    and letting \(\delta = \min \set{1 , \dfrac{\epsilon}{(2^n - n - 1)(A + 1)}}\) we arrive at the conclusion.
\end{proof}

\begin{example}
    Let \(Z: \Reals^3 \times \Reals^3 \to \Reals\) with \(\func{Z}{u,v}= u \times v\) be a bilinear function, \(f,g: \Reals \to \Reals^3\) and \(\func{h}{t} = \func{f}{t} \times \func{g}{t}\). \(h = Z \circ \phi\) where \(\func{\phi}{t} = (\func{f}{t},\func{g}{t})\). Then we have:
    \begin{align*}
        \DiffOperator \func{h}{t} & = \operatorFunc{\DiffOperator Z}{\func{\phi}{t}}\circ \DiffOperator \func{\phi}{t}                             \\
                                  & =  \operatorFunc{\DiffOperator Z}{\func{\phi}{t}} \circ (\DiffOperator \func{f}{t}, \DiffOperator \func{g}{t}) \\
                                  & = \func{Z}{\DiffOperator \func{f}{t}, \func{g}{t}} + \func{Z}{ \func{f}{t}, \DiffOperator \func{g}{t}}         \\
                                  & = \DiffOperator \func{f}{t} \times \func{g}{t} + \func{f}{t} \times \DiffOperator \func{g}{t}
    \end{align*}
\end{example}

\begin{example}
    Consider \(A = [\func{f_{ij}}{x_1, \dots , x_n}]\) where each \(f_{ij}\) is differentiable. Then
    \begin{equation*}
        \DiffOperator \func{\det}{A}
    \end{equation*}
    can be calculated using the Leibnitz rule, since determinant is \(n\)-linear function.
\end{example}

\section{Mean value theorem}
Mean value theroem of 1-dimensional does not generalize very well. For example, the continuous function \(\func{f}{t} : \clcl{0}{1} \to \Reals^2\) with
\begin{equation*}
    t \mapsto (t^2, t^3)
\end{equation*}
is differentiable on \(\opop{0}{1}\), however
\begin{align*}
    \func{f}{1} - \func{f}{0} = (1,1) & = \DiffOperator \func{f}{c} (1 - 0) \\
                                      & = (2c,3c^2)
\end{align*}
which has no solution for \(c \in \opop{0}{1}\).
Although it must be said that for \(f: U \to \Reals\) where \(U \subset V\) is convex, the mean value theorem holds.

\begin{theorem} \label{th:MultivariableMVT}
    Let \(V,W\) be normed finite dimensional vector spaces and \(f: U \to W\) is differentiable and \(A,B \in U\) are such that the line connecting in completely contained in \(U\) and for each \(p\) on that line
    \begin{equation*}
        \norm{\DiffOperator \func{f}{p}} \leq M
    \end{equation*}
    then
    \begin{equation*}
        \norm{\func{f}{B} - \func{f}{A}}_W \leq M \norm{B - A}_V
    \end{equation*}
\end{theorem}
First consider the following lemma:
\begin{lemma} \label{lm:MeanValueTheoremLemma}
    If \(\phi: \clcl{0}{1} \to W\) is continuous, differentiable on \(\opop{0}{1}\) and \(\norm{\func{\phi'}{t}} \leq M\) for all \(t \in \opop{0}{1}\) then
    \begin{equation*}
        \norm{\func{\phi}{1} - \func{\phi}{0}}_W \leq M
    \end{equation*}
\end{lemma}

\begin{prooflemma}
    We provide three proofs for the lemma
    \begin{enumerate}
        \item Assuming the norm on \(W\) is induced by an inner product. Then, let \( e = \frac{\func{\phi}{1} - \func{\phi}{0}}{\norm{\func{\phi}{1} - \func{\phi}{0}}}\) be a unit vector in \(W\) then \(\psi : \clcl{0}{1} \to \Reals\), \(\func{\psi}{t} = e \cdot \func{\phi}{t}\) is continuous and differentiable on \(\opop{0}{1}\). By the mean the value theorem
              \begin{align*}
                  \abs{\func{\psi}{1} - \func{\psi}{0}}                        & = \abs{\func{\psi'}{t_0}}       \\
                  \abs{e \cdot \left( \func{\phi}{1} - \func{\phi}{0} \right)} & = \abs{e \cdot \func{\phi'}{t}} \\
                  \norm{\func{\phi}{1} - \func{\phi}{0}}                       & \leq M
              \end{align*}
        \item Using the Hahn-Banach theorem, that is for a finite dimensional vector space \(V\) and \(e \in V\) with \(\norm{e} = 1\) there exists a linear function \(\theta : V \to \Reals\) such that \(\norm{\theta} = 1\) and \(\func{\theta}{e} = 1\). Now let \(\func{\psi}{t} = \func{\theta}{\func{\phi}{t}}\) and take \(e\) as defined above then
              \begin{align*}
                  \abs{\func{\psi}{1} - \func{\psi}{0}}                & = \abs{\func{\psi'}{t_0}}                                                             \\
                  \abs{\func{\theta}{\func{\phi}{1} - \func{\phi}{0}}} & = \operatorFunc{\DiffOperator \func{\theta}{\func{\phi}{t_0}}}{\func{\phi'}{t_0}}     \\
                  \norm{\func{\phi}{1} - \func{\phi}{0}}               & = \func{\theta}{\func{\phi'}{t_0}} \leq \norm{\theta} \norm{\func{\phi'}{t_0}} \leq M
              \end{align*}
        \item From Hoimander. For any \(\epsilon\) consider the set \(T_\epsilon\).
              \begin{equation*}
                  T_\epsilon = \set[t \in \clcl{0}{1}]{\forall s, \; 0 \leq s \leq t, \; \norm{\func{\phi}{s} - \func{\phi}{0}} \leq(M + \epsilon)s + \epsilon}
              \end{equation*}
              first note that \(T_\epsilon = \clcl{0}{c}\) for some \(c > 0\) because for \(s = 0\) the inequality is strict and both sides are continuous with respect to \(s\). We claim that \(c = 1\) because otherwise \(c < 1\) and by differentiability of \(\phi\), there exists a \(\delta < 1 - c\) such that if
              \begin{align*}
                  \norm{h} < \delta \implies \norm{\func{\phi}{c + h} - \func{\phi}{c} - \operatorFunc{\DiffOperator \func{\phi}{c}}{h}} & \leq \epsilon \norm{h}                                        \\
                  \implies \norm{\func{\phi}{c + h} - \func{\phi}{c}}                                                                    & \leq\norm{h} \left( \epsilon + \norm{\func{\phi'}{c}} \right) \\
                                                                                                                                         & \leq \norm{h} (\epsilon + M)
                  \intertext{also since \(c \in T_\epsilon\)}
                  \norm{\func{\phi}{c} - \func{\phi}{0}}                                                                                 & < (M + \epsilon)c + \epsilon                                  \\
                  \implies \norm{\func{\phi}{c + h} - \func{\phi}{0}}                                                                    & < (M + \epsilon)(c + h) + \epsilon \qquad 0 < h < \delta
              \end{align*}
              hence \(c + h \in T_\epsilon\) which is a contradiction and thus \(c = 1\).
    \end{enumerate}
\end{prooflemma}

\begin{proof}
    Let \(\sigma : \clcl{0}{1} \to U\) be a parameterization of the line connecting the point \(A\) to point \(B\), \linebreak \(\func{\sigma}{t} = (1 - t) A + tB\). Let \(\phi = f \circ \sigma\), then clearly \(\phi\) is continuous on \(\clcl{0}{1}\) and differentiable on \(\opop{0}{1}\) and we have
    \begin{align*}
        \func{\phi'}{t}                 & = \operatorFunc{\DiffOperator \func{f}{\func{\sigma}{t}}}{\func{\sigma'}{t} }         \\
                                        & = \operatorFunc{\DiffOperator \func{f}{\func{\sigma}{t}}}{B - A}                      \\
        \implies \norm{\func{\phi'}{t}} & \leq \norm{\DiffOperator \func{f}{\func{\sigma}{t}}} \norm{B-A}_V \leq M \norm{B-A}_V
    \end{align*}
    therefore by the \Cref{lm:MeanValueTheoremLemma}
    \begin{equation*}
        \norm{\func{f}{B} - \func{f}{A}}_W = \norm{\func{\phi}{1} - \func{\phi}{0}}_W \leq M \norm{B-A}_V
    \end{equation*}
    which concludes the proof.
\end{proof}

\begin{corollary} \label{cr:derivativeZeroConstant}
    Let \(U \subset V\) is connected and open and \(f: U \to W\) is differentiable and \(\DiffOperator \func{f}{u} = 0\) for all \(u \in U\) then \(f\) is constant.
\end{corollary}

\begin{proof}
    Let \(p \in U\) and \( S = \set[q \in U]{\func{f}{q} = \func{f}{p}} \). \(S\) is closed because \(f\) is continuous and hence the pre-image closed set \(\set{\func{f}{p}}\) is closed. For each \(q \in S\) there exists \(r > 0\) such that \(\func{B_r}{q} \subset U\) and since \(\func{B_r}{q}\) is convex then for each \( l \in \func{B_r}{q}\) we apply the \Cref{th:MultivariableMVT}
    \begin{equation*}
        \norm{\func{f}{l} - \func{f}{q}} \leq \sup \norm{\DiffOperator \func{f}{t}} \norm{l - q} = 0
    \end{equation*}
    which implies that \(\func{f}{l} = \func{f}{q} = \func{f}{p}\) hence \(S\) is open in \(U\) which by the connectedness of \(U\) means \(S = U\). Therefore, \(f\) is constant on \(U\).
\end{proof}

\begin{corollary}
    Let \(V_1, V_2, W\) be finite dimensional normed vector space and \(U \subset V_1 \times V_2\) is open such that for every \(y \in V_2\) the intersection \((V_1 \times \set{y}) \cap U\) is connected. Assumne \(f : U \to W\) is differentiable and \(\DiffOperator_{V_1} \func{f}{x,y} = 0\) for all \((x,y) \in U\) then for any two point \((x_1,y), (x_2,y) \in U\),\(\func{f}{x_1,y} = \func{f}{x_2,y}\).
\end{corollary}

\begin{proof}
    Fix \(y \in V_2\) and define the function \(g : V_1 \to W\)
    \begin{equation*}
        \func{g}{x} = \func{f}{x,y}
    \end{equation*}
    therefore
    \begin{equation*}
        \DiffOperator \func{g}{x} = \DiffOperator_{V_1} \func{f}{x,y} = 0
    \end{equation*}
    and since \((V_1 \times \set{y}) \cap U\), the domain of \(g\) is connected. Hence by applying the \Cref{cr:derivativeZeroConstant} we get that
    \begin{equation*}
        \func{g}{x} = c \implies \func{f}{x_1,y} = \func{f}{x_2,y}
    \end{equation*}
    for all \(y \in V_2\).
\end{proof}

\section{Fundamental theorem of calculus}
\begin{theorem}
    Let \(U\) be an open set of \(V\) such that for every \(A,B \in U\) the line segment connecting \(A\) and \(B\) remains in \(U\) and let \(\sigma : \clcl{0}{1}\to U\) be that line, \(\func{\sigma}{t} = (1-t)A + tB\), and lastly let \(f: U \to W\) is continuously differentiable. Then
    \begin{equation*}
        \func{f}{B} - \func{f}{A} = \func{T}{B - A}
    \end{equation*}
    where \(T\) is
    \begin{equation*}
        T = \int_{0}^{1} \DiffOperator \func{f}{ \func{\sigma}{t}} \diffOperator t
    \end{equation*}
\end{theorem}

\begin{proof}
    Let \(g_i : \clcl{0}{1} \to \Reals\) be
    \begin{equation*}
        \func{g_i}{t} = \pi_i \circ \func{f}{\func{\sigma}{t}}
    \end{equation*}
    is continuously differentiable then by the fundamental theorem of calculus for the real-valued functions we have
    \begin{align*}
        \func{g}{1} - \func{g}{0}                                   & = \int_{0}^{1} \func{g'}{t} \diffOperator t                                                                        \\
                                                                    & = \int_{0}^{1} \pi_i \circ \DiffOperator \func{f}{\func{\sigma}{t}} \diffOperator t                                \\
                                                                    & = \pi_i \circ \int_{0}^{1} \DiffOperator \func{f}{\func{\sigma}{t}} \DiffOperator \func{\sigma}{t} \diffOperator t \\
                                                                    & = \pi_i \circ \int_{0}^{1} \DiffOperator \func{f}{\func{\sigma}{t}} (B - A) \diffOperator t                        \\
        \implies \pi_i \circ \left(\func{f}{B} - \func{f}{A}\right) & = \pi_i \circ \func{T}{B- A}                                                                                       \\
        \implies \func{f}{B} - \func{f}{A}                          & =\func{T}{B- A}
    \end{align*}
    which was what was wanted.
\end{proof}

\begin{theorem}
    Consider the continuous function \(T: U \times U \to \func{\calL}{V,W}\) which is such that
    \begin{equation*}
        \func{f}{B} - \func{f}{A} = \operatorFunc{\func{T}{A,B}}{B-A}
    \end{equation*}
    then \(f \in \calC^1\) and \(\DiffOperator \func{f}{A} = \func{T}{A,A}\)
\end{theorem}

\begin{proof}
    We have
    \begin{equation*}
        \func{f}{A + h} - \func{f}{A} = \operatorFunc{\func{T}{A+h,A}}{h}
    \end{equation*}
    hence
    \begin{align*}
        \norm{\func{f}{A + h} - \func{f}{A} - \operatorFunc{\func{T}{A,A}}{h}} & = \norm{ \operatorFunc{\func{T}{A+h,A}}{h} - \operatorFunc{\func{T}{A,A}}{h}} \\
                                                                               & \leq \norm{\func{T}{A+h,A} - \func{T}{A,A}}\norm{h}
    \end{align*}
    now by continuity of \(T\), there exists a \(\delta > 0\) such that
    \begin{equation*}
        \norm{(h,k)} < \delta \implies \norm{\func{T}{A+h,A+k} - \func{T}{A,A}} < \epsilon
    \end{equation*}
    By letting \(k = 0\) we get \(\DiffOperator \func{f}{A} = \func{T}{A,A}\). Since \(T\) is continuous then \(f \in \calC^1\) as well.
\end{proof}

\begin{corollary}
    Let \(V\) be a normed finite dimensional vector space and \(U\) is open subset of \(V\). If
    \begin{equation*}
        f :\clcl{a}{b} \times U \to \Reals
    \end{equation*}
    is continuous then
    \begin{equation*}
        \func{F}{y} = \int_{a}^{b} \func{f}{x,y} \diffOperator x
    \end{equation*}
    is continuous. Furthermore, if \(\PDiff{f}{y_i}\) exists and is continuous then \(\PDiff{F}{y_i}\) exists and is continuous as well.
    \begin{equation*}
        \PDiff{F}{y_i} =  \int_{a}^{b} \func{\PDiff{f}{y_i}}{x,y} \diffOperator x
    \end{equation*}
\end{corollary}

\begin{proof}
    Firstly, we want to show that there exists a \(\delta > 0\) such that for each \(y \in U\)
    \begin{equation*}
        \norm{h} < \delta \implies  \norm{\func{F}{y + h} - \func{F}{y}} < \epsilon
    \end{equation*}
    we have that
    \begin{align*}
        \norm{\func{F}{y + h} - \func{F}{y}} & =  \norm{\int_{a}^{b} \func{f}{x,y+ h} \func{f}{x,y} \diffOperator x}    \\
                                             & \leq (b-a) \sup_{x \in \clcl{a}{b}} \set{\func{f}{x,y+ h} \func{f}{x,y}}
    \end{align*}
    note that from the continuity of \(f\) for each \(x \in \clcl{a}{b}\) and \(y \in U\) there are open balls \(I_{x,y}\) around \(x\) and \(J_{x,y}\) around \(y\) such that
    \begin{equation*}
        x' \in I_{x,y}, \; y' \in J_{x,y} \implies \norm{\func{f}{x',y'} - \func{f}{x,y}} < \dfrac{\epsilon}{b - a}
    \end{equation*}
    Fix \(y_0\), then \(\cup I_{x,y_0} \supset \clcl{a}{b}\) which by the compactness of the interval implies that there is a finite family of there open set the covers \(\clcl{a}{b}\). Setting \(\delta\) to the minimum radius of \(J_{x,y_0}\) yields the result.
    Secondly, we show that there exists a \(\delta > 0 \) such that
    \begin{equation*}
        \abs{h} < \delta \implies \norm{\dfrac{\func{F}{y + he_i} - \func{F}{y}}{h}} < \epsilon
    \end{equation*}
    and we have that
    \begin{align*}
        \dfrac{\func{F}{y + he_i} - \func{F}{y}}{h} & = \dfrac{1}{h}  \int_{a}^{b} \func{f}{x,y+ he_i} - \func{f}{x,y} \diffOperator x   \\
                                                    & =  \dfrac{1}{h}  \int_{a}^{b} \func{\PDiff{f}{y_i}}{x,y + the_i}h  \diffOperator x \\
                                                    & = \int_{a}^{b} \func{\PDiff{f}{y_i}}{x,y + the_i}  \diffOperator x
    \end{align*}
    from the previous part we know that we can make
    \begin{equation*}
        \norm{\func{\PDiff{f}{y_i}}{x,y'} - \func{\PDiff{f}{y_i}}{x,y}}
    \end{equation*}
    as small as we want by making \(\norm{y - y'} < \delta\) small independently of \(x\). Therefore, there exist a \(\delta > 0\) such that if \(\abs{th} < \abs{h} < \delta\) then
    \begin{equation*}
        \norm{\func{\PDiff{f}{y_i}}{x,y'} - \func{\PDiff{f}{y_i}}{x,y}} < \frac{\epsilon}{b-a}
    \end{equation*}
    hence
    \begin{equation*}
        \norm{\dfrac{\func{F}{x,y + he_i} - \func{F}{x,y}}{h} - \int_{a}^{b} \func{\PDiff{f}{y_i}}{x,y} \diffOperator x} = \norm{\int_{a}^{b} \func{\PDiff{f}{y_i}}{x,y + the_i} - \func{\PDiff{f}{y_i}}{x,y}  \diffOperator x}
        < \frac{\epsilon}{b - a}
    \end{equation*}
    and the continuity of \(\PDiff{F}{y_i}\) comes as a result of applying the first part to \(\PDiff{f}{y_i}\).
\end{proof}
\section{Higher derivative}
Let \(V,W\) be finite dimensional normed vector spaces with \((e_1, \dots , e_n)\) is an ordered basis for \(V\). Consider \(U \subset V\) is an open set and \(f: U \to W\). If \(f\) is differentiable then its partial derivatives
\begin{equation*}
    \DiffOperator_i f: U \to E \quad \text{with} \quad \operatorFunc{\DiffOperator_i f}{x} = \operatorFunc{\DiffOperator \func{f}{x}}{e_i}
\end{equation*}
Then, clearly if \(\DiffOperator_i f\) is differentiable one can define its partial derivatives \(\operatorFunc{\DiffOperator_j}{\DiffOperator_i f}\) also denoted by
\begin{equation*}
    \operatorFunc{\DiffOperator_j}{\DiffOperator_i f}=  \dfrac{\PDiffOperator^2 f}{\PDiffOperator x_j \PDiffOperator x_i} =\DiffOperator_{ji} f
\end{equation*}

For Fr\'{e}chet derivative, if \(\DiffOperator f: U \to \func{\calL}{V,W}\) is differentiable at \(x\), then \(f\) is twice differentiable and
\begin{equation*}
    \DiffOperator^2 \func{f}{x} = \operatorFunc{\func{\DiffOperator \,}{\DiffOperator f}}{x} : U \xrightarrow{\text{linear map}} \func{\calL}{V,W}
\end{equation*}
is a linear map. Therefore,
\begin{equation*}
    \DiffOperator^2 f :  U \to \func{\calL}{V,\func{\calL}{V,W}}
\end{equation*}
which by the \Cref{pr:nLinearIsmorphicLinear} is equivalent to \(\func{\calL^2}{V \times V, W}\) and one can define
\begin{equation*}
    \diffOperator^2 f : U  \to \func{\calL^2}{V \times V, W}
\end{equation*}
where \(\diffOperator^2 = \func{T}{\DiffOperator^2}\) as defined in \Cref{pr:nLinearIsmorphicLinear}. With this definition, for the higher order derivatives \(n \geq 2\)
\begin{equation*}
    \diffOperator^n : U \to \func{\calL^n}{V^n, W}
\end{equation*}

\begin{example}
    Let \(A : V \to W\) be a affine function \(\func{A}{x} = Lx + b\) where \(L\) is linear. Then, \(\DiffOperator \func{A}{x} = L\) and hence \(\DiffOperator^2 A = 0\).
\end{example}

\begin{example}
    Let \(\beta : V \times V \to W\) be a bilinear function. By the Leibnitz rule
    \begin{equation*}
        \operatorFunc{\DiffOperator \func{\beta}{x_1,x_2}}{h_1,h_2} = \func{\beta}{x_1,h_2} + \func{\beta}{h_1,x_2}
    \end{equation*}
    therefore \(\DiffOperator \beta: V \times V \to \func{\calL}{V \times V, W}\) is a linear a function itself, since
    \begin{equation*}
        \operatorFunc{\DiffOperator \func{\beta}{x_1 + x'_1,x_2 + x'_2}}{h_1,h_2} = \func{\beta}{x_1,h_2} + \func{\beta}{x'_1,h_2} + \func{\beta}{x_1,h_2} + \func{\beta}{b_1,x'_2}
    \end{equation*}
    which means \(\operatorFunc{\func{\DiffOperator \,}{\DiffOperator \beta}}{x} = \DiffOperator \beta\) independent of \(x\).
\end{example}

\begin{theorem}
    If \(f\) is twice differentiable at \(p\) then its second partial derivatives exist at \(p\). Conversely, if its second partial derivatives exist at a neighbourhood of \(p\) and they are continuous, then \(f\) is differentiable.
\end{theorem}

\begin{proof}
    Assume that \(\DiffOperator^2 \func{f}{p}\) exists. Then
    \begin{align*}
        \func{\DiffOperator_j \,}{\DiffOperator_i \func{f}{p}} & = \lim_{h \to 0} \dfrac{\operatorFunc{\DiffOperator \func{f}{p + he_j}}{e_i} - \operatorFunc{\DiffOperator \func{f}{p}}{e_i} }{h} \\
                                                               & = \operatorFunc{ \lim_{h \to 0} \dfrac{\DiffOperator \func{f}{p + he_j} - \DiffOperator \func{f}{p}}{h} }{e_i}
    \end{align*}
    which exists sincce \(\DiffOperator f\) is differentiable at \(p\). Conversely, assume that the second partials exist and are continuous at \(p\). Then
    \begin{equation*}
        \operatorFunc{\func{\DiffOperator \,}{\DiffOperator f}}{p} = \begin{bmatrix}
            \func{\PDiff{\DiffOperator f}{x_1}}{p} & \dots & \func{\PDiff{\DiffOperator f}{x_n}}{p}
        \end{bmatrix}
    \end{equation*}
    note that each \(\func{\PDiff{\DiffOperator f}{x_i}}{p}\) is in \(\func{\calL}{V,W}\). In fact, since
    \begin{equation*}
        \DiffOperator f = \begin{bmatrix}
            \PDiff{f}{x_1} & \dots & \PDiff{f}{x_n}
        \end{bmatrix}
    \end{equation*}
    then
    \begin{equation*}
        \PDiff{\DiffOperator f}{x_i} =  \begin{bmatrix}
            \dfrac{\PDiffOperator^2 f}{\PDiffOperator x_i \PDiffOperator x_1} & \dots & \dfrac{\PDiffOperator^2 f}{\PDiffOperator x_i \PDiffOperator x_n}
        \end{bmatrix}
    \end{equation*}
    which is continuous at \(p\) and hence \(\func{\PDiff{\DiffOperator f}{x_i}}{p}\) is continuous and by \Cref{th:DifferentiabilityCriteria}, \(\DiffOperator f\) is differentiable at \(p\).
\end{proof}

\begin{remark}
    In general, one can show that \(f \in \calC^r\) is equivalent to its partial being in \(\calC^r\).
\end{remark}

Let \(f: U \to \Field\) then \(\DiffOperator f: U \to \func{\calL}{V,\Field}\) which is the topological dual space \(V^\ast\) therefore
\begin{equation*}
    \DiffOperator f = \sum_{i = 1}^n \PDiff{f}{x_i} e^\ast_i
\end{equation*}
then for the second derivative of \(f\), \(\DiffOperator^2 \func{f}{x} : U \to V^\ast\)
\begin{align*}
    \DiffOperator^2 \func{f}{x}                                                  & = \sum_{i = 1}^n \func{\PDiff{\DiffOperator f}{x_i}}{x} e^\ast_i                                                     \\
                                                                                 & = \sum_{i = 1}^n \func{\PDiff{\sum_{j = 1}^n \PDiff{f}{x_j}}{x_i}}{x}  e^\ast_j e^\ast_i                             \\
                                                                                 & =  \sum_{i = 1}^n \sum_{j = 1}^n \dfrac{\PDiffOperator^2 f}{\PDiffOperator x_i \PDiffOperator x_j} e^\ast_j e^\ast_i \\
    \implies \operatorFunc{\operatorFunc{\DiffOperator^2 \func{f}{x}}{e_i}}{e_j} & = \func{\dfrac{\PDiffOperator^2 f}{\PDiffOperator x_i \PDiffOperator x_j}}{x}                                        \\
    \implies \func{\diffOperator^2 \func{f}{x}}{e_i,e_j}                         & = \func{\dfrac{\PDiffOperator^2 f}{\PDiffOperator x_i \PDiffOperator x_j}}{x}
\end{align*}

\begin{definition}[Hessian matrix]
    If for a function \(f : U \to \Field\) all of its second partial derivatives exist then \textbf{hessian matrix} is
    \begin{equation*}
        \begin{bmatrix}
            \func{\dfrac{\PDiffOperator^2 f}{\PDiffOperator x_1^2}}{x}                  & \dots  & \func{\dfrac{\PDiffOperator^2 f}{\PDiffOperator x_1 \PDiffOperator x_n}}{x} \\
            \vdots                                                                      & \ddots & \vdots                                                                      \\
            \func{\dfrac{\PDiffOperator^2 f}{\PDiffOperator x_n \PDiffOperator x_1}}{x} & \dots  & \func{\dfrac{\PDiffOperator^2 f}{\PDiffOperator x_n^2}}{x}
        \end{bmatrix}
    \end{equation*}
\end{definition}

%TODO: needs better proof
\begin{theorem}
    If \(f\) is twice differentiable at \(x\), \(\diffOperator^2 \func{f}{x}\) is symmetric. That is,
    \begin{equation*}
        \func{\diffOperator^2 \func{f}{x}}{h,k} = \func{\diffOperator^2 \func{f}{x}}{k,h}
    \end{equation*}
\end{theorem}

\begin{proof}
    Let \(\norm{h}\) and \(\norm{k}\) be sufficiently small such that \(a + th, a+  tk , a+th + tk\) and the lines connecting them stays in \(U\) for some \(t \in \Reals\). Consider
    \begin{equation*}
        \func{\Delta}{t,h,k} = \func{f}{a + th + tk} - \func{f}{a + th} - \func{f}{a  + tk} + \func{f}{a}
    \end{equation*}
    Assuming \(f\) is a real-valued twice differentiable function then if we prove
    \begin{equation*}
        \operatorFunc{\diffOperator^2 \func{f}{x}}{h,k} = \lim_{t \to 0} \dfrac{\func{\Delta}{t,h,k}}{t^2}
    \end{equation*}
    we are done since, \(\Delta\) is symmetric with respect to \(h\) and \(k\). Now consider
    \begin{equation*}
        \func{g}{s} = \func{f}{a + th + tsk} - \func{f}{a  + tsk}
    \end{equation*}
    then by the the Mean value theorem
    \begin{align*}
        \func{\Delta}{t,h,k} = \func{g}{1} - \func{g}{0} & = \func{g'}{\xi}                                                                                                       \\
                                                         & =\operatorFunc{ \DiffOperator \func{f}{a + th + t\xi k}}{tk} - \operatorFunc{ \DiffOperator \func{f}{a  + t\xi k}}{tk}
    \end{align*}
    and since \(\DiffOperator f\) is differentiable then by definition
    \begin{align*}
        \implies \DiffOperator \func{f}{a+x} & = \DiffOperator \func{f}{a} + \operatorFunc{\DiffOperator^2 \func{f}{a}}{x} - \func{R}{x}
    \end{align*}
    therefore
    \begin{multline*}
        \func{\Delta}{t,h,k} = t\operatorFunc{\DiffOperator \func{f}{a } + \operatorFunc{\DiffOperator^2 \func{f}{a}}{th + t\xi k} - \func{R}{th + t\xi k}}{k} \\- t\operatorFunc{\DiffOperator \func{f}{a} + \operatorFunc{\DiffOperator^2 \func{f}{a}}{t\xi k} - \func{R}{t\xi k}}{k}
    \end{multline*}
    then
    \begin{align*}
        \func{\Delta}{t,h,k}                       & = t \operatorFunc{\operatorFunc{\DiffOperator^2 \func{f}{a}}{th + t\xi k}-\operatorFunc{\DiffOperator^2 \func{f}{a}}{t\xi k} }{k} - t \operatorFunc{ \func{R}{t\xi k} -\func{R}{th + t\xi k}}{k}               \\
                                                   & = t^2 \operatorFunc{\operatorFunc{\DiffOperator^2 \func{f}{a}}{h}}{k} - t \operatorFunc{ \func{R}{t\xi k} -\func{R}{th + t\xi k}}{k}                                                                           \\
        \implies \dfrac{\func{\Delta}{t,h,k}}{t^2} & = \operatorFunc{\operatorFunc{\DiffOperator^2 \func{f}{a}}{h}}{k} - \dfrac{ \operatorFunc{ \func{R}{t\xi k} -\func{R}{th + t\xi k}}{k}}{t} \to \operatorFunc{\operatorFunc{\DiffOperator^2 \func{f}{a}}{h}}{k}
    \end{align*}
    which is what we wanted.
\end{proof}

%TODO: needs proof
\begin{theorem}
    The \(k_\cardinalTH\) derivative of a \(k\)-times differentiable function is a symmetric \(k\)-linear function.
\end{theorem}

\begin{proof}
    it's generalization of above.
\end{proof}

\begin{proposition}
    If \(f,g \in \calC^r\) are two functions then \(f \circ g \in \calC^r\).
\end{proposition}

\begin{proof}
    Let \(f: V' \to V''\) and \(g: V \to V'\) be two \(\calC^r\) functions and  \(\beta : \func{\calL}{V',V''} \times \func{\calL}{V,V'} \to \func{\calL}{V,V''}\) is a bilinear function such that
    \begin{equation*}
        \func{\beta}{\phi,\psi} = \phi \circ \psi
    \end{equation*}
    Now note that
    \begin{align*}
        \operatorFunc{\func{\DiffOperator \,}{ f \circ g}}{a} & = \operatorFunc{\DiffOperator f \circ g}{a}\circ \DiffOperator \func{g}{a}            \\
                                                              & = \func{\beta }{\operatorFunc{\DiffOperator f \circ g}{a}, \DiffOperator \func{g}{a}}
    \end{align*}
    Consider the following functions
    \begin{equation*}
        a \xmapsto[\calC^\infty]{\Delta} (a,a) \xmapsto[\calC^{r-1}]{(\DiffOperator f \circ g, \DiffOperator g)} (\operatorFunc{\DiffOperator f \circ g}{a}, \DiffOperator \func{g}{a}) \xmapsto[\calC^\infty]{\beta}  \operatorFunc{\func{\DiffOperator \,}{ f \circ g}}{a}
    \end{equation*}
    therefore \(\func{\DiffOperator \,}{ f \circ g} \in \calC^{r-1}\) and hence \(f \circ g \in \calC^{r}\).
\end{proof}

\begin{example}
    The inverse operator \(i : \func{\GL}{V} \to \func{\calL}{V,V}\) is in \(C^\infty\). Remember that
    \begin{equation*}
        \operatorFunc{\operatorFunc{\DiffOperator i}{A}}{M} = - A^{-1}M A^{-1}
    \end{equation*}
    Let \(\gamma : \func{\calL}{V,V} \times \func{\calL}{V,V} \to \func{\calL}{\func{\calL}{V,V} , \func{\calL}{V,V}}\) with
    \begin{equation*}
        \operatorFunc{\func{\gamma}{A,B}}{M} = - AMB
    \end{equation*}
    is a bilinear function. Therefore
    \begin{equation*}
        \operatorFunc{\operatorFunc{\DiffOperator i}{A}}{M} = \operatorFunc{\func{\gamma}{A^{-1},A^{-1}}}{M}
    \end{equation*}
    now
    \begin{equation*}
        A \xmapsto{i} A^{-1} \xmapsto[\calC^\infty]{\Delta} (A^{-1}, A^{-1}) \xmapsto[\calC^\infty]{\gamma} \operatorFunc{\DiffOperator i}{A}
    \end{equation*}
    Since we have proved that \(i\) is differentiable then \(\DiffOperator i\) is differentiable which means \(i\) is twice differentiable and so on. Hence \(i \in C^{\infty}\).
\end{example}

As a matter of notation if \(\phi : V_1 \times \dots \times V_n\) be an \(n\)-linear then
\begin{equation*}
    \phi \cdot h_1\dots h_n := \func{\phi}{h_1, \dots , h_n}
\end{equation*}
particularly if \(V_1 = \dots = V_n\)
\begin{equation*}
    \phi \cdot h^n := \func{\phi}{h, \dots , h}
\end{equation*}
Now one can describe a homogeneous polynomial of degree \(k\) with a symmetric \(k\)-linear function
\begin{align*}
    \func{p}{x} = \phi \cdot x^k
\end{align*}
Then, \(\func{p}{x}\) is differentiable since
\begin{equation*}
    x \xmapsto[\calC^\infty]{\Delta} (x,\dots,x) \xmapsto[\calC^\infty]{\phi} p
\end{equation*}
and
\begin{align*}
    \operatorFunc{\DiffOperator \func{p}{x}}{h} & = \operatorFunc{\DiffOperator \func{\phi}{\func{\Delta}{x}} \circ \DiffOperator \func{\Delta}{x}}{h} \\
                                                & = \DiffOperator \phi \cdot x^n \circ \func{\Delta}{h}                                                \\
                                                & = k \phi \cdot x^{k-1}h                                                                              \\
    \implies \DiffOperator \func{p}{x}          & = k \phi \cdot x^{k-1}
\end{align*}

\begin{theorem}[Taylor approximation]
    Let \(f : U \to W\) be \(k\)-times differentiable at \(a\), then
    \begin{equation*}
        \func{p_k}{x} = \func{f}{a} + \diffOperator \func{f}{a} \cdot (x-a) + \dfrac{1}{2!} \diffOperator^2 \func{f}{a} \cdot (x-a)^2 + \dots + \dfrac{1}{k!} \diffOperator^k \func{f}{a} (x-a)^k
    \end{equation*}
    is \(k_\cardinalTH\) degree \textbf{Taylor} polynomial. Then the followings hold
    \begin{enumerate}
        \item
              \begin{equation*}
                  \lim_{x \to a} \dfrac{\func{f}{x} - \func{p_k}{x}}{\norm{x-a}^{k}} = 0
              \end{equation*}
        \item \(\func{p_k}{x}\) is the only \(k_\cardinalTH\) degree polynomial with such property.
        \item Additionally, if \(f\) is \((k + 1)\)-times differentiable in a neighbourhood of \(a\) then the remainder
              \begin{equation*}
                  \func{R}{x} = \func{f}{x} - \func{p_k}{x}
              \end{equation*}
              can be estimated with
              \begin{equation*}
                  \norm{\func{R}{b}} \leq \dfrac{1}{(k+1)!} \sup \set{\norm {\DiffOperator^{k+1} \func{f}{\xi}}} \norm{b-a}^{k+1}
              \end{equation*}
              where \(\xi\) is on line connecting \(a\) to \(b\).
    \end{enumerate}
\end{theorem}

\begin{proof} \leavevmode
    \begin{enumerate}
        \item for \(k = 1\) it is equivalent to differentiability of \(f\). By induction, assume it is true for \(k = n -1\) and let \(\func{g}{x} = \func{f}{x} - \func{p_k}{x}\) then  \footnote{Differentiablity of order \(k\) implies differentiability of order \(k-1\) in a neighbourhood. }
              \begin{align*}
                  \DiffOperator \func{g}{x}        & = \DiffOperator \func{f}{x} - \DiffOperator \func{p_k}{x}                                                                                                                                           \\
                                                   & = \diffOperator \func{f}{x} - \DiffOperator\left[\func{f}{a} + \diffOperator \func{f}{a} \cdot (x-a)  + \dots + \dfrac{1}{n!} \diffOperator^n \func{f}{a} (x-a)^n \right]                           \\
                                                   & = \diffOperator \func{f}{x} - \left[ \diffOperator \func{f}{a}  + \dfrac{1}{1!} \diffOperator^2 \func{f}{a} \cdot (x-a) + \dots + \dfrac{1}{(n-1)!} \diffOperator^n \func{f}{a} (x-a)^{n-1} \right] \\
                  \intertext{which is equivalent to the proposition at \(n-1\) for \(\diffOperator \func{f}{a}\) and hence there exists a \(\delta > 0 \) such that if \(\norm{x-a} < \delta\)}
                  \norm{\DiffOperator \func{g}{x}} & \leq \epsilon \norm{x-a}^{n-1}
              \end{align*}
              by the \Cref{th:MultivariableMVT} we have
              \begin{align*}
                  \norm{\func{g}{x}} & = \norm{\func{g}{x} - \func{g}{a}} \leq \norm{x-a} \sup \norm{\DiffOperator \func{g}{\xi}} \\
                                     & \leq \epsilon \norm{x-a} \norm{\xi - a}^{k-1}                                              \\
                                     & \leq \norm{x-a}^{k}
              \end{align*}
        \item If there were two such polynomial \(p_1, p_2\) then for \(q = p_1 - p_2\) we have that
              \begin{equation*}
                  \lim_{x \to a} \dfrac{\func{q}{x}}{\norm{x-a}^k} = 0
              \end{equation*}
              then one can show that \(\func{q}{x} \equiv 0\). %TODO: do the rest
        \item Define \(g: \clcl{0}{1} \to W\) as such
              \begin{equation*}
                  \func{g}{t} = \func{f}{a + t(b-a)}
              \end{equation*}
              therefore
              \begin{equation*}
                  \func{g^{(n)}}{t} = \diffOperator^k \func{f}{a + t(b-a)} \cdot (b-a)^k
              \end{equation*}
              For each component of \(g\) we apply the single variable Taylor's approximation
              \begin{equation*}
                  \func{g_i}{1} - \sum_{n = 0}^{k} \dfrac{\func{g^{(n)}_i}{0}}{n!} = \dfrac{\func{g^{(k+1)}_i}{\xi_i}}{(k+1)!}
              \end{equation*}
              or equivalently
              \begin{multline*}
                  \norm{\func{R}{b}} =\norm{ \func{f}{b} - \sum_{n = 0}^{k} \dfrac{\diffOperator^n \func{f}{a} \cdot (b-a)^n}{n!} } \\= \dfrac{1}{(k+1)!} \norm{ {\begin{bmatrix}
                                  \diffOperator^{k+1} \func{f_1}{a + \xi_1(b-a)} \cdot (b-a)^k & \dots & \diffOperator^{k+1} \func{f}{a + \xi_m(b-a)} \cdot (b-a)^k
                              \end{bmatrix}} } %TODO: needs clearification.
              \end{multline*}
              which was what was wanted.
    \end{enumerate}
\end{proof}

\begin{theorem}
    Let \(f : U \to \Reals\) and \(p\) is an extremum of the function then
    \begin{equation*}
        \forall h, \ \operatorFunc{\DiffOperator \func{f}{p}}{h} = 0
    \end{equation*}
\end{theorem}

\begin{proof}
    For all \(h\) define \(g_h : \opop{-\epsilon}{\epsilon} \to \Reals\)
    \begin{equation*}
        \func{g_h}{t} = \func{f}{p + th}
    \end{equation*}
    then \(\func{g'_h}{0} = 0\).
\end{proof}

%TODO: needs proof
\begin{theorem}
    Let \(f : U \to \Reals\) be of \(\calC^2\), \(p\) be a critical point of \(f\), and \(\DiffOperator^2 \func{f}{p}\) be positive definite. Then, \(p\) is a local minimum of \(f\). (If \(\DiffOperator^2 \func{f}{p}\) is negative definite then \(p\) is local maxima.)
\end{theorem}
Assuming the following lemma
\begin{lemma} \label{lm:ContinuityOfPositiveDefinite}
    If \(\DiffOperator^2 f\) is continuous and positive definite at point \(p\) then it is positive definite in a neighbourhood of \(p\).
\end{lemma}
\begin{proof}
    We wish to prove that there exists a \(\delta > 0\) for all unit vectors in \(V\), \(e\), \( 0 < t < \delta\)
    \begin{equation*}
        \func{f}{p} \leq \func{f}{p + te}
    \end{equation*}
    To do so, define \(g_e : \opop{0}{\delta} \to \Reals\)
    \begin{equation*}
        \func{g_e}{t} = \func{f}{p + te}
    \end{equation*}
    then by the Taylor's theorem
    \begin{equation*}
        \func{g_e}{t} = \func{g}{0}  + \func{g'}{0}t + \dfrac{\func{g''}{\xi}}{2!}t^2
    \end{equation*}
    where \(\xi \in \opop{0}{t}\). Equivalently
    \begin{align*}
        \func{f}{p + te} & = \func{f}{p} + \operatorFunc{\DiffOperator \func{f}{p}}{e} + \dfrac{\diffOperator^2 \func{f}{p + t\xi} \cdot e^2}{2}t^2 \\
                         & = \func{f}{p} +  \dfrac{\diffOperator^2 \func{f}{p + t\xi} \cdot e^2}{2}t^2
    \end{align*}
    Using the \Cref{lm:ContinuityOfPositiveDefinite} there exists a neighbourhood of \(p\) such that
    \begin{equation*}
        \diffOperator^2 \func{f}{p + t\xi} \cdot h^2 > 0
    \end{equation*}
    for all \(h\) in the neighbourhood. Therefore,
    \begin{equation*}
        \func{f}{p + te} > \func{f}{p}
    \end{equation*}
    which is what we wanted.
\end{proof}

\section{Smoothness Classes}

Let \(f \in \calC^r\) then one can define the norm
\begin{equation*}
    \norm{f}_r = \max \set{\sup_{x \in U}\norm{\func{f}{x}}, \dots , \sup_{x \in U}\norm{\DiffOperator^r \func{f}{x}}}
\end{equation*}
and let the set of all such \(f\) with \(\norm{f}_r < \infty\) be denoted as \(\func{\calC^r}{U,W}\).

%TODO: needs proof
\begin{theorem}
    Uniform convergence in \(\calC^r\) is equivalent to Cauchy.
\end{theorem}

\begin{proof}

\end{proof}

\begin{theorem}
    \( \func{\calC^r}{U,W}\) under \(\norm{\cdot}_r\) is a Banach space.
\end{theorem}


\begin{definition}[Local convergence]
    A functional sequence \(f_n\) is \textbf{locally convergent} if for each \(x \in U\)  there exists a open set \(x \in V \subset U\) such that \(\left. f_n \right|_V\) is uniformly convergent.
\end{definition}

\begin{theorem}
    Let \(V,W\) be normed finite dimensional spaces, \(U \subset V\) is open and connected, \(x_0 \in U\) and \(f_n : U \to W\) is a sequence of differentiable function that
    \begin{enumerate}
        \item \(\func{f_n}{x_0}\) is convergent.
        \item \(\DiffOperator f_n : U \to \func{\calL}{V,W}\) is locally convergent to some function \(g : U \to \func{\calL}{V,W}\)
    \end{enumerate}
    then the sequence \(f_n\) is locally convergent to \(f : U \to W\) and \(\DiffOperator f = g\). Furthermore, because of connectedness of \(U\) for each \(x \in U\), \(\func{f_n}{x}\) is convergent.
\end{theorem}

%TODO: do the proof
\begin{proof}
    take open ball \(W\) around \(x_0\) such that \(\DiffOperator f_n|_W\) is uniformly convergent. then prove the first statement.
    \begin{equation*}
        \norm{\func{f_m}{x} - \func{f_n}{x}} \leq \norm{\operatorFunc{f_m - f_n}{x} - \operatorFunc{f_m - f_n}{x_0}} + \norm{\func{f_m}{x_0} - \func{f_n}{x_0}}
    \end{equation*}
    apply MVT here and make the bounds smaller using (2). Then prove the differentiability with e/3. To prove (3) use open/close argument.
\end{proof}

\section{Inverse function theorem}
Consider a function \(f\), we wish to find all the solutions to the equation
\begin{equation*}
    \func{f}{x} = y_0
\end{equation*}
To do so, we can define another function \(F_{y_0}\) such that
\begin{equation*}
    \func{F_{y_0}}{x} = x - \func{f}{x} + y_0
\end{equation*}
then if \(x\) is a solution to the equation, it is a fixed point of \(F_{y_0}\).

\begin{theorem} [Banach fixed point] \label{th:BanachFixedPoint}
    Let \(\metricSpace{X}{d}\) be a complete metric space and \(f : X \to X\) is such that for some \(0 \leq \lambda < 1\)
    \begin{equation*}
        \forall x,y \in X, \ \func{d}{\func{f}{x},\func{f}{y}} \leq \lambda \func{d}{x,y}
    \end{equation*}
    Then for each \(x \in X\) the sequence \(\set{\func{f^n}{x}}\) is convergent to \(p \in X\) such that \(\func{f}{p} = p \).
\end{theorem}

\begin{proof}
    Let \(x_n = \func{f^n}{x}\) for \(n \geq 0\) then
    \begin{equation*}
        \func{d}{x_n,x_{n+1}} \leq \lambda \func{d}{x_{n-1},x_{n}} \leq \lambda^n \func{d}{x_0,x_1}
    \end{equation*}
    thereofore
    \begin{equation*}
        \func{d}{x_n,x_m} \leq \sum_{i = n}^{m-1} \lambda^i \func{d}{x_0,x_1} \leq \func{d}{x_0,x_1} \dfrac{\lambda^n}{1 - \lambda}
    \end{equation*}
    hence \(\set{\func{f^n}{x}}\) is Cauchy and it is convergent to a point \(p\). Lastly,
    \begin{align*}
        \func{d}{\func{f}{p},p} & \leq \func{d}{\func{f}{p},\func{f}{x_n}} + \func{d}{\func{f}{x_n},p} \\
                                & \leq \lambda \func{d}{p,x_n} + \func{d}{x_{n+1},p} < \epsilon
    \end{align*}
\end{proof}

\begin{theorem}[Inverse function theorem]
    Let \(V,W\) be finite dimensional normed vector space such that \(\dim V = \dim W\) and \(U \subset V\) is open. If \(f : U \to W\) is continuously differentiable and for some \(a \in U\), \(\DiffOperator \func{f}{a}\) is invertible. Then, there are open set \(S \subset V\) and \(T \subset W\)such that \(a \in S  \subset U\) and \(\func{f}{a} \in T\) such that \(f|_S\) is bijective and \((f|_S)^{-1} = g\) where \(g \in \calC^1\) and
    \begin{equation*}
        \DiffOperator \func{g}{\func{f}{x}} = \left(\DiffOperator \func{f}{x}\right)^{-1}
    \end{equation*}
\end{theorem}

\begin{proof}
    Let \(S\) be an open convex set around \(a\) such that for all \(x \in S\)
    \begin{equation*}
        \norm{\DiffOperator \func{f}{x} - \DiffOperator \func{f}{a}} < \dfrac{1}{2} \norm{\DiffOperator \func{f^{-1}}{a}}^{-1}
    \end{equation*}
    hence \(\DiffOperator \func{f}{x}\) is invertible. Let \(T = \func{f}{S}\) then we shall prove the following
    \begin{enumerate}
        \item \(f|_S\) is bijective.

              Let \(\psi : S \to V\) with
              \begin{align*}
                  \func{\psi_y}{x}                                & = x - \left(\DiffOperator \func{f}{a}\right)^{-1} (\func{f}{x} - y)                                                            \\
                  \implies \DiffOperator \func{\psi_y}{x }        & = \DSOne_V - \left(\DiffOperator \func{f}{a}\right)^{-1} \DiffOperator \func{f}{x}                                             \\
                                                                  & = \left(\DiffOperator \func{f}{a}\right)^{-1} \circ \left( \DiffOperator \func{f}{a} - \DiffOperator \func{f}{x} \right)       \\
                  \implies \norm{\DiffOperator \func{\psi_y}{x }} & \leq \norm{\left(\DiffOperator \func{f}{a}\right)^{-1}} \norm{\DiffOperator \func{f}{a} - \DiffOperator \func{f}{x}}           \\
                                                                  & < \dfrac{1}{2} [\left(\DiffOperator \func{f}{a}\right)^{-1}] [\left(\DiffOperator \func{f}{a}\right)^{-1}]^{-1} = \dfrac{1}{2}
              \end{align*}
              therefore by mean value theorem
              \begin{equation*}
                  \norm{\func{\psi_y}{x_1} - \func{\psi_y}{x_2}} \leq \dfrac{1}{2} \norm{x_1 - x_2}
              \end{equation*}
              which follows that \(\psi_y\) has at most one fixed point because
              \begin{equation*}
                  \norm{\func{\psi_y}{x_1} - \func{\psi_y}{x_2}} = \norm{x_1 - x_2} \leq  \dfrac{1}{2} \norm{x_1 - x_2}
              \end{equation*}
              is a contradiction, and for that fixed point
              \begin{equation*}
                  \func{\psi_y}{x} = x - \left(\DiffOperator \func{f}{a}\right)^{-1} (\func{f}{x} - y) = x \implies y = \func{f}{x}
              \end{equation*}
              which means \(f\) is injective. By the definition of \(T\), \(f\) is surjective as well.

        \item  \(T\) is open.

              We wish to prove that for each \(\func{f}{x_0} = y_0 \in T\) we wish to prove there exist a \( \sigma > 0\) such that \(\func{B_\sigma}{y_0}\) is contained in \(T\). In other words, \(\forall y \in \func{B_\sigma}{y_0}\)
              \begin{align*}
                  \exists x \in S, \; \func{f}{x} = y \iff \func{\psi_y}{x} = x
              \end{align*}
              To apply the contraction fixed point we must find complete metric space \(X\) such that \(\func{\psi_y}{X} = X\). Choose \(\rho\) as small as needed that \(\overline{\func{B_\rho}{x_0}} \subset S\), which makes a complete metric space. Let \(\sigma =  \dfrac{r \rho}{2}\) where \(r = \norm{\left(\DiffOperator \func{f}{a}\right)^{-1}}^{-1}\). Lastly, we show that for each \(y \in \overline{\func{B_\sigma}{y_0}}\), \(\func{\psi_y}{ \overline{\func{B_\rho}{x_0}} }=  \overline{\func{B_\rho}{x_0}}\). That is, \(x \in \overline{\func{B_\rho}{x_0}}\) implies that \( \func{\psi_y}{x} \in \overline{\func{B_\rho}{x_0}}\).
              \begin{align*}
                   & \norm{\func{\psi_y}{x}  - x_0} \leq \norm{\func{\psi_y}{x} - \func{\psi_y}{x_0}} + \norm{\func{\psi_y}{x_0} - x_0}                              \\
                   & \leq  \dfrac{1}{2} \norm{x - x_0} +  \norm{\left(\DiffOperator \func{f}{a}\right)^{-1}(y - y_0)}  \leq \dfrac{\rho}{2} + \dfrac{\rho}{2} = \rho
              \end{align*}
        \item \(g = (f|_S)^{-1} : T \to S\) is continuously differentiable. Writting the differentiability criteria
              \begin{equation*}
                  \norm{\func{g}{y + h} - \func{g}{y} - \operatorFunc{\DiffOperator \func{g}{y}}{h}} \leq \epsilon \norm{h}
              \end{equation*}
              Let \(y = \func{f}{x}\) and \(y + h = \func{f}{x + k}\) then \(h = \func{f}{x + k} - \func{f}{x}\) and note
              \begin{equation*}
                  \norm{\func{\psi_y}{x + k} - \func{\psi_y}{x}} = \norm{k - \operatorFunc{\left(\DiffOperator \func{f}{a}\right)^{-1}}{h}} \leq \dfrac{1}{2} \norm{k}
              \end{equation*}
              which implies
              \begin{equation*}
                  \dfrac{1}{2}\norm{k} \leq \norm{\operatorFunc{\left(\DiffOperator \func{f}{a}\right)^{-1}}{h}} \leq \dfrac{3}{2} \norm{k}
              \end{equation*}
              \begin{align*}
                  \norm{k - \operatorFunc{\left(\DiffOperator \func{f}{x}\right)^{-1}}{\func{f}{x + k} - \func{f}{x}}} & =                                                                                                             \norm{\operatorFunc{\left(\DiffOperator \func{f}{x}\right)^{-1}}{\operatorFunc{\DiffOperator \func{f}{x}}{k} - \func{f}{x + k} - \func{f}{x}}} \\
                                                                                                                       & \leq \norm{\left(\DiffOperator \func{f}{x}\right)^{-1}} \norm{\func{f}{x + k} - \func{f}{x} - \operatorFunc{\DiffOperator \func{f}{x}}{k}}                                                                                                                   \\
                                                                                                                       & \leq \norm{\left(\DiffOperator \func{f}{x}\right)^{-1}} \epsilon \norm{k}                                                                                                                                                                                    \\
                                                                                                                       & \leq 2 \norm{\left(\DiffOperator \func{f}{x}\right)^{-1}} \norm{\left(\DiffOperator \func{f}{a}\right)^{-1}} \epsilon
              \end{align*}
              which proves the differentiability of \(g\) as \(\norm{\left(\DiffOperator \func{f}{x}\right)^{-1}}\) is bounded in \(S\). As shown, the inverse operator is \(i\) is continuous and therefore if \(\DiffOperator f\) is continuous, then \(\func{i}{\DiffOperator f}\) is continuous. In fact, if \(f \in C^k\) then \(g \in C^k\) as well.
    \end{enumerate}
\end{proof}

\begin{corollary}
    If \(f\) is continuously differentiable and \(\DiffOperator \func{f}{x}\) is invertible for every \(x \in U\), then for any open set \(S\), \(\func{f}{S}\) is an open set as well.
\end{corollary}

\begin{proof}
    By the inverse function theorem for each \(x \in S\) there is an open set \(U_x\) in \(S\) and \(V_x\) in \(W\) such that \(\func{f}{U_x} = V_x\), therefore
    \begin{equation*}
        \func{f}{S} =  \func{f}{\bigcup U_x} = \bigcup \func{f}{U_x} = \bigcup V_x
    \end{equation*}
    which is an open set.
\end{proof}

\section{Implicit function}
\begin{theorem}
    Let \(V,W\) be finite dimensional normed vector spaces and \(U \subset V \times W\) is open. If \(f: U \to W\), \(f \in \calC^1\) where \(\func{f}{x_0,y_0} = z_0\) and \(\operatorFunc{\DiffOperator f|_{\set{x_0} \times W}}{x_0,y_0}\) is invertible then there exist open set \(S\) around \(x_0\) and \(T\) around \(y_0\) that \(S \times T \subset U\), such that for each \(x \in S\) there exists a unique \(y \in T\) with
    \begin{equation*}
        \func{f}{x,y} = z_0
    \end{equation*}
    hence there is a continuously differentiable function \(\phi: S \to T\) such that \(\func{\phi}{x} = y\) where \(\func{f}{x,y} = z_0\) and
    \begin{equation*}
        \DiffOperator \phi = - \left(\DiffOperator_y f\right)^{-1} \DiffOperator_x f
    \end{equation*}
\end{theorem}

\begin{proof}
    To apply the inverse function theorem, we need a function whose domain and range have the same dimension. So define, \(F: U \to V \times W\)
    \begin{equation*}
        \func{F}{x,y} = (x,\func{f}{x,y})
    \end{equation*}
    Then
    \begin{equation*}
        \DiffOperator \func{F}{x_0,y_0} = \left[\begin{array}{c|c}
                I_n                                                           & \DSZero                                                       \\ \hline
                \operatorFunc{\DiffOperator f|_{\set{y_0} \times U}}{x_0,y_0} & \operatorFunc{\DiffOperator f|_{\set{x_0} \times W}}{x_0,y_0}
            \end{array}\right]
    \end{equation*}
    Since \(I_n\) and \(\operatorFunc{\DiffOperator f|_{\set{x_0} \times W}}{x_0,y_0}\) are both invertible then \(\DiffOperator \func{F}{x_0,y_0}\) is invertible as well. By inverse function theorem there are open set \(\Omega_1\) around \((x_0,y_0)\) and \(\Omega_2\) around \((x_0,z_0)\) such that \(F|_{\Omega_1}\) is \(\calC^1\) diffeomorphism from \(\Omega_1\) to \(\Omega_2\). Let \(S = \set[x]{(x,z_0) \in \Omega_2}\) and the set \(T = \set[y]{(x,y) \in \Omega_1}\) , then we shall prove that for each \(x \in S\) there exists exactly one \(y \in T\) \footnote{show that they are open}. If \(x \in S\) then there exists \((x,y) \in \Omega_1\) such that \(\func{F}{x,y} = (x,z_0)\), suppose that there are two such \(y\), \(y_1\) and \(y_2\) for \(x\)
    \begin{equation*}
        \func{F}{x,y_1} = (x,\func{f}{x,y_1}) = (x,z_0) = (x,\func{f}{x,y_2}) = \func{F}{x,y_2}
    \end{equation*}
    which since \(F|_{\Omega_1}\) is injective then \(y_1 = y_2\).

    Let \(G: \Omega_2 \to \Omega_1\) be the local inverse of \(F\) and \(\phi : S \to T\)
    \begin{equation*}
        \func{\phi}{x} = \operatorFunc{ \pi_2 \circ G}{x,z_0}
    \end{equation*}
    which implies that \(\phi\) is continuously differentiable. Lastly,
    \begin{align*}
        \DiffOperator \func{f}{x,\func{\phi}{x}} & = \DiffOperator \func{f}{x,\func{\phi}{x}} \circ (I, \DiffOperator \func{\phi}{x})                                         \\
                                                 & = \DiffOperator_x \func{f}{x,\func{\phi}{x}} + \DiffOperator_y  \func{f}{x,\func{\phi}{x}}\DiffOperator \func{\phi}{x} = 0 \\
        \implies  \DiffOperator \phi             & = - \left(\DiffOperator_y f\right)^{-1} \DiffOperator_x f
    \end{align*}
\end{proof}


\section{Rank theorem}
A generalization of \(PAQ = \begin{bmatrix}
    I_r & 0 \\
    0   & 0 \\
\end{bmatrix}\)
\begin{theorem}
    Let \(f: U \to W\) be of class \(\calC^1\) and
    \begin{equation*}
        \forall x \in U, \; \rank \DiffOperator \func{f}{x} = k
    \end{equation*}
    then for each \(p \in U\) there exist open subsets \(p \in U_0\) and \(\func{f}{p} \in W_0\) and diffeomorphisms
    \begin{align*}
        \alpha & : U_0 \to U'_0 \\
        \beta  & : V_0 \to V'_0
    \end{align*}
    such that
    \begin{equation*}
        \func{ \beta \circ f \circ a^{-1} }{x_1, \dots, x_n} = (x_1, \dots , x_k, 0 , \dots, 0)
    \end{equation*}
\end{theorem}

\begin{proof}
    Without loss of generality, assume \(V = \Reals^n\), \(W = \Reals^m\), with a transition \(p = 0 \in \Reals^n\), \(\func{f}{p} = 0 \in \Reals^m\), and with a change of basis,
    \begin{equation*}
        \DiffOperator \func{f}{p} = \begin{bmatrix}
            I_k     & \DSZero \\
            \DSZero & \DSZero \\
        \end{bmatrix}
    \end{equation*}
    Suppose
    \begin{equation*}
        \func{f}{x,y} = (\func{f_1}{x,y}, \func{f_2}{x,y})
    \end{equation*}
    where \(x \in \Reals^k\), \(y \in \Reals^{n-k}\), \(\func{f_1}{x,y} \in \Reals^k\), and \(\func{f_2}{x,y} \in \Reals^{m-k}\). Then
    \begin{equation*}
        \DiffOperator \func{f}{0,0} = \begin{bmatrix}
            I_k     & \DSZero \\
            \DSZero & \DSZero
        \end{bmatrix}
    \end{equation*}

    Let \(\func{H}{x,y,u} := u - \func{f_1}{x,y}\), where \(u \in \Reals^k\). Note that \(H \in \calC^1\) and \(\DiffOperator_x H = -\DiffOperator_x f_1\) is invertible at \((0,0,0)\). Therefore, there are open set \(\Omega_1\) around \((0,0) \in \Reals^{k} \times \Reals^{n-k}\) and \(\Omega_2\) around \(0 \in \Reals^k\) such that \(H|_{\Omega_1}\) is bijective and there exists a function \(\phi: \Omega_1 \to \Omega_2\) such that
    \begin{equation*}
        x = \func{\phi}{u,y}
    \end{equation*}
    whenever \(\func{H}{x,y,u} = 0\). Then let \(\alpha^{-1} : \Omega_1 \to \Omega_2 \times \Reals^{n-k}\) be
    \begin{equation*}
        \func{\alpha^{-1}}{u,y} = (\func{\phi}{u,y},y)
    \end{equation*}
    which is a diffeomorphism around \((0,0) \in \Reals^{k} \times \Reals^{n-k}\) and
    \begin{equation*}
        \func{f \circ \alpha^{-1}}{u,y} = (\func{f_1}{\func{\phi}{u,y},y}, \func{f_2}{\func{\phi}{u,y},y}) = (u, \func{\eta}{u,y})
    \end{equation*}
    \(\DiffOperator_y \eta \equiv 0\) that is \(\func{\eta}{u,y} = \func{\eta}{u,0}\) since, \(\func{\rank}{\DiffOperator \func{f \circ \alpha^{-1}}{u,y}} = \func{\rank}{\DiffOperator \func{f}{u,y}}\) and thus
    \begin{equation*}
        \DiffOperator \func{f \circ \alpha^{-1}}{u,y} = \begin{bmatrix}
            I_k                  & \DSZero              \\
            \DiffOperator_u \eta & \DiffOperator_y \eta
        \end{bmatrix}
    \end{equation*}
    Let \(\beta : \Reals^k \times \Reals^{m-k} \to \Reals^k \times \Reals^{m-k}\)
    \begin{equation*}
        \func{\beta}{u,z} = (u,z - \func{\eta}{u,z})
    \end{equation*}
    which is clear diffeomorphism around \((0,0) \in \Reals^k \times \Reals^{m-k}\) which means that
    \begin{equation*}
        \func{ \beta \circ f \circ \alpha^{-1} }{x,y} = (x,0)
    \end{equation*}
\end{proof}

\section{Lagrange Multiplier}
\begin{theorem}
    Let \(f: U \subset \Reals^n \to \Reals\) is differentiable and \(g : U \to \Reals\) is continuously differentiable, \(S = \func{g^{-1}}{0} \subset U\) and \(\nabla \func{g}{s} \neq 0\), \(\forall s \in S\). Assume \(x_0 \in S\) 
    \begin{equation*}
        \max_{x \in S} \func{f}{x} = \func{f}{x_0}
    \end{equation*}
    then there exists \(\lambda \in \Reals\) such that 
    \begin{equation*}
        \nabla f|_{x_0} = \lambda \nabla g|_{x_0}
    \end{equation*}
\end{theorem}

\begin{proof}
    Since \(\rank \DiffOperator g = 1\) everywhere, then there are diffeomorphism \(\alpha,\beta\) such that 
    \begin{equation*}
        \func{\beta \circ g \circ \alpha^{-1}}{t_1, \dots ,t_n} = t_n
    \end{equation*}
\end{proof}

{\Large\textbf{Exercises}}
\begin{enumerate}
    \item Using l'Hopital's rule show that
          \begin{equation*}
              \lim_{t \to 0} \dfrac{\func{\Delta}{t,h,k}}{t^2} = \dfrac{\operatorFunc{\diffOperator \func{f}{a}}{h,k} + \operatorFunc{\diffOperator \func{f}{a}}{k,h}}{2}
          \end{equation*}
\end{enumerate}
\newpage

\end{document}