\documentclass[12pt]{book}
\usepackage[a4paper,bindingoffset=0.2in,%
            left=0.75in,right=0.75in,top=1in,bottom=1in,%
            footskip=.25in]{geometry}
\usepackage{fancyhdr}
\setlength{\headheight}{15.2pt}
\usepackage[utf8]{inputenc}
\pagestyle{fancy}

\renewcommand{\chaptermark}[1]{\markboth{\thechapter.\ #1}{}}
\renewcommand{\sectionmark}[1]{\markright{\thesection\ #1}}
\fancyhead[LE,RO]{\textbf{\thepage}}
\fancyhead[LO]{\textbf{\rightmark}}
\fancyhead[RE]{\textbf{\leftmark}}
\fancyfoot{}
\fancypagestyle{plain}
{
    \fancyhf{}
}

\usepackage{hyperref}
\usepackage{amsmath}
\usepackage{amssymb}
\usepackage{mathtools}
\usepackage{xcolor}
\usepackage{enumitem}
\usepackage[ruled,noline]{algorithm2e}

\usepackage{common}
\usepackage{english-theorems}
\setcounter{tocdepth}{1}

\DeclareMathOperator{\LEXP}{EXP}
\DeclareMathOperator{\LPR}{PR}

\begin{document}
\tableofcontents
\allowdisplaybreaks
\clearpage
\ifodd\value{page}\else
\thispagestyle{empty}
\fi
%\part{Automata and Computation Machines}
\chapter{Regular Languages}

\section{Finite automata}
\subsection{Formal definition}
\begin{definition}
    A \textbf{finite automaton} is a 5-tuple \((Q,\Sigma, \delta, q_0, F)\) where
    \begin{enumerate}
        \item \(Q\) is a finite set called \textbf{states}.
        \item \(\Sigma\) is a finite set called the \textbf{alphabet}.
        \item \(\delta: Q \times \Sigma \to \Sigma\) is the \textbf{transition function}.
        \item \(q_0 \in Q\) is the \textbf{start state}.
        \item \(F \subset Q\) is the \textbf{set of accept states}.
    \end{enumerate}
\end{definition}
A finite automaton machine \(M\) \textbf{accepts} a string \(w = w_1\dots w_n\) with \(w_i \in \Sigma\), when there exists states \(r_0 , \dots , r_n \in Q\) such that 
\begin{enumerate}
    \item \(r_0 = q_0\).
    \item \(r_{i+1} = \func{\delta}{r_i, w_{i+1}}\) for \(i = 0, \dots, n-1\).
    \item \(r_n \in F\).
\end{enumerate}
The set of strings that \(M\) accept is the \textbf{language of machine} \(M\), denoted by \(\func{L}{M}\). We say that \(M\) \textbf{recognizes} \(\func{L}{M}\).

\begin{definition}
    A language is called a \textbf{regular language} if some finite automaton recognizes it.
\end{definition}

\subsection{Regular operations}
\begin{definition}
    Let \(A\) and \(B\) be languages. We define the regular operations \textbf{unioun}, \textbf{concatenation}, and \textbf{star} as follows:
    \begin{description}
        \item[Union:] \(A \cup B = \set<x>{x \in A \lor x \in B}\).
        \item[Concatenation:] \(A \circ B = \set<xy>{x \in A, y \in B}\).
        \item[Star:] \(A^{\ast} = \set<x_1x_2\dots x_k>{k \geq 0, x_i \in A}\).   
    \end{description}
\end{definition}

\begin{theorem}
    The class of regular languages is closed under regular operations.
\end{theorem}

\section{Nondeterminstic finite automaton}
\subsection{Formal definition}
\begin{definition}
    A \textbf{nondeterminstic finite automaton} is a 5-tuple \((Q,\Sigma, \delta, q_0, F)\) where
    \begin{enumerate}
        \item \(Q\) is a finite set called \textbf{states}.
        \item \(\Sigma\) is a finite set called the \textbf{alphabet}.
        \item \(\delta: Q \times \Sigma_{\epsilon} \to \powerSet{Q}\) is the \textbf{transition function}.
        \item \(q_0 \in Q\) is the \textbf{start state}.
        \item \(F \subset Q\) is the \textbf{set of accept states}.
    \end{enumerate}
    where \(\Sigma_{\epsilon} = \Sigma \cup \set{\epsilon}\).
\end{definition}
A nondeterminstic finite automaton machine \(M\) \textbf{accepts} a string \(w = w_1\dots w_n\) with \(w_i \in \Sigma_{\epsilon}\), when there exists states \(r_0 , \dots , r_n \in Q\) such that 
\begin{enumerate}
    \item \(r_0 = q_0\).
    \item \(r_{i+1} \in \func{\delta}{r_i, w_{i+1}}\) for \(i = 0, \dots, n-1\).
    \item \(r_n \in F\).
\end{enumerate}

\subsection{Equivalence of NFA and DFA}
Two machine are \textbf{equivalent} if they recognize the same language.

\begin{theorem}
    Every nondeterminstic finite automaton has an equivalent deterministic finite automaton.
\end{theorem}

\begin{proof}
    Define \(\func{E}{R}\) the \textbf{epsilon neighborhood of} \(R \subset Q\) to be 
    \begin{equation*}
        \func{E}{R} = \set<q>{q \in \func{\delta}{r,\epsilon^n} \ \mathrm{for some} \ r \in R, n \geq 0}
    \end{equation*}
\end{proof}

\begin{corollary}
    A language is regular language if and only if some nondeterminstic finite automaton recognizes it.
\end{corollary}

\begin{proof}
    Add.
\end{proof}

\subsection{Closure under regular operations}

\begin{theorem}
    The class of regular languages is closed under union operation.
\end{theorem}

\begin{proof}
    Add.
\end{proof}

\begin{theorem}
    The class of regular languages is closed under concatenation operation.
\end{theorem}

\begin{proof}
    Add.
\end{proof}

\begin{theorem}
    The class of regular languages is closed under star operation operation.
\end{theorem}

\begin{proof}
    Add.
\end{proof}
\section{Regular expression}
\subsection{Formal defintion}
\begin{definition}
    \(R\) is a \textbf{regular expression} if \(R\) is 
    \begin{enumerate}
        \item \(\set{a}\) for some \(a \in \Sigma\),
        \item \(\set{\epsilon}\),
        \item \(\emptyset\),
        \item \(R_1 \cup R_2\), where \(R_1\) and \(R_2\) are regular expressions,
        \item \(R_1 \circ R_2\), where \(R_1\) and \(R_2\) are regular expressions, or
        \item \(R_1^{\ast}\), where \(R_1\) is a regular expressions,
    \end{enumerate}
\end{definition}
For convenience, we let \(R^+\) be shorthance for \(RR^{\ast}\). i.e. \(R^{\ast} = R^{+} \cup \set{\epsilon}\).
\begin{align*}
    \emptyset^{\ast} &= \set{\epsilon}& R \cup \emptyset &= R \\
    R \circ \emptyset &=  \emptyset & R \circ \set{\epsilon} &= R 
\end{align*}

\subsection{Equivalence with DFA}

\begin{theorem}
    A language is regular if and only if some regular expression describes it.
\end{theorem}

\begin{lemma}
    If a language is described by a regular expression, then it is regular.
\end{lemma}

\begin{lemma}
    If a language is regular, then it is described by a regular expression.
\end{lemma}

\begin{definition}
    A \textbf{generalized nondeterminstic finite automaton}, GNFA for short, is a 5-tuple, \((Q,\Sigma, \delta, q_{\start}, q_{\accept})\) where
    \begin{enumerate}
        \item \(Q\) is a finite set called \textbf{states}.
        \item \(\Sigma\) is a finite set called the \textbf{alphabet}.
        \item \(\delta: Q - \set{q_{\accept}}\times  Q - \set{q_{\start}} \to \calR\) where \(\calR\) is the set of regular expressions, is the \textbf{transition function}.
        \item \(q_{\start} \in Q\) is the \textbf{start state}.
        \item \(q_{\accept} \in Q\) is the \textbf{accept states}.
    \end{enumerate}
\end{definition}
A generalized nondeterminstic finite automaton machine \(M\) \textbf{accepts} a string \(w = w_1\dots w_n\) with \(w_i \in \Sigma^{\ast}\), when there exists states \(r_0 , \dots , r_n \in Q\) such that 
\begin{enumerate}
    \item \(r_0 = q_{\start}\).
    \item \(w_{i} \in \func{L}{\func{\delta}{r_{i-1}, r_{i}}}\) for \(i = 0, \dots, n-1\).
    \item \(r_n = q_{\accept}\).
\end{enumerate}

\begin{prooflemma}
    Convert DFA into a GNFA and then reduce that GNFA to a two state GNFA.
\end{prooflemma}

\section{Nonregular languages}
\begin{theorem}[Pumping lemma]
    If \(A\) is a regular language, then there is a number \(p\) where if \(s\) is any string in \(A\) of length at least \(p\), then \(s\) may be divided into three piece, \(s = xyz\), satisfying the following conditions:
    \begin{enumerate}
        \item \(\abs{y} > 0\).
        \item \(\abs{xy} \geq p\).
        \item \(xy^nz \in A\) for all \(n \geq 0\).
    \end{enumerate}
\end{theorem}

\begin{proof}
    Add.
\end{proof}
% TODO: add proofs, add examples and problems 
% TODO: Right linear and left linear grammars
\chapter{Context Free Languages}
\section{Context free languages}
\subsection{Formal definition}
\begin{definition}
    A context-free grammar is a 4-tuple \((V,\Sigma,R,S)\) where 
    \begin{enumerate}
        \item \(V\) is a finite set called the \textbf{variables}.
        \item \(\Sigma\) is a finite set disjoint from \(V\), called the \textbf{terminals}.
        \item \(R\) is a finite set of rule, with each rule being a variable and a string of variables and terminals.
        \item \(S \in V\) is the start state. 
    \end{enumerate}
\end{definition}

We say \(u\) derives \(v\), denoted by \(u \xRightarrow{\ast} v\), if \(u = v\) or if a sequence \(u_1,u_2, \dots, u_k\) exists for \(k\geq 0\) such that 
\begin{equation*}
    u \Rightarrow u_1 \Rightarrow u_2 \Rightarrow \dots \Rightarrow u_k \Rightarrow v
\end{equation*}
The language of the grammar is \(\set<w \in \Sigma^{\ast}>{S \xRightarrow{\ast} w }\).
\subsection{Designing CFL}
Suppose \(G_i\) for \(i=1,\dots,k\) are context free grammars with start state \(S_i\). Then, the union of these grammar \(G\) can be obtained by the following rule 
\begin{equation*}
    S \rightarrow S_1 | S_2 | \dots | S_k
\end{equation*}
where \(S\) is the start state of \(G\).

Moreover we can construct a context free grammar equivalent of a DFA. For each state \(q_i\), consider a variable \(R_i\). Then, \(R_i \rightarrow aR_j\) if \(\func{\delta}{q_i,a} = q_j\).

\subsection{Ambguity}
A string \(w\) is derived \textbf{ambiguously} in a context free grammar \(G\) if it has two or more distince leftmost derivation. Grammar \(G\) is \textbf{ambiguous} if it generate some strings ambiguously. In leftmost derivation, at each step the left most variable is replace.

Some grammars can only be generated by ambiguous CFLs. These are said to be inherently ambiguous. 
\subsection{Chomsky normal formal}
\begin{definition}
    A context free grammar is in \textbf{Chomsky normal form} if every rule is of the form 
    \begin{equation*}
        A \to BC \qquad \qquad A \to a
    \end{equation*}
    where \(a\) is any terminal and \(A,B,\) and \(C\) are any terminals -- except that \(B\) and \(C\) may not be the start state. In addition, we permit \(S \rightarrow \epsilon\).
\end{definition}

\begin{theorem}
    Any context free language is generated by a context free grammar in Chomsky normal form.
\end{theorem}

\begin{proof}
    Let \(u\) and \(v\) be any strings of terminals and variables.
    \begin{itemize}
        \item Add a new start variable.
        \item For \(\epsilon\)-rules, such as \(A \rightarrow \epsilon\), remove the rule and replace any \(R \rightarrow uAv\) with \(R \rightarrow uv\). If \(R \rightarrow A\), then add \(R \rightarrow \epsilon\) unless it had been previously removed. 
        \item For unit rules, such as \(A \rightarrow B\), remove the rule and replace any \(B \rightarrow u\) with \(A \rightarrow u\) unless this was a unit rule previously removed.
        \item Lastly, consider \(A \rightarrow u_1\dots u_k\) where \(u_i\) are either a variable or a terminal. We can replace this rule with the following 
        \begin{align*}
            A &\to U_1 A_1\\
            A_i &\to U_{i+1} A_{i+1}\qquad \qquad i = 1, \dots , k-3\\
            A_{k-2} &\to U_{k-1}U_{k}
        \end{align*} 
        where \(U_i = u_i\) if \(u_i\) is a variable, otherwise we must add \(U_i \rightarrow u_i\).
    \end{itemize}
\end{proof}

\section{Pushdown automata}
A pushdown automaton is a nondeterminstic finite automaton with a stack. 
\subsection{Formal definition}
\begin{definition}
    A pushdown automaton is a 6-tuple \((Q,\Sigma,\Gamma, \delta, q_0,F)\) where 
    \begin{enumerate}
        \item \(Q\) is a finite set of states.
        \item \(\Sigma\) is a fintie set of input alphabet.
        \item \(\Gamma\) is a finite set of stack alphabet.
        \item \(\delta: Q \times \Sigma_{\epsilon} \times \Gamma_{\epsilon} \to \powerSet{Q \times \Gamma_{\epsilon}}\) is the transition function.
        \item \(q_0 \in Q\) is the start state.
        \item \(F \subset Q\) is the set of accept states.
    \end{enumerate}
\end{definition}

It accepts \(w = w_1 \dots w_n\) where \(w_i \in \Sigma_{\epsilon}\) if there exists a sequence of states \(r_0, \dots , r_n \in Q\) and string \(s_0, \dots , s_n \in \Gamma^{\ast}\) exists that satisfy the following 
\begin{enumerate}
    \item \(r_0 = q_0\) and \(s_0 = \epsilon\).
    \item \((r_{i+1},b) \in \func{\delta}{r_i, w_{i+1},a}\) where \(s_i = at\) and \(s_{i+1} = bt\) for some \(a,b \in \Gamma_{\epsilon}\) and \(t \in \Gamma^{\ast}\).
    \item \(r_n \in F\).
\end{enumerate}
\subsection{Equivalence with context free grammars}
\begin{theorem}
    A language is context free if and only if some pushdown automaton recognizes it.
\end{theorem}

\begin{corollary}
    Every regular language is context free. 
\end{corollary}

\section{Non-context free languages}
\begin{theorem}[Pumping lemma for CFL]
    If \(G\) is a context free language, then there is a number \(p\) where if \(s\) is any string in \(G\) of length at least \(p\), then \(s\) may be divided into five pieces \(s =uvxyz\) satisfying 
    \begin{enumerate}
        \item \(\abs{vy} > 0\).
        \item \(\abs{vxy} \leq p\).
        \item \(uv^i x y^i z \in A\) for all \(i \geq 0\).
    \end{enumerate}   
\end{theorem}

\section{Deterministic context free languages}
\begin{definition}
    A nondeterminstic pushdown automaton is a 6-tuple \((Q,\Sigma,\Gamma,\delta,q_0,F)\) where 
    \begin{enumerate}
        \item \(Q\) is a finite set of states.
        \item \(\Sigma\) is a fintie set of input alphabet.
        \item \(\Gamma\) is a finite set of stack alphabet.
        \item \(\delta: Q \times \Sigma_{\epsilon} \times \Gamma_{\epsilon} \to Q \times \Gamma_{\epsilon} \cup \set{\emptyset}\) is the transition function.
        \item \(q_0 \in Q\) is the start state.
        \item \(F \subset Q\) is the set of accept states.
    \end{enumerate}
    and for all \(q \in Q\), \(a \in \Sigma\), and \(x \in \Gamma\) exactly one the 
    \begin{equation*}
        \func{\delta}{q,a,x} \quad \func{\delta}{q,\epsilon,x} \quad \func{\delta}{q,a,\epsilon} \quad \func{\delta}{q,\epsilon,\epsilon} 
    \end{equation*}
    is not \(\emptyset\).
\end{definition}

\begin{lemma}
    Every DPDA has an equivalent DPDA that always reads the entire string.
\end{lemma}
\subsection{Closure properties}
\begin{theorem}
    The class of DFCL is closed under complementation.
\end{theorem}

Let \(A \dashv = \set<w \dashv>{w \in A}\). 
\begin{theorem}
     \(A\) is DFCL if and only if \(A \dashv \) is DCFL.
\end{theorem}
% TODO: ambiguity, quotient of CFL with regular languages
% TODO: add proofs, add examples and problems 
% TODO: add some more about DCFC
\chapter{Church-Turing Thesis}
\section{Turing machine}
\subsection{Formal definition}
\begin{definition}
    A Turing machine is a 7-tuple \((Q,\Sigma,\Gamma,\delta,q_0,q_{acc},q_{rej})\) where \(Q,\Sigma,\Gamma\) are all finite sets 
    \begin{enumerate}
        \item \(Q\) is the set of states.
        \item \(\Sigma\) is the input alphabet not containing the blank symbol \(\blankSymbol\).
        \item \(\Gamma\) is the tape alphabet, where \(\blankSymbol \in \Gamma\) and \(\Sigma \subset \Gamma\).
        \item \(\delta: Q \times \Gamma \to Q \times \Gamma \times \set{L,R}\) is the transition function. 
        \item \(q_0 \in Q\) is the start state.
        \item \(q_{acc} \in Q\) is the accept state. 
        \item \(q_{rej} \in Q\) is the reject state and \(q_{rej} \neq q_{acc}\).
    \end{enumerate}
\end{definition}
Turing machine \(M = (Q,\Sigma,\Gamma,\delta,q_0,q_{acc},q_{rej})\) receives input \(w = w_1 \dots w_n \in \Sigma^{\ast}\) on the leftmost \(n\) squares of the tape. And initially the rest of the tape is blank. The content tape together with the current state and head's position is called the \textbf{configuration}. A configuration may be denoted as \(uqv\) where \(uv\) is the content of the tape, \(q\) is the current state, and the head is on the first symbol of \(v\). For example, \(q_0w\) is the start configuration and \(q_{acc}\) and \(q_{rej}\) are the accepting and rejecting configuration respectively. The accepting configuration and rejecting configuration are called the \textbf{halting configuration}.

Suppose \(C_1\) and \(C_2\) are two configuration. \(C_1\) \textbf{yields} \(C_2\) if the Turing machine can legally go from \(C_1\) to \(C_2\) in a single step. 

A Turing machine accepts input \(w\) if a sequence of configuration \(C_1, \dots , C_k\) exists where 
\begin{enumerate}
    \item \(C_1\) is the start configuration of \(M\) on input \(w\).
    \item Each \(C_i\) yields \(C_{i + 1}\).
    \item \(C_k\) is an accepting configuration.
\end{enumerate}
The collection of strings that \(M\) accepts is the language of \(M\).
\begin{definition}
    A language is \textbf{Turing recognizable} or \textbf{recursively enumeratable} if some Turing machine recognizes it.

    \textbf{Deciders} are Turing machines that halt on all iputs. A decider that recognizes some language is also said to decide that language.

    A language is \textbf{Turing decidable} or \textbf{recursive} if some Turing machine decides it. 
\end{definition}
%TODO: difference between a decider and TM
\subsection{Examples}
%TODO: give some examples
\section{Variants of Turing machine}
\subsection{Stay put}
Any Turing machine with stay put order \(S\) can be emulated by a Turing machine without.
\subsection{Multipath Turing machine}
each tape has its own head for reading and writing. 
\begin{equation*}
    \delta: Q \times \Gamma^k \to  Q \times \Gamma^k \times \set{L,R,S}^l
\end{equation*}
\begin{theorem}
    Every Multipath Turing machine has an equivalent single Turing machine.
\end{theorem}
\begin{corollary}
    A language is Turing recognizable if and only if some multipath Turing machine recognizes it. 
\end{corollary}

\subsection{Nondeterministic Turing machine}
\begin{equation*}
    \delta: Q \times \Gamma \to \powerSet{Q \times \Gamma \times \set{L,R}}
\end{equation*}
\begin{theorem}
    Every nondeterministic Turing machine has an equivalent determinstic Turing machine.
\end{theorem}
\begin{corollary}
    A language is Turing recognizable if and only if some nondeterministic Turing machine recognizes it. 
\end{corollary}
\subsection{Enumerators}
%TODO: definitions
\begin{theorem}
    Every enumerator has an equivalent determinstic Turing machine.
\end{theorem}
\begin{corollary}
    A language is Turing recognizable if and only if some enumerator recognizes it. 
\end{corollary}

\section{The definition of algorithm}
\subsection{Hilber's \(10_{\cardinalTH}\) problem}
Provide an algorithm for determining if a polynomial has an integral root. This is equivalent to 
\begin{equation*}
    D = \set<p>{p\ \text{is a polynomial with an integral root}}
\end{equation*}
being decidable which is not. Although, it is Turing recognizable.
% TODO: add proofs, add examples and problems 

\part{Logic}
\chapter{Introduction}
some stuff on propositional logic, induction, well-formed formula. -- need to be completed.
For all proposition \(A\) we can define the set of all its sub-proposition, \(\subprop {A}\), and it can be defined inductively. Order of operation. A meaning is a function \(I : \mathrm{PR} \to \set{0,1}\) such that;
\begin{enumerate}
    \item \(\func{I}{\perp} = 0 \).
    \item \(\func{I}{A \land B} = \func{I}{A} \func{I}{B}\).
    \item \(\func{I}{\neg A} = 1 - \func{I}{A}\). (a book on negation)
    \item \(\func{I}{A \lor B} = \max \set{\func{I}{A}, \func{I}{B}}\).
    \item \(\func{I}{A \to B} = \func{I}{\neg A \lor B}\). A point of contention among logician.
\end{enumerate}
An evaluation is a meaning function restricted to the atomes, \(\nu : P \to \set{0,1}\).

\begin{theorem}
    For each evaluation function there is unique extension to a meaning function.
\end{theorem}

\(I \models A\) if \(\func{I}{A} = 1\). \(\models A\) means \(I \models A\) for all \(I\), a tautology. \(\not\models A\) if \(\func{I}{A} = 0\) for all \(I\). If \(\Gamma\) is a subset of proposition then, \(\Gamma \models A\) when for all \(I \models \Gamma\) then \(I \models A\).

some propositions regarding meaning and evaluation.

\(A[P|b]\)
 substituition  theorem.
 \section{Inference rules}
 -- Hilbert's method
 \subsection{Natural Deduction}
 Everything is a rule. Two types of rules, introduction rules and elimination rules.
\part{Computability Theory}
\chapter{Lecture Notes}
\section{Computation}
Abstractly, a \textit{computation} is the ``work done'' by a \textit{computation machine}. For example, we can consider our brains as computation machines and whatever they do is considered a computation. A computation machine \(M\) can be viewed as a black box that takes an input \(w\) and acts on the input in a step by a step manner. Suppose a brain is tasked with calculating a formula. This formula is the input and at each step the brain may calculate or simplify a certain part of the formula. 

A \textit{configuration} is a description of the internals and the state of the machine. Note that the state of the machine may be described in many ways, however, we usually fix a method of describing so that the resulting configurations are unique. A configuration of a brain computing a formula might be the state of neurons in the brain and the steps of the calculation as written in a piece of paper.

In each unit of time the configuration of the machine changes based on its \textit{configuration graph}. This is like saying that in each time unit the brain does an algebraic step such as addition or multiplication and then changes the content of the paper to the new intermediate result. This graph is a directed graph with all the possible configurations of \(M\) as nodes. There is an edges from configuration \(C\) to \(C'\) if the machine go from \(C\) to \(C'\) in one unit of time. In our case, the brain can go from \(C\) to \(C'\) if the corresponding formula of \(C'\) can be arrived by doing one algebraic step from the corresponding formula of \(C\). This graph might be infinite, and in fact, finite configuration graphs can be dealt with tools from graph theory and thus do not provide a rich theory of computation.

In this way, we can formalize the notion of computation machines as a discrete dynamical system. 

\begin{definition}
    A discrete dynamical system is a system whose state varies with a discrete variable, time, according to a directed graph \(G = (\calV,\calE)\) such that \(\calV\), the set of states of the system, is a discrete set. The set \(\calE\) determines how the state of the system varies with time. Let \(V_t \in \calV\) be the state of the system at time \(t\), then the state of the system a unit of time later, \(V_{t+1}\), must be a neighbor of \(V_t\). That is,
    \begin{equation*}
        V_{t+1} \in \func{N}{V_t} = \set<V \in \calV>{(V_t, V) \in \calE}
    \end{equation*}
\end{definition}

\begin{remark}
    In general, \(\func{N}{V_t}\) might have zero, one, or more than one states. When \(\func{N}{V_t}\) then the system can not progess and it stops. When \(V_t\) has exactly one neighbouring state, the next state is uniquely determined. However, when it has more than one neighbouring states, then the next state is selected non-determinstically. Such selections are not based on some probability distribution or any other rules. 
\end{remark}

\begin{definition}
    A \textbf{computation machine} is a discrete dynamical system \(M = (\calC,\calE)\). The state \(C \in \calC\) is called a \textbf{configuration} of the machine and \(E \in \calE\) is called a \textbf{transition}. On an input \(w\), the machine will be placed in an \textit{initial configuration} \(C_0^w \in \calC\). At each unit of time, a neighbouring configuration is chosen if it exists. If no neighbouring configurations exists, then the machine stops. A computation machine \(M\) is deterministic for any \(C \in \calC\) there is at most one neighbour.
\end{definition}

The possible configurations that may be attained by an input \(w\), form a tree. If the computation is deterministic, then the tree is a path. 

\begin{remark}
    In computation, we usually distinguish between halting and crashing. Loosely defined, halting is an expected stopping and crashing is an unexpected stopping. To distinguish between these two scenarios, we may consider \(\calC_h \subset \calC\) as the set of halting configuration and add it to the description of \(M\). Therefore, if a non-halting configuration has no neighbour, the computation machine crahses.
\end{remark}

When viewing our current definition of the computation machines as computer programs, we see certain similarities. For example, a computer program is a set of variables and instructions. At each step, an instruction is executed which will change the value of the some the variables-- e.g. the instruction pointer. The input of the program is considered as one of the variables. Then, the values of the variables is the configuration of the program. The instruction are the transition between different configurations.

However, there is a major difference between computer programs and computation machines. Computer programs are finite, however, computation machines are not. As stated above, the set of configurations of any useful machine is infinite. As a result, the set of transitions is also infinite. On the other hand, a practical theory of computation requires a certain finite limitation. This finite limitation is placed on the description of the computation machine. That is, there exists a finite set symbols such that the computation machines can be described by a finite string of those symbols.

Therefore, if we can describe a machine by a string, it makes sense to assume that its configurations and transitions correspond to some strings. As a result, a computation machine can also be viewed as a string processor. We want our string processor to have a finite description.

A string processor's configurations are strings and its transition describe how these strings change. There are infinitely many possible strings. Thus, the challange is to give the correct result/appropriate action for infinitely many input by a finitely describable entity. Hence, finite description is the same as saying that configuration graph can be constructed by finitely many rules. Finitely many rules implies that configurations change locally. Local property is necessary for finite description. Basic operations are changes in bounded number of memory.

\subsection{Problems}
A problem is a query on some given data and constants. To answer the query by a computation machine, we need to encode the data of the problem into some string. This string is then given to the computation machine. The computation machine takes into consideration the constants and query of the problem so that on any valid input it returns the answer to the query. 

The coding of the problem must not have any pre-processing. We enforce this by making the coding essentially trivial. The reason being that different coding may result in better or faster computation.


\section{Formal Languages}
\begin{itemize}
    \item Suppose we want to compute a function \(f :\Sigma^{\ast} \to \Pi^{\ast}\) with yes/no answer. That is, \(\forall w \in \Sigma^{\ast},\; \func{f}{w} \in \set{0,1}\). Then, computing \(\func{f}{w}\) is equivalent to determining \(w \in L_f\) where \(L_f = \set<w>{\func{f}{w} = 1}\).
    \item Generally a problem type is described by a triple of constants, given data, and a query. Yes/no problems have queries which are either true or false.
    \item To have a well-defined yes/no problem, the set \(L\) must have a finite description.
\end{itemize}
\subsection{Finite description of formal languages}
\begin{enumerate}
    \item Logical: \(L = \set<x>{\func{P}{x}}\) for some proposition \(P\) on strings.
    \item Grammar: \(L = \angleBracket{A,\tau_1, \dots, \tau_n}\) where \(A\) is a set of axioms and \(\tau_i\) are the rules that allow us to combine the axoim, inference rules.
    \item Computation Machine: \(L = \) is the language of a computation machine, a program in a compiler.
    \item Enumerator: \(L = \) is the language of a an enumerator, a generator in a compiler.
    \item Formal Game: \(L = \).?
    \item Transducer?
\end{enumerate}
In general, the classes of languages that can be described by several of the above's methods are interesting.

A definition is given by specifying a type, the data, and the properties.

\section{Computation Machines}

Computation is a finite description for the formal language and an evaluation process.
Types of machines:
\begin{itemize}
    \item Deterministic vs Nondetermninistic. Technically, any deterministic machine is also nondetermninistic.
\end{itemize}
Machines may have some halting configurations. Based on that we can convene the certain halting configuration are accepting configuration and others are rejecting configuration. Based on that we can define certain languages for the machines, for example, the input that are accepted. Then, we have two further classifications.
\begin{itemize}
    \item Acceptors vs Deciders.
\end{itemize}

Therefore, a computation is described by its finite description and its standard language.
Deciders are closed under inversion of output.

The main problem in theory of computation is determining the relationships of the subclasses of a model. For example, whether DetDec are a proper subset of DetAcc or are they equal. 


\subsection{Finite automaton}
\subsection{Pushdown automaton}
\subsection{Turing machine}
\subsection{Enumerators}
\section{Complexity of the languages}
We want to measure the complexity of the problems wrt some model. One way is to reduce the problems to each other.

many-to-one reduction. \(B\) must be serializable.
hardness and completeness.
isomorphism conjecture. Berman-Hartmanis conjecture
\section{Gap problems}
Gap problems between NATM and DATM: No.
Gap problems between DATM and DDTM: No.
Gap problems between NATM and DDTM: Yes. G\"{o}del's theorem.

Halting problem. Complete problem.

\begin{enumerate}
    \item Fix a coding for a TMs in \(\set{0,1}\).
    \item It is like numbering them.
    \item Determining if a number is TM are not is a simple syntax checking.
    \item Construct a language that does not have a DTM. Done by the Cantor's diagonalization.
    \item If lucky, the language has a acceptor.
\end{enumerate}

\section{Constraining Turing machines}
\subsection{Time constraints}
redefine the TM as binary and single taped. define a the TIME as the length of the computation. Define \(\func{f}{n}\)-boundedness of a machine. Now limit by choosing \(f\). 

Write the G\"{o}del's proof for \(P\) vs \(NP\).

\(NP \subset DDTM\).

Combination of two poly TM is another poly TM.

\(NP\) is like having a poly certifier.

\subsection{Space constraints}
redefine the TM as 3-taped: input RO, worktap RW, output WO. Argue that this does not affect our computation capability and results in the same class of languages (almost).

LBAs and the context sensitive grammers.


\part{Advanced Algorithms}

\chapter{Introduction}
The topics are advanced data structures -- Heap, Splay tree, suffix tree and suffix array-- and advanced algorithms -- string mathcing and network flow. There is also an overview of complexity classes, reducibility, and completeness. Approximation and randomized algorithms are also considered.

\section{Computation Variants}
Computation models include main memory, external memory, and streaming memory. Consider the sequence \(x=x_1x_2\dots x_n\) where each \(x_i \in \calU = \Naturals_m\), count the number of unique elements in \(x\), \(\func{D}{x}\).

\begin{enumerate}
    \item naive approach.
    \item hash approach.
    \item Sample based approach. relative error \(\sqrt{\dfrac{n-r}{2r} \ln \dfrac{1}{\delta}}\). proof is wtf
    \item Streaming model. deterministic memory \(M = \bigOmega{n \log m}\).
    \item Randomized approximation. with a factor of \(1+\epsilon\) and with probability of \(\delta\), the order of some shit is \(\bigO{\log \dfrac{n}{\epsilon^2} \log \dfrac{1}{\delta}}\).
    \item 
\end{enumerate}

-- Distinct sampling.
\chapter{Amortized Analysis}
\section{Table insertion}
some stuff about potential
with deletion we could have an obvious strategy and a simple startegy.

% \part{Complexity Theory}
\chapter{Time Complexity}
\section{Defintion of time}
One of the most important resource in computation is time. Hence it is natural to quantify the time required to compute a given task with a computing model. Consider a computation model \(\calM\), and let \(\func{\DD}{\calM}, \func{\DA}{\calM},\func{\ND}{\calM},\)  and \(\func{\NA}{\calM}\) be its four fundamental classes based on its standard language. Fix a machine \(M \in \calM\) and an input \(w \in \Sigma^{\ast}\). Given a computation path \(\calP_w\) on \(M\),  the most obvious measure of time is the length of the path.
\begin{equation*}
    \func{\Time_M}{\calP_w} = \abs{\calP_w}
\end{equation*} 
Then, we naturallly define the computation time of \(w\) on \(M\) to be the minimum time of accepting path.
\begin{equation*}
    \func{\Time_M}{w} = \min_{\substack{\calP_w \text{ ends in}\\\text{an accepting state}}}\func{\Time_M}{\calP_w} 
\end{equation*}
We set \(\min \emptyset = \infty\), thus when \(w \in \func{L}{M}\), \(\func{\Time_M}{w} = \infty\). So far, we have a function \(\Time_M : \Sigma^{\ast} \to \Naturals \cup \set{\infty}\) that measures the time complexity of a string \(w\) on \(M\). The goal developing such function is to then compare the efficiency two machines \(M_1\) and \(M_2\) that compute the same task \(L\). Ofcourse, this is usually an impossible task. Firstly, computing \(\func{\Time_M}{w}\) itself is a computationally challanging endeavor. Even, when the exact time measure is computed, it is not obvious how to compare two functions on strings over \(\Sigma\). For example, we can easily construct a `better' algorithm by hardcoding the acceptance of a string \(w\), however, such the new algorithm does not inherently improve the initial algorithm. 

For these reason, and much more, we shall develop our theory through a coarser lens. Our intuition is that a better algorithm works `better' on `more complicated' inputs. We now formalize our intuition.

\begin{enumerate}
    \item What are the `complicated' inputs? Usually, but ofcourse not always, longer inputs are harder to solve. For example, consider the following linear systems. 
    \begin{align*}
        &\begin{cases}
          2x + y =& 1\\
          3x - 2y =& 2  
        \end{cases}\ (1) & 
        &\begin{cases}
            x + y + z =& 1\\
            x - 2y + z =&  2\\
            3x - y - z =&  3
          \end{cases}\ (2)
        &\begin{cases}
            x  =& 1\\
            y =&  2\\
            z =&  3 \\
            w =& 4
          \end{cases}\ (3)
    \end{align*}
    In a glance, we can see that system \((3)\) is the easiest to solve, then \((1)\), and lastly \((2)\). Even though, generally, a \(4\)-variable linear system is harder than a \(2\)-variable linear system. System \((1)\) is described with \(13\) characters while system \((3)\) is described with \(12\) characters. Thus, it is not hard to argue that shorter inputs are easier to so solve and thus less complicated that longer inputs. 

    By the above reasoning, we partition the time measure over the length of the input, which gives the following defintion for \(\Time_M : \Naturals \to \Naturals \cup \set{\infty}\).
    \begin{equation*}
        \func{\Time_M}{n} = \max_{\substack{\abs{w} = n \\ w \in \func{L}{M}}} \func{\Time_M}{w}
    \end{equation*}
    \item Suppose we are given two functions \(\Time_{M_1}\) and \(\Time_{M_2}\), how do we compare them over \(\Naturals\). Our intuition tells us that \(\Time_{M_1}\) is better than \(\Time_{M_2}\) if it is smaller for larger inputs. That is, for sufficiently large \(n\), \(\func{\Time_{M_1}}{n} \leq \func{\Time_{M_2}}{n}\). This is closely related to the idea of asymptotic growth. Let \(f : \Naturals \to \Naturals\), then 
    \begin{itemize}
        \item \(\bigO{f} = \set<g : \Naturals \to \Naturals>{ \exists N,c \in \Naturals\; \suchThat \; \forall n \geq N, \; \func{g}{n} \leq c \func{f}{n}}\). When \(g \in \bigO{f}\) we say that \(g\) is asymptotically bounded above by \(f\).
        \item \(\bigOmega{f} = \set<g : \Naturals \to \Naturals>{ \exists N,c \in \Naturals\; \suchThat \; \forall n \geq N, \; \func{g}{n} \geq c \func{f}{n}}\). When \(g \in \bigOmega{f}\) we say that \(g\) is asymptotically bounded below by \(f\).
        \item  \(\bigTheta{f} = \set<g : \Naturals \to \Naturals>{ \exists N,c,d \in \Naturals\; \suchThat \; \forall n \geq N, \; d \func{f}{n} \leq \func{g}{n} \leq c \func{f}{n}}\). When \(g \in \bigTheta{f}\) we say that \(g\) is asymptotically bounded by \(f\).
        \item \(\littleO{f} = \set<g : \Naturals \to \Naturals>{ \forall \epsilon > 0, \exists N \in \Naturals\; \suchThat \; \forall n \geq N, \; \func{g}{n} \leq \epsilon \func{f}{n}}\). When \(g \in \littleO{f}\) we say that \(g\) is asymptotically dominated above by \(f\).
        \item  \(\littleOmega{f} = \set<g : \Naturals \to \Naturals>{ \forall k > 0, \exists N \in \Naturals\; \suchThat \; \forall n \geq N, \;  \func{g}{n} \geq k \func{f}{n}}\). When \(g \in \littleOmega{f}\) we say that \(g\) is asymptotically dominated above by \(f\).
    \end{itemize}
    In our treatment of time, we typically use the big-\(O\) notation, since when an algorithm with an efficient upperbound can be considered efficient, even though, its exact time function might not be efficient. 

    For practical and historical reasons, the class of polynomial functions are considered efficient. Thus, an algorithm that has polynomially bounded above is consider efficient.
\end{enumerate}

\subsection{Accptors vs Deciders}
So far we have considered \(\Time_M\) for when \(M\) is an acceptor. Ofcourse, this is fine since every decider is an acceptor. However, when we \(M\) a decider, then the reject states shall also be considered. One way to amend the definition of time for deciders is 
\begin{align*}
    \func{\Time_M}{w} &= \min_{\substack{\calP_w \text{ ends in}\\\text{a halting state}}}\func{\Time_M}{\calP_w} \\
    \func{\Time_M}{n} &= \max_{\abs{w} = n } \func{\Time_M}{w}
\end{align*}
We can note that, when a decider \(M\) is viewed as an acceptor, its time complexity is lower than we it is viewed as a decider. This is because, it might be the case that rejecting is computationally more expensive than just accepting. For example, consider the following language,
\begin{equation*}
    L = \set<1^n>{n \text{ is composite}.}
\end{equation*}
and the algorithm \(M\) that does the following.
\begin{enumerate}
    \item Set \(m = 2\)
    \item  Check if \(m\) divides \(n\) and it does accept immediately. Otherwise increment \(m\) by one and repeat.
    \item When \(m = n\) reject.
\end{enumerate}
In this machine, we only reject when we can not accept. Moreover, rejecting requires us to divide \(n\) by all the numbers between \(2\) and \(n-1\), whereas, to accept \(n\) we only divide until we find the first smallest prime divisor of \(n\) which is substantially smaller than \(n\).

One way to reconcile these two definitions is to argue that an acceptor with a time function is equivalent to a decider. Suppose an acceptor \(M\) has an asymptotic upperbound \(f\) on its time complexity. Then, we can construct another machine that first computes \(\func{f}{n}\), then works exactly like \(M\) for \(\func{f}{n}\) clocks. Although this construction might have signficant overheads, we have other constructions -- discussed in the next section-- with negligible overhead which does not affect the time complexity of the machine. However, even with in those construction, the new machine is not a decider since \(M\) may reject even for strings in its language. 

As a result, the time definitions for acceptors and deciders seem to be irreconcilable. 
\section{Simulation}
Throughout this section we argue that the standard Turing machine is good model for computation as it can efficiently simulate all other Turing machine models. 
\begin{definition}
    A single-tape Turing machine \(M = (\Sigma = \set{0,1},\Gamma=\set{0,1,\blankSymbol},\delta,q_0,q_{\acc}, q_{\rej})\) is called an standard Turing machine. We denote the class of all standard Turing machines by \(\calM\).
\end{definition}

\begin{theorem}[Alphabet expansion]
    Suppose \(M\) is a single-tape binary Turing machine with tape alphabet \(\Gamma\) and time complexity \(\bigO{\func{T}{n}}\). There exists a standard Turing machine that accepts the same language in time complexity of \(\bigO{n^2 + \func{T}{n}}\).
\end{theorem}

\begin{proof}
    Consider a coding \(\sigma : \Gamma \to \set{0,1}^{\ast}\) such that for all \(x \in \Gamma\), \(\func{\sigma}{x}\) and has length \(\lg \abs{\Gamma}\). Consider a standard Turing machine \(N\) that does the following.
    \begin{enumerate}
        \item First \(N\) computes the coding of the input such that the content of tape becomes \(w \blankSymbol \func{\sigma}{w}\) with the head of machine being on the top of the first symbol of \(\func{\sigma}{w}\). This steps takes \(\bigO{n^2}\) steps.
        \item For each step of \(M\), \(N\) must read \(\lg \abs{\Gamma}\) characters, write another \(\lg \abs{\Gamma}\) characters, and then move to left or right. Suppose \(N\)'s head is always at the leftmost character of the code. Thus, it read \(\lg \abs{\Gamma}\) and moves to rightmost character of the code. Then, it writes from there (backwards) in \(\lg \abs{\Gamma}\) steps. Then, to move to left or right it must move its head another \(\lg \abs{\Gamma}\) steps. In total, each step on \(M\), takes \(3 \lg \abs{\Gamma} = \bigO{1}\) steps in \(N\).
    \end{enumerate}
    Therefore, \(N\) simulates \(M\) in \(\bigO{n^2 + \func{T}{n}}\).
\end{proof}

\begin{remark}
    Note that, by the above construction we can show that there exists a standard Turing mahcine that accepts \(\func{\sigma}{L}\) in \(\bigO{\func{T}{n}}\).
\end{remark}

\subsection{Universal machine}
Turing machines are assumed to be the ultimate computing machines, as it seems that every algorithm can be implemented on Turing machine. Ofcourse, we can think of an algorithm that takes in a description of another algorithm with an input and runs that algorithm on the input. For example, interpreters are programs that take in a code with an input and execute the code with the input. Thus, is there a Turing machine that can simulate all other Turing machines? Consider an encoding of Turing machines \(T\) into binary strings \(\angleBracket{T}\), then we are asking if \(\univ\) as defined below is Turing recognizable.
\begin{equation*}
    \univ = \set< \bracket{\angleBracket{T},w}>{ T \text{ is a standart Turing machine and it accepts } w \in \set{0,1}^{\ast}}
\end{equation*}
For the sake of simplicity, suppose \(\bracket{\angleBracket{T},w} = \angleBracket{T} \# w\) is given as input. In essence, the Turing machine \(U\) for \(\univ\) does the following.

\begin{enumerate}
    \item \(U\) goes over the input until it reaches \(\#\) and then copies down the rest on the second tape. Thus, the content of second tape is \(\#w\).
    \item Suppose the encoding is such that the binary of representation of \(q_0,q_{\acc}, q_{\rej}\) are easily (construct such a representation as an exercise). Copy the binary representation of \(q_0\) on the third tape. 
    \item Based on the content of second tape which corresponds to the tape of \(T\) and the third tape which is the state of \(T\), find the correct action from the description of \(T\) -- use nondeterminism. 
    \item \(T\) halts whenever its state is \(q_{\acc}\) or \(q_{\rej}\). After each move, check if the state of the third is equal to either halting state or not.
\end{enumerate}
Is this an efficient universal machine?

\begin{enumerate}
    \item The first step takes \(\bigO{\angleBracket{T} + \abs{w}}\).
    \item The second step takes \(\bigO{\angleBracket{T}}\) as only need to go over the representation of \(T\) and copy \(q_0\).
    \item Each time we execute the third step, we read the character after \(\#\) on the second tape, the state on third tape, and nondeterministically find corresponding move on the representation of \(T\). Then, we just apply the transition on the second and third tape. This is done in the order of \(\bigO{\angleBracket{T}}\). 
    \item We do third step until \(T\) halts on \(w\). If \(T\) halts in \(t\) steps on \(w\), then the overall time complexity is 
    \begin{equation*}
        \bigO{t \angleBracket{T} + \abs{w} } = \bigO{t + \abs{w}}
    \end{equation*}
\end{enumerate}

\section{Time constructiblity}
When designing a Turing machine with some time constraint \(\func{t}{n}\), we ought to make sure the Turing machine halts after \(\func{t}{n}\) or \(\bigO{\func{t}{n}}\) clocks. 

The first way to achieve this is to devise a Turing machine that given \(1^n\) runs for \(\func{t}{n}\) steps and then halts. We can then equip our original Turing machine with this timer so that it halts in \(\func{t}{n}\) clocks.
\begin{definition}
    A \(t\)-timer is a multi-tape Turing machine that given any \(w\) with lenght \(n\), halts exactly after \(\func{t}{n}\) clocks. A function \(t:\Naturals \to \Naturals\) is said to be time constructible if there exists a \(t\)-timer. 
\end{definition}

\begin{example}
    The function \(\func{t}{n} = n\) is time constructible. The machine simply reads the input until it reaches a blank symbol and then halts.
    \begin{center}
        \begin{tikzpicture}
            \node[state, initial,initial text = ] (q0) {$q_0$};
            \draw
            (q0) edge[loop above] node{$ 1 \to 1, R $} (q0);
        \end{tikzpicture}
    \end{center}
\end{example}

\begin{example}
    The function \(\func{t}{n} = n^2\) is time constructible. The machine starts with an empty second tape and works as follows.
    \begin{center}
        \begin{tikzpicture}
            \node[state, initial,initial text = ] (q0) {$q_0$};
            \node[state] (q1) [right of = q0] {$q_1$};
            \draw
            (q0) edge[loop above] node{$\substack{1 \to 1, S \\ 1 \to 1, L}$} (q0)
            (q0) edge[bend left] node[above]{$\substack{1 \to 1, R \\ \blankSymbol\to \blankSymbol, R}$} (q1)
            (q1) edge[loop above] node{$\substack{1 \to 1, S \\ 1 \to 1, R}$} (q0)
            (q1) edge[bend left] node[below]{$\substack{1 \to 1, S \\ \blankSymbol\to 1, S}$} (q0);
        \end{tikzpicture}
    \end{center}
\end{example}

\begin{remark}
    We claim that we can transform the unary the representation of \(n\) into its binary representation and vice versa in \(\bigO{n}\) steps. This is done by a 2-tape Turing machine. To map binary to unary, the Turing machine decrements the binary number and add a \(1\) in the output tape. To map unary to binary, the Turing machine removes a \(1\) and increments the binary representation on the output tape.
\end{remark}

\begin{definition}
    A function \(f: \Naturals \to \Naturals\) is computable if there exists a multi-tape Turing machine that on input \(1^{n}\) computes \(1^{\func{f}{n}}\) in the output tape. Moreover, \(f\) is said to be in-time computable if a Turing machine exists than can compute it in \(\bigO{\func{f}{n}}\) steps. 
\end{definition}

In the rest of this section we present the Kobayashi's proof of equivalency of time constructible functions and in-time computable functions.

Let \(\calF_0\) be class of all natural functions, \(\calF_1\) be the class of natural function \(f\) such that \(\func{f}{n} \geq n\), lastly let \(\calF_2\) be the class of functions in \(\calF_1\) that \(\exists \epsilon > 0 \; \suchThat\; \forall n,\; \func{f}{n} \geq (1+\epsilon) n\). 


\begin{theorem}
    If both \(f_1 + f_2\) and \(f_2\) are time constructible, \(f_1 \in \calF_1\), and 
    \begin{equation*}
        \exists \epsilon > 0 \; \suchThat \; \forall n,\; \func{f_1}{n} \geq \epsilon \func{f_2}{n} + (1+\epsilon)n
    \end{equation*}
    Then, \(f_1\) is time constructible.
\end{theorem}


\begin{proof}
    Suppose \(M_1\) and \(M_2\) are the timers for \(f_1+f_2\) and \(f_2\) both with same alphabet \(\Gamma\). Consider the timer \(M = \func{M}{k}\) with integer parameters \(k\) with \(k > 7\) that does the following.
    \begin{enumerate}
        \item It is given an input of length \(n\).
        \item On the second tape it compress the input \(n\) by a factor of \(k\), i.e. each \(k\) symbol is represented in 1 symbol, for the simulation of \(M_1\). The resulting string is of length \(\ceil{\frac{n}{k}}\). Also, it puts a dot on the last character to represnet the head tape, the machines will be working ``backwards''.
        \item On the third tape it compress the input \(n\) by a factor of \(l = k-7\) for the simulation of \(M_2\). The resulting string is of length \(\ceil{\frac{n}{k}}\). 
        \item On the fourth tape it writes another string of length  \(\ceil{\frac{n}{l}}\). Here the machine learn the value \(i\) such that \(\ceil{\frac{n}{l}} = \frac{n+i}{l}\)
        \item The last three steps can be done simultaneously in \(n +1\) steps, since it needs to read \(n\) charcters and the first blank symbol. Moreover, \(M\) can compute \(k\) steps on \(M_1\) and \(l\) steps in \(M_2\) in seven steps. Suppose the head of each tape is on the dotted character. \(M\) read the adjecents cells of each tape -- by going left, right,right--, then it repleces the content of these three cells -- going right,right--, and lastly puts the head on the correct place, i.e. the dotted character, which takes at most 2 steps (if it less then it just stays).
        \item \(M\) simulates \(M_1\) and \(M_2\) until \(M_2\) halts. This requires \(\ceil{\frac{\func{f_2}{n}}{l}}\) clocks. Here the machine learn the value \(j\) such that \(\ceil{\frac{\func{f_2}{n}}{l}} = \frac{\func{f_2}{n}+j}{l}\)
        \item  Then, \(M\) simulates \(M_1\) for \(\ceil{\frac{n}{l}}+1\) steps. The third taped is used as counter for here.
        \item Then, \(M\) simulates \(M_1\) until it halts, where in each step it only simulates one step of \(M_1\).
        \item Finally, \(M\) counts upto \(i+j+k-8\) and halts. 
    \end{enumerate}
    The total number of steps is as follows.
    \begin{align*}
        &n+1 + 7\ceil{\frac{\func{f_2}{n}}{l}} + 7\ceil{\frac{n}{l}}+ 7 + \bracket{\func{f_2}{n} + \func{f_1}{n} - k\ceil{\frac{\func{f_2}{n}}{l}} - k\ceil{\frac{n}{l}} - k} + i+j+k-8 \\
        &= \func{f_1}{n} + \func{f_2}{n} + (7-k)\ceil{\frac{\func{f_2}{n}}{l}} + n + (7-k)\ceil{\frac{n}{l}} + i + j\\
        &= \func{f_1}{n} 
    \end{align*}
    This construction only requires that 
    \begin{equation*}
        \func{f_2}{n} + \func{f_1}{n} - k\ceil{\frac{\func{f_2}{n}}{l}} - k\ceil{\frac{n}{l}} - k \geq 0 
    \end{equation*}
    Note that, 
    \begin{align*}
        &\func{f_2}{n} + \func{f_1}{n} - k\ceil{\frac{\func{f_2}{n}}{l}} - k\ceil{\frac{n}{l}} - k\\
        &= \func{f_1}{n} - \dfrac{k-l}{l}\func{f_2}{n} - \dfrac{k}{l}n - k\dfrac{i+j+l}{l}\\
        &\geq \func{f_1}{n} - \dfrac{k-l}{l}\func{f_2}{n} - \dfrac{k}{l}n - 3k \\
        &\geq (1+\epsilon)(\func{f_2}{n} + n) - \dfrac{k}{l} (\func{f_2}{n} + n) - 3k\\
        &= \bracket{1+\epsilon - \dfrac{k}{k-7}}(\func{f_2}{n} + n) -3k \geq \dfrac{\epsilon}{2}n - 3k
    \end{align*}
    for \(k \geq 7(1+ 2\epsilon^{-1})\). And for \(n\geq 6k\epsilon^{-1}\) the value is positive. Thus, we have constructed a timer for \(\func{f_1}{n}\) for \(n\geq N\) some sufficiently large \(N\). Note that since \(\func{f_1}{n} \geq n\) for all \(n\), then we can build a timer \(M'\) such that,
    \begin{enumerate}
        \item It works like \(M\), however, in the first step it also determines if the length of the input exceeds \(N\). 
        \item If it does it continues with \(M\).
        \item If it does not, it clocks for exactly \(n - \func{f_1}{n}\).
    \end{enumerate} 
\end{proof}

\begin{theorem}
    For \(f \in \calF_2\), \(f\) is time constructible if and only if \(f\) is in-time computable.
\end{theorem}

\begin{proof}
    Suppose \(f\) is time constructible. Then, construct a Turing machine such that it simulates the \(f\)-timers and adds a \(1\) on the output tape in each clock. 

    Now, suppose \(f\) is in-time computable and the Turing machine \(M\) computes \(f\) in \(\bigO{f}\). Let \(N_1\) be a timer such that, it runs \(M\) until it halts, and \(N_2\) be a timer that run \(M\) and then goes over the output tape which takes exactly \(1^{\func{f}{n}}\) clocks. Suppose \(N_1\) is a \(g\)-timer. Then, \(g\) and \(g + f\) are both time constructible. Moreover, \(g \leq c f\) for some constant \(c\). Since \(f \in \calF_2\) thus \(f \in \calF_1\) as well. Lastly, for sufficiently small \(\delta > 0\)
    \begin{equation*}
        \delta \func{g}{n} + (1+\delta) n \leq \delta c \func{f}{n} + \dfrac{1+\delta}{1 + \epsilon} \func{f}{n} \leq \func{f}{n}
    \end{equation*}
\end{proof}

\section{\(\compClass{P}\) vs \(\compClass{NP}\)}
\begin{definition}
    The class of all polynomially bounded standard Turing machines is denoted by \(\calM_{\poly}\).
\end{definition}

\begin{definition}
    The class \(\compClass{P}\) is the class of languages that have polynomially bounded standard deterministic decider, \(\compClass{P} = \func{\DD}{\calM_{\poly}}\). The class \(\compClass{NP}\) is the class of languages that have polynomially bounded standard nondeterministic acceptor, \(\compClass{NP} = \func{\NA}{\calM_{\poly}}\). 
\end{definition}

Let \(t: \Naturals \to \Naturals\) be a time-constructible function and define \(\calM_t\) to be the class of standard Turing machines that are equipped with a \(t\)-timer. That is, on input \(w\) of length \(n\), after \(\func{t}{n}\) clocks, the machine halts and rejects the input. 

\begin{theorem}
    \(\compClass{P} = \bigcup_{p \text{ is poly}} \func{\DD}{\calM_p}\) and  \(\compClass{NP} = \bigcup_{p \text{ is poly}} \func{\NA}{\calM_p}\).
\end{theorem}

By the virtue of the above's theorem, we can define \(\compClass{P}\) and \(\compClass{NP}\) in terms of standard Turing machines that always halt, i.e. do not loop.
\chapter{\(\compClass{P}\) vs \(\compClass{NP}\), and \(\compClass{NP}\)-completeness} 
In order to determine whether \(\compClass{P} = \compClass{NP}\), one may try to emulate the proof of \(\compClass{r} \neq \compClass{re}\). Firstly, we defined a notion hardness with many-to-one reduction. Then, we find a hardess problem in \(\compClass{r}\), the halting problem. Lastly, we showed that this problem can not be in \(\compClass{re}\).

\begin{definition}
    A language \(L\) is polynomial-time reducible \(K\), denoted by \(L \leq_p K\) if there exists a polynomial-time computable function \(f: \Sigma^{\ast} \to \Sigma^{\ast}\) such that \(w \in L \iff \func{f}{w} \in K\).
\end{definition}

\begin{definition}
    A language \(L\) is said to be \(\compClass{NP}\)-hard, if for all \(K \in \compClass{NP}\), \(K \leq_p L\). If \(L \in \compClass{NP}\), then \(L\) is called \(\compClass{NP}\)-complete.
\end{definition}

Consider this modified version of the Halting problem.
\begin{equation*}
    H_{np} = \set<\angleBracket{T,w,1^t}>{w \text{ is accepted by } T \text{ in at most } t \text{ clocks.}}
\end{equation*}

\begin{theorem}
    \(H_{np}\) is \(\compClass{NP}\)-hard.
\end{theorem}

\begin{proof}
    Suppose \(L\) is in \(\compClass{NP}\). As a consequence, there exists a polynomial \(p\) and Turing machine \(T\) such that \(T\)'s time is bounded by \(p\) and \(T\) accepts \(L\). Moreover, there exists a function \(f_p\) such that \(\func{f_p}{w} = \angleBracket{T,w,1^{\func{p}{\abs{w}}}}\). This encoding can be computed in polynomial time? and obviously \(w \in L \iff \func{f_p}{w} \in H_{np}\). Hence, \(L \leq_p H_{np}\).
\end{proof}

\begin{theorem}
    \(H_{np}\) is \(\compClass{NP}\)-complete. 
\end{theorem}

\begin{proof}
    The nondeterministic acceptor for \(H_{np}\) is the universal machine that also counts down from \(t\) to \(0\). The time complexity of the universal machine is 
    \begin{equation*}
        \bigO{t\angleBracket{T} + \abs{w}} 
    \end{equation*}
    which is polynomial in size of the input \(\abs{\angleBracket{T}} + \abs{w} + t\).
\end{proof}
\end{document}