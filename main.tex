\documentclass[12pt]{book}
\usepackage[a4paper,bindingoffset=0.2in,%
            left=0.75in,right=0.75in,top=1in,bottom=1in,%
            footskip=.25in]{geometry}
\usepackage{fancyhdr}
\setlength{\headheight}{15.2pt}
\usepackage{inputenc}
\pagestyle{fancy}

\renewcommand{\chaptermark}[1]{\markboth{\thechapter.\ #1}{}}
\renewcommand{\sectionmark}[1]{\markright{\thesection\ #1}}
\fancyhead[LE,RO]{\textbf{\thepage}}
\fancyhead[LO]{\textbf{\rightmark}}
\fancyhead[RE]{\textbf{\leftmark}}
\fancyfoot{}
\fancypagestyle{plain}
{
    \fancyhf{}
}


\usepackage{amsmath}
\usepackage{amssymb}
\usepackage{mathtools}
\usepackage{xcolor}
\usepackage{enumitem}
\usepackage{breqn}


\usepackage{common}
\usepackage{english-theorems}
\setcounter{tocdepth}{1}

\newcommand{\scdot}{\; \cdot \;}

\newcommand*{\textcal}[1]{%
  % family qzc: Font TeX Gyre Chorus (package tgchorus)
  % family pzc: Font Zapf Chancery (package chancery)
  \textit{\Large{{\fontfamily{pzc}\selectfont#1}}}%
}

\begin{document}
\tableofcontents
\clearpage
\ifodd\value{page}\else
\thispagestyle{empty}
\fi
\part{Real Analysis}
\chapter{Real Numbers}
\thispagestyle{headings}
\section{Axiomatic Formulation of Real Numbers}
The building axioms of real numbers is devided into three groups based on the properties they are describing.
\begin{enumerate}
    \item Field axioms.
    \item Order axioms.
    \item Completeness axiom.
\end{enumerate}
\subsection{Field Axioms}
A field is a non-empty set \(\Field\) with two binary operations \textit{addition}, \(+\), and \textit{multiplication}, \(\cdot\). For all \(x,y,z \in  \Field\):
\begin{enumerate}[wide,start=1,label={Axiom \arabic*.}]
    \item Addition and multiplication are commutitive.
          \[x + y = y + x, \quad \ x \cdot y = y \cdot x\]
    \item Addition and multiplication are associative.
          \[x + (y + z) = (x + y) + z, \quad x\cdot (y\cdot z) = (x \cdot y) \cdot z\]
    \item  Multiplication distributes over addition.
          \[x \cdot (y + z) = x \cdot y + x \cdot z\]
    \item There exists a number \(0\) such that for every number \(x\):
          \[ x + 0  = 0 + x = x\]
    \item There exists a number \(1\) such that for every number \(x\):
          \[ x \cdot 1 = 1 \cdot x = x\]
    \item  For every number \(x\), there exists a number \(y\) such that:
          \[x + y = 0\]
          \(y\) is called the negative of \(x\) and is denoted by \(-x\).
    \item For every number \(x \neq 0\), there exists a number \(y\) such that:
          \[x \cdot y = 1\]
          \(y\) is called the reciprocal of \(x\) and is denoted by \(x^{-1}\) or \(\dfrac{1}{x}\).
\end{enumerate}
\subsection{Order Axioms}
The order axioms establishes an ordering on the numbers of \(\Field\) to determine which element is larger or smaller. To achieve an ordering, we define the set of positive real numbers \(\Field^+ \subset \Field\).
\begin{enumerate}[wide,resume,label={Axiom \arabic*.}]
    \item The \(\Field^+\) is closed under addition and multiplication.
          \[\forall x,y \in \Field^+, \quad (x + y) \in \Field^+ \text{ and } (x \cdot y) \in \Field^+\]
    \item \(0 \notin \Field^+\).
    \item For every number \(x \neq 0\), either \(x \in \Field^+\) or \(-x \in \Field^+\).
\end{enumerate}
We then define the binary operator \( > \) such that \(x > y \) whenever \((x - y) \in \Field^+\).
\subsection{Completeness Axiom}
Given that \((\Field, + ,\cdot , >)\) is an ordered field, we define the followings:
\begin{definition} [Upper bound]
    A set \(S \in \Field\) has an upper bound if for some \(a \in \Field\) is greater or equal to all element of \(S\). That is, \(\forall x \in S\: a \geq x\). We say that \(S\) is bounded from above.
\end{definition}
\begin{definition} [The least upper bound]
    \(a \in \Field\) is the least upper bound of a set \(S \subset \Field\) if it is smaller than every upper bound of \(S\). We say \(a\) is the supremum of \(S\), denoted by \(a = \sup{S}\).
\end{definition}
Note that, if the least upper bound exists, it must be unique.
\begin{enumerate}[wide,resume,label={Axiom \arabic*.}]
    \item If \(S\) is a non-empty set that bounded from above that it has supremum.
\end{enumerate}
\begin{theorem}
    There exists a unique set that satisifies all the axioms above. It is denoted by \(\Reals\), the set of real numbers.
\end{theorem}
\begin{proof}
    The existence of \(\Reals\) is proved in many ways. One way to construct real numbers uses \textit{Dedekind Cuts}. Let the pair of rational sets \((A,B)\) be a partition of \(\Rationals\) such that:
    \begin{enumerate}
        \item \(A \neq \emptyset\) and \(A \neq \Rationals\).
        \item \(\forall x,y \in \Rationals \; \mathrm{s.t.} \; x < y,\: y \in A \implies x \in A\).
        \item \(\nexists x \in A \; \mathrm{s.t.} \; \forall y \in A, x \geq y\).
    \end{enumerate}
    For convenience we let \(A\) represent the pair \((A,B)\) as \(A\) completely determines \(B\).
    We define \(+\), \(\cdot\), and \( > \) as follows:
    \begin{flalign*}
        A + B &= \set< a + b >{ a \in A, b\in B} &&\\
        \textcal{0} &= \set< a>{ a < 0}&& \\
        -A &= \set<a'>{\forall a \in A, a' < -a }&&
        \intertext{For \(\cdot\), we first take two set \(A\) and \(B\) that have some positive elements.}
        A \cdot B &= \set< a \cdot b >{a \in A \land a \leq 0, b\in B \land b \leq 0 } \: \cup \: \textcal{0} &&\\
        \intertext{If \(A\) or \(B\) did not have any positive elements, we first take the negative of the set, and then multiply the two sets and take the negative of the product. Similarly, we define the reciprocal of \(A\) if \(A\) has a positive element.}
        \textcal{1} &= \set<a >{ a < 1}&&\\
        A^{-1} &= \set< a' >{\forall a \in A, a > 0, a' < \dfrac{1}{a} } &&
        \intertext{Lastly:}
        A &> B \text{ if } A \supset B &&
    \end{flalign*}
    Also, if a non-empty set \(S\) of real numbers is bounded from above, then it has a supremum in \(\Reals\) equal to \( \bigcup S\). It is left to the reader that the \((\Reals, + , \cdot, >)\) satisifies the axioms above.

    The set of real numbers is unique in sense that if \((\Reals, + , \cdot, >)\) and \((\Reals',+',\cdot', >')\) both satisify the axioms, then there exists bijective mapping \(\alpha : \Reals \to \Reals'\) such that:
    \begin{flalign*}
        \func{\alpha}{x + y} &= \func{\alpha}{x} +' \func{\alpha}{y} &&\\
        \func{\alpha}{0} &= 0'&&\\
        \func{\alpha}{x\cdot y} &= \func{\alpha}{x}\cdot'\func{\alpha}{y}&&\\
        \func{\alpha}{1} &= 1' &&\\
        x < y &\iff \func{\alpha}{x} <' \func{\alpha}{y} &&
    \end{flalign*}
    Lastly, if \(S\) is a non-empty set in \(\Reals\) and \( \func{\alpha}{S} = \set<\func{\alpha}{x}>{ x \in S} \), then \(S\) is has an upper bound if and only if \(\func{\alpha}{S}\) has an upper bound. Furthermore \( \func{\alpha}{\sup{S}} = \sup{\func{\alpha}{S}}\).
\end{proof}
\begin{results}
    \leavevmode
    \begin{enumerate}
        \item The set of natural numbers \(\Naturals\) in \(\Reals\) is not bounded from above.
        \item Let \(x \in \Reals\) be such that for all \(n \in \Naturals\)
              \[ 0\leq x \leq \dfrac{1}{n}\]
              then \(x = 0\).
        \item (Archimedean Property) For all \(a,b > 0\) there exists \(n \in \Naturals\):
              \[ na > b\]
        \item Consider \(I_n = \clcl{a_n}{b_n} \;\forall n \in \Naturals \) such that \(I_1 \supset I_2 \supset \dots\;\). Then \(\cap{I_n}\) is not empty. Moreover, if for each \(e > 0\) there exists \( n\) such that \(b_n - a_n < e\), then \(\cap{I_n}\) is a single point.
        \item \(\sqrt{2} \in \Reals\). In addition, for all \(p > 0\), there is a positive real number \(q \) such that \(q^2 = p\)
    \end{enumerate}
\end{results}
\newpage
{\Large\textbf{Exercises}}
\begin{enumerate}
    \item Prove that the addition and multiplication identity elements are unique.
    \item Show that the Trichotomy law holds for \( (>) \). That is, exactly one of the following three is true.
          \[ x > y \qquad x = y \qquad y > x\]
    \item Show that \(1 \in \Field^+\).
    \item Show that if \(x > -1\) and \(n \in \Naturals\):
          \[(1 + x)^n \geq 1 + nx\]
          and equality only holds when \(n = 1\).
    \item Let \(F_p = \set{ 0, 1, \dots, p -1 }\) where \(p\) is a prime number. Define \(+\) and \(\cdot\) to be the modular addition and product modulus \(p\), respectively. Investigate whether if \(F_p\) can be ordered.
    \item Consider the set of all rational polynomials \(\Rationals[x]\):
          \[\Rationals[x] = \set< \frac{a_m x^m + \dots + a_1 x + a_0}{b_n x^n + \dots + b_1 x + b_0}>{ a_i, b_j \in \Rationals, b_n \neq 0}\]
          Show that \(\Rationals[x]\) under the normal addition and multiplication is a field. Furthermore, show that \(\Rationals^+[x] = \set< q \in \Rationals[x] >{ a_m \cdot b_n > 0 }\) constitutes an ordering on \(\Rationals[x]\).
\end{enumerate}


\chapter{Topology and Metric Spaces}
\section{Topology}

Let \(X\) be a set. A \textbf{topology} on \(X\) is a collection \(\scrT\)  of subsets of \(X\) called \textbf{open set} having the following properties
\begin{enumerate}
    \item If \(U_{\alpha} \in \scrT\) where \(\alpha \in A\) for any set \(A\) then, \(\cup_{\alpha \in A} U_{\alpha} \in \scrT\).
    \item If \(U_{\alpha} \in \scrT\) where \(\alpha \in A\) for any finite set \(A\) then, \(\cap_{\alpha \in A} U_{\alpha} \in \scrT\).
    \item \(X,\emptyset \in \scrT\).
\end{enumerate}
A topological space is a pair \((X,\scrT)\), where \(\scrT\) is a topology on \(X\).

\begin{example}
    On any set \(X\) we can define two topologies. The \textit{trivial topology} on \(X\) consists of \(\set{\emptyset, X}\) and the \textit{discrete topology} on \(X\) is \(\powerSet{X}\).
\end{example}

If \(\scrT_1\) and \(\scrT_2\) are topologies on \(X\), we say that \(\scrT_1\) is \textit{weaker}/\textit{coarser} than \(\scrT_2\), or that \(\scrT_2\) is \textit{stronger}/\textit{finer} than \(\scrT_1\), if \(\scrT_1 \subset \scrT_2\).

\begin{definition}
    Let \(X\) be a topological space and \(x \in X\), \(N\) is called a \textbf{neighbourhood} of \(x\) if there exists an open set \(G\) such that \(x \in G \subset N\). We say that \(x\) is an interior point of \(N\) if \(N\) is neighbourhood of \(x\).
\end{definition}

\begin{proposition}
    A subset \(U\) is open if and only if \(U\) is a neighbourhood for all \(x \in U\).
\end{proposition}

\begin{proof}
    Let \(U\) be a neighbourhood for all of its points. That is, for every \(x \in U\) there is an open set \(G_x\) such that \(x \in G_x \subset U\). Then, \(\bigcup_x G_x \subset U\) however, \(U \subset \bigcup_x G_x\). Therefore, \(U = \bigcup_x G_x\) and by the first axiom \(U\) is an open set. If \(U\) is an open set, then \(U\) is a neighbourhood for each point \(x \in U\).
\end{proof}

\begin{definition}
    A subset \(F\) of \(X\) is called \textbf{closed} if \(F^c\) is open. The \textbf{Closure} of a set \(E\) is the intersection of all closed sets that include \(E\) and it is denoted by \(\closure E\) or \(\bar{E}\).
\end{definition}

\begin{proposition}
    Let \(X\) be a topological space and \(E\) a subset of \(X\). Then, \(\closure E\) is closed.
\end{proposition}

\begin{proof}
    We know that \(\closure E = \bigcap_{E \subset F} F\) where \(F\) are closed set. Then, \((\closure E)^c = \bigcup_{E \subset F} F^c\) which is an open set and hence \(\closure E\) is closed.
\end{proof}

\begin{proposition}
    If \(E\) is subset of a topological space \(X\) and \(x in X\), then \(x \in \closure E\) if and only if \(U \cap E = \emptyset\) for every open neighbourhood of \(U\) of \(x\).
\end{proposition}

\begin{proof}
    Suppose there exists an open neighbourhood of \(x\), \(U\) such that, \(U \cap E = \emptyset\). Then, \(F = \closure E \cap U^c\) is a closed that constains \(E\) hence, \(\closure E \subset F\) and \(x \notin \closure E\). If \(x \notin \closure E\), then \((\closure E)^c\) is an open neighbourhood of \(x\) which does not meet \(E\).
\end{proof}

\begin{definition}
    A \textbf{limit point} of \(E\) is a point \(x \in X\) such that \(E \cap U \backslash \set{x} \neq \emptyset\) for every neighbourhood \(U\) of \(x\). The set of all limit points of \(E\) is denoted by \(E'\) or \(\lim E\). If \(x \in E\) but \(x \notin \closure E\), then \(x\) is called an \textbf{isolated point} of \(E\).
\end{definition}

\begin{definition}
    A subset \(E\) of a topological space \(X\) is \textbf{perfect} if every point of \(E\) is a limit point.
\end{definition}

\begin{definition}
    Let \(E\) be subset of a topological space \(X\) then, the \textbf{interior} of \(E\) is the union of all open sets that are contained in \(E\), denoted by \(E^{\circ}\) or \(\interior E\).
\end{definition}

\begin{proposition}
    For any set \(E\) in topological space \(X\), interior of  \(E\) is the set of all interior points of \(E\) and \(\closure E =( \interior E)^c\).
\end{proposition}

\begin{definition}
    The \textbf{boundary} of \(E\) is defined as \(\closure E \backslash \interior E\) and it is denoted by \(\boundary E\) or \(\partial E\).
\end{definition}

\begin{proposition}
    Let \(E\) be subset of a topological space \(X\). \(E\) is closed if and only if \(E = \closure E\) and \(E\) is open if and only if \(E = \interior E\). Furthermore, \(\boundary E = \emptyset\) if and only if \(E\) is both open and closed.
\end{proposition}

\begin{proof}
    Note that \(E \subset \closure E\) and when \(E\) is closed, \(\closure E \subset E\) therefore, \(E = \closure E\).
\end{proof}

\begin{definition}
    A set \(D\) in a topological space \(X\) is called \textbf{dense} when, \(\closure D = X\). More generally, \(D\) is dense in subset \(E\), if \(E \subset \closure D\). A topological space in which there exists a countable dense set is called \textbf{separable}.
\end{definition}

\begin{proposition}
    \(D\) is dense in \(E\) if and only if for all \(x \in E\), \(D \cap U \neq \emptyset\) for any open neighbourhood \(U\) of \(x\).
\end{proposition}

\begin{proof}
    If there exists \(x \in E\) such that an open neighbourhood \(U\) of \(x\) does not intersect \(D\), then \(x \notin \closure D\) and hence \(D \not\supset E\). If for each \(x \in E\) every neighbourhood \(U\) of \(x\) intersects \(D\), then \(x \in \closure D\) hence \(E \subset \closure D\).
\end{proof}

Let \((X,\scrT)\) be a topological space and \(Y \subset X\). Then, \((Y,\scrT_y)\) is a \textbf{topological subspace} where \(\scrT_y = \set<U \cap Y>{U \in \scrT}\) is the \textbf{relative topology} of \(Y\).

\begin{definition}
    A \textbf{base} for a topolgy \(\scrT\) is a collection of open set \(\scrB \subset \scrT\) such that each \(U \in \scrT\) is a union of open sets in \(\scrB\). That is, \(U = \bigcup_{G \in \scrB} G\).
\end{definition}

\section{Metric spaces}
Let \(X\) be a non-empty set and \(x,y \in X\) then if there exists a non-negative real number \(\func{d}{x,y}\) with following three properties:
\begin{enumerate}
    \item (Positive definiteness.) \(\func{d}{x,y} = 0 \text{ if and only if } x = y\).
    \item (Symmetry.) \(\func{d}{x,y} = \func{d}{y,x}\).
    \item (Triangle inequality.)\(\func{d}{x,y} \leq \func{d}{x,z} + \func{d}{z,y}\).
\end{enumerate}
the combination \((X,d)\) is called a \textbf{metric space} and \(\func{d}{x,y}\) is called the \textbf{metric}, or also \textbf{distance} function.
\begin{example}
    The Euclidean space \(\Reals^n = \set<x_1,x_2,\dots,x_n>{ x_i \in \Reals}\) with the metric \(\func{d}{x,y}=\sqrt{(x_1-y_1)^2 + \dots + (x_n - y_n)^2} \) makes a metric space. To prove this we must show the above properties hold.
    \begin{enumerate}
        \item if \(\func{d}{x,y} = 0\) then:
              \begin{equation*}  \sqrt{(x_1-y_1)^2 + \dots + (x_n - y_n)^2} = 0  \end{equation*}
              Therefore eac  terms must be zero:
              \begin{align*}
                  (x_i - y_i)^2 & = 0 \quad  \forall i \leq n \\
                  x_i - y_i     & = 0 \implies x_i = y_i
              \end{align*}
              Thus \(x = y\).
        \item  It is obvious that \((x_i - y_i)^2 = (y_i - x_i)^2\) and hence \(\func{d}{x,y} = \func{d}{y,x}\)
        \item  The triangle inequality immediately follows from the Cauchy-Schwartz inequality.
    \end{enumerate}
\end{example}

We can expand the Euclidean norm by defining Minkowski \textit{\(p\)-norm} also called \textit{\(L^p\)-norm} for \(1 \leq p \leq \infty\) as follows:
\begin{equation*}
     \func{d_p}{x,y} = \bracket{\sum_{i}{|x_i - y_i|^p}}^{\frac{1}{p}} 
\end{equation*}
and by taking the limit, \(p \to \infty\) we find out that:
\begin{equation*} 
    \func{d_\infty}{x,y} = \max_{i} \set{|x_i - y_i|} 
\end{equation*}

\begin{example}
    We can define \textbf{discrete distance} as follows:
    \begin{equation*}\func{d}{x,y} =
        \begin{cases}
            1 & x\neq y \\
            0 & x=y
        \end{cases}
    \end{equation*}
    and it is pretty straightforward to show that the three properties hold.
\end{example}

\begin{definition}
    The \textbf{open ball} \(\func{B_r}{a}\) with radius \(r\) centered at \(a\) is the set of all points:
    \begin{equation*}  
        \func{B_r}{a} = \set< x \in X>{ \func{d}{x,a} < r}
     \end{equation*}
    and the \textbf{closed ball} \(\func{\overline{B}_r}{a}\) with radius \(r\) centered at \(a\) is the set of all points:
    \begin{equation*} 
        \func{\overline{B}_r}{a} = \set< x \in X>{\func{d}{x,a} \leq r}  
    \end{equation*}
    The \textbf{sphere} \(\func{S_r}{a}\) with radius \(r\) centered at \(a\) is the set of all the points:
    \begin{equation*}  
        \func{S_r}{a} = \set< x \in X>{ \func{d}{x,a}  = r}  
    \end{equation*}
\end{definition}

\begin{definition}
    Let \((X,d)\) be a metric space. A subset \(U \subset X\) is an open set if for all \(a \in U\) there exists \(\rho > 0\) such that \( \func{B_\rho}{a} \subset U\).
\end{definition}

Given this defintion of open sets, we can define a topolgy on metric space \(X\).
\begin{enumerate}
    \item Firstly, we need to show that every union of open sets is open itself. Let \(U_{\alpha}\) be some open sets indexed by \(A\) and let \(x \in \bigcup_{\alpha} U_{\alpha}\). Then, there exists a \(\alpha \in A\) such that \(x \in U_{\alpha}\). Since, \(U_{\alpha}\) is open, there exists a ball \(\func{B_r}{x}\) which is contained in \(U_{\alpha}\). Clearly, \(\func{B_r}{x} \in \bigcup_{\alpha} U_{\alpha}\) and hence \(\bigcup_{\alpha} U_{\alpha}\) is open.
    \item Secondly, we show that intersection of finite collection of open sets in open. Let \(U_{\alpha}\) be open sets indexed by a finite set \(A\) and let \(x \in \bigcap_{\alpha} U_{\alpha}\). For each \(\alpha \in A\), there exists a ball \(\func{B_{r_{\alpha}}}{x}\) such that \(\func{B_{r_{\alpha}}}{x} \subset U_{\alpha}\). Let \(r = \min_{\alpha} r_{\alpha}\) and note that \(\func{B_r}{x} \subset \func{B_{r_{\alpha}}}{x} \subset U_{\alpha}\) for all \(\alpha \in A\). Thus, \(\func{B_{r}}{x} \subset \bigcap_{\alpha} U_{\alpha}\) hence, \(\bigcap_{\alpha} U_{\alpha}\) is open.
    \item Thirdly, we show that \(X\) and \(\emptyset\) are open. \(\emptyset\) is trivially open as it has no element. And \(\func{B_r}{x} \subset X\)  by defintion for all \(r\) hence, \(X\) is open as well.
\end{enumerate}

Consider the following re-defintions of concepts introduced in the previous section.

\begin{definition} [Internal Point]
    A point \(a \in X\) is called an internal point of \(U\) if \(\exists \rho > 0\) that the ball \(\func{B_\rho}{a}\) contained in \(U\).
\end{definition}

\begin{definition} [Interior]
    The interior of a set \(U\) denoted by \(U^\circ\) or \(\interior (U)\) is the set of all its interior points.
\end{definition}

\begin{definition} [Adherent Point]
    A point \(a \in X\) is called an adherent point of \(U\) if \(\forall \rho > 0\) the ball \(\func{B_\rho}{a}\) contains a point in \(U\).
\end{definition}

\begin{definition} [Limit Point]
    A point \(a \in X\) is called a limit point of \(U\) if \(\forall \rho > 0\) the set \(\func{B_\rho}{a} - \{ a\}\) contains a point in \(U\). The set of all limit points is denoted by \(S'\) or \(\lim S\).
\end{definition}

\begin{note}
    For any limit point \(a \in U\) every open ball \(\func{B_r}{a}\) contains infinitely many points in \(U\).
\end{note}

\begin{definition} [Closed Set]
    A subset \(C \subset X\) is closed set if it contains all of its adherent point.
\end{definition}

\begin{definition} [Closure]
    The closure of a set \(U\) denoted by \(\overline{U}\) or \(\closure U\) is set of all its adherent points.
\end{definition}

\begin{note}
    The closure of a set is a closed set.
\end{note}

We then, show that these re-definitions are equivalent to the topological defintions.

\begin{theorem}
    Subset \(C \subset X\) is closed if and only if \(X \backslash C\) is open.
\end{theorem}

\begin{proof}
    Firstly, we prove the necessity condition that is \(C\) is closed if \(X \backslash C\) is open . We employ proof by contradiction. Let \(C\) be a closed subset of \(X\) such that its complement is not open. That is, for some \(a \in (X \backslash C)\) there is no \(\rho > 0\) exists such that \(\func{B_\rho}{a} \subset (X \backslash C)\). In other words, for all \(\rho,\: \exists p \in \func{B_\rho}{a} \;\mathrm{ s.t }\; p_\rho \in C\). Which implies that \(a\) is an adherent point of \(C\) but since \(C\) is closed then \(a \in C\) which is a contradiction. Similarly, one can show the sufficiency condition.
\end{proof}

\begin{corollary}
    \(X\)  and \(\emptyset\) are both closed and open.
\end{corollary}

\begin{remark} (Equivalent Definitions)
    \begin{enumerate}
        \item An open set is a union of open balls. Conversely, a union of open balls is an open set.
              \begin{prooflemma}
                  For every \(a \in U\) there is a ball \(\func{B_\rho}{a} \subset U\) thus \(\bigcup_{a \in U}{\func{B_\rho}{a}} \subset U\) and since \(a \in \func{B_\rho}{a}\) we must have \(\bigcup_{a \in U}{\func{B_\rho}{a}} \supset U\) hence \(U = \bigcup_{a \in U}{\func{B_\rho}{a}}\).

                  Now let \(U = \bigcup {\func{B_\rho}{a}}\) we need to show that \(U\) is open. Let \(b \in U\) then \(b\) must be a point in at least one of those balls. Let \(b \in \func{B_r}{c}\) and \(\rho = r - \func{d}{b,c}\). We will show that \(\func{B_{\rho}}{b} \subset \func{B_r}{c} \subset U\), for any \(x \in \func{B_{\rho}}{b}\) by triangle inequality we have \(\func{d}{x,c} \leq \func{d}{x,b} + \func{d}{b,c} < \rho + \func{d}{b,c} = r\) which means \(x \in \func{B_r}{c}\).
              \end{prooflemma}
        \item A set is open if and only if all of its members are interior points. Therefore, \(U = \interior U\).
        \item Let \(I = \set< S \subset U>{ S \text{ is open}}\) then \(\interior U =  \bigcup_{S \in I} S\)/
        \item Let \(I = \set< U \subset S >{S \text{ is closed}}\) then \(\closure U =  \bigcap_{S \in I} S\).
    \end{enumerate}
\end{remark}

Let \((X,d)\) be a metric space and \(Y \subset X\) then \(Y\) may inherit its metric from \(X\) and \((Y,d)\) would also be a metric space and is called a \textbf{metric subspace} of \(X\). We will investigate the nature of open and closed sets in subspaces. Let \(\func{B_\rho^Y}{y} = \set< p \in Y >{ \func{d}{y,p} < \rho}\) Then, it is easy to see that:
\begin{equation*} 
    \func{B_\rho^Y}{y} = \func{B_{\rho}}{y} \bigcap Y
\end{equation*}

\begin{corollary} \label{InheritancePrinciple}
    Let \((X,d)\) be a metric space and \(Y \subset X\) is a metric subspace of \(X\) then \(U \subset Y\) is an open subset of \(Y\) if and only if there is a open set \(V \subset X\) such that \(U = V \cap Y\). Similarly, for any closed set \(C \subset Y\) there is a closed set \(D \subset X\) such that \(C = D \cap Y\).
\end{corollary}

\begin{proof}
    Ofcourse, if \(U \subset Y\) is open in \(Y\) then by definition it can be represent as a union of open ball \(\func{B_r^Y}{a} \). Each of these balls is the intersection of a \(\func{B_r^X}{a} \cap Y\). Therefore
    \begin{equation*}
        U = \bigcup \func{B_r^Y}{a} = \bigcup \bracket{\func{B_r^X}{a} \cap Y} = \bracket{ \bigcup \func{B_r^X}{a} } \cap Y = V \cap Y
    \end{equation*}
    Furthermore, if \(a \in V \cap Y\) then there exists a ball \(\func{B_r^X}{a} \subset V\). Therefore
    \begin{equation*}
        \func{B_r^Y}{a} = \func{B_r^X}{a} \cap Y \subset V \cap Y = U
    \end{equation*}
    The case for closed subsets can be proved using the complements.
\end{proof}

\begin{exercise}
    \item Show that \(\closure S = S \cup \lim S\)
\end{exercise}
\newpage

\section{Convergence}
Let \(X\) be a topological space. A \textbf{sequence} is a function of form  \( a : \set{k, k+1, k+2, \dots } \to X\) where \(k \in \Integers\). Conventionally, instead of \(\func{a}{n}\), \(a_n\) is used. The sequence \(\set{ a_n}\) is \textbf{convergent} to \(a \in X\) if for all neighbourhood \(U\) of \(a\) there exists \(N\) such that:
\begin{equation*}
    n \geq N \implies a_n \in U
\end{equation*}
and it is denoted by \(a_n \to a\) or \(a = \lim_{n \to \infty}{a_n}\). Given a topolgy, convergence is not necessarily well-behaved. For example, in the trivial topolgy, if \(a_n \to a\), then \(a_n \to b\) for any \(b \in X\). To do away with this we consider \textbf{Hausdorff spaces}.

\begin{definition}
    Let \(X\) be a topological space. \(X\) is a Hausdorff space if for any two \(x,y \in X\) where \(x \neq y\), there exists disjoint open sets \(U\) and \(V\) such that \(x \in U\) and \(y \in V\).
\end{definition}

\begin{proposition}
    Let \(X\) be a Hausdorff space and \(x_n\) is a sequence in \(X\). If \(x_n \to x\) and \(x_n \to y\) as \(n \to \infty\), then \(x = y\).
\end{proposition}

\begin{proof}
    If \(x \neq y\), then there are disjoint open set \(U\) and \(V\) with \(x \in U\) and \(y \in V\). If \(x_n \to x\), then there exists \(N\) such that for \(n \geq N\), \(x_n \in U\). But this implies that for \(n \geq N\), \(x_n \notin V\). Meaning, there exists no \(N'\) that for \(n \geq N'\), \(x_n \in V\) hence, \(x_n\) does not converge to \(y\).
\end{proof}

In the case of a metric space, the set \(\set{ a_k, a_{k + 1} , \dots }\) is bounded in \(X\), that is, there exist \(K > 0\) and a point \(b \in X\) such that \(\forall n,\; a_n \in \func{B_K}{b}\).

Another problem with the defintion of convergence is its dependence on a convergence point. So naturally the following question comes up. Is there a way to show the convergence of sequence based on itself?
For that, we need to define \textbf{Cauchy sequence}. A sequence \(\set{a_n}\) in a metric space \(X\) is a Cauchy sequence if:
\begin{equation*}
     \forall \epsilon > 0,\: \exists N \quad \suchThat \quad n, m \geq N \implies \func{d}{a_n,a_m} < \epsilon  
\end{equation*}

\begin{lemma}
    In a metric space \(X\), convergence of \(a_n\) to \(a\) is equivalent to
    \begin{equation*}
        \forall \epsilon > 0,\: \exists N \quad \suchThat \quad n \geq N \implies \func{d}{a_n , a} < \epsilon
    \end{equation*}
\end{lemma}
\begin{proof}
    If \(a_n \to a\), then for \(\func{B_{\epsilon}}{a}\) there exists \(N\) such that \(n \geq N \implies a_n \in \func{B_{\epsilon}}{a}\). On the other hand, for any neighbourhood \(U\) we can find \(\epsilon > 0\) such that \(\func{B_{\epsilon}}{a} \subset U\). Hence, if the metric convergence condition holds, then topological convergence holds, as well.
\end{proof}

\begin{theorem} \label{ConvergenceCauchy}
    Every convergent sequence is a Cauchy sequence.
\end{theorem}

\begin{proof}
    For a given \(\epsilon > 0 \) we know there exist \(N\) such that:
    \begin{equation*}  
        n \geq N \implies \func{d}{a_n,a} < \frac{\epsilon}{2} 
    \end{equation*}
    and equivalently:
    \begin{equation*}  
        m \geq N \implies \func{d}{a_m,a} < \frac{\epsilon}{2} 
    \end{equation*}
    and since by the triangle inequality we have:
    \begin{equation*}
        \func{d}{a_m,a_n} \leq  \func{d}{a_m,a} +  \func{d}{a_n,a} \leq \frac{\epsilon}{2} + \frac{\epsilon}{2} = \epsilon
    \end{equation*}
\end{proof}

\begin{definition} [Subsequence]
    We call \(\set{ b_n }\) a \textbf{subsequence} of \(\set{ a_n }\) if there is a sequence of positive integers \(n_1 < n_2 < n_3 < \dots \) such that for each \(k\), \(b_k = a_{n_k}\).
\end{definition}

\begin{exercise}
    \item Show that if a sequence \(\set{a_n }\) is convergent, then its limit is unique. That is, if \(a_n \to a \) and \(a_n \to b\) as \(n \to \infty\) then \(a = b\).
    \item Prove that every subsequence of a convergent sequence converges and it converges to the same limit.
\end{exercise}
\newpage

\section{Completeness}
A metric space \((X,d)\) is \textbf{complete} if every Cauchy sequence converges.

\begin{proposition}
    \(\Reals\) with the normal Euclidean norm is a complete metric space.
\end{proposition}

To prove it, we need the following lemmas.

\begin{lemma} \label{Bounded}
    If \(\set{ a_n }\) is a Cauchy sequence in a metric space \((X,d)\) then the set \(S = \set{ a_k , a_{k + 1}, \dots }\) is bounded.
\end{lemma}

\begin{prooflemma}
    For a fixed \(\epsilon > 0\) we know there exists \(N\) such that:
    \begin{equation*}
        m,n \geq N \implies \func{d}{a_n,a_m} < \epsilon
    \end{equation*}
    especially:
    \begin{equation*}
        n \geq N \implies \func{d}{a_n,a_N} < \epsilon
    \end{equation*}
    Since there is only finitely many indices less than \(N\) then we can determine the largest \(\func{d}{a_N, a_m}\)for all \(m\) less than \(N\) lets denote it by \(A\). Finally, let \(K = \max \set{ \epsilon, A}\) then, \(\func{B_K}{a_N} \) contains all the elements of sequence.
\end{prooflemma}

\begin{lemma} \label{convergenceSubsequence}
    If one of the subsequences of Cauchy sequence is convergent, then the Cauchy sequence is convergent to the same element.
\end{lemma}

\begin{prooflemma}
    Let \(a_{n_k} \to a \) when \(k \to \infty\) That is, for a given \(\epsilon > 0,\: \exists N_1\) such that:
    \begin{equation*}
        k \geq N_1 \implies \func{d}{a_{n_k},a} < \frac{\epsilon}{2}
    \end{equation*}
    and since \(\set{a_n} \) is a Cauchy sequence then we also know that there exists \(N_2\) such that:
    \begin{equation*}
        q,m \geq N_2 \implies \func{d}{a_m,a_q} < \frac{\epsilon}{2}
    \end{equation*}
    Let \(N = \max \set{ N_1, N_2} \) and \(n_q \geq N\) consequently:
    \begin{equation*}
        n_q,m \geq N \implies \func{d}{a_m,a_{n_q}} < \frac{\epsilon}{2}
    \end{equation*}

    and by the triangle inequality we have:
    \begin{equation*}
        \func{d}{a_m,a} \leq \func{d}{a_m, a_{n_q}} + \func{d}{a_{n_q},a} \leq \frac{\epsilon}{2} + \frac{\epsilon}{2} = \epsilon
    \end{equation*}
    which proves the convergence of \(a_n \to a\).
\end{prooflemma}

We present a proof for the completeness of \(\Reals\) under the Euclidean norm.

\begin{proof} \label{RealComplete}
    Let \(\set{ a_n }\) be a Cauchy sequence. Then by \Cref{Bounded}, the sequence is bounded and there is a closed interval \(I_0 = \clcl{a}{b} \) in which all \(a_n\) lie. Consider the closed intervals \(\clcl*{a}{\dfrac{a + b}{2}} \) and \(\clcl*{\dfrac{a + b}{2}}{b}\). Since the sequence has infinitely many terms then there are infinitely many terms in at least one of the two intervals. Let that interval be \(I_1\) and choose \(x_1 \in I_1\) where \(x_1 = a_{n_1}\) for some \(n_1\).
    Repeat the process for \(I_1\) to get \(I_2 \) and \(x_2 = a_{n_2}\) where \(n_2 > n_1\). Since there are infinitely many terms in \(I_2\) we can find such \(n_2\). By continuing this process we have a subsequence \(\{ x_k \}\) and a sequence of nested closed sets \(\set{ I_k = \clcl{a_k}{b_k}}\). Since for all \(\epsilon > 0\) there exists \(K\) such that \(b_K - a_K < \epsilon\) then the intersection of \(\set{I_k}\) is a point, say \(y\). We claim that that \(x_k \to y\), that is:
    \begin{equation*}
        \forall \epsilon > 0 \; \exists N \in \Naturals \quad \suchThat \quad n \geq N \implies \abs{x_n - y} < \epsilon
    \end{equation*}

    Since \(y = \bigcap I_k\) then \(y \in I_k\) for all \(k\), especially \(y \in I_n\). Therefore, \(\abs{x_n - y}\) is smaller than or equal to the length of \(I_n\) which is \(\dfrac{b - a}{2^n} \leq \dfrac{b-a}{2^N}\). By setting \(N > \log_2{\dfrac{b-a}{\epsilon}}\) we have:
    \begin{equation*}
        \abs{x_n - y} \leq \dfrac{b - a}{2^n} \leq \dfrac{b-a}{2^N} < \epsilon
    \end{equation*}
    Therefore \(\Reals\) is a complete metric under Euclidean norm.
\end{proof}


Let \((X,d)\) and \((X',d')\)be two metric spaces. Define the following norms on the Cartesian product \(X \times X'\):
\begin{enumerate}
    \item \(\func{D_1}{(x,x'),(y,y')} = \func{d}{x,y} + \func{d'}{x,y}\)
    \item \(\func{D_2}{(x,x'),(y,y')} = \sqrt{\func{d}{x,y}^2 +\func{d'}{x',y'}^2}\)
    \item \(\func{D_3}{(x,x'),(y,y')} = \max \{ \func{d}{x,y} , \func{d'}{x',y'}\}\)
\end{enumerate}

Let \(p_1 = (x,x')\) and \(p_2 = (y,y')\):
\begin{equation*}
    \func{D_3}{p_1,p_2} \leq \func{D_2}{p_1,p_2} \leq \func{D_1}{p_1,p_2} \leq 2\func{D_3}{p_1,p_2}
\end{equation*}

Then, it is easy to see that if a sequence \(\{ a_n \}\) is convergent under one of these norms, it is convergent to the same value under the other two.
The same is true if the sequence is a Cauchy sequence.

By induction we can generalize it to \(X_1 \times X_2 \times \dots \times X_n\). For example, \(\Reals^n\) is complete metric under all the three norms introduced above. That is, every Cauchy sequence in \(\Reals^n\) is convergent. To show this assume the sequence \(\set{ x_i }\) is a Cauchy sequence under, WLOG, \(D_1\):
\begin{equation*}
    \forall \epsilon > 0, \, \exists N \quad \suchThat \quad i,j \geq N \implies \func{D_1}{x_i,x_j} < \epsilon
\end{equation*}

Then for the \(k\)-th coordinate:
\begin{equation*}
    \abs{x_{i_k} - x_{j_k}}  < \func{D_1}{x_i,x_j}  < \epsilon
\end{equation*}

Therefore, for every coordinate, the image of the sequence on that coordinate is a Cauchy sequence. Since \(\Reals\) is complete then \(\set{ x_{i_k} } _i\) is convergent to some \(x_k\) for all \(k\). We claim that \(x_i \to x = (x_1, \dots, x_n)\) as \(i \to \infty\):
\begin{equation*}
    \func{D_1}{x,x_i} = \abs{x_{i_1} - x_1} + \abs{x_{i_2} - x_2}  + \dots + \abs{x_{i_n} - x_n}
\end{equation*}

We have shown that \(\set{ x_{i_k}} _i\) is convergent to \(x_k\) then there must be \(N_1,N_2, \dots N_n\) such that for all \(k\):
\begin{equation*}
    \forall \epsilon,\quad i \geq N_k \implies \abs{x_{i_k} - x_k} < \frac{\epsilon}{n}
\end{equation*}
Setting \( N = \underset{1 \leq k \leq n}{\max}{N_k} \):
\begin{equation*}
    \func{D_1}{x,x_i} < n \cdot \frac{\epsilon}{n} = \epsilon
\end{equation*}

\begin{theorem}
    Let \((X,d)\) be a complete metric space and \(Y \subset X\) is a complete metric space if and only \(Y\) is a closed subset of \(X\).
\end{theorem}

\begin{proof}
    It is clear that \(Y\) being closed is necessary for \(Y\) being a complete metric subspace. To show that is also sufficient, we need ot show that if \(Y\) is a complete metric subspace then it is closed. Assume the contrary, that is there exists an adherent point of \(Y\), \(a \notin Y\). Since \(a\) is an adherent point of \(Y\) then for all \(\rho > 0\) there exists a point \(x \in \func{B_\rho}{a}\) such that \(x \in Y\). For each \(n\) let \(\rho = \frac{1}{n}\) and choose a point \(x_n \in Y\)
    It is clear that \(\set{ x_n }\) is convergent to \(a\). From \Cref{ConvergenceCauchy} \(\set{ x_n }\) is a Cauchy sequence. Since \(Y\) is complete then \(a\) must be in \(Y\) which is a contradiction.
\end{proof}

\begin{theorem}[Baire Category Theorem]
    Let \(X\) be a complete metric space and \(G_n\) be a sequence of dense open sets in \(X\). Then, \(\cap_{n = 1}^{\infty} G_n\) is dense in \(X\).
\end{theorem}

\begin{theorem}
    Let \((X,d)\) be a complete metric space, and suppose that \(f:X \to X\) has the property that there exists \(\alpha < 1 \) such that
    \begin{equation*}
        \func{d}{\func{f}{x}, \func{f}{y}} < \alpha \func{d}{x,y}
    \end{equation*}
    for all \(x,y \in X\). Then, there exists a unique point \(x \in X\) such that \(\func{f}{x} x\). Moreover, if \(x_0 \in X\), and \(x_{n+1} = \func{f}{x_n}\), then \(x = \lim_{n \to \infty} x_n\).
\end{theorem}
\begin{exercise}
    \item Show that if a sequence \(\set{ a_n }\) is convergent, then its limit is unique. That is, if \(a_n \to a \) and \(a_n \to b\) as \(n \to \infty\) then \(a = b\).
    \item Prove that every subsequence of a convergent sequence converges and it converges to the same limit.
\end{exercise}
\newpage

\section{Continuity}
\begin{definition} [Continuity]
    Let \((X,\scrT_X)\) and \((Y,\scrT_Y)\) be two topological spaces and \(f : X \to Y\) be a function. We say \(f\) is continuous if for every open subset \(V\) of \(Y\) the pre-image of it is an open set in \(X\).
    \begin{equation*} 
        \func{f^{\mathrm{pre}}}{V}= \func{f^{-1}}{V} = \{ x \in X : \func{f}{x} \in V\}
    \end{equation*}

    Furthermore, \(f\) is continuous at a point \(x \in X\) when for all subset \(W\) of  \(Y\) that \( \func{f}{x} \) is an internal point of, there is an open set \(U\) containing \(x\) such that \(\set<f(y)> {y \in U }\subset W\). In other words \(x\) is an internal point of \(\func{f^{\mathrm{pre}}}{W}\).
\end{definition}

\begin{proposition}
    \(f\) is continuous if and only if \(f\) is continuous at every point \(x \in X\).
\end{proposition}

\begin{proof}
    Firstly, if \(f\) is continuous we show that \(f\) is continuous at every point \(x \in X\). Let \(V\) be an open set around \( \func{f}{x}\) then \(x \in \func{f^{\mathrm{pre}}}{V}\) must be an internal point since \(\func{f^{\mathrm{pre}}}{V}\) is open.
    Secondly, if \(f\) is continuous at every point \(x \in X\) then \(f\) is continuous. Let \(V = \set<\func{f}{x} >{ x \in U}\) be an open set in \(Y\). For any \(x \in U\), \(\func{f}{x}\) is an internal point of \(V\) and since \(f\) is continuous at \(x\), \(x\) is an internal point of \(U\) which means every point \(x \in U\) is an internal point of \(U\) and thus \(U = \func{f^{\mathrm{pre}}}{V}\) is open.
\end{proof}

\begin{theorem} [\(\epsilon-\delta\) condition]
    Let \((X,d)\) and \((Y,d')\) be two metric space. Continuity at a point \(x\) is equivalent to the existence a \(\delta > 0\) for all \(\epsilon > 0\) such that:
    \begin{equation*}
        \func{d}{x,y} < \delta \implies \func{d'}{\func{f}{x},\func{f}{y}} < \epsilon  
    \end{equation*}
\end{theorem}

\begin{proof}
    Let \(V = \set< \func{f}{y} >{\func{d'}{\func{f}{x},\func{f}{y}} < \epsilon}\) then \(V\) is open and hence \(\func{f}{x}\) is an internal point of \(V\). By continuity at point \(x\), \(x\) must be an internal point of \(\func{f^{\mathrm{pre}}}{V}\). In other words, there exists a \(\delta > 0\) such that \(U = \set<y>{\func{d}{x,y} < \delta } \subset f^{\mathrm{pre}}(V)\).
    Take an open set \(U \subset Y\), then assuming the \(\epsilon-\delta\) condition, we will show that \(\func{f^{\mathrm{pre}}}{U}\) is open. Let \(y \in U\) then there is \(x \in \func{f^{\mathrm{pre}}}{U}\) such that \(\func{f}{x} = y\). From openness of \(U\), there is a \( \epsilon > 0\) such that \(\func{B_\epsilon}{y} \subset U\), also by continuity condition, there exists a \(\delta > 0 \) such that:
    \begin{equation*}
        \func{d}{x,z} < \delta \implies \func{d'}{\func{f}{x},\func{f}{z}} < \epsilon
    \end{equation*}
    The openness of \(\func{f^{\mathrm{pre}}}{U}\) is equivalent to \(\func{B_\delta}{x} \subset \func{f^{\mathrm{pre}}}{U}\), which clearly holds, since for any \(z \in \func{B_\delta}{x} \implies \func{f}{z} \in \func{B_\epsilon}{y} \subset U\).
\end{proof}

\begin{example}
    Let \((X,d)\) be a metric space with \(\func{d}{x,y}\) being the discrete metric, \(f : X \to X'\) where \((X',d')\) is an arbitary metric space. Then \(f\) is always continuous. Since for every point \(a\) the open ball \(\func{B_{\frac{1}{2}}}{a} = \set{ a }\), and union of open sets is an open set itself, then every subset of \(X\) is open.
\end{example}
Equivalently, \(f\) is continuous at \(a\) if for all \(\epsilon > 0\), \(a\) is an internal point of \(\func{f^{\mathrm{pre}}}{\func{B_\epsilon}{\func{f}{a}}}\). That is there exists \(\delta > 0\) such that, \(\func{B_\delta}{a}  \subset \func{f^{\mathrm{pre}}}{\func{B_\epsilon}{\func{f}{a}}}\). More generally, if \(X\) has the discrete topolgy or \(X'\) has the trivial topolgy, then \(f:X\to X'\) is always continous.

\begin{proposition}
    Let \(f: (X,\scrT_1) \to (X,\scrT_2)\) be the identity function. \(f\) is continous if and only if \(\scrT_1\) is stronger that \(\scrT_2\).
\end{proposition}

\begin{proof}
    If \(f\) is continous, then every open set \(V \in \scrT_2\) is an open set in \(\scrT_1\). Hence, \(\scrT_2 \subset \scrT_1\). If \(\scrT_2 \subset \scrT_1\) and \(V\) is an open set in \(\scrT_2\), then \(V = \func{f^{\mathrm{pre}}}{V} \in \scrT_1\) is open and thus \(f\) is continous.
\end{proof}

\begin{theorem}
    Let \((X,d)\) and \((X',d')\) be two metric spaces and \( f: X \to X' \). \(f\) is continuous at \( a \in X\) if and only if for every sequence \(\set{ a_n}\) in \(X\) with \(a_n \to a\) we have \(\func{f}{a_n} \to \func{f}{a}\).
\end{theorem}

\begin{proof}
    Let \(f\) be continuous at \(a\) and \(a_n \to a\). From continuity of \(f\), for each given \(\epsilon\), there is a \(\delta\) such that:
    \begin{equation*}
        \func{d}{x,a} < \delta \implies \func{d'}{\func{f}{x},\func{f}{a}} < \epsilon
    \end{equation*}
    From the convergence of \(\set{a_n}\), for each given \(\delta\), there is a \(N\) such that:
    \begin{equation*}
        \forall n \geq N \implies \func{d}{a_n,a} < \delta
    \end{equation*}
    By merging these two equations we will get:
    \begin{equation*}
        \forall n \geq N \implies  \func{d}{a_n,a} < \delta \implies \func{d'}{\func{f}{a_n},\func{f}{a}} < \epsilon
    \end{equation*}
    which was what was wanted.

    If \(f\) is not continuous, there must be an \(\epsilon > 0\) that for all \(\delta > 0 \), for some  \(x \in \func{B_\delta}{a}\), \( \func{d'}{\func{f}{x},\func{f}{a}} \geq \epsilon\). Especially, for each \(n \in \Naturals\), let \(\delta = \frac{1}{n}\) and \(x_n\) have the described property. Since \(x_n \to a\) by our assumption \(\func{f}{x_n} \to \func{f}{a}\), which is a contradiction and thus \(f\) is continuous.
\end{proof}

\begin{definition}
    Let \(X\) and \(Y\) be two topological spaces and \(f:X \to Y\). \(f\) is a \textbf{homeomorphism} of \(X\) to \(Y\) if \(f\) is bijective and, \(f\) and \(f^{-1}\) are continous. Furthermore, \(X\) and \(Y\) are \textbf{homeomorphic} if there exists a homeomorphism between them.
\end{definition}

\begin{exercise}
    \item Let \((X,d), (X',d'),\)   and \((X'',d'')\) be metric spaces and \(f : X \to X'\), \(g : X' \to X''\) be two functions. If \(f\) is continous at \(a\) and \(g\) is continous at \(\func{f}{a}\), then \(g \circ f\) is continuous at \(a\).
    \item Let \((X_i,d_i), \; i = 1, \dots k\) be metric spaces. Define \(D\) to be any of the three discussed metric over \( X = X_1 \times X_2 \times \dots \times X_k\). Then, the projection function, \(\func{\pi_j}{x} : X \to X_j\) is continuous for all \(j\).
          \begin{equation*}
              \func{\pi_j}{x_1,x_2, \dots, x_n} = x_j
          \end{equation*}

    \item Let \((X,D)\) be defined as above and let \((X',d')\) be a matic space, and \(f : X' \to X\). \(f\) is continous at \(a' \in X'\) if and only if \(\pi_j \circ f\) is continuous for all \(j = 1, \dots, k\).
    \item The four algebraic operations are continuous on their domain.
          \begin{flalign*}
              \text{\large{$+$}}        & : \Reals \times \Reals \to \Reals, \quad \text{\large{$+$}}(x,y) = x + y                            &  & \\
              \text{\large{$-$}}        & : \Reals \times \Reals \to \Reals, \quad \text{\large{$-$}}(x,y) = x - y                            &  & \\
              \text{\large{\(\times\)}} & : \Reals \times \Reals \to \Reals, \quad \text{\large{\(\times\)}}(x,y) = x \times y                &  & \\
              \text{\large{\(\div\)}}   & : \Reals \times \left(\Reals-\{0\}\right) \to \Reals, \quad \text{\large{\(\div\)}}(x,y) = x \div y &  &
          \end{flalign*}
          Where the metric of \(\Reals\) on the right hand side is the common Euclidean metric, and on the left hand side is any of the three metric.
\end{exercise}

\newpage
\section{Covering compactness}
\begin{definition}
    Let \(X\) be a topological space. A \textbf{covering} for s set \(E \subset X\) is a collection of \(U_\alpha\) of open subsets of \(X\) such that \(E \subset \bigcup U_\alpha \).
\end{definition}

\begin{definition}
    A  \textbf{subcovering} of \(\scrU = \set<U_{\alpha}>{\alpha \in A}\) is a collection \(\scrV = \set<U_{\alpha}>{\alpha \in B}\) where \(B \subset A\) and \(\scrV\) covers the same set.
\end{definition}

\begin{definition}
    A subset \(K\) of \(X\) is called \textbf{compact} if every open cover of \(K\) has a finite subcover.
\end{definition}

\begin{example}
    \(\scrU = \set<\opop{x-1}{x+1}>{x \in \Reals}\) is covering for \(\Reals\). It Obviously has a countable subcovering, however, it does not have finite subcovering.
\end{example}

\begin{theorem}
    Closed and bounded intervals in \(\Reals\) are compact.
\end{theorem}

\begin{proposition}
    If \(X\) is a compact space and \(\set{K_n}\) is a sequence of non-empty closed subsets of \(X\), with \(K_{n+1} \subset K_n\) for all \(n\), then \(\bigcap_{n = 1}^{\infty} K_n\) is non-empty.
\end{proposition}

\begin{proposition}
    If \(E\) is an infinite subset of a compact set, then \(E\) has a limit point in \(K\).
\end{proposition}

\begin{proposition}
    A subset of a topological space is compact if and only if it is compact in iteself with the relative topology.
\end{proposition}

\begin{proposition}
    If \(X\) is a Hausdorff space and \(K \subset X\) is compact, then \(K\) is closed.
\end{proposition}

\begin{definition}
    Let \(X\) be a metric space and \(E\) a subset of \(X\). The \textbf{diameter} of \(E\) is defined to be \(\diam E = \sup \set<\func{d}{x,y}>{x,y \in E}\). \(E\) is said to be bounded if its diameter is finite.
\end{definition}

\begin{definition}
    Let \(E\) be subset of a metric space \(X\). \(E\) is \textbf{totally bounded} if for all \(\epsilon > 0\),  there exists a finite subset \(\set{x_1, \dots , x_n}\) of \(X\) such that \(E \subset \bigcup_{k= 1}^n \func{B_{\epsilon}}{x_k}\). A set \(F\) such that \(E \subset  \bigcup_{x \in F} \func{B_{\epsilon}}{x}\) is called an  \(\epsilon\)-\textbf{net} for \(E\).
\end{definition}

\begin{proposition}
    If \(E\) is totally bounded, then:
    \begin{enumerate}
        \item \(E\) is bounded.
        \item \(\closure E\) is totally bounded.
        \item Any subset of \(E\) is totally bounded.
    \end{enumerate}
\end{proposition}

\begin{theorem}
    A metric space \(X\) is complete if and only if \(X\) is complete and totally bounded.
\end{theorem}

\begin{definition}
    A subset \(E\) of topological space is \textbf{relatively compact} or \textbf{precompact} if \(\closure E\) is compact.
\end{definition}

\begin{corollary}
    A subset of a complete metric space is relatively compact if and only if it is totally bounded.
\end{corollary}

\begin{theorem}
    A set \(E\) in \(\Reals^k\) is closed and bounded  if and only if it is compact.
\end{theorem}

\begin{note}
    This is generally not true for other metric spaces.
\end{note}

\begin{theorem}[Weierstrass]
    Every bounded infinite subset of \(\Reals^k\) has a limit point in \(\Reals^k\).
\end{theorem}

\begin{exercise}
    \item Show that \(\Rationals \cap [0,1]\) is not covering compact, directly from the definition.
\end{exercise}
\newpage

\section{Sequential compactness}

A subset \(K \subset X\) is \textbf{compact} if it has the Boltzano-Weierstrass property, if for all sequences \(\set{a_n}\) in \(K\) there exists a subsequence of \(\set{a_n}\) that converges to a point \(a \in K\).

\begin{theorem}
    Compactness is equivalent to the existence a finite subcovering for every covering.
\end{theorem}
\begin{proof}
    To prove the theorem, let us define:

    We will show for metric spaces compactness is equivalent to covering compact.
    lebegue number
\end{proof}

% \begin{corollary}
%     If \(K\) is compact then \(K\) must be closed and bounded.
% \end{corollary}

% \begin{proof}
%     Obviously if \(K\) is not closed then there must be a limit point \(a \notin K\) such that the sequence \(\{a_n\}\) converges to \(a\). We have shown every subsequence of a convergent sequence converges to the same value, therefore \(K\) is not compact.
%     If \(K\) is unbounded then for each point \(a \in K\) for all \(n \in \Naturals\), the ball \(\func{B_n}{a}\) has a point other than \(a\) in \(K\). Then we can select \(a_n\) to be a point. Cleary no subsequence of \(\{a_n\}\) can be convergent.
% \end{proof}

% \begin{theorem}
%     If \(K \subset X\) is compact and \(C\) is a closed subset of \(X\) such that \(C \subset K\), then \(C\) is compact.
% \end{theorem}

% \begin{proof}
%     Take a sequence \(\{a_n\} \in C\). Since \(\{a_n\} \in K\) then it has a convergent subsequence \(b_k = a_{n_k}\). Let \(b \in K\) be the point of convergence of \(\{b_k\}\). Since \(\{b_k\} \in C\) and \(C\) is closed then \(b \in C\) and therefore \(C\) is compact.
% \end{proof}

% \begin{proposition}
%     A subset in \(\Reals^n\) is compact if and only if it is closed and bounded.
% \end{proposition}

% \begin{proof}
%     Using the same idea as \Cref{RealComplete} one can show the case for \(n = 1\). Assume the propsition is true for \(n = k - 1\) and let \(K \in \Reals^k\) be a closed and bounded set and \(\{a_n\} \in K\). Furthermore, let \(\{b_n\}\) be the projection of \(\{a_n\}\) onto \(\Reals^{k-1}\) and \(\{c_n\}\) be the projection of \(\{a_n\}\) on to \(k_{\cardinalTH}\) dimension. By induction, there exists a convergent subsequence \(\{b_{n_m}\}\). For \(\{c_{n_m}\}\) there exists a convergent subsequence \(\{c_{n_m}^i\}_i\) as well. It is easy to see that \(\{ a_{n_m}^i\}_i \) is a convergent subsequence of \(\{a_n\}\).
% \end{proof}

% \begin{corollary}
%     \([a,b]\) is compact in \(\Reals\).
% \end{corollary}

Let \(\set{a_n}\) be a sequence in \(\Reals\). We define:
\begin{align*}
    \limsup a_n & = \overline{\lim} \; a_n = \lim_{n \to \infty} \bracket[[\Big]]{\sup{\{a_k : k\geq n\}}}  \\
    \liminf a_n & = \underline{\lim} \; a_n = \lim_{n \to \infty} \bracket[[\Big]]{\inf{\{a_k : k\geq n\}}}
\end{align*}

\begin{note}
    The limits, \(\limsup a_n \) and \(\liminf a_n\), always exists. Albeit they might be infinite.
\end{note}

Let \(\set{a_n}\) be a bounded sequence in \(\Reals\), and \(A^*\) is the set of all limit points of all subsequence of  \(\set{a_n}\). We know that \(A^*\) is not empty and since  \(\set{a_n}\) is bounded and then \(A^*\) must be bounded as well. Thus, by completeness axiom, \(A^*\) has infimum and supremum. Moreover, \(\sup A^*, \inf A^* \in A^*\).

\begin{proposition}
    A bounded sequence \(\set{a_n}\) is convergent if and only if \(\;\limsup a_n = \liminf a_n\).
\end{proposition}

\begin{corollary}
    If \(K\) is a compact subset of \(\Reals\) then \(K\) has minimum and maximum. That is, there are \(M,m \in K\) such that \(\forall x \in K,\; m \leq x \leq M\).
\end{corollary}

\begin{proof}
    Since \(K\) is bounded then it has supremum and infimum in \(\Reals\). Obviously, there are convergent sequences \(\set{a_n}\) and \(\set{b_n}\) such that \(a_n \to m = \inf K\) and \(b_n \to M = \sup K\). By compactness of \(K\), \(M,m \in K\).
\end{proof}

\begin{theorem}
    \((X,d)\) and \((X',d')\) are metric spaces and \(K \subset X\) is compact. If \(f : X \to X'\) is continuous, then \(\func{f}{K}\) is a compact subset of \(X'\).
\end{theorem}

\begin{proof}
    Let \(\set{y_n} \in \func{f}{K}\) and \(\set{x_n} \in K\) are such that \(\func{f}{x_n} = y_n\). Since \(K\) is compact there is a convergent subsequence \(\set{x_{n_k}}\) and since \(f\) is continous \( \set{ y_{n_k} = \func{f}{x_{n_k}}}\) is also convergent. Hence \(\func{f}{K}\) is compact.
\end{proof}

\begin{corollary}
    Let \((X,d)\) be a metric space and \(f : X \to \Reals\) is continuous. If \(K\) is a compact subset of \(X\). Then \(f\) attains maximum and minimun in \(\Reals\).
\end{corollary}

\begin{note}
    For a continuous function \(f : X \to X'\) it is not necessary that the image of an open/closed set to be open/closed.
\end{note}

\begin{definition} [Uniform continuity]
    Let \((X,d)\) and \((X',d')\) be metric spaces.  \(f :X \to X'\) is uniformly continuous if:
    \begin{equation*}
        \forall \epsilon > 0\; \exists \delta > 0, \; x,y \in X, \; \func{d}{x,y} < \delta \implies \func{d'}{\func{f}{x},\func{f}{y}} < \epsilon
    \end{equation*}
\end{definition}

\begin{proposition}
    \(f: X \to X'\) is uniformly continous if and only if for every pair sequence \(\set{\pair{x_n}{y_n}}\) in \(X\) satisfying \(\func{d}{x_n,y_n} \to 0\) we have \(\func{d'}{\func{f}{x_n}, \func{f}{y_n}} \to 0\).
\end{proposition}

\begin{proof}
    Necessity: We have
    \begin{align*}
         & \forall \epsilon \; \exists \delta \  \suchThat \ \forall x,y \in X, \func{d}{x,y} < \delta \implies \func{d'}{\func{f}{x}, \func{f}{y}} < \epsilon \\
         & \forall \delta \; \exists N \ \suchThat \ n \geq N \implies \func{d}{x,y} < \delta
    \end{align*}
    combining the two brings us at the conclusion.

    Sufficiency: Suppose for the sake of contradtion that:
    \begin{equation*}
        \exists \epsilon \; \forall \delta \; \exists x,y \in X  \ \suchThat \ \func{d}{x,y} < \delta \land \func{d'}{\func{f}{x}, \func{f}{y}} \geq \epsilon
    \end{equation*}
    then let \(\delta = \frac{1}{n}\) and make the sequence pair \(\set{\pair{x_n}{y_n}}\). Clearly, \(\func{d}{x_n,y_n} \to 0\) therefore, \(\func{d'}{\func{f}{x}, \func{f}{y}} \to 0\). Which is a contradition since \(\func{d'}{\func{f}{x}, \func{f}{y}} \geq \epsilon \).
\end{proof}

\begin{proposition}
    \((X,d)\) and \((X',d')\) are matric spaces and \(X\) is compact. If \(f: X \to X'\) is continuous then it is uniformly continuous.
\end{proposition}

\begin{proof}
    Similarly, for the sake of contradicition suppose
    \begin{equation*}
        \exists \epsilon \; \forall \delta \; \exists x,y \in X \ \suchThat \ \func{d}{x,y} < \delta \land \func{d'}{\func{f}{x}, \func{f}{y}} \geq \epsilon
    \end{equation*}
    and let \(\delta = \frac{1}{n}\) and make the sequence pair \(\pair{x_n}{y_n}\). By compactness of \(X\), there are two convergent subsequence \(\set{x_{n_k}}\) and \(\set{y_{n_k}}\). Since \(\func{d}{x_n,y_n} \to 0\) then if \(x_{n_k} \to x\), \(y_{n_k} \to x\) as well. By continuity of \(f\), \(\func{f}{x_{n_k}} \to \func{f}{x}\) and \(\func{f}{y_{n_k}} \to \func{f}{x}\) and thus \(\func{d'}{\func{f}{x_{n_k}}, \func{f}{y_{n_k}}} \to 0 \). Which is a contradicition as for sufficiently large \(K\), \(k \geq K \implies \func{d'}{\func{f}{x}, \func{f}{y}} \geq \epsilon\)
\end{proof}

% Define the \textbf{diamter} of a set \(S\) to be:
% \begin{equation*} \diam{S} = \sup{\{\func{d}{s,s'} : s,s' \in S\}} \end{equation*}
% the cleary for bounded sets we have:
% \begin{equation*} \diam{S} < +\infty \end{equation*}

% \begin{proposition}
%     Let \((X,d)\) be a metric space and \(\{K_n\}\) is a sequence of compact subset of \(X\) with \(K_1 \supset K_2 \supset \dots\).
%     \begin{enumerate}
%         \item \(\bigcap K_n\) is not empty.
%         \item If \(\diam{K_n} \to 0\) then \(\bigcap K_n\) is a singular point.
%     \end{enumerate}
% \end{proposition}

% \begin{proof} \leavevmode
%     \begin{enumerate}
%         \item Consider the sequence \(\{a_n\}\) such that \(a_n \in K_n\). Since \(a_n \in K_1\) for all \(n\), then there is a convergent subsequence \(\{a_{n_k}\}\) with \(a_{n_k} \to a\). \(a \in K_1\), however, \(a \in K_2\) and so on, as well. Therefore \(a \in \bigcap K_n\).
%         \item Let \(a,b \in \bigcap K_n\). Then, \(a,b \in K_n\) for all \(n\) and we must have that \(\func{d}{a,b} \leq \diam K_n\). Therefore, \(a = b \).
%     \end{enumerate}
% \end{proof}

\begin{exercise}
    \item Prove that \(\sqrt{\mid x \mid} :\Reals \to \Reals\) is uniformly continuous.
\end{exercise}

\newpage
\section{Connectedness}
\begin{definition}
    Let \(X\) be a topological space. \(X\) is \textbf{disconnected} if there are non-empty open sets \(A\) and \( B\) are found such that
    \begin{equation*}
        A \bigcap B = \emptyset, \quad A \bigcup B = X
    \end{equation*}
    \(X\) is said to be \textbf{connected} if it is not disconnected. A subset \(S\) of \(X\) is connected if it is connected when considering its relative topolgy.
\end{definition}

\begin{definition}

\end{definition}
\begin{example}
    The following subsets of \(\Reals\) are disconnected:
    \begin{enumerate}
        \item \(S =\clop{-1}{0} \;\cup\; \opcl{0}{1}\).
        \item \(\Rationals\).
        \item \(S = \clcl{1}{0} \cup \clcl{1}{2}\).
    \end{enumerate}
\end{example}

\begin{definition}
    Two sets \(A\) and \(B\) in a topological space \(X\) are said to be \textbf{separable} if \(A \cap \closure B = \closure A \cap B = \emptyset\).
\end{definition}

\begin{proposition}
    \(X\) is connected if and only if it can not be written in form of two non-empty separable sets.
\end{proposition}

\begin{definition}
    \(S \subset \Reals\) is an \textbf{intervals} if when \(a , c \in S\) and \(a < b < c\) then \(b \in S\).
\end{definition}
\begin{example}
    \(\Reals\) and its intervals are connected. In fact the only connected subsets of \(\Reals\) are its intervals.
\end{example}
\begin{theorem}
    Let \(X\) and \(X'\) be two topological spaces. Let \(f : X \to X'\) be continuous and \(S\) be a connected subset of \(X\). Then, \(\func{f}{S}\) is connected in \(X'\).
\end{theorem}

\begin{corollary} [Mean value theorem]
    If \(f : \clcl{a}{b} \to \Reals\) is a continous function and \(\func{f}{a} = A, \func{f}{b} = B\) then for every \(C\) between \(A\) and \(B\) there exists a \(c \in \clcl{a}{b}\) such that \(\func{f}{c} = C\).
\end{corollary}
\begin{proposition}
    If \(S \subset X\) is a connected set then every \(S \subset T \subset \bar{S}\) is connected.
\end{proposition}

\begin{proposition}
    Let \(\set{E_{\alpha}}\) be collection of connected sets with \(\cap_{\alpha} E_{\alpha} \neq \emptyset\). Then, \(\cap_{\alpha} E_{\alpha}\) is connected.
\end{proposition}

\begin{definition}
    Let \(x\) be a point in a topological space \(X\). The \textbf{connected component} of \(x\), \(C_x\), is the union of the all the connected set including \(x\).
\end{definition}

\begin{proposition}
    For any \(x,y \in X\)
    \begin{enumerate}
        \item \(C_x\) is connected and closed.
        \item \(C_x\) and \(C_y\) are either disjoint or equal to each other.
    \end{enumerate}
\end{proposition}

\begin{definition}
    A topological space \(X\) is said to be \textbf{totally disconceted} if \(C_x = \set{x}\) for all \(x \in X\). A subset \(S\) of \(X\) is totally disconceted if it is totally disconceted when considering its relative topology.
\end{definition}

\begin{definition}
    The \textit{graph} of a function \(f:M \to N\), \(G_f\), given by \(G_f = \set<(x,\func{f}{x})>{ x \in M}\).
\end{definition}

\begin{theorem}
    The graph of a continous function over a connected set is connected.
\end{theorem}
\begin{example}
    Topological curve is connected and also its closure is connected.
\end{example}


\begin{definition}[Path connected]
    A set \(S\) is path connected if for every pair of points \(p,q \in S\) there exists a continous function \(\gamma: \clcl{a}{b} \to S\) such that \(\func{\gamma}{a} = p\) and \(\func{\gamma}{b} = q\).
\end{definition}
\begin{theorem}
    If a set \(S\) is path connected, then it is connected but the inverse is not true.
\end{theorem}
\begin{example}
    Infinite broom is path connected but toplogical sine curve is not.
\end{example}
\begin{proposition}
    If \(f\) is continuous function and \(S\) is a path connected set, then the image of \(S\) under \(f\) is path connected.
\end{proposition}
\begin{proposition}
    Every open set of \(\Reals\) is the union of countably many disjoint open intervals.
\end{proposition}

\section{Cantor set}
\begin{definition}
    define cantor set
\end{definition}

\begin{proposition}
    Cantor set is a perfect set.
\end{proposition}

\begin{proposition}
    Cantor set is totally disconnected.
\end{proposition}
\begin{theorem}
    Let \(K\) be a complete, totally disconnected, and compact metric space. Then, \(K\) is homeomorphic to Cantor set.
\end{theorem}

\begin{theorem}
    Let \(P\) be a non-empty perfect set in \(\Reals^k\). Then, \(P\) is uncountable.
\end{theorem}

%\chapter{Differentiation}
\thispagestyle{headings}

\begin{definition}
    Let \(I\) be an interval in \(\mathbb{R}\). If \(a\) is an interior point of \(I\), then we say that \(f: I \to \mathbb{R}\) is differentiable at \(a\) when the following limit exists:
    \[\lim_{x \to a} \dfrac{f(x) - f(a)}{x - a} \]
    The limit, if exists, is denoted by \(f'(a)\).
    If \(a\) is an end point and the length of the interval is greater than zero, then the limit only exists from one direction.

    Equivalently, there exists a line \(l\), not parallel to \(y\)-axis, in form of \(l : A(x) = mx + b\), that is tangent to \(f\) at \(x = a\). In this case:
    \[\lim_{x \to a}\dfrac{f(x) - [mx + b]}{x - a } = 0 \qquad A(a) = f(a) \]
\end{definition}
In a general case, two functions \(f,g\) are tangent to each other at \(x = a\) if:
\begin{equation}
    f(a) = g(a) \qquad \lim_{x \to a}{\dfrac{f(x) - g(x)}{x - a} = 0}
\end{equation}
\begin{corollary}\leavevmode
    \begin{enumerate}
        \item \(f\) is differentiable at \(a\) if it is continuous at \(a\).
        \item \label{extrma} If \(f'(a) > 0\), there exists \(\delta > 0\) such that for \(x \in \; ]a- \delta,a[\; \cap I \implies f(x) < f(a) \) and for \(x \in \; ]a, a +  \delta[\; \cap I \implies f(x) < f(a)\). And if \(f'(a) < 0\) the inequality sign are reversed. Therefore, if \(f\) has a local extremum at \(a\), then in case \(f'(a)\) exists, \(f'(a) = 0\).
    \end{enumerate}
\end{corollary}
\begin{example}
    a function that its derivate is not continuous (with \(\sin\frac{1}{x}\)).
\end{example}
\begin{theorem}[Rolle's theorem] \label{Rolle}
    Let \(f: [a,b] \to \mathbb{R}\) be a continuous and differentiable on the interval. If \(f(a) = 0, f(b) = 0\), then there exists \(c \in \; ]a,b[\) such that:
    \[f'(c) = 0\]
\end{theorem}
\begin{proof}
    If \(f \equiv 0\) on \([a,b]\) then its derivative \(f'(x) \equiv 0\) on \([a,b]\). If \(f(x) \neq 0\) for some \(x \in ]a,b[\) then it must have a non-zero maximum or minimum at some \(c \in ]a,b[\). Since \([a,b]\) is compact then by continuity of \(f\), \(f([a,b])\) is also compact in \(\mathbb{R}\) and therefore \(f\) attains its maximum or minimum. We know that at least one of its extremities must lie in \(]a,b[\), say point \(c\), hence by \Cref{extrma} \(f'(c) = 0\).
\end{proof}
\begin{theorem}[Mean value theorem]\label{MVT}
    Let \(f: [a,b] \to \mathbb{R}\) be a continuous and differentiable on the interval,then there exists \(c \in \; ]a,b[\) such that:
    \[f'(c) = \dfrac{f(b) - f(a)}{b - a}\]
\end{theorem}
\begin{proof}
    Define \(g(x) = f(x)  - f(a) - \dfrac{f(b) - f(a)}{b-a}(x-a)\). Then it is clear that \(g(a) = g(b) = 0\) and \(g\) is continous and differentiable on the interval. Then by \Cref{Rolle} there exists \(c \in ]a,b[\) such that \(g'(c) = 0\). Equivalently:
    \begin{align*}
         & g'(c) = f'(c) -  \dfrac{f(b) - f(a)}{b-a} = 0 \\
         & \implies f'(c) =  \dfrac{f(b) - f(a)}{b-a}
    \end{align*}
    which concludes the proof.
\end{proof}
\begin{corollary}[Growth Estimate]
    If \(|f'(x)| \leq M \) in \(]a,b[\) then \(f\) satisfies the global lipschitz condition for all \(x,y \in [a,b]\) \(|f(x) - f(y) | \leq M |x-y|  \).
\end{corollary}
\begin{corollary}
    Let \(f: [a,b] \to \mathbb{R}\) is continuous and \(f'(x) < 0\) (or \(f'(x) > 0\)) for all \(x \in ]a,b[\) then \(f\) is strictly increasing (or decreasing) on \([a,b]\).
\end{corollary}
\begin{theorem}
    \(f: [a,b] \to \mathbb{R}\) is continuous and differentiable on \(]a,b[\) then for \(f'(]a,b[)\) the intermediate value theorem holds and thus it is an interval.
\end{theorem}
\begin{proof}
    Let \(x_1, x_2 \in ]a,b[\ \). WLOG assume \(f'(x_1) < f'(x_2)\), we wish to prove that for all \(y^* \in ]f'(x_1), f'(x_2)[\) there is a \(x^* \in ]x_1,x_2[ \) such that \(f(x^*) = y^*\). Put \(g(x) = f(x) - y^*x\). By differentiability of \(f\) on \([a,b]\), \(g\) is differentiable on \([a,b]\). Then, \(g'(x_1) = f'(x_1) - y^* < 0\) and \(g'(x_2) =  f'(x_2) - y^* > 0\), therefore there are \(t_1,t_2 \in ]x_1,x_2[\) such that \(g(t_1) < g(x_1)\) and \(g(t_2) < g(x_2)\). Since \(g\) is continuous on \([x_1,x_2]\) then it must attains its minimum at some \(x^* \in [x_1,x_2]\). However \(x^* \) can't be \(x_1\) or \(x_2\) and hence \(x^* \in ]x_1,x_2[\). It is then easy to see that \(f'(x^*) = y^*\).
\end{proof}
\begin{definition}[Darboux continous]
    A function \(f\) is Darboux continous if it posseses the intermediate value property.
\end{definition}
For example \(f'\) of differentiable function is Darboux continuous.
\begin{theorem}[Cauchy's mean value theorem]
    \(f,g : [a,b] \to \mathbb{R}\) are continuous then there exists a \(c \in \; ]a,b[\), such that:
    \begin{equation*}
        f'(c)(g(b) - g(a)) = g'(c)(f(b) - f(a))
    \end{equation*}
\end{theorem}
\begin{proof}
    Define \(h(x) = (f(x) - f(a))(g(b) - g(a)) -  (g(x) - g(a))(f(b) - f(a))\), then clearly \(h(a) =0, h(b) =0\) and \(h(x)\) is continous and differentiable on \([a,b]\). Hence by applying the \cref{Rolle} for some \(c \in ]a,b[\) we have:
    \begin{align*}
         & h'(c)  = 0                                            \\
         & \implies f'(c)(g(b) - g(a)) -  g'(c)(f(b) - f(a)) = 0 \\
         & \implies  f'(c)(g(b) - g(a)) =  g'(c)(f(b) - f(a))
    \end{align*}
\end{proof}

\begin{theorem}[L'Hopital's rule]
    Suppose that \(\lim_{x \to a^+}{f(x)} = 0,\lim_{x \to a^+}{g(x)} = 0 \) where \(f,g\) are differentiable on a open interval \(I = ]a,b[\) for some \(b\) such that \(g'(x) \neq 0\) in \(I\) except maybe at \(x = a\) and the limit
    \begin{equation*}
        \lim_{x \to a^+}{\dfrac{f'(x)}{g'(x)}} = L
    \end{equation*}
    exists, then:
    \begin{equation*}
        \lim_{x \to a^+}{\dfrac{f(x)}{g(x)}} = L
    \end{equation*}
\end{theorem}
\begin{proof}
    For a fixed \(\epsilon > 0\) there exists a \(\delta > 0\) such that:
    \begin{equation*}
        \abs{x - a} < \delta \implies \abs{\dfrac{f'(x)}{g'(x)} - L} < \dfrac{ \epsilon}{2}
    \end{equation*}
    then since \(f(t), g(t) \to 0\) as \(t \to a\) from right side then there must be a \(t \in ]a,x[\) such that
    \begin{equation*}
        \abs{\dfrac{f(x) - f(t)}{g(x) - g(t)} - \dfrac{f(x)}{g(x)}} < \dfrac{\epsilon}{2}
    \end{equation*}
    then simply:
    \begin{align}
        \abs{\dfrac{f(x)}{g(x)} - L} & \leq \abs{\dfrac{f(x)}{g(x)} - \dfrac{f(x) - f(t)}{g(x) - g(t)} } + \abs{\dfrac{f(x) - f(t)}{g(x) - g(t)} - L} \\
                                     & < \dfrac{\epsilon}{2} + \abs{\dfrac{f'(\theta)}{g'(\theta)} - L}                                               \\
                                     & < \epsilon
    \end{align}
    Note that \(\theta \in \; ]t,x[\) and thus \(\abs{\theta - a} < \delta \)
\end{proof}
\begin{definition}[Higher order derivatives]
    \(f\) is said to be \(r_{\text{th}}\)-differentiable at \(x\) if it is differentiable \(r\) times. The \(r_{\text{th}}\) derivative of \(f\) is denoted as \(f^{(r)}\). If \(f^{(r)}\) exists for all \(r\) and \(x\) then \(f\) is said to be infinitely differentiable or smooth.
\end{definition}
\begin{definition}[Smoothness classes]
    The set of all \(f\) is continuosly \(r_{\text{th}}\)-differentiable is called class \(\mathcal{C}^r\).
\end{definition}
\begin{definition}[Taylor polynomial]
    The \(r_{\text{th}}\)-order Taylor polynomial of an \(r_{\text{th}}\)-order differentiable function \(f\) at \(x\) is
    \begin{equation*}
        P_r(x,h) =f(x) + f'(x)h +  \dfrac{f''(x)}{2}h^2 + \dots +  \dfrac{f^{(r)}(x)}{r!} h^r = \sum_{n = 0}^{r}\dfrac{f^{(n)}(x)}{n!} h^n
    \end{equation*}
\end{definition}
\begin{theorem}[Taylor approximation theorem]
    Let \(f\) be a \(r\)-differentiable function at \(x\) then:
    \begin{enumerate}
        \item
              \begin{equation*}
                  \dfrac{f(x+h) - P_r(x,h)}{h^r} \to 0 \text{ as } h \to 0
              \end{equation*}
        \item
              and \(P_r\) is the only \(r_{\text{th}}\) degree polynomial that has such property.
        \item
              Furthermore, if \(f\) is \(r\)-differentiable on an interval \(I\) for every \(x,y \in I\), there exists \(\xi\) between \(x,y\) such that:
              \begin{equation*}
                  f(y) - P_{r-1}(x,y-x) = \dfrac{f^{(r)}(\xi )}{(r)!}(y-x)^{r}
              \end{equation*}
    \end{enumerate}
\end{theorem}
\begin{proof} \leavevmode
    \begin{enumerate}
        \item
              For the base case \(r = 1\)
              \begin{align*}
                  \lim\limits_{h \to 0}{\dfrac{f(x+h) - f(x) -f'(x)h}{h}} = f'(x) - f'(x) = 0
              \end{align*}

              and by induction we prove the case \(r = n \geq 2\)
              \begin{align*}
                   & \lim\limits_{h \to 0}{\dfrac{f(x+h) - P_n(x,h)}{h^n}} = 0                                                                      \\
                   & \iff \forall \epsilon >0, \ \exists \delta > 0\ \text{such that } \ |h| < \delta \implies |f(x+h) - P_n(x,h)| < \epsilon |h^n|
              \end{align*}

              Let \(g(h) = f(x+h) - P_n(x,h)\) then since both \(f(x+h)\) and \(P_n(x,h)\) are differentiable then we apply \Cref{MVT}
              \begin{align*}
                  g(h) - g(0) & = g(h) = h(g'(c))                                                   \\
                              & = h(f'(x+c) - \sum_{k = 1}^{n}{\dfrac{f^{(k)}(x)}{(k-1)!}c^{k-1}})  \\
                              & =  h(f'(x+c) - \sum_{k = 0}^{n-1}{\dfrac{f^{(k+1)}(x)}{k!}c^{k}})   \\
                              & =  h(f'(x+c) - \sum_{k = 0}^{n-1}{\dfrac{f^{'^{(k)}}(x)}{k!}c^{k}})
              \end{align*}
              for some \(c \in ]0,h[\). Note that \(f'\) is \((n-1)\)-differentiable at \(x\) thus by induction for any \(\epsilon > 0\) there exists a \(\delta\) such that if \(c < \delta \) then:
              \begin{equation*}
                  |f'(x+c) - \sum_{k = 0}^{n-1}{\dfrac{f^{'^{(k)}}(x)}{k!}c^{k}}| < \epsilon |c^{n-1}|
              \end{equation*}

              which means
              \begin{align*}
                  |f(x+h) - P_n(x,h)| & = |g(h)| = |h| |f'(x+c) - \sum_{k = 0}^{n-1}{\dfrac{f^{'^{(k)}}(x)}{k!}c^{k}}| \\
                                      & < |h| \epsilon |c^{n-1}|< \epsilon |h^n|
              \end{align*}

              Therefore for any \(\epsilon\) if \(h < \delta\) then \(c < \delta\) and the result holds.
        \item
              Let \(Q_r(x,h)\) be another \(r_{\text{th}}\) degree polynomial such that
              \begin{equation*}
                  \lim\limits_{h \to 0}{\dfrac{f(x+h) -Q_r(x,h)}{h^r}} = 0
              \end{equation*}

              then
              \begin{equation*}
                  \lim\limits_{h \to 0}{\dfrac{P_r(x,h) -Q_r(x,h)}{h^r}} = 0
              \end{equation*}

              however this can only happen if \(Q_r(x,h) = P_r(x,h)\).
        \item
              Again for the base case \(r = 1\)
              \begin{equation*}
                  f(y) - f(x) = f'(\xi)(y-x)
              \end{equation*}

              which is the \Cref{MVT}. for \(r = n\) we have that
              \begin{equation*}
                  g(h) = f(x+h) - P_{n-1}(x,h) + Ch^n \implies g(0) = g'(0) = \dots = g^{(n-1)} =0
              \end{equation*}
              Set \(C\) such that \(g(y-x) = 0\). Then by applying \Cref{Rolle} \((n-1)\) times

              \begin{flalign*}
                  &&g(0) = g(y-x) = 0 &\implies g'(c_1) = 0 \quad c_1\in ]0,y-x[ &&\\
                  &&g'(0) = g'(c_1) = 0& \implies g'(c_2) = 0 \quad c_2 \in ]0,c_0[ &&\\
                  && &\vdots &&\\
                  &&g^{(n-2)}(0) = g^{(n-2)}(c_{n-2}) = 0 &\implies g^{(n-1)}(c_{n-1}) = 0 \quad c_{n-1} \in ]0,c_{n-2}[&&\\
                  &&g^{(n-1)}(0) = g^{(n-1)}(c_{n-1}) = 0 &\implies g^{(n)}(\xi - x) = 0 \quad \xi - x \in ]0,c_{n-1}[ \;\subset\; ]0,y-x[&&\\
                  &&\implies  g^{(n)}(\xi - x) =  f^{(n)}(\xi) + Cn! &= f^{(n)}(\xi) - \dfrac{n!}{(y-x)^n}(f(y) - P_{n-1}(x,y-x) ) = 0 &&\\
                  &&\implies f(y) - P_{n-1}(x,y-x)  &=  \dfrac{f^{(n)}(\xi)}{n!}(y-x)^n \qquad \xi \in ]x,y[&&
              \end{flalign*}
    \end{enumerate}
    which completes the proof.
\end{proof}

\begin{theorem}[Inverse function]
    Let \(I\) be an open set and \(f : I \to \mathbb{R}\) is continuous and differentiable such that its derivate is non-zero. Thus, \(f\) is either monotonic. Furthermore, it is one to one then it has a differentiable inverse \(f^{-1}\):
    \[f^{-1}(x) = \dfrac{1}{f'(f^{-1}(x))}\]
\end{theorem}
\begin{proof}
    limit algebra
\end{proof}

%\chapter{Integration}
\thispagestyle{headings}

\begin{definition}[Partition]
    \(I = [a,b] \in \mathbb{R}\). A partition of I is a finite ordered sequence of points in \(I\).
    \begin{equation*}
        P = \{x_0, x_1, \dots, x_n | a = x_0 \leq x_1 \leq \dots \leq x_n = b\}
    \end{equation*}
    A partition pair \((P,T)\) is set
    \begin{equation*}
        (P,T) = \{x_0, t_1,x_1, t_2,x_2 \dots x_{n-1}, t_n, x_n | a = x_0 \leq t_1 \leq x_1 \leq \dots \leq t_n \leq x_n = b\}
    \end{equation*}
    Moreover, define \(\|P\| = \max{(x_i - x_{i-1})}\).
\end{definition}
The Riemann sum of a function \(f\) on the interval \([a,b]\) with respect to the pair partition \((P,T)\) is:
\begin{equation*}
    R(f,P,T)= \sum_{i = 1}^{n} {f(t_i)(x_i - x_{i-1})}
\end{equation*}
\begin{definition}[Riemann Integrability]
    If there exist a number \(S\) that for all \(\epsilon > 0\) there exist a \(\delta > 0\) such that for all partition \((P,T)\) that if \(\|P\| < \delta\) implies \(|S - R(f,P,T)| < \epsilon\), then \(f\) is Riemann integrable and \(S\) is the integral of \(f\) on \(I\) denoted by
    \begin{equation*}
        \int_{a}^{b}{f}
    \end{equation*}
    . Furthermore, if \(S\) exists then it is unique.
    Denote the set of all Riemann integrable function on an interval \(I\) as \(\mathcal{R}_I\).
\end{definition}
\begin{theorem}
    Suppose \(f \in \mathcal{R}_I\) then \(f\) is bounded.
\end{theorem}
\begin{proof}
    By Riemann integrability of \(f\), for a \(\epsilon > 0\) there is \(\delta > 0\) such that for any partition pair \((P,T)\) that \(\|P\| < \delta\) then \(|S - R(f,P,T)| < \epsilon\). Consider twp partition pair \((P,T)\) and \((P,T')\) on \(I\) with  \(\|P\| < \delta\) and \(t_i = t'_i\) for all \(i\) except \(j\). Then by triangle inequality:
    \begin{align*}
         & |R(f,P,T) - R(f,P,T')| \leq |S - R(f,P,T)| + |S - R(f,P,T')| \leq 2 \epsilon       \\
         & \implies |R(f,P,T) - R(f,P,T')| = (x_j - x_{j-1})|f(t_j) - f(t'_j)| \leq 2\epsilon
    \end{align*}
\end{proof}
\begin{corollary}  \leavevmode
    \begin{enumerate}
        \item 	If \(f,g \in \mathcal{R}_I\) and \(c \in \mathbb{R}\) then \(f + cg \in \mathcal{R}_I\) and
              \begin{equation*}
                  \int_a^b{f +cg } =  \int_a^b{f} + c\int_a^b{g}
              \end{equation*}
        \item For a constant function \(f(x) = c \) its integral is \(c(b-a)\)
        \item If \(f(x) \geq 0\) then \(\int_{a}^{b}{f} \geq 0\)
    \end{enumerate}
\end{corollary}
\begin{definition}[Commen refinement]
    Let \(P_1,P_2\) be two partitions on an interval \(I\). Their common refinement \(P^* = P_1 \lor P_2\) is defined as
    \begin{equation*}
        P^* = \{z_0 \leq z_1 \leq \dots \leq z_{m} | z_i \in P_1 \cup P_2\}
    \end{equation*}
\end{definition}
\begin{definition}[Darboux Integral]
    Suppose \(f : [a,b] \to \mathbb{R}\) is a bounded function. Define the upper Darboux and lower Darboux sums with respect to a partition \(P\) as follow
    \begin{align*}
        \text{U}(f,P) & = \sum_{i = 1}^n{M_i(x_i - x_{i-1})} & M_i = \sup\limits_{x \in [x_{i-1},x_i]}{\{f(x)\}} \\
        \text{L}(f,P) & = \sum_{i = 1}^n{m_i(x_i - x_{i-1})} & m_i = \inf\limits_{x \in [x_{i-1},x_i]}{\{f(x)\}}
    \end{align*}
    Consider \(P'\) a refinement of \(P\), then the following inequalities hold:
    \begin{equation*}
        \text{L}(f,P) \leq \text{L}(f,P') \leq 	\text{U}(f,P') \leq \text{U}(f,P)
    \end{equation*}
    Therefore as the partition gets refined the upper sum decrease and the lower sum increase. Since both of these sums are bounded then by the completeness axiom the upper and lower integral
    \begin{align*}
        \overline{\int_{a}^{b}}f  & = \inf{\{\text{U}(f,P)\}} \\
        \underline{\int_{a}^{b}}f & = \sup{\{\text{L}(f,P)\}}
    \end{align*}
    exist. In case they are equal, \(f\) is said to be Darboux integrable.
\end{definition}
\begin{theorem}
    Darboux integrability is equivalent to Riemann integrability and the value of integrals are equal.
\end{theorem}

\begin{proof}
    Firstly, assume \(f\) is bounded and Darboux integrable. Equivalently, for any \(\epsilon_1 > 0\) there exists a partition \(P\) such that

    \begin{equation*}
        \text{U}(f,P) - \text{L}(f,P) <\epsilon_1
    \end{equation*}

    Let \(\|P\| = \delta_1 \) and \(0< \delta < \delta_1\) such that if a partition \(Q\) has \(\|Q\| < \delta\) then for all partition pairing \(|I - R(f,Q,T) | < \epsilon\). Consider \(P^* = Q \lor P\). It is clear that

    \begin{equation*}
        \text{U}(f,P^*) - \text{L}(f,P^*) <\epsilon_1 \quad \text{and} \quad \|P^*\| < \delta
    \end{equation*}

    To estimate \(\text{U}(f,Q)\) and \(\text{L}(f,Q)\) consider their difference with \(\text{U}(f,P^*)\) and \(\text{L}(f,P^*)\), respectively.

    \begin{equation*}
        \text{U}(f,Q) - \text{U}(f,P^*) = \sum_{i = 1}^{n}{M^Q_i (x^Q_i - x^Q_{i-1})} - \sum_{i = 1}^{n^*}{M^*_i (x^*_i - x^*_{i-1})}
    \end{equation*}

    The sums are different only in \(x^*_i \in P\). Therefore, their difference is in the intervals that are have an endpoint in \(P\) and for each of these interval the difference is \((M_j^Q - M_i^*)(x^*_i - x^*_{i-1})\), note that \(j\) is dependent on \(i\), hence

    \begin{equation*}
        \text{U}(f,Q) - \text{U}(f,P^*) = \sum_{i = 1}^{n^P}{((M_j^Q - M_i^*)(x^*_i - x^*_{i-1})} < 2Mn^P\delta
    \end{equation*}

    where \(M\) is the bound of \(f\). Similary for the lower bounds we get:

    \begin{equation*}
        \text{L}(f,P^*) - \text{L}(f,Q) = \sum_{i = 1}^{n^P}{((m_i^*- m_j^Q )(x^*_i - x^*_{i-1})} < 2Mn^P\delta
    \end{equation*}

    As a result if set \(\delta\) such that \(\text{U}(f,Q) - \text{L}(f,Q) < \epsilon\) we will be done, since for any partition \(T\) \(R(f,Q,T),I \in [ \text{L}(f,Q),\text{U}(f,Q)]\) hence \(| I - R(f,Q,T) < \epsilon\). To do so notice

    \begin{equation*}
        \text{U}(f,Q) - \text{L}(f,Q) = \text{U}(f,Q) - \text{U}(f,P^*) + \text{U}(f,P^*) - \text{L}(f,P^*)  + \text{L}(f,P^*) -  \text{L}(f,Q) < 4Mn^P\delta + \epsilon_1
    \end{equation*}

    which will be less \(\epsilon\) if

    \begin{equation*}
        \delta = \min{(\dfrac{\epsilon}{6Mn^P}, \delta_1)} , \qquad \epsilon_1 = \dfrac{\epsilon}{3}
    \end{equation*}

    Secondly, assum \(f\) is Riemann integrable. Then for any fixed \(\epsilon > 0\) then for any two pair partition \((P,T), (P,T')\) such that \(\|P\| < \delta\) then
    \begin{equation*}
        R(f,P,T) - R(f,P,T') < \dfrac{\epsilon}{3}
    \end{equation*}

    Then choose \(T\) such that
    \begin{equation*}
        \text{U}(f,P) - R(f,P,T) < \dfrac{\epsilon}{3}
    \end{equation*}
    that is, choose \(t_i\) such that
    \begin{align*}
         & M_i - f(t_i) < \dfrac{\epsilon}{3(b-a)}                                                                                                \\
         & \implies \sum_{i=1}^{n}{(M_i - f(t_i))(x_i - x_{i-1})} <\dfrac{\epsilon}{3(b-a)} \sum_{i = 1}^{n}{x_i - x_{i-1}} < \dfrac{\epsilon}{3}
    \end{align*}
    Similarly one can choose \(T'\) so that
    \begin{equation*}
        R(f,P,T')- \text{L}(f,P)  < \dfrac{\epsilon}{3}
    \end{equation*}
    Therefore:
    \begin{equation*}
        \text{U}(f,P) -  \text{L}(f,P) = \text{U}(f,P) - R(f,P,T) + R(f,P,T) - R(f,P,T') +  R(f,P,T')- \text{L}(f,P) < \epsilon
    \end{equation*}
\end{proof}
example: f and g differ in only one point.
\begin{definition}[Zero set]
    A set \(A \subset \mathbb{R}\) is a zero set if for each \(\epsilon > 0\) there is a countable covering of \(A\) of open intervals \(]a_i, b_i[\) such that:
    \begin{equation}
        \sum_{i = 1}^{\infty}{b_i - a_i} \leq \epsilon
    \end{equation}
    If a property holds for all points except those in a zero set then one says that the property holds almost everywhere.
\end{definition}

\begin{proposition}
    The following properties hold for zero sets:
    \begin{enumerate}
        \item
              Covering of \(A\) with open intervals is equivalent to covering with closed interval.
        \item
              A finit set is a zero set.
        \item
              A countable union of zero set is a zero set.
    \end{enumerate}
\end{proposition}
\begin{definition}[Oscillation]
    Suppose \(f : I \to \mathbb{R}\) where \(I\) is an interval and \(x \in I\) then the oscillation of \(f\) at \(x\) is
    \begin{align*}
        \text{Osc}(f,x) & =\limsup\limits_{t \to x} f(t) - \liminf\limits_{t \to x} f(t) \\
                        & = \lim\limits_{h \to 0} \diam{f([x-h,x+h])}
    \end{align*}
\end{definition}
\begin{proposition}
    \(f\) is continuous at \(x\) if and only if \(\text{Osc}(f,x) = 0\).
\end{proposition}
\begin{theorem}[Riemann-Lebesgue theorem]
    The function \(f\) is Riemann integrable if and only if it is bounded and the set of its discountinuities is zero set.
\end{theorem}
\begin{proof}
    First assume \(f\) is Riemann integrable. Let \(\mathcal{D}\) be the set of all its discontinuities. Moreover, \(\mathcal{D}_n = \{x | \text{Osc}(f,x) \geq \frac{1}{n}\}\). Thus it is clear that \(\mathcal{D} = \bigcup \mathcal{D}_n \). We will show that each \(\mathcal{D}_n\) is a zero set. By Riemann integrability of \(f\), for any \(\epsilon >0\) we have a partition \(P\) such that
    \begin{equation*}
        \text{U}(f,P) - \text{L}(f,P) < \dfrac{\epsilon}{n}
    \end{equation*}
    We call \([x_{i-1},x_i] \in P\) a bad interval if there exist \(x \in [x_{i-1},x_i]\) an interior point, such that \(x \in \mathcal{D}_n\).
    \begin{align*}
         & \sum_{\text{bad}}{(M_i - m_i)(x_i - x_{i-1})} < 	\text{U}(f,P) - \text{L}(f,P) < \dfrac{\epsilon}{n}                      \\
         & \dfrac{1}{n} \sum_{\text{bad}}{(x_i - x_{i-1}) } < \sum_{\text{bad}}{(M_i - m_i)( x_i - x_{i-1}) } < \dfrac{\epsilon}{n} \\
         & \qquad \implies \sum_{\text{bad}}{(x_i - x_{i-1}) } < \epsilon
    \end{align*}
    and since the endpoints are finite then \(\mathcal{D}_n\) is a zero set and therefore, \(\mathcal{D}\) is zero set.
    Seconde assume that \(f:[a,b] \to \mathbb{R}\) is bounded and \(\mathcal{D}\) is a zero set. Choose \(n\) such that:
    \begin{equation*}
        \dfrac{1}{n} < \epsilon_1
    \end{equation*}
    for \(\epsilon_1\) that is to be determined. Since \(\mathcal{D}_n \subset \mathcal{D} \) then it is a zero set as well. In other words for any \(\epsilon_2\) there is covering of \(\mathcal{D}_n\), \(I_1, I_2, \dots\) such that
    \begin{equation*}
        \sum{\diam{I_i}} < \epsilon_2
    \end{equation*}
    For any \(x \notin \mathcal{D}_n\) we know that is an open interval \(J_x\) such that \(M_{J_x} - m_{J_x} < \dfrac{1}{n}\). Let \(I = \bigcup I_i\) and  \(J = \bigcup J_x\).
    It is clear than \(I \cup J\) is a covering of \([a,b]\). Since \([a,b]\) is compact then the open covering has a Lebesgue number\(\lambda\). Let \(P\) be a partition such that \(\|P\| < \lambda\) then an interval \([x_{i-1}, x_i]\) is bad if it is wholly within a \(I_i\) and it is good if it is not.
    \begin{align*}
        \text{U}(f,P) - \text{L}(f,P) & = \sum_{\text{good}}{(M_i - m_i)(x_i - x_{i-1})}+ \sum_{\text{bad}}{(M_i - m_i)(x_i - x_{i-1})} \\
                                      & < \dfrac{1}{n} \sum_{\text{good}}{(x_i - x_{i-1})} + 2M \sum_{\text{bad}}{(x_i - x_{i-1})}      \\
                                      & < \dfrac{b-a}{n} + 2M\epsilon_2 < (b-a)\epsilon_1 +  2M\epsilon_2 = \epsilon
    \end{align*}
    by setting \(\epsilon_1 = \dfrac{\epsilon}{2(b-a)}\) and \(\epsilon_2 = \dfrac{\epsilon}{4M}\).
\end{proof}
\begin{corollary}
    \leavevmode
    \begin{enumerate}
        \item
              Any continuous function \(f:[a,b] \to \mathbb{R}\) is integrable.
              \begin{prooflemma}
                  Since there is no point of discontinuity then it is a zero set.
              \end{prooflemma}
        \item
              Any monotonic function \(f:[a,b] \to \mathbb{R}\) is integrable.
        \item
              Product of two integrable function is integrable.
    \end{enumerate}
\end{corollary}
\begin{theorem}[Fundamental theorem of calculus]
    If \(f\) is an integrable function then its indefinite integral
    \begin{equation*}
        F(x) = \int_{a}^{x}{f(t)dt}
    \end{equation*}
    is continuous at \(x\). Furthermore, its derivative is equal to \(f(x)\) at every point \(x\) that \(f\) is continuous.
\end{theorem}
\begin{definition}
    \(F(x)\) is anti-derivate of \(f(x): [a,b] \to \mathbb{R}\) if
    \begin{equation*}
        F'(x) = f(x)
    \end{equation*}
    for all \(x \in [a,b]\).
\end{definition}
\begin{corollary}
    Every continuous function has an anti-derivative.
\end{corollary}
\begin{theorem}
    Anti-derivate of a Riemann integrable function if exists differs from its indefinite integral by a constant.
\end{theorem}

%\chapter{Series}
\thispagestyle{headings}
\begin{theorem}
    A series \(s_n\) is convergent if and only if for each \(\epsilon > 0\) there exists a \(N \in \mathbb{N}\) such that:
    \begin{equation*}
        m,n \geq N \implies \abs{\sum_{i = n}{m}{a_m}} \leq \epsilon
    \end{equation*}
\end{theorem}
\begin{proof}
    Obviously any convergent sequence is Cauchy. Furthermore, due to completeness of \(\mathbb{R}\) every Cauchy sequence is convergent.
\end{proof}
\begin{corollary}
    The series \(s_n\) is convergent if \(a_n \to 0\).
\end{corollary}
\begin{theorem}	\leavevmode
    \begin{enumerate}
        \item
              If \(|a_n| < b_n\) for all \(n > N\) for a sufficiently large \(N\) then convergence of \(\sum{b_n}\) implies the convergence \(\sum{a_n}\).
        \item
              If \(0 < b_n < a_n\) for all \(n > N\) for a sufficiently large \(N\) then divergence of \(\sum{b_n}\) implies the divergence \(\sum{a_n}\).
    \end{enumerate}
\end{theorem}
\begin{corollary}
    Absolute convergence implies convergence.
\end{corollary}
\begin{theorem}[Integral Test]
    Consider the improper integral \(\int_{0}^{\infty}{f}\) and the series \(\sum_{i = 1}^{\infty}{a_k}\)
    \begin{enumerate}
        \item \(0 \leq a_k \leq f(x)\) for sufficiently large \(k\) and each \(x \in \; ]k-1, k]\), then the convergence of integral implies the convergence of the series.
        \item Similarly if \(0 \leq f(x) \leq a_k\) for sufficiently large \(k\) and each \(x \in \; [k,k+1[\), then the divergence of integral implies the divergence of the series.
    \end{enumerate}
\end{theorem}
\begin{definition}
    The exponential growth rate of the series \(\sum{a_n}\)
    \begin{equation*}
        \alpha = \limsup_{k \to \infty}{\sqrt[k]{a_k}}
    \end{equation*}
\end{definition}
\begin{theorem}[Root test]
    If \(\alpha < 1\) the series is convergent and if \(\alpha > 1\) it is divergent. If \(\alpha = 1\) the test is inconclusive.
\end{theorem}
\begin{theorem}[Ratio test]
    Let the ratio between successive terms of the series \(a_k\) be \({r_k = |\frac{a_{k+1}}{a_k}|}\)
    \begin{equation*}
        \rho = \limsup {r_k} \qquad \lambda  = \liminf{r_k}
    \end{equation*}
    If \(\rho < 1\) then the series converges, if \(\lambda > 1\)  then the series diverges, and otherwise the ratio test is inconclusive.
\end{theorem}
\begin{theorem}
    Let \(a_1 \geq a_2 \geq \dots \geq 0\) be a decreasing non-negative sequence then the alternating series
    \begin{equation}
        \sum_{n = 1}^{\infty}{a_n (-1)^n}
    \end{equation}
    is convergent.
\end{theorem}
\begin{theorem}
    Suppose \(\sum{c_k x^k}\) is a power series. Its radius of convergence \(R\) is unique and is such that for \(|x| < R\) the power series converges and for \(|x| > R\) diverges.
    \begin{equation*}
        R = \dfrac{1}{\limsup {\sqrt[k]{c_k}}}
    \end{equation*}
\end{theorem}

%\chapter{Function Spaces}
\thispagestyle{headings}

\begin{definition}[Point Convergence]
    for each point there is a \(\epsilon\)
\end{definition}
example \(x^n\), 1/2 lines, \(sqrt(x^2 + 1/n)\), rationals
\begin{definition}[Uniform Convergence]
    there is a \(\epsilon\) for all points
\end{definition}
\begin{corollary}
    Uniform convergence implies point convergence.
\end{corollary}
example x/n with restriction
\begin{theorem}
    \(f_n\) are uniformly convergent and continuous then \(f\) is continuous
\end{theorem}

\begin{definition}
    \(\mathcal{C}^0(X,\mathbb{R}):\) all continuous function from \(X \) to \(\mathbb{R}\). \(d(f,g) = \sup{\{|f(x) - g(x)| : x \in X\}}\)
\end{definition}
\begin{definition}
    \(\|f\| = \sup\{|f(x)| : x \in X\}\) therefore \(d(f,g) = \|f - g\|\)
\end{definition}

\begin{theorem}
    \(f_n:[a,b] \to \mathbb{R}\) uniformly convergent if \(f_n\) is Riemann integrable then \(f\) is Riemann integrable
    \[\int_{a}^{b} \underbrace{\lim{f_n}}_{f} = \lim{\int_{a}^{b}{f_n}}\]
\end{theorem}
\begin{lemma}
    metric spaces and \(f_n: X \to X'\) are uniformly convergent if \(f_n\) is bounded then \(f\) is bounded.
\end{lemma}

\begin{proposition}
    \((\mathcal{C}^0_b,d)\) is a complete metric space.
\end{proposition}
\begin{definition}
    \(\mathcal{C}^0_b(X,\mathbb{R})\) and 	\(\mathcal{C}_b(X,\mathbb{R})\) closed subset of \(\mathcal{B}(X,\mathbb{R})\)
\end{definition}
exampele: any compact metric space
\begin{theorem}
    \(f_n\) are differentiable functions
    \begin{enumerate}
        \item \(f_n'\) are uniformly convergent to \(g\)
        \item \(f_n\) are point convergent
    \end{enumerate}
\end{theorem}
\begin{proposition}
    \(f_n:[a,b] \to \mathbb{R}\) consider \(\sum{f_n}\)
    \begin{enumerate}
        \item \(f_n\) are riemann integrable and the series uniformly convergent then the \(\sum{f_n}\) is riemann integrable and
              \[ \int_{a}^{b}{\sum{f_n}} = \sum {\int_{a}^{b}{f_n}}\]
        \item
              Similarly for derivative
    \end{enumerate}
\end{proposition}
\begin{definition}[Wierstrass M test]
    This is super cool
\end{definition}
power series and convergence
radius of convergence of integral/derivative of power series is equal to the radius of convergence of the originalo series.
\begin{theorem}
    in the convergence cirle the power series is inifinitely integrable and differentiable, and coefficients are taylor coefficients
\end{theorem}
analytical definition
proof of analytical means analytical in interval using a pair sequence



\part{Multivariate Analysis}
\chapter{Linear Algebra}
\section{Vector Spaces}

\begin{definition}[Normed vector space]
    Let \(V\) be a vector space. A \textbf{norm} is a real valued function \(\norm{ \scdot }: V \to \Reals\) which has the following properties
    \begin{enumerate}
        \item \(\forall x \in V, \; \norm{x} > 0\).
        \item \(\norm{x} = 0 \implies x = 0\).
        \item \(\forall x \in V \; \forall \alpha \in \Field, \; \norm{\alpha x} = \abs{\alpha} \norm{x}\).
        \item \(\forall x,y \in V \; \norm{x+y} \leq \norm{x} + \norm{y}\).
    \end{enumerate}
\end{definition}

Each normed vector space induces a metric space \(\metricSpace{V}{d}\) where \(\func{d}{x,y} = \norm{y - x}\).

\begin{theorem}
    In every normed space \(\normedSpace{V}{\norm{\scdot}}\) we have
    \begin{equation*}
        \abs{ {\norm{v} - \norm{w}}} \leq \norm{v - w}
    \end{equation*}
    Hence the norm is Lipschitz continuous.
\end{theorem}


\begin{definition}
    Assume \(V\) is a vector space and let \(\norm{\scdot}_1, \; \norm{\scdot}_2\) be two norms for \(V\). They are said to be equivalent when
    \begin{equation*}
        \exists c_1,c_2 > 0 \; \forall x : \qquad c_1 \norm{x}_1 \leq \norm{x}_2 \leq c_2 \norm{x}
    \end{equation*}
\end{definition}

To check if the above definition is indeed an equivalence relation, we must show that following:
\begin{description}
    \item [Reflexive] \(\norm{\scdot}_1 \sim \norm{\scdot}_1\).
    \item [Symmetric] \(\norm{\scdot}_1 \sim \norm{\scdot}_2 \implies \norm{\scdot}_2 \sim \norm{\scdot}_1\).
    \item [Transitive] \( \norm{\scdot}_1 \sim \norm{\scdot}_2 , \; \norm{\scdot}_2 \sim \norm{\scdot}_3 \implies \norm{\scdot}_1 \sim \norm{\scdot}_3\).
\end{description}

\begin{remark}
    Equivalent norms induce equivalent metrics, hence they induce the same topology.
\end{remark}

\begin{theorem} \label{th:normsEquivalent}
    All norms defined on a finite dimensional vector space \(V\) are equivalent.
\end{theorem}

\begin{proof}
    Let \(\norm{\cdot}\) be an arbitrary norm on \(V\) and \(\{e_1, e_2, \dots , e_n\} \) be a basis of \(V\). Let \(\norm{\cdot}_2\) be \(L_2\)-norm (Euclidean norm). It will suffice to show \(\norm{\cdot} \sim \norm{\cdot}_2\). Let
    \begin{equation*}
        M = \max \! \left( \norm{e_1}, \dots , \norm{e_n} \right)
    \end{equation*}
    Take \(x \in V\), writing \(x = \sum_{i = 1}^n {\xi_i e_i}\) we have:
    \begin{equation*}
        \norm{x} = \norm{\sum_{i = i}^n {\xi_i e_i}} \leq \sum_{i = 1}^n \abs{\xi_i} \norm{e_i} \leq M \sqrt{n} \norm{x}_2
    \end{equation*}
    Taking \(c_2 = M \sqrt{n}\) proves the right inequality. For the left inequality we need the following lemma
    \begin{lemma} \label{lm:ContinuityOfNorm}
        If \(V\) is a normed vector space with \(\norm{\cdot}_2\), as defined above, is viewed as metric space \(\metricSpace{V}{\norm{\cdot}_2}\) then \(\norm{\cdot} : V \to \Reals\) is continuous.
    \end{lemma}

    \begin{prooflemma}
        Let \(x_0 \in V\) and \(M\) be defined as above. For any \(\epsilon > 0\) consider \(\delta = \frac{\epsilon}{M \sqrt{n}}\) then if \(\norm{x - x_0}_2 < \delta\)
        \begin{equation*}
            \abs{{\norm{x} - \norm{x_0}}} \leq \norm{x - x_0} \leq M \sqrt{n} \norm{x - x_0} \leq \epsilon
        \end{equation*}
    \end{prooflemma}

    Now consider the sphere of radius \(r = 1\) centered at \(0\), \(\func{S_1}{0} = S_1 = \{x \in V : \norm{x}_2 = 1\}\). One can show that \(S\) is compact (\Cref{th:CompactnessOfFiniteDimensional}). Therefore, \(\norm{x}\) assumes its minimum on \(S\). Let \( a = \norm{x_0}\) be the minimum. Since \(0 \notin S\) then \(a > 0\). By letting \(y = x / \norm{x}_2 \), we have \(y \in S\) and thus \(a \leq \norm{y}\) which is
    \begin{equation*}
        a \norm{x}_2 \leq \norm{x}
    \end{equation*}
    Taking \(c_1 = a\) proves the theorem.
\end{proof}

\begin{theorem}\label{th:CompactnessOfFiniteDimensional}
    Let \(\normedSpace{V}{\norm{\cdot}}\) be a normed space over a normed complete field \(\Field\). The following are equivalent
    \begin{enumerate}
        \item \(V\) is finite dimensional. \label{it:COFD_1}
        \item every bounded closed set in \(V\) is compact. \label{it:COFD_2}
        \item the closed unit ball in \(V\) is compact. \label{it:COFD_3}
    \end{enumerate}
\end{theorem}
\begin{proof}
    \Cref{it:COFD_1} \(\implies\) \Cref{it:COFD_2}: It is similar to proving a closed set \(\Reals^n\) is compact using the fact a closed interval is compact in \(\Reals\).

    \Cref{it:COFD_2} \(\implies\) \Cref{it:COFD_3}: Trivial.

    \Cref{it:COFD_3} \(\implies\) \Cref{it:COFD_1}: Requires the following lemma:
    \begin{lemma}[Riesz's lemma] \label{lm:RieszsLemma}
        If \(V\) is a normed vector space and \(W\) is a closed proper space of \(V\) and \(\alpha \in \Reals\) with \(0 < \alpha < 1\), then there exists an \(v \in V\) with \(\norm{v} = 1\) such that \(\norm{v- w} \geq \alpha \) for all \(w \in W\).
    \end{lemma}
    Now suppose \(V\) were to be an infinite dimensional vector space. Then by the \Cref{lm:RieszsLemma} there is sequence of unit vectors \({x_n}\) such that \(\forall m,n \in \Naturals, \; \norm{x_n - x_m} > \alpha\) for some \(0 <\alpha < 1\). Which implies that no subsequence of \(\set{x_n}\) is convergent and hence the closed unit ball can not be compact.
\end{proof}

\begin{example}
    The closed unit ball in the infinite dimensional vector space \(\func{C}{\clcl{0}{1},\Reals}\) with \(\norm{f} = \max \func{f}{x}\) is not compact.  Take \(\func{f_n}{x} = x^n\). Obviously \(\norm{f_n} = 1\), however \(f_n\) doesn't uniformly converge and hence \(f_n\) doesn't have a limit in \(\func{C}{\clcl{0}{1},\Reals}\) with the \(\max\) norm. Consider the following norm
    \begin{equation*}
        \norm{f}_I = \int_0^1 \abs{\func{f}{x}} \diffOperator x
    \end{equation*}
    Note that \(\norm{\cdot}_I\) and \(\norm{\cdot}_{\max} \) are not equivalent. Let \(\func{g}{x} = 0\) for all \(x \in \clcl{0}{1}\). Then
    \begin{equation*}
        \norm{f_n - g}_I = \dfrac{1}{n+1} \to 0 \quad \text{as} n \to \infty.
    \end{equation*}
\end{example}

\begin{definition}[Banach space]
    A normed vector space \(V\) that is complete is a \textbf{Banach space}. A \textbf{Hilbert Space} is a Banach space whose norm is induced by an inner product.
\end{definition}


\begin{proposition} \label{pr:FiniteIsBanach}
    A normed finite dimensional vector space \(V\) over a normed complete field \(\Field\), is Banach space.
\end{proposition}

\begin{proof}
    Let \(\set{v_i} \in V\) be a Cauchy sequence, and \(\set{e_1, \dots ,e_n}\) be a basis for \(V\) with the norm \(L^1\), that is if \(v = (\xi^1 , \dots ,\xi^n)\) then \(\norm{v} = \sum_{m = 1}^n \abs{\xi^m}\). Then if \(v_i = (\xi^1_i , \dots , \xi^n_i)\)
    \begin{equation*}
        \abs{\xi^m_i - \xi^m_j} \leq \sum_{m=1}^n \abs{\xi^m_i - \xi^m_j} \leq \norm{v_i - v_j} < \epsilon
    \end{equation*}
    then \(\set{\xi^m_i}_i\) are a Cauchy sequence in \(\Field\) and hence they converge \(\xi^m_i \to \xi^m\). Then, clearly \(v_i \to v = (\xi^1 , \dots , \xi^n)\) as each component converges.
\end{proof}

\begin{example}
    \(\Rationals\) form a vector space itself over itself. It is finite dimensional as \(\set{\DSOne_\Rationals}\) is the basis, however the sequence
    \begin{equation*}
        1 ,\; 1.4 ,\; 1.41 ,\; \dots
    \end{equation*}
    does not converge even though it is Cauchy.
\end{example}

\section{Linear Maps}
Let \(V\) and \(W\) be a vector spaces over \(\Field\). A map \(T: V \to W\) is \textbf{linear} if
\begin{equation*}
    \func{T}{x + \lambda y} = \func{T}{x} + \lambda \func{T}{y}
\end{equation*}
for all \(x,y \in V\) and \(\lambda \in \Field\).

\begin{definition}
    Let \(\normedSpace{V}{\norm{\cdot}_V}\) and \(\normedSpace{W}{\norm{\cdot}_W}\) be normed spaces then, a linear transformation \(T : V \to W\) is \textbf{bounded} if there exists a constant \(C > 0\) such that
    \begin{equation*}
        \norm{Tv}_W \leq C \norm{v}_V
    \end{equation*}
    for all \(v \in V\). We denote the set of all linear map from \(V \to W\) as \(\func{\calL}{V,W}\) and the set of all bounded linear maps as \(\func{\calB}{V,W}\). If \(T \in \func{\calL}{V,W}\) is bijective such that \(T^{-1} \in \func{\calL}{V,W}\), then \(T\) is called an \textbf{isomorphism} and \(V,W\) are \textbf{isomorphic}. An operator \(T \in \func{\calL}{V,W}\) is called \textbf{isometric} if \(\norm{Tv}_W = \norm{v}_V\) for all \(v \in V\).
\end{definition}

\begin{definition}
    If \(\normedSpace{V}{\norm{\cdot}_V},\normedSpace{W}{\norm{\cdot}_W}\) are normed spaces then the \textbf{operator norm} of a linear transformation \(T : V \to W\) is
    \begin{equation*}
        \norm{T} = \sup \left\{\dfrac{\norm{Tv}_W}{\norm{v}_V} \middle| v \neq 0 \right\}
    \end{equation*}
\end{definition}

\begin{proposition}
    Let \(T : U \to V\) and \(T' : V \to W\) be two linear transformations.
    \begin{equation*}
        \norm{T' \circ T} \leq \norm{T} \norm{T'}
    \end{equation*}
\end{proposition}

\begin{proof}
    for an arbitrary non-zero \(x \in U\)
    \begin{equation*}
        \norm{\func{T' \circ T}{x}}_W \leq \norm{T'} \norm{Tx}_V \leq \norm{T'} \norm{T} \norm{x}_U
    \end{equation*}
    which implies
    \begin{equation*}
        \norm{T' \circ T} \leq \norm{T} \norm{T'}
    \end{equation*}
\end{proof}

\begin{theorem} \label{th:linearTransformation}
    Let \(\normedSpace{V}{\norm{\cdot}_V}\) and \(\normedSpace{W}{\norm{\cdot}_W}\) be normed spaces and \(T: V \to W\) be a linear transformation. The following are equivalent
    \begin{enumerate}
        \item \(\norm{T}\) is finite. \label{it:LT_1}
        \item \(T\) is bounded. \label{it:LT_2}
        \item \(T\) is Lipschitz continuous. \label{it:LT_3}
        \item \(T\) is continuous at a point. \label{it:LT_4}
        \item \(\sup_{\norm{v}_V = 1} \norm{Tv}_W < \infty\). \label{it:LT_5}
    \end{enumerate}
\end{theorem}

\begin{proof}
    \cref{it:LT_1} \(\Rightarrow\) \cref{it:LT_2}: Obviously
    \begin{align*}
        \dfrac{\norm{Tv}_W}{\norm{v}_V} & \leq \norm{T}            \\
        \implies \norm{Tv}_W            & \leq \norm{T} \norm{v}_V
    \end{align*}
    note that if \(v = 0\) then \(Tv = 0\) as well and thus the last inequality holds for all \(v \in V\).

    \cref{it:LT_2} \(\Rightarrow\) \cref{it:LT_3}:
    \begin{equation*}
        \norm{Tv - Tu}_W = \norm{T(u - v)}_W \leq C \norm{u - v}_V
    \end{equation*}

    \cref{it:LT_3} \(\Rightarrow\) \cref{it:LT_4}: Trivial.

    \cref{it:LT_4} \(\Rightarrow\) \cref{it:LT_5}: Let \(T\) be continuous at \(u \in V\). Then there is  a \(\delta > 0 \) such that
    \begin{equation*}
        \norm{v-u} < \delta \implies \norm{Tv - Tu}_W = \norm{T(v-u)}_W < 1
    \end{equation*}
    Now for an arbitrary non-zero \(v\) we have
    \begin{equation*}
        \norm{\left( \dfrac{\delta v}{2\norm{v}_V} + u \right) - u}_V < \delta
    \end{equation*}
    Therefore
    \begin{align*}
         & \norm{\func{T}{\dfrac{\delta v}{2\norm{v}_V}}}_W  < 1         \\
         & \norm{\func{T}{\dfrac{v}{\norm{v}_V}}}_W  < \dfrac{2}{\delta}
    \end{align*}

    \cref{it:LT_5} \(\Rightarrow\) \cref{it:LT_1}: Let \(v \in V\) be an arbitrary vector. Then
    \begin{align*}
                 & \sup \norm{\func{T}{\dfrac{v}{\norm{v}_V}}}_W < \infty \\
        \implies & \sup \dfrac{\norm{Tv}_W}{\norm{v}_W} < \infty
    \end{align*}

\end{proof}




\begin{theorem} \label{th:finiteDimensionTransformationContinuous}
    If \(V\) is a finite dimensional normed vector space then any linear transformation \(T : V \to W\) is continuous.
\end{theorem}

\begin{proof}
    Since \(V\) is finite dimensional, according to \Cref{th:normsEquivalent}, any two norms are equivalent. Hence, take \(\norm{\cdot}_2\) to be Euclidean norm over a basis \(\{e_1, \dots , e_n\}\). Let \(x\) be such that \(\norm{x}_2 < \delta\) for some \(\delta > 0\). Therefore, \(\abs{\xi_i} < \delta^2\)
    \begin{equation*}
        \norm{Tx}_W = \norm{\sum_{i = 1}^n \xi_i \func{T}{e_i}}_W \leq \sum_{i = 1}^n \abs{\xi_i} \norm{\func{T}{e_i}}_W \leq \delta^2 K
    \end{equation*}
    where \(K = \max \norm{\func{T}{e_i}}_W \). By letting \(\delta = \sqrt{\frac{\epsilon}{K}}\) we proved continuity at \(0\) and hence the continuity by \Cref{th:linearTransformation}.
\end{proof}

Another proof of \Cref{pr:FiniteIsBanach}

\begin{proof}
    Let \(\{e_1, \dots , e_n\}\) be a basis for \(V\) and \(\phi : V \to \Field^n\) be the representation map for the basis. Since \(\phi\) is a linear map and a bijection then \(\phi\) is homeomorphism. Consider a Cauchy sequence \(\set{v_k} \in V\) and let \(x_k = \func{\phi}{v_k}\) then by continuity of \(\phi\) and \(\phi^{-1}\) we have
    \begin{equation*}
        \abs{x_i - x_j} = \abs{\func{\phi}{v_i} - \func{\phi}{v_j}} \leq \norm{\phi} \norm{v_i - v_j} \leq \norm{\phi} \norm{\func{\phi^{-1}}{x_i} - \func{\phi^{-1}}{x_j}} \leq \norm{\phi} \norm{\phi^{-1}} \abs{x_i - x_j}
    \end{equation*}
    hence \(\set{x_k}\) are Cauchy in \(\Field^n\) which by completeness of \(\Field\) implies that they are convergent, \(x_k \to x\). Let \(v = \func{\phi^{-1}}{x}\) then by the right side of the inequality \(v_k \to v\).
\end{proof}

\begin{remark}
    As seen in the last proof, for a bijective linear transformation \(T\)
    \begin{equation*}
        1 \leq \norm{T} \norm{T^{-1}}
    \end{equation*}
\end{remark}

\begin{definition}[Dual space]
    Let \(V\) be a normed space over the normed field \(\Field\), then the \textbf{topological/continuous dual space} of the normed space \(V\) is
    \begin{equation*}
        V^\ast  = \func{\calL}{V,\Field}
    \end{equation*}
    elements of \(V^\ast\) are called \textbf{bounded functionals} on \(V\).
\end{definition}

\begin{remark}
    Dual space is defined for all vector spaces, however, in analysis we study the topological dual space which only in the finite dimensional case coincide with the algebraic dual space.
\end{remark}

\begin{proposition}
    For a finite dimensional normed vector space \(V\), \(\dim V^\ast = \dim V\).
\end{proposition}

\begin{proof}
    Let \(\set{e_1, \dots , e_n}\) be a basis for \(V\) then, consider the following linear functions
    \begin{equation*}
        e^\ast_1, \dots , e^\ast_n \in V^\ast
    \end{equation*}
    where
    \begin{equation*}
        \func{e^\ast_i}{e_j} = \begin{cases}
            1 & i = j    \\
            0 & i \neq j
        \end{cases}
    \end{equation*}
    we claim that \(\set{e^\ast_1, \dots , e^\ast_n}\) is a basis for \(V^\ast\). It is easy to see that they are as for each \(j\)
    \begin{equation*}
        \left[\sum_{i = 1}^n c_i e^\ast_i\right] e_j = c_j
    \end{equation*}
    and for each \(\phi \in \func{\calL}{V,\Field}\) we have
    \begin{equation*}
        \func{\phi}{e_j} = \alpha_i =  \sum_{i = 1}^n \alpha_i \func{e^\ast_o}{e_i}\\
    \end{equation*}
    hence \(\dim V^\ast = n = \dim V\).
\end{proof}

\begin{theorem}
    For two normed vector spaces \(V,W\), \(\normedSpace{\func{\calB}{V,W}}{\norm{T}}\) is a normed vector space. Moreover, it is a Banach space when \(W\) is a Banach space.
\end{theorem}


\begin{proof}
    Clearly \(\func{\calB}{V,W}\) is a vector space. For its norm \(\norm{T}\) we have
    \begin{enumerate}
        \item \(\norm{T} \geq 0\) by definition.
        \item if \(\alpha \in \Field_W\) then
              \begin{equation*}
                  \norm{\alpha T} = \sup \left\{ \dfrac{\norm{(\alpha T)v}_W}{\norm{v}_V} \middle| v \neq 0 \right\} = \abs{\alpha} \sup \left\{ \dfrac{\norm{Tv}_W}{\norm{v}_V} \middle| v \neq 0 \right\} = \abs{\alpha} \norm{T}
              \end{equation*}
        \item for the triangle inequality
              \begin{align*}
                  \norm{T_1 + T_2} & = \sup \left\{ \dfrac{\norm{(T_1 + T_2)v}_W}{\norm{v}_V} \right\}                                                     \\
                                   & \leq \sup \left\{ \dfrac{\norm{T_1v}_W + \norm{T_2v}_W}{\norm{v}_V} \right\}                                          \\
                                   & = \sup \left\{ \dfrac{\norm{T_1v}_W}{\norm{v}_V} \right\} + \sup \left\{ \dfrac{\norm{ T_2 v}_W}{\norm{v}_V} \right\} \\
                                   & = \norm{T_1} + \norm{T_2}
              \end{align*}
    \end{enumerate}
    Suppose \(W\) is a Banach space and \(\{T_i\} \in \func{\calB}{V,W}\) is a Cauchy sequence. Then for all \(v \in V\)
    \begin{equation*}
        \forall \epsilon \, \exists N \; \suchThat \; m,n > N \implies \norm{T_m v - T_n v}_W \leq \norm{T_m - T_n}\norm{v}_V < \epsilon
    \end{equation*}
    \(\{T_i v\}\) is a Cauchy sequence. Since \(W\) is complete then \(T_i v \to Tv\) for some function \(T\). We claim that \(T\) is a bounded linear map and is the limit of \(T_i \to T\).
    \begin{align*}
        \func{T}{v + cu} & = \lim_{i \to \infty} \func{T_i}{v + cu} = \lim_{i \to \infty} T_i v + c T_i u \\
                         & = Tv + c Tu
    \end{align*}
    Note that  \( \abs{{\norm{T_m} - \norm{T_n}}} \leq \norm{T_m - T_n}\) and hence \(\norm{T_i}\) is a Cauchy in sequence in \(\Reals\) that has a limit \(t\). There exists a \(N\) such that \(\abs{{\norm{T_n} - t}} < 1\) for all \(n \geq N\).
    \begin{equation*}
        \dfrac{\norm{Tv}_W}{\norm{v}_V} = \lim_{i \to \infty}  \dfrac{\norm{T_i v}_W}{\norm{v}_V} < t + 1
    \end{equation*}
    hence \(T\) is bounded and \(T \in \func{\calB}{V,W}\). Finally, we show that \(T_i \to T\). For an arbitrary \(v \neq 0\) and \(\epsilon > 0\) there exist \(N\) such that
    \begin{align*}
        n \geq N \implies \norm{T_i v - Tv}_W < \epsilon \norm{v}_V
    \end{align*}
    which means that
    \begin{equation*}
        \norm{T_i - T} = \sup \dfrac{\norm{T_i v - Tv}_W}{\norm{v}_V} < \epsilon
    \end{equation*}
    Therefore \(T_i \to T\) as desired.
\end{proof}

\begin{theorem}
    Let \(\normedSpace{V}{\norm{\cdot}}\) be a normed space. Then any linear transformation \(T: \Reals^n \to V\) is continuous. Furthermore, if \(T\) is a bijection, it is a homeomorphism.
\end{theorem}

\begin{proof}
    Since \(\Reals^n\) is finite then by \Cref{th:finiteDimensionTransformationContinuous}, \(T\) is continuous. Assuming \(T\) is bijective, we must show that its inverse \(T^{-1}\) is continuous as well. Since \(T\) is a bijection then \(T\) is a linear isomorphism and \(\dim V = \dim \Reals^n = n\) hence \(T^{-1}: V \to \Reals^n\) is a continuous map.
\end{proof}


\begin{theorem} \label{th:LinearInvertibility}
    Let \(V,W\) be two finite dimensional normed vector spaces. \(T : V \to W\) linear transformation is invertible if and only if there exists a \(c\) such that:
    \begin{equation*}
        c \norm{v}_V \leq \norm{Tv}_W
    \end{equation*}
\end{theorem}

\begin{proof}
    If \(T\) is invertible then \(T^{-1} : W \to V \) is bounded and thus
    \begin{equation*}
        \norm{T^{-1}w}_V \leq c \norm{w}_W
    \end{equation*}
    and since \(T\) is bijective then there exists \(v\) such that \(w = Tv\) which implies
    \begin{equation*}
        \norm{y}_V \leq c\norm{Ty}_W
    \end{equation*}
    If there exists such \(c\) then \(\norm{Tx} > 0\) for all non-zero \(x\) and hence \(\ker T  = 0\) which implies that \(T\) is a bijection and is invertible.
\end{proof}

\begin{remark}
    the supremum of such \(c\) is \(\norm{T^{-1}}^{-1}\) which is called the \textbf{conorm} of \(T\).
\end{remark}

\begin{definition}[General linear group]
    The \textbf{general linear group} of a vector space, written \(\func{\GL}{V}\) is the set of all bijective linear transformation.
\end{definition}

\begin{proposition}
    If \(V\) is a finite (also works for infinite) vector space then \(\func{\GL}{V}\) is open in \(\func{\calL}{V,V}\), in fact, if \(f \in \func{\GL}{V}\) then the open ball centered at \(f\) with radius \(\norm{f^{-1}}^{-1}\) remains in \(\func{\GL}{V}\). Furthermore, the inverse operator \(i : \func{\GL}{V} \to \func{\GL}{V}\), \(\func{i}{T} = T^{-1}\) is continuous.
\end{proposition}

\begin{proof}
    First assume \(f = \DSOne_V\) then we prove that any linear \(g\) that \(\norm{\DSOne_V - g} < 1\) is invertible which then implies bijectivity (true for linear maps). Let \(\norm{v} = 1\) then
    \begin{equation*}
        \abs{\norm{v} - \norm{gv}} \leq \norm{v - gv} \leq \norm{\DSOne_V - g} \norm{v} < 1
    \end{equation*}
    Therefore
    \begin{equation*}
        0 < \norm{gv} < 2
    \end{equation*}
    which means \(\ker g = \set{0}\) and since \(V\) is finite then then \(g\) is invertible. For a general \(f\), we have that
    \begin{equation*}
        \norm{1 - f^{-1} \circ g} \leq \norm{f^{-1}}\norm{f -g} < 1
    \end{equation*}
    therefore \(f^{-1} \circ g\) is invertible and as a consequence \(g = f \circ f^{-1} \circ g\) is invertible. To prove inverse operator is continuous, fix \(\epsilon > 0\) then for a \(\delta > 0\) if \(\norm{T-S} < \delta\) then
    \begin{align*}
        \norm{\DSOne_V- T^{-1} \circ S}= \norm{T^{-1} \circ T  - T^{-1} \circ S}           & \leq \norm{T^{-1}} \norm{T-S} < \delta \norm{T^{-1}} \\
        \implies  \norm{T^{-1} - S^{-1}} \leq \norm{T^{-1}\circ S - \DSOne_V}\norm{S^{-1}} & < \delta \norm{T^{-1}} \norm{S^{-1}}
    \end{align*}
    note that by letting \(\delta = \norm{T^{-1}}^{-1}/2\) then
    \begin{equation*}
        \norm{S} > -\dfrac{\norm{T^{-1}}^{-1}}{2} + \norm{T} > \dfrac{\norm{T^{-1}}^{-1}}{2}
    \end{equation*}
    also if for any invertible linear map \(R\)
    \begin{equation*}
        \norm{R} > a \implies \norm{Rx} > a\norm{x} \implies \dfrac{\norm{y}}{a} = \dfrac{\norm{R\circ \func{R^{-1}}{y}}}{a} > \norm{R^{-1}y}
    \end{equation*}
    which means that \(\norm{S^{-1}} < 2 \norm{T^{-1}}\), hence by letting
    \begin{equation*}
        \delta = \min \set{\dfrac{\epsilon \norm{T^{-1}}^2}{2} , \dfrac{\norm{T^{-1}}^{-1}}{2}}
    \end{equation*}
    we proved the continuity.
\end{proof}


\begin{definition}
    Let \(V_1, V_2 ,\dots , V_n\) be  normed vector spaces. Then \(\phi : V_1 \times \dots \times V_n \to W\) is \textbf{\(n\)-linear} if by fixing any \(n-1\) component, \(\phi\) is linear relative to the remaining component.
\end{definition}

\begin{proposition}
    If \(V_1, V_2, \dots , V_n\) are  normed vector spaces and \(\ \phi : V_1 \times \dots \times V_n \to W \) is a \(n\)-linear then the followings are equivalent
    \begin{enumerate}
        \item \(\phi\) is continuous. \label{it:nLP_1}
        \item \(\phi\) is continuous at 0. \label{it:nLP_2}
        \item \(\phi\) is bounded, that is there exists a constant \(C > 0\) such that \label{it:nLP_3}
              \begin{equation*}
                  \norm{\func{\phi}{v_1, \dots ,v_n}}_W \leq C \norm{v_1}_{V_1} \dots \norm{v_n}_{V_n}
              \end{equation*}
    \end{enumerate}
\end{proposition}

\begin{remark}
    As oppose to linear transformation, \(n\)-linear function's continuity does not imply uniform continuity.
\end{remark}

\begin{proof}
    \Cref{it:nLP_1} \(\implies\) \Cref{it:nLP_2}: Trivial.

    \Cref{it:nLP_2} \(\implies\) \Cref{it:nLP_3}: For the sake of contradiction, suppose \Cref{it:nLP_3} is false. That is, for every \(k \in \Naturals\) there exists a point \(v_k = (v^1_k, \dots , v^n_k)\) such that
    \begin{equation*}
        \norm{\func{\phi}{v^1_k, \dots , v^n_k}}_W > n^n \norm{v^1_k}_{V_1} \dots \norm{v^n_k}_{V_k}
    \end{equation*}
    Note that \(v^m_k\) can not be zero for any \(k\) and \(m\), otherwise \(\func{\phi}{v_k} = 0 \). Define
    \begin{equation*}
        w^m_k = \dfrac{v^m_k}{n\norm{v^m_k}_{V_k}} \to 0
    \end{equation*}
    which from the continuity at 0 implies that \(w_k = (w^1_k, \dots , w^n_k) \to 0\). However,
    \begin{equation*}
        \norm{ \func{\phi}{w_k} - \func{\phi}{0}}_W > n^n \frac{1}{n} \dots \frac{1}{n} = 1
    \end{equation*}
    which is a contradiction.

    \Cref{it:nLP_3} \(\implies\) \Cref{it:nLP_1}. Let \(v_n \to v\) and define the points
    \begin{equation*}
        \bar{v}^m_k = (v^1 , \dots , v^m, v^{m+1}_k , \dots , v^n_k) , \qquad \bar{v}^0_k = v_k
    \end{equation*}
    and \(\bar{v}^n_k = v\). Note that \(v^m_k\) are bounded for sufficiently large \(k \geq N_1\), therefore there exists \(M\) such that \(\forall m, \; \norm{v^m_k}_{V_m} \leq M\). Also, pick \(M\) such that \(\forall m, \; \norm{v^m}_{V_m} \leq M\) as well. Then
    \begin{align*}
        \norm{\func{\phi}{v_k} - \func{\phi}{v}}_W & \leq \sum_{i = 1}^n \norm{\func{\phi}{\bar{v}^{i-1}_k  }- \func{\phi}{\bar{v}^i_k}}_W                                                              \\
                                                   & = \sum_{i = 1}^n \norm{\func{\phi}{\bar{v}^{i - 1}_k - \bar{v}^{i}_k }}_W                                                                          \\
                                                   & \leq \sum_{i = 1}^n C \norm{v^1}_{V_1} \dots \norm{v^{i-1}}_{V_{i-1}} \norm{v^i_k - v^i}_{V_i} \norm{v^{i+1}_k}_{V_{i+1}} \dots \norm{v^n_k}_{V_n} \\
                                                   & \leq CM^{n-1} \sum_{i = 1}^n \norm{v^i_k - v^i}_{V_i}
        \intertext{pick \(N_2\) such that for all \(k \geq N_2\), for each \(i, \; \norm{v^i_k - v^i}_{V_i} < \frac{\epsilon}{nCM^{n-1}}\) then}
        \norm{\func{\phi}{v_k} - \func{\phi}{v}}_W & < CM^{n-1}  \sum_{i = 1}^n \frac{\epsilon}{nCM^{n-1}} = \epsilon
    \end{align*}
\end{proof}

We denote  the set of all \(n\)-linear functions from \(V_1 \times \dots \times V_n \to W\) by \(\func{\calL^n}{V_1 \times \dots \times V_n, W}\).
\begin{proposition} \label{pr:nLinearIsmorphicLinear}
    Let \(V_1, \dots , V_n, W\) be normed vector spaces. Then \(\func{\calL^n}{V_1 \times \dots \times V_n, W}\) and \(\func{\calL}{V_1 , \func{\calL}{V_2, \dots ,\func{\calL}{V_n,W}}}\) are isomorphic.
\end{proposition}

\begin{proof}
    We want to prove
    \begin{equation*}
        \func{\calL^n}{V_1 \times \dots \times V_n, W} \cong \func{\calL}{V_1 , \func{\calL}{V_2, \dots ,\func{\calL}{V_n,W}}}
    \end{equation*}
    consider the mapping \(T : \func{\calL}{V_1 ,\func{\calL}{V_2, \dots ,\func{\calL}{V_n,W}}} \to \func{\calL^n}{V_1 \times \dots \times V_n, W}\), such that for any \(v_1 \in V_1,\; \dots, \; v_n \in V_n\)
    \begin{equation*}
        \func{\alpha}{(v_1)(v_2) \dots (v_n)} = \func{\func{T}{\alpha}}{v_1,v_2, \dots, v_n}
    \end{equation*}
    First note that \(T\) is linear. Then if \(\func{T}{\alpha} = 0\) implies \(\alpha  = 0\), thus \(T\) is injective and hence bijective.
\end{proof}

\begin{definition}[Positive definite]
    Matrix \(A \in \func{M_n}{\Reals}\) is \textbf{positive definite} whenever \(A\) is symmetric and 
    \begin{equation*}
        \forall x \in \Reals^n \backslash \set{0}, \ x^T A x > 0
    \end{equation*}
\end{definition}

\begin{theorem}
     Every positive definite matrix \(A\) is diagonizable. In face, there exists an orthognal matrix \(P\) such that 
     \begin{equation*}
         PAP^T = \begin{bmatrix}
             \lambda_1 & 0&\dots & 0 \\
             0 & \lambda_2 & \dots & 0\\
             \vdots & \ddots & & \vdots\\
             0 & \dots & 0 & \lambda_n
         \end{bmatrix}
     \end{equation*}
     where \(\lambda_i > 0\) for each \(i\).
\end{theorem}

{\Large\textbf{Exercises}}
\begin{enumerate}
    \item Show that for a linear transformation \(T\), \(\norm{T} = \sup_{\norm{v}_V \leq 1} \norm{Tv}_W\).
    \item Prove or disprove that if \(x^T A x = x^T A^T x\) for all \(x \in \Reals^n \backslash \set{0}\) then \(A = A^T\).
\end{enumerate}
\newpage


\chapter{Differentiation}
Let \(V,W\) be finite dimensional vector spaces and \(f: U \subset V \to W\) where \(U\) is open. Then \(f\) is differentiable at \(x_0\) when a linear transformation \(T : V \to W\) such that
\begin{equation*}
    \lim_{\norm{h} \to 0} \dfrac{\norm{\func{f}{x_0 + h} - \func{f}{x_0} - \func{T}{h}}}{\norm{h}}= 0
\end{equation*}
Or equivalently there exists a sublinear function \(\func{R}{h}\) such that
\begin{equation*}
    \func{f}{x_0 + h} - \func{f}{x_0} - Th = \func{R}{h} \qquad \frac{\func{R}{h}}{\norm{h}} \to 0
\end{equation*}
\(T\) if it exists is unique, represented by \(\func{f'}{x_0}\), \(\DiffOperator f\), or \(\func{\diffOperator f}{x}\) and called the \textbf{total derivative} or \textbf{Fr\'{e}chet derivative}.

\begin{example}
    Any linear function \(f : V \to W\) with \(\func{f}{v} = Tv + b\) where \(b \in W\) is differentiable and \(\func{\DiffOperator f}{v} = T\). Since
    \begin{equation*}
        \norm{h}_V < \delta \implies \norm{\func{f}{v + h} - \func{f}{v} - \operatorFunc{\DiffOperator \func{f}{v}}{h}}_W = \norm{T(v+h) - Tv - Th}_W = 0 < \epsilon \norm{h}_V
    \end{equation*}
    Hence, the derivative of any linear function is constant.
    Consider \(S : V \times V \to V\) with \(\func{S}{v,u} = v + u\). \(S\) is differentiable because \(S\) is linear (why?). We claim that \(\DiffOperator S = S\) as
    \begin{equation*}
        \norm{\func{S}{(v + h),(u + k)} - \func{S}{v,u} - \func{S}{h,k}} = 0
    \end{equation*}
\end{example}

\begin{example}
    Let \(\mu : \Reals \times V \to V\) with \(\func{\mu}{r,x} = rx\). Then \(\mu\) is differentiable and \(\operatorFunc{\func{\DiffOperator \mu}{r,x}}{t,h} = rh + tx\) as
    \begin{align*}
        \norm{\func{\mu}{(r + t),(x + h)} - \func{\mu}{r,x} - \operatorFunc{\func{\DiffOperator \mu}{r,x}}{t,h}} & = \norm{rx + rh + tx + th - rx - rh - tx}     \\
                                                                                                                 & = \abs{t} \norm{h} \leq \epsilon \norm{(t,h)}
    \end{align*}
    by letting \(\norm{(t,h)} = \sqrt{t^2 + \norm{h}^2}\) and \(\delta = \epsilon\).
\end{example}

\begin{proposition}
    Differentiability of \(f\) at \(x\) implies continuity at \(x\).
\end{proposition}

\begin{proof}
    \begin{equation*}
        \norm{\func{f}{x + h} - \func{f}{x}} = \norm{\operatorFunc{\DiffOperator \func{f}{x}}{h} + \func{R}{h}} \leq \norm{\DiffOperator \func{f}{x}}\norm{v} + \norm{\func{R}{v}} \to 0
    \end{equation*}
    as \(v \to 0\).
\end{proof}

\begin{proposition} \label{eq:partialDerivative}
    Assume \(f: U \subset V \to W\) is differentiable at \(x_0\) and let \(u \in V\) be a non-zero vector then
    \begin{equation*}
        \func{f'}{x_0} (u) = \lim_{t \to 0} \dfrac{\func{f}{x_0 + tu} - \func{f}{x_0}}{t}
    \end{equation*}
\end{proposition}

\begin{proof}
    Let \(h = tu\) then
    \begin{align*}
        \func{R}{tu}                     & = \func{f}{x_0 + tu} - \func{f}{x_0} - \func{T}{tu}            \\
                                         & = \func{f}{x_0 + tu} - \func{f}{x_0} - t\func{T}{u}            \\
        \implies \dfrac{\func{R}{tu}}{t} & = \dfrac{ \func{f}{x_0 + tu} - \func{f}{x_0}}{t} - \func{T}{u} \\
        \implies \lim_{t \to 0}          & \dfrac{ \func{f}{x_0 + tu} - \func{f}{x_0}}{t} = \func{T}{u}
    \end{align*}
\end{proof}

\begin{definition}[Directional derivative]
    If we let \(\norm{u} = 1\) then the limit in \Cref{eq:partialDerivative} becomes the \textbf{directional derivative} of \(f\) in the direction of \(u\) and is denoted by \(\DiffOperator_u f\).
\end{definition}

\begin{remark}
    The existence of all directional derivatives of \(f\) doesnt imply its differentiability or even its continuity.
\end{remark}

\begin{remark}
    If \(\DiffOperator f: U \to \func{\calL}{V,W}\) is continuous then each \(\PDiff{f_i}{x_j}\) is continuous. Since
    \begin{equation*}
        \DiffOperator \func{f}{x} = \begin{bmatrix}
            \func{\PDiff{f_1}{x_1}}{x} & \dots  & \func{\PDiff{f_1}{x_n}}{x} \\
            \vdots                     & \ddots &                            \\
            \func{\PDiff{f_m}{x_1}}{x} & \dots  & \func{\PDiff{f_m}{x_n}}{x}
        \end{bmatrix}
    \end{equation*}
    and the reverse is true as well.
\end{remark}

\begin{theorem} \label{th:DifferentiabilityCriteria}
    \(f : V \to W\) has all of its partial derivative in a neighbourhood of \(u \in U\) and they're continuous at \(u\) then \(f\) is differentiable at \(u\). Especially, if \(\PDiff{f_i}{x_j}\) exist and are continuous at every point of \(U\) then \(f \in \calC^1\).
\end{theorem}

\begin{proof}
    We prove that each \(f_i\) is differentiable. Let \(\set{e_1, \dots , e_n}\) be a basis for \(V\) and take \(\norm{x} = \sum \abs{\xi_j}\). Consider a convex neighbourhood \(E\) of \(a\). Then, for a given \(\epsilon > 0\) we will show there exists a \(\delta > 0\) such that
    \begin{equation*}
        \norm{h} < \delta \implies \norm{\func{f_i}{a + h} - \func{f_i}{a} - \sum_{j = 1}^n \operatorFunc{\DiffOperator_{e_j} \func{f_i}{a}}{h_j}} \leq \epsilon \norm{h}
    \end{equation*}
    Cosider the point sequence \(a^k =\sum_{j < k} a_j e_j + \sum_{j \geq k} (a_j + h_j)e_j \) where \(a^1 = a + h\) and \(a^{n + 1} = a\) then
    \begin{equation*}
        \norm{\func{f_i}{a + h} - \func{f_i}{a} - \sum_{j = 1}^n \operatorFunc{\DiffOperator_{e_j} \func{f_i}{a}}{h_j}}  \leq \sum_{k = 1}^{n} \norm{\func{f_i}{a^k} - \func{f_i}{a^{k+1}} - \operatorFunc{\DiffOperator_{e_k} \func{f_i}{a}}{h_k}}
    \end{equation*}
    hence we are done if
    \begin{equation*}
        \norm{\func{f_i}{a^k} - \func{f_i}{a^{k+1}} - \operatorFunc{\DiffOperator_{e_k} \func{f_i}{a}}{h_k}} \leq \epsilon \abs{h_k}
    \end{equation*}
    for \(k = n\)
    \begin{equation*}
        \norm{\func{f_i}{a^n} - \func{f_i}{a} - \operatorFunc{\DiffOperator_{e_n} \func{f_i}{a}}{h_n}}
    \end{equation*}
    which equivalent to the existence \(n_\cardinalTH\) partial derivative of \(a\). and for \(k < n\)
    \begin{align*}
         & \norm{\func{f_i}{a^k} - \func{f_i}{a^{k+1}} - \operatorFunc{\DiffOperator_{e_k} \func{f_i}{a}}{h_k}}                                                                                                                                  \\
         & \leq \norm{\func{f_i}{a^k} - \func{f_i}{a^{k+1}} - \operatorFunc{\DiffOperator_{e_k} \func{f_i}{a^k}}{h_k}}  + \norm{\operatorFunc{\DiffOperator_{e_k} \func{f_i}{a^k}}{h_k} - \operatorFunc{\DiffOperator_{e_k} \func{f_i}{a}}{h_k}}
    \end{align*}
    which uses the existence of partial derivatives in neighbourhood and its continuity.
\end{proof}

\begin{proposition}
    Let \(f,g : V \to W\) be differentiable at \(x\) and \(h : W \to U\) be differentiable at \(y = \func{f}{x}\). Furthermore, let \(c\) be an scalar then
    \begin{enumerate}
        \item \(\func{\DiffOperator \,}{ f + cg} = \DiffOperator f + c \DiffOperator g\).
        \item  \(h \circ f\) is differentiable at \(x\) and
              \begin{equation*}
                  \func{\DiffOperator \,}{ h \circ f} =  \left( (\DiffOperator h) \circ f \right) \circ \DiffOperator f
              \end{equation*}
    \end{enumerate}
\end{proposition}

\begin{proof} \leavevmode
    \begin{enumerate}
        \item we have
              \begin{align*}
                   & \norm{ \operatorFunc{f + cg}{x + k} - \operatorFunc{f + cg}{x} - \operatorFunc{\DiffOperator \func{f}{x} + c \DiffOperator \func{g}{x}}{k}}                                          \\
                   & \leq \norm{ \func{f}{x + k} - \func{f}{x} - \operatorFunc{\DiffOperator \func{f}{x}}{h}} + \abs{c}\norm{\func{g}{x + k} - \func{g}{x} - \operatorFunc{\DiffOperator \func{g}{x}}{h}}
              \end{align*}
        \item we know that
              \begin{equation*}
                  \begin{cases}
                      \func{f}{x + k} - \func{f}{x} - \operatorFunc{\DiffOperator \func{f}{x}}{k}  = \func{R}{k} \\
                      \func{h}{y + l} - \func{h}{y} - \operatorFunc{\DiffOperator \func{h}{y}}{l}  = \func{S}{l}
                  \end{cases}
              \end{equation*}
              and we wish to prove that
              \begin{equation*}
                  \func{h \circ f}{x + k} - \func{h \circ f}{x} - \operatorFunc{\DiffOperator \func{h}{\func{f}{x}} \circ \DiffOperator \func{f}{x}}{k} = \func{T}{k}
              \end{equation*}
              where \(\norm{\func{T}{k}} \leq \epsilon \norm{k}\) whenever \(\norm{k} < \delta\). Let \(l = \func{f}{x + k} - \func{f}{x}\) and substituting into the second equation
              \begin{align*}
                  \func{h}{\func{f}{x + k}} & - \func{h}{\func{f}{x}} - \operatorFunc{\DiffOperator \func{h}{y}}{\func{f}{x + k} - \func{f}{x}}                                                                                         \\
                                            & =  \func{h}{\func{f}{x + k}} - \func{h}{\func{f}{x}} - \operatorFunc{\DiffOperator \func{h}{y}}{\operatorFunc{\DiffOperator \func{f}{x}}{k}  + \func{R}{k}}                               \\
                                            & = \func{h}{\func{f}{x + k}} - \func{h}{\func{f}{x}} - \operatorFunc{\DiffOperator \func{h}{y} \circ \DiffOperator \func{f}{x}}{k} - \operatorFunc{\DiffOperator \func{h}{y}}{\func{R}{k}} \\
                                            & = \func{T}{k} - \operatorFunc{\DiffOperator \func{h}{y}}{\func{R}{k}} = \func{S}{l}                                                                                                       \\
                  \implies \func{T}{k}      & = \func{S}{l} + \operatorFunc{\DiffOperator \func{h}{y}}{\func{R}{k}}
              \end{align*}
    \end{enumerate}
\end{proof}


\begin{proposition}
    \(f : U \subset V \to W_1 \times \dots \times W_n\) is differentiable at \(x_0\) if and only if all its component is differentiable at \(x_0\). Furthermore, \(\DiffOperator f = (\DiffOperator f_1, \dots , \DiffOperator f_n)\).
\end{proposition}

\begin{proof}
    Define the following norm on \(W_1 \times \dots \times W_n\)
    \begin{equation}
        \norm{(w_1, \dots w_n)} = \sum_{i = 1}^n \norm{w_i}_{W_i}
    \end{equation}
    then
    \begin{equation*}
        \norm{ \func{f}{x_0 + h} - \func{f}{x_0} - \operatorFunc{\DiffOperator \func{f}{a}}{h}} = \sum_{i = 1}^n \norm{ \func{f_i}{x_0 + h} - \func{f_i}{x_0} - \operatorFunc{\DiffOperator \func{f_i}{a}}{h}}
    \end{equation*}
    and since every other norm is equivalent to the norm defined above, we are done.
\end{proof}

\begin{theorem}[Leibnitz rule]
    Let \(V_1, V_2, \dots , V_n\) be finite dimensional vector spaces and \(f: V_1 \times \dots \times V_n \to W\) is a \(n\)-linear function. \(f\) is differentiable at \(a = (a_1, \dots , a_n)\) and
    \begin{equation*}
        \operatorFunc{\func{\DiffOperator f}{a}}{h_1, \dots h_n} = \func{f}{h_1, a_2, \dots, a_n} + \func{f}{a_1, h_2, \dots, a_n} + \dots + \func{f}{a_1, a_2, \dots, h_n}
    \end{equation*}
\end{theorem}

\begin{proof}
    we have that
    \begin{equation*}
        \func{f}{a + h} = \sum_{\xi_i \in \set{a_i,h_i}} \func{f}{\xi_1, \dots , \xi_n}
    \end{equation*}
    therefore
    \begin{equation*}
        \func{f}{a + h} - \func{f}{a} - \sum_{i = 1}^{n} \func{f}{a_1, \dots, a_{i-1}, h_i, a_{i+1}, \dots , a_n} = \sum_{\substack{\xi_i \in \set{a_i,h_i} \\ \text{at least two \(h_i\)}}} \func{f}{\xi_1, \dots , \xi_n}
    \end{equation*}
    Let \(\delta = 1\) then  \(\norm{h} = \sum \norm{h_i} < 1\) also \(i,j, \; \norm{h_i}\norm{h_j} \leq \norm{h}^2\). Hence if we define
    \begin{equation*}
        A = \max \set[\prod_{i \in I} {\norm{a_i}}]{I \subset \Naturals_n}
    \end{equation*}
    then
    \begin{equation*}
        \sum_{\substack{\xi_i \in \set{a_i,h_i} \\ \text{at least two \(h_i\)}}} \func{f}{\xi_1, \dots , \xi_n} \leq (2^n - n - 1)A \norm{h}^2
    \end{equation*}
    and letting \(\delta = \min \set{1 , \dfrac{\epsilon}{(2^n - n - 1)(A + 1)}}\) we arrive at the conclusion.
\end{proof}

\begin{example}
    Let \(Z: \Reals^3 \times \Reals^3 \to \Reals\) with \(\func{Z}{u,v}= u \times v\) be a bilinear function, \(f,g: \Reals \to \Reals^3\) and \(\func{h}{t} = \func{f}{t} \times \func{g}{t}\). \(h = Z \circ \phi\) where \(\func{\phi}{t} = (\func{f}{t},\func{g}{t})\). Then we have:
    \begin{align*}
        \DiffOperator \func{h}{t} & = \operatorFunc{\DiffOperator Z}{\func{\phi}{t}}\circ \DiffOperator \func{\phi}{t}                             \\
                                  & =  \operatorFunc{\DiffOperator Z}{\func{\phi}{t}} \circ (\DiffOperator \func{f}{t}, \DiffOperator \func{g}{t}) \\
                                  & = \func{Z}{\DiffOperator \func{f}{t}, \func{g}{t}} + \func{Z}{ \func{f}{t}, \DiffOperator \func{g}{t}}         \\
                                  & = \DiffOperator \func{f}{t} \times \func{g}{t} + \func{f}{t} \times \DiffOperator \func{g}{t}
    \end{align*}
\end{example}

\begin{example}
    Consider \(A = [\func{f_{ij}}{x_1, \dots , x_n}]\) where each \(f_{ij}\) is differentiable. Then
    \begin{equation*}
        \DiffOperator \func{\det}{A}
    \end{equation*}
    can be calculated using the Leibnitz rule, since determinant is \(n\)-linear function.
\end{example}

\section{Mean value theorem}
Mean value theroem of 1-dimensional does not generalize very well. For example, the continuous function \(\func{f}{t} : \clcl{0}{1} \to \Reals^2\) with
\begin{equation*}
    t \mapsto (t^2, t^3)
\end{equation*}
is differentiable on \(\opop{0}{1}\), however
\begin{align*}
    \func{f}{1} - \func{f}{0} = (1,1) & = \DiffOperator \func{f}{c} (1 - 0) \\
                                      & = (2c,3c^2)
\end{align*}
which has no solution for \(c \in \opop{0}{1}\).
Although it must be said that for \(f: U \to \Reals\) where \(U \subset V\) is convex, the mean value theorem holds.

\begin{theorem} \label{th:MultivariableMVT}
    Let \(V,W\) be normed finite dimensional vector spaces and \(f: U \to W\) is differentiable and \(A,B \in U\) are such that the line connecting in completely contained in \(U\) and for each \(p\) on that line
    \begin{equation*}
        \norm{\DiffOperator \func{f}{p}} \leq M
    \end{equation*}
    then
    \begin{equation*}
        \norm{\func{f}{B} - \func{f}{A}}_W \leq M \norm{B - A}_V
    \end{equation*}
\end{theorem}
First consider the following lemma:
\begin{lemma} \label{lm:MeanValueTheoremLemma}
    If \(\phi: \clcl{0}{1} \to W\) is continuous, differentiable on \(\opop{0}{1}\) and \(\norm{\func{\phi'}{t}} \leq M\) for all \(t \in \opop{0}{1}\) then
    \begin{equation*}
        \norm{\func{\phi}{1} - \func{\phi}{0}}_W \leq M
    \end{equation*}
\end{lemma}

\begin{prooflemma}
    We provide three proofs for the lemma
    \begin{enumerate}
        \item Assuming the norm on \(W\) is induced by an inner product. Then, let \( e = \frac{\func{\phi}{1} - \func{\phi}{0}}{\norm{\func{\phi}{1} - \func{\phi}{0}}}\) be a unit vector in \(W\) then \(\psi : \clcl{0}{1} \to \Reals\), \(\func{\psi}{t} = e \cdot \func{\phi}{t}\) is continuous and differentiable on \(\opop{0}{1}\). By the mean the value theorem
              \begin{align*}
                  \abs{\func{\psi}{1} - \func{\psi}{0}}                        & = \abs{\func{\psi'}{t_0}}       \\
                  \abs{e \cdot \left( \func{\phi}{1} - \func{\phi}{0} \right)} & = \abs{e \cdot \func{\phi'}{t}} \\
                  \norm{\func{\phi}{1} - \func{\phi}{0}}                       & \leq M
              \end{align*}
        \item Using the Hahn-Banach theorem, that is for a finite dimensional vector space \(V\) and \(e \in V\) with \(\norm{e} = 1\) there exists a linear function \(\theta : V \to \Reals\) such that \(\norm{\theta} = 1\) and \(\func{\theta}{e} = 1\). Now let \(\func{\psi}{t} = \func{\theta}{\func{\phi}{t}}\) and take \(e\) as defined above then
              \begin{align*}
                  \abs{\func{\psi}{1} - \func{\psi}{0}}                & = \abs{\func{\psi'}{t_0}}                                                             \\
                  \abs{\func{\theta}{\func{\phi}{1} - \func{\phi}{0}}} & = \operatorFunc{\DiffOperator \func{\theta}{\func{\phi}{t_0}}}{\func{\phi'}{t_0}}     \\
                  \norm{\func{\phi}{1} - \func{\phi}{0}}               & = \func{\theta}{\func{\phi'}{t_0}} \leq \norm{\theta} \norm{\func{\phi'}{t_0}} \leq M
              \end{align*}
        \item From Hoimander. For any \(\epsilon\) consider the set \(T_\epsilon\).
              \begin{equation*}
                  T_\epsilon = \set[t \in \clcl{0}{1}]{\forall s, \; 0 \leq s \leq t, \; \norm{\func{\phi}{s} - \func{\phi}{0}} \leq(M + \epsilon)s + \epsilon}
              \end{equation*}
              first note that \(T_\epsilon = \clcl{0}{c}\) for some \(c > 0\) because for \(s = 0\) the inequality is strict and both sides are continuous with respect to \(s\). We claim that \(c = 1\) because otherwise \(c < 1\) and by differentiability of \(\phi\), there exists a \(\delta < 1 - c\) such that if
              \begin{align*}
                  \norm{h} < \delta \implies \norm{\func{\phi}{c + h} - \func{\phi}{c} - \operatorFunc{\DiffOperator \func{\phi}{c}}{h}} & \leq \epsilon \norm{h}                                        \\
                  \implies \norm{\func{\phi}{c + h} - \func{\phi}{c}}                                                                    & \leq\norm{h} \left( \epsilon + \norm{\func{\phi'}{c}} \right) \\
                                                                                                                                         & \leq \norm{h} (\epsilon + M)
                  \intertext{also since \(c \in T_\epsilon\)}
                  \norm{\func{\phi}{c} - \func{\phi}{0}}                                                                                 & < (M + \epsilon)c + \epsilon                                  \\
                  \implies \norm{\func{\phi}{c + h} - \func{\phi}{0}}                                                                    & < (M + \epsilon)(c + h) + \epsilon \qquad 0 < h < \delta
              \end{align*}
              hence \(c + h \in T_\epsilon\) which is a contradiction and thus \(c = 1\).
    \end{enumerate}
\end{prooflemma}

\begin{proof}
    Let \(\sigma : \clcl{0}{1} \to U\) be a parameterization of the line connecting the point \(A\) to point \(B\), \linebreak \(\func{\sigma}{t} = (1 - t) A + tB\). Let \(\phi = f \circ \sigma\), then clearly \(\phi\) is continuous on \(\clcl{0}{1}\) and differentiable on \(\opop{0}{1}\) and we have
    \begin{align*}
        \func{\phi'}{t}                 & = \operatorFunc{\DiffOperator \func{f}{\func{\sigma}{t}}}{\func{\sigma'}{t} }         \\
                                        & = \operatorFunc{\DiffOperator \func{f}{\func{\sigma}{t}}}{B - A}                      \\
        \implies \norm{\func{\phi'}{t}} & \leq \norm{\DiffOperator \func{f}{\func{\sigma}{t}}} \norm{B-A}_V \leq M \norm{B-A}_V
    \end{align*}
    therefore by the \Cref{lm:MeanValueTheoremLemma}
    \begin{equation*}
        \norm{\func{f}{B} - \func{f}{A}}_W = \norm{\func{\phi}{1} - \func{\phi}{0}}_W \leq M \norm{B-A}_V
    \end{equation*}
    which concludes the proof.
\end{proof}

\begin{corollary} \label{cr:derivativeZeroConstant}
    Let \(U \subset V\) is connected and open and \(f: U \to W\) is differentiable and \(\DiffOperator \func{f}{u} = 0\) for all \(u \in U\) then \(f\) is constant.
\end{corollary}

\begin{proof}
    Let \(p \in U\) and \( S = \set[q \in U]{\func{f}{q} = \func{f}{p}} \). \(S\) is closed because \(f\) is continuous and hence the pre-image closed set \(\set{\func{f}{p}}\) is closed. For each \(q \in S\) there exists \(r > 0\) such that \(\func{B_r}{q} \subset U\) and since \(\func{B_r}{q}\) is convex then for each \( l \in \func{B_r}{q}\) we apply the \Cref{th:MultivariableMVT}
    \begin{equation*}
        \norm{\func{f}{l} - \func{f}{q}} \leq \sup \norm{\DiffOperator \func{f}{t}} \norm{l - q} = 0
    \end{equation*}
    which implies that \(\func{f}{l} = \func{f}{q} = \func{f}{p}\) hence \(S\) is open in \(U\) which by the connectedness of \(U\) means \(S = U\). Therefore, \(f\) is constant on \(U\).
\end{proof}

\begin{corollary}
    Let \(V_1, V_2, W\) be finite dimensional normed vector space and \(U \subset V_1 \times V_2\) is open such that for every \(y \in V_2\) the intersection \((V_1 \times \set{y}) \cap U\) is connected. Assumne \(f : U \to W\) is differentiable and \(\DiffOperator_{V_1} \func{f}{x,y} = 0\) for all \((x,y) \in U\) then for any two point \((x_1,y), (x_2,y) \in U\),\(\func{f}{x_1,y} = \func{f}{x_2,y}\).
\end{corollary}

\begin{proof}
    Fix \(y \in V_2\) and define the function \(g : V_1 \to W\)
    \begin{equation*}
        \func{g}{x} = \func{f}{x,y}
    \end{equation*}
    therefore
    \begin{equation*}
        \DiffOperator \func{g}{x} = \DiffOperator_{V_1} \func{f}{x,y} = 0
    \end{equation*}
    and since \((V_1 \times \set{y}) \cap U\), the domain of \(g\) is connected. Hence by applying the \Cref{cr:derivativeZeroConstant} we get that
    \begin{equation*}
        \func{g}{x} = c \implies \func{f}{x_1,y} = \func{f}{x_2,y}
    \end{equation*}
    for all \(y \in V_2\).
\end{proof}

\section{Fundamental theorem of calculus}
\begin{theorem}
    Let \(U\) be an open set of \(V\) such that for every \(A,B \in U\) the line segment connecting \(A\) and \(B\) remains in \(U\) and let \(\sigma : \clcl{0}{1}\to U\) be that line, \(\func{\sigma}{t} = (1-t)A + tB\), and lastly let \(f: U \to W\) is continuously differentiable. Then
    \begin{equation*}
        \func{f}{B} - \func{f}{A} = \func{T}{B - A}
    \end{equation*}
    where \(T\) is
    \begin{equation*}
        T = \int_{0}^{1} \DiffOperator \func{f}{ \func{\sigma}{t}} \diffOperator t
    \end{equation*}
\end{theorem}

\begin{proof}
    Let \(g_i : \clcl{0}{1} \to \Reals\) be
    \begin{equation*}
        \func{g_i}{t} = \pi_i \circ \func{f}{\func{\sigma}{t}}
    \end{equation*}
    is continuously differentiable then by the fundamental theorem of calculus for the real-valued functions we have
    \begin{align*}
        \func{g}{1} - \func{g}{0}                                   & = \int_{0}^{1} \func{g'}{t} \diffOperator t                                                                        \\
                                                                    & = \int_{0}^{1} \pi_i \circ \DiffOperator \func{f}{\func{\sigma}{t}} \diffOperator t                                \\
                                                                    & = \pi_i \circ \int_{0}^{1} \DiffOperator \func{f}{\func{\sigma}{t}} \DiffOperator \func{\sigma}{t} \diffOperator t \\
                                                                    & = \pi_i \circ \int_{0}^{1} \DiffOperator \func{f}{\func{\sigma}{t}} (B - A) \diffOperator t                        \\
        \implies \pi_i \circ \left(\func{f}{B} - \func{f}{A}\right) & = \pi_i \circ \func{T}{B- A}                                                                                       \\
        \implies \func{f}{B} - \func{f}{A}                          & =\func{T}{B- A}
    \end{align*}
    which was what was wanted.
\end{proof}

\begin{theorem}
    Consider the continuous function \(T: U \times U \to \func{\calL}{V,W}\) which is such that
    \begin{equation*}
        \func{f}{B} - \func{f}{A} = \operatorFunc{\func{T}{A,B}}{B-A}
    \end{equation*}
    then \(f \in \calC^1\) and \(\DiffOperator \func{f}{A} = \func{T}{A,A}\)
\end{theorem}

\begin{proof}
    We have
    \begin{equation*}
        \func{f}{A + h} - \func{f}{A} = \operatorFunc{\func{T}{A+h,A}}{h}
    \end{equation*}
    hence
    \begin{align*}
        \norm{\func{f}{A + h} - \func{f}{A} - \operatorFunc{\func{T}{A,A}}{h}} & = \norm{ \operatorFunc{\func{T}{A+h,A}}{h} - \operatorFunc{\func{T}{A,A}}{h}} \\
                                                                               & \leq \norm{\func{T}{A+h,A} - \func{T}{A,A}}\norm{h}
    \end{align*}
    now by continuity of \(T\), there exists a \(\delta > 0\) such that
    \begin{equation*}
        \norm{(h,k)} < \delta \implies \norm{\func{T}{A+h,A+k} - \func{T}{A,A}} < \epsilon
    \end{equation*}
    By letting \(k = 0\) we get \(\DiffOperator \func{f}{A} = \func{T}{A,A}\). Since \(T\) is continuous then \(f \in \calC^1\) as well.
\end{proof}

\begin{corollary}
    Let \(V\) be a normed finite dimensional vector space and \(U\) is open subset of \(V\). If
    \begin{equation*}
        f :\clcl{a}{b} \times U \to \Reals
    \end{equation*}
    is continuous then
    \begin{equation*}
        \func{F}{y} = \int_{a}^{b} \func{f}{x,y} \diffOperator x
    \end{equation*}
    is continuous. Furthermore, if \(\PDiff{f}{y_i}\) exists and is continuous then \(\PDiff{F}{y_i}\) exists and is continuous as well.
    \begin{equation*}
        \PDiff{F}{y_i} =  \int_{a}^{b} \func{\PDiff{f}{y_i}}{x,y} \diffOperator x
    \end{equation*}
\end{corollary}

\begin{proof}
    Firstly, we want to show that there exists a \(\delta > 0\) such that for each \(y \in U\)
    \begin{equation*}
        \norm{h} < \delta \implies  \norm{\func{F}{y + h} - \func{F}{y}} < \epsilon
    \end{equation*}
    we have that
    \begin{align*}
        \norm{\func{F}{y + h} - \func{F}{y}} & =  \norm{\int_{a}^{b} \func{f}{x,y+ h} \func{f}{x,y} \diffOperator x}    \\
                                             & \leq (b-a) \sup_{x \in \clcl{a}{b}} \set{\func{f}{x,y+ h} \func{f}{x,y}}
    \end{align*}
    note that from the continuity of \(f\) for each \(x \in \clcl{a}{b}\) and \(y \in U\) there are open balls \(I_{x,y}\) around \(x\) and \(J_{x,y}\) around \(y\) such that
    \begin{equation*}
        x' \in I_{x,y}, \; y' \in J_{x,y} \implies \norm{\func{f}{x',y'} - \func{f}{x,y}} < \dfrac{\epsilon}{b - a}
    \end{equation*}
    Fix \(y_0\), then \(\cup I_{x,y_0} \supset \clcl{a}{b}\) which by the compactness of the interval implies that there is a finite family of there open set the covers \(\clcl{a}{b}\). Setting \(\delta\) to the minimum radius of \(J_{x,y_0}\) yields the result.
    Secondly, we show that there exists a \(\delta > 0 \) such that
    \begin{equation*}
        \abs{h} < \delta \implies \norm{\dfrac{\func{F}{y + he_i} - \func{F}{y}}{h}} < \epsilon
    \end{equation*}
    and we have that
    \begin{align*}
        \dfrac{\func{F}{y + he_i} - \func{F}{y}}{h} & = \dfrac{1}{h}  \int_{a}^{b} \func{f}{x,y+ he_i} - \func{f}{x,y} \diffOperator x   \\
                                                    & =  \dfrac{1}{h}  \int_{a}^{b} \func{\PDiff{f}{y_i}}{x,y + the_i}h  \diffOperator x \\
                                                    & = \int_{a}^{b} \func{\PDiff{f}{y_i}}{x,y + the_i}  \diffOperator x
    \end{align*}
    from the previous part we know that we can make
    \begin{equation*}
        \norm{\func{\PDiff{f}{y_i}}{x,y'} - \func{\PDiff{f}{y_i}}{x,y}}
    \end{equation*}
    as small as we want by making \(\norm{y - y'} < \delta\) small independently of \(x\). Therefore, there exist a \(\delta > 0\) such that if \(\abs{th} < \abs{h} < \delta\) then
    \begin{equation*}
        \norm{\func{\PDiff{f}{y_i}}{x,y'} - \func{\PDiff{f}{y_i}}{x,y}} < \frac{\epsilon}{b-a}
    \end{equation*}
    hence
    \begin{equation*}
        \norm{\dfrac{\func{F}{x,y + he_i} - \func{F}{x,y}}{h} - \int_{a}^{b} \func{\PDiff{f}{y_i}}{x,y} \diffOperator x} = \norm{\int_{a}^{b} \func{\PDiff{f}{y_i}}{x,y + the_i} - \func{\PDiff{f}{y_i}}{x,y}  \diffOperator x}
        < \frac{\epsilon}{b - a}
    \end{equation*}
    and the continuity of \(\PDiff{F}{y_i}\) comes as a result of applying the first part to \(\PDiff{f}{y_i}\).
\end{proof}
\section{Higher derivative}
Let \(V,W\) be finite dimensional normed vector spaces with \((e_1, \dots , e_n)\) is an ordered basis for \(V\). Consider \(U \subset V\) is an open set and \(f: U \to W\). If \(f\) is differentiable then its partial derivatives
\begin{equation*}
    \DiffOperator_i f: U \to E \quad \text{with} \quad \operatorFunc{\DiffOperator_i f}{x} = \operatorFunc{\DiffOperator \func{f}{x}}{e_i}
\end{equation*}
Then, clearly if \(\DiffOperator_i f\) is differentiable one can define its partial derivatives \(\operatorFunc{\DiffOperator_j}{\DiffOperator_i f}\) also denoted by
\begin{equation*}
    \operatorFunc{\DiffOperator_j}{\DiffOperator_i f}=  \dfrac{\PDiffOperator^2 f}{\PDiffOperator x_j \PDiffOperator x_i} =\DiffOperator_{ji} f
\end{equation*}

For Fr\'{e}chet derivative, if \(\DiffOperator f: U \to \func{\calL}{V,W}\) is differentiable at \(x\), then \(f\) is twice differentiable and
\begin{equation*}
    \DiffOperator^2 \func{f}{x} = \operatorFunc{\func{\DiffOperator \,}{\DiffOperator f}}{x} : U \xrightarrow{\text{linear map}} \func{\calL}{V,W}
\end{equation*}
is a linear map. Therefore,
\begin{equation*}
    \DiffOperator^2 f :  U \to \func{\calL}{V,\func{\calL}{V,W}}
\end{equation*}
which by the \Cref{pr:nLinearIsmorphicLinear} is equivalent to \(\func{\calL^2}{V \times V, W}\) and one can define
\begin{equation*}
    \diffOperator^2 f : U  \to \func{\calL^2}{V \times V, W}
\end{equation*}
where \(\diffOperator^2 = \func{T}{\DiffOperator^2}\) as defined in \Cref{pr:nLinearIsmorphicLinear}. With this definition, for the higher order derivatives \(n \geq 2\)
\begin{equation*}
    \diffOperator^n : U \to \func{\calL^n}{V^n, W}
\end{equation*}

\begin{example}
    Let \(A : V \to W\) be a affine function \(\func{A}{x} = Lx + b\) where \(L\) is linear. Then, \(\DiffOperator \func{A}{x} = L\) and hence \(\DiffOperator^2 A = 0\).
\end{example}

\begin{example}
    Let \(\beta : V \times V \to W\) be a bilinear function. By the Leibnitz rule
    \begin{equation*}
        \operatorFunc{\DiffOperator \func{\beta}{x_1,x_2}}{h_1,h_2} = \func{\beta}{x_1,h_2} + \func{\beta}{h_1,x_2}
    \end{equation*}
    therefore \(\DiffOperator \beta: V \times V \to \func{\calL}{V \times V, W}\) is a linear a function itself, since
    \begin{equation*}
        \operatorFunc{\DiffOperator \func{\beta}{x_1 + x'_1,x_2 + x'_2}}{h_1,h_2} = \func{\beta}{x_1,h_2} + \func{\beta}{x'_1,h_2} + \func{\beta}{x_1,h_2} + \func{\beta}{b_1,x'_2}
    \end{equation*}
    which means \(\operatorFunc{\func{\DiffOperator \,}{\DiffOperator \beta}}{x} = \DiffOperator \beta\) independent of \(x\).
\end{example}

\begin{theorem}
    If \(f\) is twice differentiable at \(p\) then its second partial derivatives exist at \(p\). Conversely, if its second partial derivatives exist at a neighbourhood of \(p\) and they are continuous, then \(f\) is differentiable.
\end{theorem}

\begin{proof}
    Assume that \(\DiffOperator^2 \func{f}{p}\) exists. Then
    \begin{align*}
        \func{\DiffOperator_j \,}{\DiffOperator_i \func{f}{p}} & = \lim_{h \to 0} \dfrac{\operatorFunc{\DiffOperator \func{f}{p + he_j}}{e_i} - \operatorFunc{\DiffOperator \func{f}{p}}{e_i} }{h} \\
                                                               & = \operatorFunc{ \lim_{h \to 0} \dfrac{\DiffOperator \func{f}{p + he_j} - \DiffOperator \func{f}{p}}{h} }{e_i}
    \end{align*}
    which exists sincce \(\DiffOperator f\) is differentiable at \(p\). Conversely, assume that the second partials exist and are continuous at \(p\). Then
    \begin{equation*}
        \operatorFunc{\func{\DiffOperator \,}{\DiffOperator f}}{p} = \begin{bmatrix}
            \func{\PDiff{\DiffOperator f}{x_1}}{p} & \dots & \func{\PDiff{\DiffOperator f}{x_n}}{p}
        \end{bmatrix}
    \end{equation*}
    note that each \(\func{\PDiff{\DiffOperator f}{x_i}}{p}\) is in \(\func{\calL}{V,W}\). In fact, since
    \begin{equation*}
        \DiffOperator f = \begin{bmatrix}
            \PDiff{f}{x_1} & \dots & \PDiff{f}{x_n}
        \end{bmatrix}
    \end{equation*}
    then
    \begin{equation*}
        \PDiff{\DiffOperator f}{x_i} =  \begin{bmatrix}
            \dfrac{\PDiffOperator^2 f}{\PDiffOperator x_i \PDiffOperator x_1} & \dots & \dfrac{\PDiffOperator^2 f}{\PDiffOperator x_i \PDiffOperator x_n}
        \end{bmatrix}
    \end{equation*}
    which is continuous at \(p\) and hence \(\func{\PDiff{\DiffOperator f}{x_i}}{p}\) is continuous and by \Cref{th:DifferentiabilityCriteria}, \(\DiffOperator f\) is differentiable at \(p\).
\end{proof}

\begin{remark}
    In general, one can show that \(f \in \calC^r\) is equivalent to its partial being in \(\calC^r\).
\end{remark}

Let \(f: U \to \Field\) then \(\DiffOperator f: U \to \func{\calL}{V,\Field}\) which is the topological dual space \(V^\ast\) therefore
\begin{equation*}
    \DiffOperator f = \sum_{i = 1}^n \PDiff{f}{x_i} e^\ast_i
\end{equation*}
then for the second derivative of \(f\), \(\DiffOperator^2 \func{f}{x} : U \to V^\ast\)
\begin{align*}
    \DiffOperator^2 \func{f}{x}                                                  & = \sum_{i = 1}^n \func{\PDiff{\DiffOperator f}{x_i}}{x} e^\ast_i                                                     \\
                                                                                 & = \sum_{i = 1}^n \func{\PDiff{\sum_{j = 1}^n \PDiff{f}{x_j}}{x_i}}{x}  e^\ast_j e^\ast_i                             \\
                                                                                 & =  \sum_{i = 1}^n \sum_{j = 1}^n \dfrac{\PDiffOperator^2 f}{\PDiffOperator x_i \PDiffOperator x_j} e^\ast_j e^\ast_i \\
    \implies \operatorFunc{\operatorFunc{\DiffOperator^2 \func{f}{x}}{e_i}}{e_j} & = \func{\dfrac{\PDiffOperator^2 f}{\PDiffOperator x_i \PDiffOperator x_j}}{x}                                        \\
    \implies \func{\diffOperator^2 \func{f}{x}}{e_i,e_j}                         & = \func{\dfrac{\PDiffOperator^2 f}{\PDiffOperator x_i \PDiffOperator x_j}}{x}
\end{align*}

\begin{definition}[Hessian matrix]
    If for a function \(f : U \to \Field\) all of its second partial derivatives exist then \textbf{hessian matrix} is
    \begin{equation*}
        \begin{bmatrix}
            \func{\dfrac{\PDiffOperator^2 f}{\PDiffOperator x_1^2}}{x}                  & \dots  & \func{\dfrac{\PDiffOperator^2 f}{\PDiffOperator x_1 \PDiffOperator x_n}}{x} \\
            \vdots                                                                      & \ddots & \vdots                                                                      \\
            \func{\dfrac{\PDiffOperator^2 f}{\PDiffOperator x_n \PDiffOperator x_1}}{x} & \dots  & \func{\dfrac{\PDiffOperator^2 f}{\PDiffOperator x_n^2}}{x}
        \end{bmatrix}
    \end{equation*}
\end{definition}

%TODO: needs better proof
\begin{theorem}
    If \(f\) is twice differentiable at \(x\), \(\diffOperator^2 \func{f}{x}\) is symmetric. That is,
    \begin{equation*}
        \func{\diffOperator^2 \func{f}{x}}{h,k} = \func{\diffOperator^2 \func{f}{x}}{k,h}
    \end{equation*}
\end{theorem}

\begin{proof}
    Let \(\norm{h}\) and \(\norm{k}\) be sufficiently small such that \(a + th, a+  tk , a+th + tk\) and the lines connecting them stays in \(U\) for some \(t \in \Reals\). Consider
    \begin{equation*}
        \func{\Delta}{t,h,k} = \func{f}{a + th + tk} - \func{f}{a + th} - \func{f}{a  + tk} + \func{f}{a}
    \end{equation*}
    Assuming \(f\) is a real-valued twice differentiable function then if we prove
    \begin{equation*}
        \operatorFunc{\diffOperator^2 \func{f}{x}}{h,k} = \lim_{t \to 0} \dfrac{\func{\Delta}{t,h,k}}{t^2}
    \end{equation*}
    we are done since, \(\Delta\) is symmetric with respect to \(h\) and \(k\). Now consider
    \begin{equation*}
        \func{g}{s} = \func{f}{a + th + tsk} - \func{f}{a  + tsk}
    \end{equation*}
    then by the the Mean value theorem
    \begin{align*}
        \func{\Delta}{t,h,k} = \func{g}{1} - \func{g}{0} & = \func{g'}{\xi}                                                                                                       \\
                                                         & =\operatorFunc{ \DiffOperator \func{f}{a + th + t\xi k}}{tk} - \operatorFunc{ \DiffOperator \func{f}{a  + t\xi k}}{tk}
    \end{align*}
    and since \(\DiffOperator f\) is differentiable then by definition
    \begin{align*}
        \implies \DiffOperator \func{f}{a+x} & = \DiffOperator \func{f}{a} + \operatorFunc{\DiffOperator^2 \func{f}{a}}{x} - \func{R}{x}
    \end{align*}
    therefore
    \begin{multline*}
        \func{\Delta}{t,h,k} = t\operatorFunc{\DiffOperator \func{f}{a } + \operatorFunc{\DiffOperator^2 \func{f}{a}}{th + t\xi k} - \func{R}{th + t\xi k}}{k} \\- t\operatorFunc{\DiffOperator \func{f}{a} + \operatorFunc{\DiffOperator^2 \func{f}{a}}{t\xi k} - \func{R}{t\xi k}}{k}
    \end{multline*}
    then
    \begin{align*}
        \func{\Delta}{t,h,k}                       & = t \operatorFunc{\operatorFunc{\DiffOperator^2 \func{f}{a}}{th + t\xi k}-\operatorFunc{\DiffOperator^2 \func{f}{a}}{t\xi k} }{k} - t \operatorFunc{ \func{R}{t\xi k} -\func{R}{th + t\xi k}}{k}               \\
                                                   & = t^2 \operatorFunc{\operatorFunc{\DiffOperator^2 \func{f}{a}}{h}}{k} - t \operatorFunc{ \func{R}{t\xi k} -\func{R}{th + t\xi k}}{k}                                                                           \\
        \implies \dfrac{\func{\Delta}{t,h,k}}{t^2} & = \operatorFunc{\operatorFunc{\DiffOperator^2 \func{f}{a}}{h}}{k} - \dfrac{ \operatorFunc{ \func{R}{t\xi k} -\func{R}{th + t\xi k}}{k}}{t} \to \operatorFunc{\operatorFunc{\DiffOperator^2 \func{f}{a}}{h}}{k}
    \end{align*}
    which is what we wanted.
\end{proof}

%TODO: needs proof
\begin{theorem}
    The \(k_\cardinalTH\) derivative of a \(k\)-times differentiable function is a symmetric \(k\)-linear function.
\end{theorem}

\begin{proof}
    it's generalization of above.
\end{proof}

\begin{proposition}
    If \(f,g \in \calC^r\) are two functions then \(f \circ g \in \calC^r\).
\end{proposition}

\begin{proof}
    Let \(f: V' \to V''\) and \(g: V \to V'\) be two \(\calC^r\) functions and  \(\beta : \func{\calL}{V',V''} \times \func{\calL}{V,V'} \to \func{\calL}{V,V''}\) is a bilinear function such that
    \begin{equation*}
        \func{\beta}{\phi,\psi} = \phi \circ \psi
    \end{equation*}
    Now note that
    \begin{align*}
        \operatorFunc{\func{\DiffOperator \,}{ f \circ g}}{a} & = \operatorFunc{\DiffOperator f \circ g}{a}\circ \DiffOperator \func{g}{a}            \\
                                                              & = \func{\beta }{\operatorFunc{\DiffOperator f \circ g}{a}, \DiffOperator \func{g}{a}}
    \end{align*}
    Consider the following functions
    \begin{equation*}
        a \xmapsto[\calC^\infty]{\Delta} (a,a) \xmapsto[\calC^{r-1}]{(\DiffOperator f \circ g, \DiffOperator g)} (\operatorFunc{\DiffOperator f \circ g}{a}, \DiffOperator \func{g}{a}) \xmapsto[\calC^\infty]{\beta}  \operatorFunc{\func{\DiffOperator \,}{ f \circ g}}{a}
    \end{equation*}
    therefore \(\func{\DiffOperator \,}{ f \circ g} \in \calC^{r-1}\) and hence \(f \circ g \in \calC^{r}\).
\end{proof}

\begin{example}
    The inverse operator \(i : \func{\GL}{V} \to \func{\calL}{V,V}\) is in \(C^\infty\). Remember that
    \begin{equation*}
        \operatorFunc{\operatorFunc{\DiffOperator i}{A}}{M} = - A^{-1}M A^{-1}
    \end{equation*}
    Let \(\gamma : \func{\calL}{V,V} \times \func{\calL}{V,V} \to \func{\calL}{\func{\calL}{V,V} , \func{\calL}{V,V}}\) with
    \begin{equation*}
        \operatorFunc{\func{\gamma}{A,B}}{M} = - AMB
    \end{equation*}
    is a bilinear function. Therefore
    \begin{equation*}
        \operatorFunc{\operatorFunc{\DiffOperator i}{A}}{M} = \operatorFunc{\func{\gamma}{A^{-1},A^{-1}}}{M}
    \end{equation*}
    now
    \begin{equation*}
        A \xmapsto{i} A^{-1} \xmapsto[\calC^\infty]{\Delta} (A^{-1}, A^{-1}) \xmapsto[\calC^\infty]{\gamma} \operatorFunc{\DiffOperator i}{A}
    \end{equation*}
    Since we have proved that \(i\) is differentiable then \(\DiffOperator i\) is differentiable which means \(i\) is twice differentiable and so on. Hence \(i \in C^{\infty}\).
\end{example}

As a matter of notation if \(\phi : V_1 \times \dots \times V_n\) be an \(n\)-linear then
\begin{equation*}
    \phi \cdot h_1\dots h_n := \func{\phi}{h_1, \dots , h_n}
\end{equation*}
particularly if \(V_1 = \dots = V_n\)
\begin{equation*}
    \phi \cdot h^n := \func{\phi}{h, \dots , h}
\end{equation*}
Now one can describe a homogeneous polynomial of degree \(k\) with a symmetric \(k\)-linear function
\begin{align*}
    \func{p}{x} = \phi \cdot x^k
\end{align*}
Then, \(\func{p}{x}\) is differentiable since
\begin{equation*}
    x \xmapsto[\calC^\infty]{\Delta} (x,\dots,x) \xmapsto[\calC^\infty]{\phi} p
\end{equation*}
and
\begin{align*}
    \operatorFunc{\DiffOperator \func{p}{x}}{h} & = \operatorFunc{\DiffOperator \func{\phi}{\func{\Delta}{x}} \circ \DiffOperator \func{\Delta}{x}}{h} \\
                                                & = \DiffOperator \phi \cdot x^n \circ \func{\Delta}{h}                                                \\
                                                & = k \phi \cdot x^{k-1}h                                                                              \\
    \implies \DiffOperator \func{p}{x}          & = k \phi \cdot x^{k-1}
\end{align*}

\begin{theorem}[Taylor approximation]
    Let \(f : U \to W\) be \(k\)-times differentiable at \(a\), then
    \begin{equation*}
        \func{p_k}{x} = \func{f}{a} + \diffOperator \func{f}{a} \cdot (x-a) + \dfrac{1}{2!} \diffOperator^2 \func{f}{a} \cdot (x-a)^2 + \dots + \dfrac{1}{k!} \diffOperator^k \func{f}{a} (x-a)^k
    \end{equation*}
    is \(k_\cardinalTH\) degree \textbf{Taylor} polynomial. Then the followings hold
    \begin{enumerate}
        \item
              \begin{equation*}
                  \lim_{x \to a} \dfrac{\func{f}{x} - \func{p_k}{x}}{\norm{x-a}^{k}} = 0
              \end{equation*}
        \item \(\func{p_k}{x}\) is the only \(k_\cardinalTH\) degree polynomial with such property.
        \item Additionally, if \(f\) is \((k + 1)\)-times differentiable in a neighbourhood of \(a\) then the remainder
              \begin{equation*}
                  \func{R}{x} = \func{f}{x} - \func{p_k}{x}
              \end{equation*}
              can be estimated with
              \begin{equation*}
                  \norm{\func{R}{b}} \leq \dfrac{1}{(k+1)!} \sup \set{\norm {\DiffOperator^{k+1} \func{f}{\xi}}} \norm{b-a}^{k+1}
              \end{equation*}
              where \(\xi\) is on line connecting \(a\) to \(b\).
    \end{enumerate}
\end{theorem}

\begin{proof} \leavevmode
    \begin{enumerate}
        \item for \(k = 1\) it is equivalent to differentiability of \(f\). By induction, assume it is true for \(k = n -1\) and let \(\func{g}{x} = \func{f}{x} - \func{p_k}{x}\) then  \footnote{Differentiablity of order \(k\) implies differentiability of order \(k-1\) in a neighbourhood. }
              \begin{align*}
                  \DiffOperator \func{g}{x}        & = \DiffOperator \func{f}{x} - \DiffOperator \func{p_k}{x}                                                                                                                                           \\
                                                   & = \diffOperator \func{f}{x} - \DiffOperator\left[\func{f}{a} + \diffOperator \func{f}{a} \cdot (x-a)  + \dots + \dfrac{1}{n!} \diffOperator^n \func{f}{a} (x-a)^n \right]                           \\
                                                   & = \diffOperator \func{f}{x} - \left[ \diffOperator \func{f}{a}  + \dfrac{1}{1!} \diffOperator^2 \func{f}{a} \cdot (x-a) + \dots + \dfrac{1}{(n-1)!} \diffOperator^n \func{f}{a} (x-a)^{n-1} \right] \\
                  \intertext{which is equivalent to the proposition at \(n-1\) for \(\diffOperator \func{f}{a}\) and hence there exists a \(\delta > 0 \) such that if \(\norm{x-a} < \delta\)}
                  \norm{\DiffOperator \func{g}{x}} & \leq \epsilon \norm{x-a}^{n-1}
              \end{align*}
              by the \Cref{th:MultivariableMVT} we have
              \begin{align*}
                  \norm{\func{g}{x}} & = \norm{\func{g}{x} - \func{g}{a}} \leq \norm{x-a} \sup \norm{\DiffOperator \func{g}{\xi}} \\
                                     & \leq \epsilon \norm{x-a} \norm{\xi - a}^{k-1}                                              \\
                                     & \leq \norm{x-a}^{k}
              \end{align*}
        \item If there were two such polynomial \(p_1, p_2\) then for \(q = p_1 - p_2\) we have that
              \begin{equation*}
                  \lim_{x \to a} \dfrac{\func{q}{x}}{\norm{x-a}^k} = 0
              \end{equation*}
              then one can show that \(\func{q}{x} \equiv 0\). %TODO: do the rest
        \item Define \(g: \clcl{0}{1} \to W\) as such
              \begin{equation*}
                  \func{g}{t} = \func{f}{a + t(b-a)}
              \end{equation*}
              therefore
              \begin{equation*}
                  \func{g^{(n)}}{t} = \diffOperator^k \func{f}{a + t(b-a)} \cdot (b-a)^k
              \end{equation*}
              For each component of \(g\) we apply the single variable Taylor's approximation
              \begin{equation*}
                  \func{g_i}{1} - \sum_{n = 0}^{k} \dfrac{\func{g^{(n)}_i}{0}}{n!} = \dfrac{\func{g^{(k+1)}_i}{\xi_i}}{(k+1)!}
              \end{equation*}
              or equivalently
              \begin{multline*}
                  \norm{\func{R}{b}} =\norm{ \func{f}{b} - \sum_{n = 0}^{k} \dfrac{\diffOperator^n \func{f}{a} \cdot (b-a)^n}{n!} } \\= \dfrac{1}{(k+1)!} \norm{ {\begin{bmatrix}
                                  \diffOperator^{k+1} \func{f_1}{a + \xi_1(b-a)} \cdot (b-a)^k & \dots & \diffOperator^{k+1} \func{f}{a + \xi_m(b-a)} \cdot (b-a)^k
                              \end{bmatrix}} } %TODO: needs clearification.
              \end{multline*}
              which was what was wanted.
    \end{enumerate}
\end{proof}

\begin{theorem}
    Let \(f : U \to \Reals\) and \(p\) is an extremum of the function then
    \begin{equation*}
        \forall h, \ \operatorFunc{\DiffOperator \func{f}{p}}{h} = 0
    \end{equation*}
\end{theorem}

\begin{proof}
    For all \(h\) define \(g_h : \opop{-\epsilon}{\epsilon} \to \Reals\)
    \begin{equation*}
        \func{g_h}{t} = \func{f}{p + th}
    \end{equation*}
    then \(\func{g'_h}{0} = 0\).
\end{proof}

%TODO: needs proof
\begin{theorem}
    Let \(f : U \to \Reals\) be of \(\calC^2\), \(p\) be a critical point of \(f\), and \(\DiffOperator^2 \func{f}{p}\) be positive definite. Then, \(p\) is a local minimum of \(f\). (If \(\DiffOperator^2 \func{f}{p}\) is negative definite then \(p\) is local maxima.)
\end{theorem}
Assuming the following lemma
\begin{lemma} \label{lm:ContinuityOfPositiveDefinite}
    If \(\DiffOperator^2 f\) is continuous and positive definite at point \(p\) then it is positive definite in a neighbourhood of \(p\).
\end{lemma}
\begin{proof}
    We wish to prove that there exists a \(\delta > 0\) for all unit vectors in \(V\), \(e\), \( 0 < t < \delta\)
    \begin{equation*}
        \func{f}{p} \leq \func{f}{p + te}
    \end{equation*}
    To do so, define \(g_e : \opop{0}{\delta} \to \Reals\)
    \begin{equation*}
        \func{g_e}{t} = \func{f}{p + te}
    \end{equation*}
    then by the Taylor's theorem
    \begin{equation*}
        \func{g_e}{t} = \func{g}{0}  + \func{g'}{0}t + \dfrac{\func{g''}{\xi}}{2!}t^2
    \end{equation*}
    where \(\xi \in \opop{0}{t}\). Equivalently
    \begin{align*}
        \func{f}{p + te} & = \func{f}{p} + \operatorFunc{\DiffOperator \func{f}{p}}{e} + \dfrac{\diffOperator^2 \func{f}{p + t\xi} \cdot e^2}{2}t^2 \\
                         & = \func{f}{p} +  \dfrac{\diffOperator^2 \func{f}{p + t\xi} \cdot e^2}{2}t^2
    \end{align*}
    Using the \Cref{lm:ContinuityOfPositiveDefinite} there exists a neighbourhood of \(p\) such that
    \begin{equation*}
        \diffOperator^2 \func{f}{p + t\xi} \cdot h^2 > 0
    \end{equation*}
    for all \(h\) in the neighbourhood. Therefore,
    \begin{equation*}
        \func{f}{p + te} > \func{f}{p}
    \end{equation*}
    which is what we wanted.
\end{proof}

\section{Smoothness Classes}

Let \(f \in \calC^r\) then one can define the norm
\begin{equation*}
    \norm{f}_r = \max \set{\sup_{x \in U}\norm{\func{f}{x}}, \dots , \sup_{x \in U}\norm{\DiffOperator^r \func{f}{x}}}
\end{equation*}
and let the set of all such \(f\) with \(\norm{f}_r < \infty\) be denoted as \(\func{\calC^r}{U,W}\).

%TODO: needs proof
\begin{theorem}
    Uniform convergence in \(\calC^r\) is equivalent to Cauchy.
\end{theorem}

\begin{proof}

\end{proof}

\begin{theorem}
    \( \func{\calC^r}{U,W}\) under \(\norm{\cdot}_r\) is a Banach space.
\end{theorem}


\begin{definition}[Local convergence]
    A functional sequence \(f_n\) is \textbf{locally convergent} if for each \(x \in U\)  there exists a open set \(x \in V \subset U\) such that \(\left. f_n \right|_V\) is uniformly convergent.
\end{definition}

\begin{theorem}
    Let \(V,W\) be normed finite dimensional spaces, \(U \subset V\) is open and connected, \(x_0 \in U\) and \(f_n : U \to W\) is a sequence of differentiable function that
    \begin{enumerate}
        \item \(\func{f_n}{x_0}\) is convergent.
        \item \(\DiffOperator f_n : U \to \func{\calL}{V,W}\) is locally convergent to some function \(g : U \to \func{\calL}{V,W}\)
    \end{enumerate}
    then the sequence \(f_n\) is locally convergent to \(f : U \to W\) and \(\DiffOperator f = g\). Furthermore, because of connectedness of \(U\) for each \(x \in U\), \(\func{f_n}{x}\) is convergent.
\end{theorem}

%TODO: do the proof
\begin{proof}
    take open ball \(W\) around \(x_0\) such that \(\DiffOperator f_n|_W\) is uniformly convergent. then prove the first statement.
    \begin{equation*}
        \norm{\func{f_m}{x} - \func{f_n}{x}} \leq \norm{\operatorFunc{f_m - f_n}{x} - \operatorFunc{f_m - f_n}{x_0}} + \norm{\func{f_m}{x_0} - \func{f_n}{x_0}}
    \end{equation*}
    apply MVT here and make the bounds smaller using (2). Then prove the differentiability with e/3. To prove (3) use open/close argument.
\end{proof}

\section{Inverse function theorem}
Consider a function \(f\), we wish to find all the solutions to the equation
\begin{equation*}
    \func{f}{x} = y_0
\end{equation*}
To do so, we can define another function \(F_{y_0}\) such that
\begin{equation*}
    \func{F_{y_0}}{x} = x - \func{f}{x} + y_0
\end{equation*}
then if \(x\) is a solution to the equation, it is a fixed point of \(F_{y_0}\).

\begin{theorem} [Banach fixed point] \label{th:BanachFixedPoint}
    Let \(\metricSpace{X}{d}\) be a complete metric space and \(f : X \to X\) is such that for some \(0 \leq \lambda < 1\)
    \begin{equation*}
        \forall x,y \in X, \ \func{d}{\func{f}{x},\func{f}{y}} \leq \lambda \func{d}{x,y}
    \end{equation*}
    Then for each \(x \in X\) the sequence \(\set{\func{f^n}{x}}\) is convergent to \(p \in X\) such that \(\func{f}{p} = p \).
\end{theorem}

\begin{proof}
    Let \(x_n = \func{f^n}{x}\) for \(n \geq 0\) then
    \begin{equation*}
        \func{d}{x_n,x_{n+1}} \leq \lambda \func{d}{x_{n-1},x_{n}} \leq \lambda^n \func{d}{x_0,x_1}
    \end{equation*}
    thereofore
    \begin{equation*}
        \func{d}{x_n,x_m} \leq \sum_{i = n}^{m-1} \lambda^i \func{d}{x_0,x_1} \leq \func{d}{x_0,x_1} \dfrac{\lambda^n}{1 - \lambda}
    \end{equation*}
    hence \(\set{\func{f^n}{x}}\) is Cauchy and it is convergent to a point \(p\). Lastly,
    \begin{align*}
        \func{d}{\func{f}{p},p} & \leq \func{d}{\func{f}{p},\func{f}{x_n}} + \func{d}{\func{f}{x_n},p} \\
                                & \leq \lambda \func{d}{p,x_n} + \func{d}{x_{n+1},p} < \epsilon
    \end{align*}
\end{proof}

\begin{theorem}[Inverse function theorem]
    Let \(V,W\) be finite dimensional normed vector space such that \(\dim V = \dim W\) and \(U \subset V\) is open. If \(f : U \to W\) is continuously differentiable and for some \(a \in U\), \(\DiffOperator \func{f}{a}\) is invertible. Then, there are open set \(S \subset V\) and \(T \subset W\)such that \(a \in S  \subset U\) and \(\func{f}{a} \in T\) such that \(f|_S\) is bijective and \((f|_S)^{-1} = g\) where \(g \in \calC^1\) and
    \begin{equation*}
        \DiffOperator \func{g}{\func{f}{x}} = \left(\DiffOperator \func{f}{x}\right)^{-1}
    \end{equation*}
\end{theorem}

\begin{proof}
    Let \(S\) be an open convex set around \(a\) such that for all \(x \in S\)
    \begin{equation*}
        \norm{\DiffOperator \func{f}{x} - \DiffOperator \func{f}{a}} < \dfrac{1}{2} \norm{\DiffOperator \func{f^{-1}}{a}}^{-1}
    \end{equation*}
    hence \(\DiffOperator \func{f}{x}\) is invertible. Let \(T = \func{f}{S}\) then we shall prove the following
    \begin{enumerate}
        \item \(f|_S\) is bijective.

              Let \(\psi : S \to V\) with
              \begin{align*}
                  \func{\psi_y}{x}                                & = x - \left(\DiffOperator \func{f}{a}\right)^{-1} (\func{f}{x} - y)                                                            \\
                  \implies \DiffOperator \func{\psi_y}{x }        & = \DSOne_V - \left(\DiffOperator \func{f}{a}\right)^{-1} \DiffOperator \func{f}{x}                                             \\
                                                                  & = \left(\DiffOperator \func{f}{a}\right)^{-1} \circ \left( \DiffOperator \func{f}{a} - \DiffOperator \func{f}{x} \right)       \\
                  \implies \norm{\DiffOperator \func{\psi_y}{x }} & \leq \norm{\left(\DiffOperator \func{f}{a}\right)^{-1}} \norm{\DiffOperator \func{f}{a} - \DiffOperator \func{f}{x}}           \\
                                                                  & < \dfrac{1}{2} [\left(\DiffOperator \func{f}{a}\right)^{-1}] [\left(\DiffOperator \func{f}{a}\right)^{-1}]^{-1} = \dfrac{1}{2}
              \end{align*}
              therefore by mean value theorem
              \begin{equation*}
                  \norm{\func{\psi_y}{x_1} - \func{\psi_y}{x_2}} \leq \dfrac{1}{2} \norm{x_1 - x_2}
              \end{equation*}
              which follows that \(\psi_y\) has at most one fixed point because
              \begin{equation*}
                  \norm{\func{\psi_y}{x_1} - \func{\psi_y}{x_2}} = \norm{x_1 - x_2} \leq  \dfrac{1}{2} \norm{x_1 - x_2}
              \end{equation*}
              is a contradiction, and for that fixed point
              \begin{equation*}
                  \func{\psi_y}{x} = x - \left(\DiffOperator \func{f}{a}\right)^{-1} (\func{f}{x} - y) = x \implies y = \func{f}{x}
              \end{equation*}
              which means \(f\) is injective. By the definition of \(T\), \(f\) is surjective as well.

        \item  \(T\) is open.

              We wish to prove that for each \(\func{f}{x_0} = y_0 \in T\) we wish to prove there exist a \( \sigma > 0\) such that \(\func{B_\sigma}{y_0}\) is contained in \(T\). In other words, \(\forall y \in \func{B_\sigma}{y_0}\)
              \begin{align*}
                  \exists x \in S, \; \func{f}{x} = y \iff \func{\psi_y}{x} = x
              \end{align*}
              To apply the contraction fixed point we must find complete metric space \(X\) such that \(\func{\psi_y}{X} = X\). Choose \(\rho\) as small as needed that \(\overline{\func{B_\rho}{x_0}} \subset S\), which makes a complete metric space. Let \(\sigma =  \dfrac{r \rho}{2}\) where \(r = \norm{\left(\DiffOperator \func{f}{a}\right)^{-1}}^{-1}\). Lastly, we show that for each \(y \in \overline{\func{B_\sigma}{y_0}}\), \(\func{\psi_y}{ \overline{\func{B_\rho}{x_0}} }=  \overline{\func{B_\rho}{x_0}}\). That is, \(x \in \overline{\func{B_\rho}{x_0}}\) implies that \( \func{\psi_y}{x} \in \overline{\func{B_\rho}{x_0}}\).
              \begin{align*}
                   & \norm{\func{\psi_y}{x}  - x_0} \leq \norm{\func{\psi_y}{x} - \func{\psi_y}{x_0}} + \norm{\func{\psi_y}{x_0} - x_0}                              \\
                   & \leq  \dfrac{1}{2} \norm{x - x_0} +  \norm{\left(\DiffOperator \func{f}{a}\right)^{-1}(y - y_0)}  \leq \dfrac{\rho}{2} + \dfrac{\rho}{2} = \rho
              \end{align*}
        \item \(g = (f|_S)^{-1} : T \to S\) is continuously differentiable. Writting the differentiability criteria
              \begin{equation*}
                  \norm{\func{g}{y + h} - \func{g}{y} - \operatorFunc{\DiffOperator \func{g}{y}}{h}} \leq \epsilon \norm{h}
              \end{equation*}
              Let \(y = \func{f}{x}\) and \(y + h = \func{f}{x + k}\) then \(h = \func{f}{x + k} - \func{f}{x}\) and note
              \begin{equation*}
                  \norm{\func{\psi_y}{x + k} - \func{\psi_y}{x}} = \norm{k - \operatorFunc{\left(\DiffOperator \func{f}{a}\right)^{-1}}{h}} \leq \dfrac{1}{2} \norm{k}
              \end{equation*}
              which implies
              \begin{equation*}
                  \dfrac{1}{2}\norm{k} \leq \norm{\operatorFunc{\left(\DiffOperator \func{f}{a}\right)^{-1}}{h}} \leq \dfrac{3}{2} \norm{k}
              \end{equation*}
              \begin{align*}
                  \norm{k - \operatorFunc{\left(\DiffOperator \func{f}{x}\right)^{-1}}{\func{f}{x + k} - \func{f}{x}}} & =                                                                                                             \norm{\operatorFunc{\left(\DiffOperator \func{f}{x}\right)^{-1}}{\operatorFunc{\DiffOperator \func{f}{x}}{k} - \func{f}{x + k} - \func{f}{x}}} \\
                                                                                                                       & \leq \norm{\left(\DiffOperator \func{f}{x}\right)^{-1}} \norm{\func{f}{x + k} - \func{f}{x} - \operatorFunc{\DiffOperator \func{f}{x}}{k}}                                                                                                                   \\
                                                                                                                       & \leq \norm{\left(\DiffOperator \func{f}{x}\right)^{-1}} \epsilon \norm{k}                                                                                                                                                                                    \\
                                                                                                                       & \leq 2 \norm{\left(\DiffOperator \func{f}{x}\right)^{-1}} \norm{\left(\DiffOperator \func{f}{a}\right)^{-1}} \epsilon
              \end{align*}
              which proves the differentiability of \(g\) as \(\norm{\left(\DiffOperator \func{f}{x}\right)^{-1}}\) is bounded in \(S\). As shown, the inverse operator is \(i\) is continuous and therefore if \(\DiffOperator f\) is continuous, then \(\func{i}{\DiffOperator f}\) is continuous. In fact, if \(f \in C^k\) then \(g \in C^k\) as well.
    \end{enumerate}
\end{proof}

\begin{corollary}
    If \(f\) is continuously differentiable and \(\DiffOperator \func{f}{x}\) is invertible for every \(x \in U\), then for any open set \(S\), \(\func{f}{S}\) is an open set as well.
\end{corollary}

\begin{proof}
    By the inverse function theorem for each \(x \in S\) there is an open set \(U_x\) in \(S\) and \(V_x\) in \(W\) such that \(\func{f}{U_x} = V_x\), therefore
    \begin{equation*}
        \func{f}{S} =  \func{f}{\bigcup U_x} = \bigcup \func{f}{U_x} = \bigcup V_x
    \end{equation*}
    which is an open set.
\end{proof}

\section{Implicit function}
\begin{theorem}
    Let \(V,W\) be finite dimensional normed vector spaces and \(U \subset V \times W\) is open. If \(f: U \to W\), \(f \in \calC^1\) where \(\func{f}{x_0,y_0} = z_0\) and \(\operatorFunc{\DiffOperator f|_{\set{x_0} \times W}}{x_0,y_0}\) is invertible then there exist open set \(S\) around \(x_0\) and \(T\) around \(y_0\) that \(S \times T \subset U\), such that for each \(x \in S\) there exists a unique \(y \in T\) with
    \begin{equation*}
        \func{f}{x,y} = z_0
    \end{equation*}
    hence there is a continuously differentiable function \(\phi: S \to T\) such that \(\func{\phi}{x} = y\) where \(\func{f}{x,y} = z_0\) and
    \begin{equation*}
        \DiffOperator \phi = - \left(\DiffOperator_y f\right)^{-1} \DiffOperator_x f
    \end{equation*}
\end{theorem}

\begin{proof}
    To apply the inverse function theorem, we need a function whose domain and range have the same dimension. So define, \(F: U \to V \times W\)
    \begin{equation*}
        \func{F}{x,y} = (x,\func{f}{x,y})
    \end{equation*}
    Then
    \begin{equation*}
        \DiffOperator \func{F}{x_0,y_0} = \left[\begin{array}{c|c}
                I_n                                                           & \DSZero                                                       \\ \hline
                \operatorFunc{\DiffOperator f|_{\set{y_0} \times U}}{x_0,y_0} & \operatorFunc{\DiffOperator f|_{\set{x_0} \times W}}{x_0,y_0}
            \end{array}\right]
    \end{equation*}
    Since \(I_n\) and \(\operatorFunc{\DiffOperator f|_{\set{x_0} \times W}}{x_0,y_0}\) are both invertible then \(\DiffOperator \func{F}{x_0,y_0}\) is invertible as well. By inverse function theorem there are open set \(\Omega_1\) around \((x_0,y_0)\) and \(\Omega_2\) around \((x_0,z_0)\) such that \(F|_{\Omega_1}\) is \(\calC^1\) diffeomorphism from \(\Omega_1\) to \(\Omega_2\). Let \(S = \set[x]{(x,z_0) \in \Omega_2}\) and the set \(T = \set[y]{(x,y) \in \Omega_1}\) , then we shall prove that for each \(x \in S\) there exists exactly one \(y \in T\) \footnote{show that they are open}. If \(x \in S\) then there exists \((x,y) \in \Omega_1\) such that \(\func{F}{x,y} = (x,z_0)\), suppose that there are two such \(y\), \(y_1\) and \(y_2\) for \(x\)
    \begin{equation*}
        \func{F}{x,y_1} = (x,\func{f}{x,y_1}) = (x,z_0) = (x,\func{f}{x,y_2}) = \func{F}{x,y_2}
    \end{equation*}
    which since \(F|_{\Omega_1}\) is injective then \(y_1 = y_2\).

    Let \(G: \Omega_2 \to \Omega_1\) be the local inverse of \(F\) and \(\phi : S \to T\)
    \begin{equation*}
        \func{\phi}{x} = \operatorFunc{ \pi_2 \circ G}{x,z_0}
    \end{equation*}
    which implies that \(\phi\) is continuously differentiable. Lastly,
    \begin{align*}
        \DiffOperator \func{f}{x,\func{\phi}{x}} & = \DiffOperator \func{f}{x,\func{\phi}{x}} \circ (I, \DiffOperator \func{\phi}{x})                                         \\
                                                 & = \DiffOperator_x \func{f}{x,\func{\phi}{x}} + \DiffOperator_y  \func{f}{x,\func{\phi}{x}}\DiffOperator \func{\phi}{x} = 0 \\
        \implies  \DiffOperator \phi             & = - \left(\DiffOperator_y f\right)^{-1} \DiffOperator_x f
    \end{align*}
\end{proof}


\section{Rank theorem}
A generalization of \(PAQ = \begin{bmatrix}
    I_r & 0 \\
    0   & 0 \\
\end{bmatrix}\)
\begin{theorem}
    Let \(f: U \to W\) be of class \(\calC^1\) and
    \begin{equation*}
        \forall x \in U, \; \rank \DiffOperator \func{f}{x} = k
    \end{equation*}
    then for each \(p \in U\) there exist open subsets \(p \in U_0\) and \(\func{f}{p} \in W_0\) and diffeomorphisms
    \begin{align*}
        \alpha & : U_0 \to U'_0 \\
        \beta  & : V_0 \to V'_0
    \end{align*}
    such that
    \begin{equation*}
        \func{ \beta \circ f \circ a^{-1} }{x_1, \dots, x_n} = (x_1, \dots , x_k, 0 , \dots, 0)
    \end{equation*}
\end{theorem}

\begin{proof}
    Without loss of generality, assume \(V = \Reals^n\), \(W = \Reals^m\), with a transition \(p = 0 \in \Reals^n\), \(\func{f}{p} = 0 \in \Reals^m\), and with a change of basis,
    \begin{equation*}
        \DiffOperator \func{f}{p} = \begin{bmatrix}
            I_k     & \DSZero \\
            \DSZero & \DSZero \\
        \end{bmatrix}
    \end{equation*}
    Suppose
    \begin{equation*}
        \func{f}{x,y} = (\func{f_1}{x,y}, \func{f_2}{x,y})
    \end{equation*}
    where \(x \in \Reals^k\), \(y \in \Reals^{n-k}\), \(\func{f_1}{x,y} \in \Reals^k\), and \(\func{f_2}{x,y} \in \Reals^{m-k}\). Then
    \begin{equation*}
        \DiffOperator \func{f}{0,0} = \begin{bmatrix}
            I_k     & \DSZero \\
            \DSZero & \DSZero
        \end{bmatrix}
    \end{equation*}

    Let \(\func{H}{x,y,u} := u - \func{f_1}{x,y}\), where \(u \in \Reals^k\). Note that \(H \in \calC^1\) and \(\DiffOperator_x H = -\DiffOperator_x f_1\) is invertible at \((0,0,0)\). Therefore, there are open set \(\Omega_1\) around \((0,0) \in \Reals^{k} \times \Reals^{n-k}\) and \(\Omega_2\) around \(0 \in \Reals^k\) such that \(H|_{\Omega_1}\) is bijective and there exists a function \(\phi: \Omega_1 \to \Omega_2\) such that
    \begin{equation*}
        x = \func{\phi}{u,y}
    \end{equation*}
    whenever \(\func{H}{x,y,u} = 0\). Then let \(\alpha^{-1} : \Omega_1 \to \Omega_2 \times \Reals^{n-k}\) be
    \begin{equation*}
        \func{\alpha^{-1}}{u,y} = (\func{\phi}{u,y},y)
    \end{equation*}
    which is a diffeomorphism around \((0,0) \in \Reals^{k} \times \Reals^{n-k}\) and
    \begin{equation*}
        \func{f \circ \alpha^{-1}}{u,y} = (\func{f_1}{\func{\phi}{u,y},y}, \func{f_2}{\func{\phi}{u,y},y}) = (u, \func{\eta}{u,y})
    \end{equation*}
    \(\DiffOperator_y \eta \equiv 0\) that is \(\func{\eta}{u,y} = \func{\eta}{u,0}\) since, \(\func{\rank}{\DiffOperator \func{f \circ \alpha^{-1}}{u,y}} = \func{\rank}{\DiffOperator \func{f}{u,y}}\) and thus
    \begin{equation*}
        \DiffOperator \func{f \circ \alpha^{-1}}{u,y} = \begin{bmatrix}
            I_k                  & \DSZero              \\
            \DiffOperator_u \eta & \DiffOperator_y \eta
        \end{bmatrix}
    \end{equation*}
    Let \(\beta : \Reals^k \times \Reals^{m-k} \to \Reals^k \times \Reals^{m-k}\)
    \begin{equation*}
        \func{\beta}{u,z} = (u,z - \func{\eta}{u,z})
    \end{equation*}
    which is clear diffeomorphism around \((0,0) \in \Reals^k \times \Reals^{m-k}\) which means that
    \begin{equation*}
        \func{ \beta \circ f \circ \alpha^{-1} }{x,y} = (x,0)
    \end{equation*}
\end{proof}

\section{Lagrange Multiplier}
\begin{theorem}
    Let \(f: U \subset \Reals^n \to \Reals\) is differentiable and \(g : U \to \Reals\) is continuously differentiable, \(S = \func{g^{-1}}{0} \subset U\) and \(\nabla \func{g}{s} \neq 0\), \(\forall s \in S\). Assume \(x_0 \in S\) 
    \begin{equation*}
        \max_{x \in S} \func{f}{x} = \func{f}{x_0}
    \end{equation*}
    then there exists \(\lambda \in \Reals\) such that 
    \begin{equation*}
        \nabla f|_{x_0} = \lambda \nabla g|_{x_0}
    \end{equation*}
\end{theorem}

\begin{proof}
    Since \(\rank \DiffOperator g = 1\) everywhere, then there are diffeomorphism \(\alpha,\beta\) such that 
    \begin{equation*}
        \func{\beta \circ g \circ \alpha^{-1}}{t_1, \dots ,t_n} = t_n
    \end{equation*}
\end{proof}

{\Large\textbf{Exercises}}
\begin{enumerate}
    \item Using l'Hopital's rule show that
          \begin{equation*}
              \lim_{t \to 0} \dfrac{\func{\Delta}{t,h,k}}{t^2} = \dfrac{\operatorFunc{\diffOperator \func{f}{a}}{h,k} + \operatorFunc{\diffOperator \func{f}{a}}{k,h}}{2}
          \end{equation*}
\end{enumerate}
\newpage


\part{Complex Analysis}
\chapter{Topology}
\section{Topology}
A set \(N \subset S\) is called a neighbourhood of \(x \in S\) if it contains a ball \(\func{B_r}{x}\). A set is open if it is a neighbourhood of all its points. A point \(x \in X\) is an isolated point if it has a neighbourhood whose intersection with \(X\) reduces to \(x\). An accumulation point is a point that is not isolated. 

\begin{proposition}
      A non-empty open set in plane is connected if and only if any two points can be joined by a polygon which lies in the set.
\end{proposition}

\begin{definition}
      A non-empty connected open set is called a region. A component or a maximal connected set of a set \(S\) is a connected subset which is not contained in any larger connected subset.
\end{definition}

\begin{proposition}
      Every set has a unique decomposition into component.
\end{proposition}

\begin{proposition}
      In \(\Reals^n\) the components of any open set are open.
\end{proposition}

\begin{definition}
      A set \(A\) is dense in \(X\) if \(\closure E = X\). A metric space is separable if there exists a countable subset union of disjoint regions. A topological space is locally connected if for each neighbourhood of its points there is a connected open subset for that neighbourhood. A metric space that open balls \(\func{B_r}{x}\) are connected is locally connected.
\end{definition}

\begin{proposition}
      In a locally connected separable space every open set is union of disjoint regions.
\end{proposition}

\begin{definition}
      A set \(S\) is totally bounded if for every \(\epsilon > 0\), \(S\) can be covered by finitely many balls of radius \(\epsilon\).
\end{definition}


\section{Compact sets}
\begin{definition}[Point of accumulation]
      point \(v\) is a \textbf{point of accumulation} for the sequence \(\set{z_n}\) if for given \(\epsilon > 0\) there exists infinitely many \(n\) such that 
      \begin{equation*}
            \abs{z_n - v} < \epsilon
      \end{equation*}
      Similarly, a point of accumulation of an infinite set \(S\) is a point \(v\) that for each open set \(U\) containing \(v\) there are infinitely many elements of \(S\).
\end{definition}

\begin{theorem}[Weierstrass-Bolzano theorem]
      If \(S\) is an infinite bounded set of real numbers, then \(S\) has a point of accumulation.
\end{theorem}

\begin{definition}
      A \textbf{compact} set \(S\) is a set that every sequence of its elements has a point of accumulation in \(S\). The following defintions are equivalent 
      \begin{enumerate}
            \item Every infinite subset of \(S\) has a point of accumulation in \(S\).
            \item Every sequence of elements of \(S\) has a convergent subsequent whose limit is in \(S\).
      \end{enumerate}
\end{definition}

\begin{theorem}
      A complex set is compact if and only if it is closed and bounded.
\end{theorem}

\begin{theorem}
      Let \(S_1 \supset S_2 \supset \dots\) be a sequence of non-empty closed subsets of a compact set \(S\). Then, the interestion of all \(S_n\) is not empty.
\end{theorem}

\begin{theorem}
      Let \(S\) be a compact set and \(f\) be continuous function on \(S\). Then the image of \(f\) is compact.
\end{theorem}

\begin{theorem}
      Let \(S\) be a compact set and \(f\) be continuous function on \(S\). Then  \(f\) is uniformly continuous.
\end{theorem}


\begin{definition}
      Let \(A,B\) be two sets of complex numbers. The \textbf{distance} between them is 
      \begin{equation*}
            \func{d}{A,B} = \min_{\substack{\alpha \in A \\ \beta \in B}} \abs{\alpha - \beta}
      \end{equation*}
\end{definition}

\begin{theorem}
      Let \(S\) be a closed set and let \(v\) be a complex number. There exists a point \(w \in S\) such that 
      \begin{equation*}
            \func{d}{S,\set{v}} = \abs{w - v}
      \end{equation*}
\end{theorem}

\begin{theorem}
      Let \(K\) be a compact set and let be \(S\) a closed set. Then, there are elements \(\alpha \in K\) and \(\beta \in S\) such that 
      \begin{equation*}
            \func{d}{K,S} = \abs{\alpha - \beta}
      \end{equation*}
\end{theorem}

\begin{theorem}
      A set is compact if and only if it is complete and totally bounded.
\end{theorem}

\begin{corollary}
      Let \(K\) be compact. Let \(r\) be a real number greater than zero. There exists a finite number of discs of radius \(r\) whose union contains \(K\).
\end{corollary}

We say that a family of open set \(\set{U_i}\) covers a set \(S\) when for every \(z \in S\), \(z \in U_i\) for some \(i\) as well. A subcovering of \(S\) is a covering of \(S\) with a subfamily of \(\set{U_i}\). If that subfamily is finite we say that it is a finite subcovering of \(S\). \(S\) is \textbf{covering compact} if every open convering can reduce to a finite subcovering.

\begin{theorem}
      Let \(S\) be a set then \(S\) is sequentially comapct if and only if covering compact.
\end{theorem}


\section{Connectedness}
\begin{proposition}
      A path connected set is connected but the converse is true when the set is open.
\end{proposition}
\chapter{Complex Numbers}
\section{Algebra of Complex Numbers}
We define complex numbers to be all the pairs of real numbers \((x,y)\) with following addition and multiplication:
\[(x_1,y_1) +' (x_2,y_2) = (x_1 + x_2 , y_1 + y_2)\]
\[ (x_1,y_1) \cdot' (x_2,y_2) = (x_1x_2 - y_1y_2, x_1y_2 + x_2y_1)\]
where \(+\) and \(\cdot\) are real addition and multiplication, respectively.
We denote the set of complex numbers with \(\Complex\).
It is easy to check that addition and multiplication defined above have the following:
\begin{enumerate}
        \item Addition and multiplication are commutitive:
              \[z + w = w + z \quad z,w \in \Complex\]
              \[ z\cdot w = w \cdot z \quad z,w \in \Complex\]

        \item Addition and multiplication are associative:
              \[(z + w) + u = z + (w + u) \quad z,w,u \in \Complex\]
              \[ (z \cdot w) \cdot u = z \cdot (w \cdot u) \quad z,w,u \in \Complex\]
        \item Addition and multiplication are distributive:
              \[(z + w) \cdot u = z \cdot u + w \cdot u \quad z,w,u \in \Complex\]
        \item Addition and multiplication have unique identity elements \(0 = (0,0)\) and \(1 = (1,0)\), respectively.
        \item Every complex number \(z\) has a unique addition inverse. Denoted by \(-z\).
        \item Every non-zero complex number \(z\) has a unique multiplication inverse. Denoted by \(z^{-1}\) or \(\dfrac{1}{z}\).
\end{enumerate}
Which means \(\Complex\) is a field. 

We can represent in many other forms. Two of the most commonly for \(z = (x,y)\) used are :
\[ z = x + iy \quad \text{where} \; i = (0,1)\]
\[ z = re^{i\theta} \quad r \geq 0 \,, \theta \in \Reals \]
\[ z = \begin{bmatrix}
                x & -y \\
                y & x
        \end{bmatrix} \]
It is easy to see that \(i^2 = -1\). We also define the functions \(\func{\Re}{z} = x\) and \(\func{\Im}{z} = y\). In the second representation \(r\) is the distance from origin and \(\theta\) is the angle between the positive real axis and the ray passing through \(z\). One can also view complex number as an extension of real numbers in which every polynomial has a root.

\begin{theorem}
      If \(K\) is field such that every odd degree polynomial has at a least a root and for all \(a \in K\), either \(x^2 = a\) or \(x^2 = -a\) has a root then it is sufficient to \textit{add} the root of \(x^2 = -1\) to \(K\) so that every polynomial has a root.
\end{theorem}

For every complex number \(z = x + iy\) there exists the mapping \(\bar{z}: \Complex \to \Complex\) where \(\bar{z} = x - iy\) and is called the \(\emph{conjugate}\) of \(z\).
\begin{proposition}
        The following properties are satisfied.
        \begin{enumerate}
                \item \(\overline{z + w} = \bar{z} + \bar{w}\).
                \item \(\overline{zw} = \bar{z}\bar{w}\).
                \item \(z = \bar{\bar{z}}\).
                \item \(z = \bar{z}\) if and only if \(z \in \mathbb{R}\).
                \item The following relations hold:
                      \[ \func{\Re}{z} = \dfrac{z + \bar{z}}{2} \quad,\quad \func{\Im}{z} = \dfrac{z - \bar{z}}{2i}\]
        \end{enumerate}
\end{proposition}
Another mapping is the \(\emph{norm}\) or \(\emph{modulus}\) function, \(\abs{z}: \Complex \to \Reals \) where \(\abs{z} = \sqrt{x^2 + y^2}\). Geometrically speaking the norm of \(z\) gives the distance of \(z\) form origin.
\begin{proposition}
        The following properties are satisfied.
        \begin{enumerate}
                \item \(\abs{z} \geq 0 \quad \forall z \in \Complex\) and especially \(\abs{z} = 0\) if and only if \(z = 0\)
                \item \(\abs{zw} = \abs{z}\abs{w}\).
                \item \(\abs{z}^2 = z\bar{z}\).
                \item \(\abs{z} = \abs{-z} = |\bar{z}|\)
                \item The following inequalities hold:
                      \[ -\abs{z} \leq \func{\Re}{z} \:,\: \func{\Im}{z} \leq \abs{z}\]
                \item Triangle inequality:
                      \[ \abs{z + w} \leq \abs{z} + \abs{w} \]
                      \[ \abs{\abs{z} - \abs{w}} \leq \abs{z- w}\]
        \end{enumerate}
\end{proposition}

The point at infinity, \(\infty\), is informally a point that is unboundedly far from the origin. The extended complex plane is defined as \(\Complex \cup \set{\infty}\). Every line passes through \(\infty\) but no half plane contains it.

\section{Riemann sphere}
\textbf{Riemann sphere} is a unit sphere centered at the origin complex plane. We can map every point (except \(\bracket{0,0,1}\)) on the sphere to a point in complex plane by the following bijective transformation.
\begin{equation*}
      \func{\phi}{z} = \bracket{x_1, x_2, x_3}= \bracket{\dfrac{z + \bar{z}}{1 + \abs{z}^2} , \dfrac{z - \bar{z}}{i \bracket{1 + \abs{z}^2}} , \dfrac{ \abs{z}^2 - 1}{\abs{z}^2 + 1} }
\end{equation*}
Geometrically, this transformation maps every point \(z\) to the interestion of the line connecting it to the \(\bracket{0,0,1}\) and the sphere. Furthermore, we can define \(\func{\phi{\infty}} = \bracket{0,0,1}\).

\section{Limits and continuity}

Let \(f : U \to \Complex\), \(\alpha\) be an adherent point of \(U\), and \(w\) be a complex number. Then 
\begin{equation*}
      w = \lim_{z \to \alpha} \func{f}{z} 
\end{equation*}
when 
\begin{equation*}
      \forall \epsilon > 0, \; \exists \delta > 0 \ \suchThat \ z \in U , \abs{z - \alpha} < \delta \implies \abs{\func{f}{z} - w} < \epsilon
\end{equation*}
By defining the \(\epsilon-\)neighbourhood of \(\infty\) to the 
\begin{equation*}
      \set<z>{\dfrac{1}{z} < \epsilon}
\end{equation*}
we can extend the defintion of limit to when \(\alpha\) or \(w\) are \(\infty\).
We say that \(f\) is continuous at \(\alpha\) if 
\begin{equation*}
      \lim_{z \to \alpha} \func{f}{z} = \func{f}{\alpha}
\end{equation*}

\begin{definition}
      A function \(f\) is said to be uniformly continuous if 
      \begin{equation*}
            \forall \epsilon > 0 , \; \exists \delta > 0 \ \suchThat \ z,w \in S , \abs{z - w} < \delta \implies \abs{\func{f}{z} - \func{f}{w}}
      \end{equation*}
\end{definition}

\section{Sequences and convergence}
A Sequence \(\set{z_n}_{n \in \Naturals}\) is said to be convergent to \(z\) if:
\begin{equation*} 
      \forall \epsilon > 0 , \; \exists N \ \suchThat \ n \geq N \implies \abs{z - z_n} < \epsilon
\end{equation*}
and it is denoted as 
\begin{equation*}
      z = \lim_{n \to \infty} z_n
\end{equation*}

A sequence \(\set{z_n}\) is a \textbf{Cauchy sequence} if 
\begin{equation*}
      \forall \epsilon, \; \exists N \ \suchThat \ n,m \geq N \implies \abs{z_m - z_n} < \epsilon
\end{equation*}
If we write \(z_n = x_n + iy_n\), since \(\abs{z_n - z_m} = \sqrt{(x_n - x_m)^2 + (y_n - y_m)^2}\) and \(\abs{x_n - x_m}\leq \abs{z_n - z_m} \), \( \abs{y_n - y_m} \leq \abs{z_n - z_m}\), we can conclude that \(\set{z_n}\) is Cauchy if and only if \(\set{x_n}\) and \(\set{y_n}\) are Cauchy sequences. Thus, since for real sequences a Cauchy sequence is convergent then, a complex Cauchy sequence is convergent as well.

Due to similarities between the definition given above and their real counterpart, most results dealing with limits can be easily extended to complex numbers.

A series \(\sum a_n\) is absolutely convergent if \(\sum \abs{a_n}\) is convergent. 
\begin{proposition}
      If the series \(\sum a_n\) is absolutely convergent then it can be summed in any order.
\end{proposition}

\begin{proposition}
      If double sum \(\sum_m \sum_n a_{mn}\) is absolutely convergent then the summation order can be interchanged. 
      \begin{equation*}
            \sum_{m} \sum_n a_{mn} = \sum_n \sum_m a_{mn}
      \end{equation*}
      The resulting series obtained is absolutely convergent and converges to the same value.
\end{proposition}

\begin{definition}
      Let \(\set{a_n}\) and \(\set{b_n}\) be two sequences of positive real numbers. Then 
      \begin{equation*}
            a_n \equiv b_n \qquad \text{for} \quad n \to \infty
      \end{equation*} 
      if for each \(n\), there exists a \(u_n \in \Reals^+\) such that \(\lim_{n \to \infty} u_n^{\frac{1}{n}} = 1\) and \(a_n = b_n u_n\).
\end{definition}
\section{Function spaces and power series}
A family of functions \(\set{\func{f_n}{x} : X \to Y}\) converges uniformly to \(f : X \to Y\), denoted \(f_n \rightrightarrows f\)
\begin{equation*}
      \forall \epsilon, \; \exists N \ \suchThat \ n \geq N \implies \func{d_Y}{\func{f_n}{x} - \func{f}{x}} < \epsilon, \; \forall x
\end{equation*}
if \(Y\) is a complete metric space then Cauchy convergence becomes equivalent to uniform convergence 
\begin{equation*}
      \forall \epsilon, \; \exists N \ \suchThat \ n,m \geq N \implies \func{d_Y}{\func{f_n}{x} - \func{f_m}{x}} < \epsilon, \; \forall x
\end{equation*}

\(\sum \func{f_n}{x}\) converges uniformly if the partial sums \(s_m = \sum^m f_n\) converges uniformly.

\begin{theorem}[Weierstrass M-test]
      Suppose that \(\set{\func{f_n}{x}}\) is a sequence of real/complex-valued functions defined on a set \(X\), and there is a sequence of non-negative numbers \(M_n\) satisfying the conditions 
      \begin{itemize}
            \item \(\abs{\func{f_n}{x}} \leq M_n\) for all \(n \geq 1\) and all \(x \in X\)
            \item \(\sum_{n = 1}^{\infty} M_n\) converges 
      \end{itemize}
      then the series \(\sum_{n=1}^{\infty} \func{f_n}{x}\) converges absolutely and uniformly on \(X\).
\end{theorem}


\begin{theorem}
      Let \(\set{\func{f_n}{x}}\) be a sequence of continuous functions that \(f_n \rightrightarrows f\). Then \(f\) is continuous.
\end{theorem}
\chapter{Analytic Functions}
\section{Differentiability}
Let \(f: U \overset{open}{\subset} \Complex \to \Complex\) and \(z\) be a point in \(U\). \(f\) is said to be complex differentiable at \(z\) if the following limit exists.
\begin{equation*}
      \func{f'}{z} = \lim_{h \to 0} \dfrac{\func{f}{z + h} - \func{f}{z}}{h}
\end{equation*}
Equivalently, there exists a \(c \in \Complex\) such that 
    \begin{equation} 
        \func{f}{z_0 + h} - \func{f}{z_0} - hc = \littleO{h} 
\end{equation}
Differentiation rules are similar to real differentiation and are proved the same way. 

\(f\) is \textbf{holomorphic} if it is differentiable at every point of \(U\). Taking the limit along the real and imaginary axes and equating them gives the \textbf{Cauchy-Riemann equations}

\begin{equation*}
      \PDiff{u}{x} = \PDiff{v}{y} ,\quad \PDiff{u}{y} = - \PDiff{v}{x}
\end{equation*}

Assuming that \(u\) and \(v\) are twice differentiable, we can see that 
\begin{align*}
      \Delta u &= \dfrac{\PDiffOperator^2 u}{\PDiffOperator x^2} + \dfrac{\PDiffOperator^2 u}{\PDiffOperator y^2}\\
      &= \dfrac{\PDiffOperator^2 v}{\PDiffOperator x \PDiffOperator y} - \dfrac{\PDiffOperator^2 v}{\PDiffOperator y \PDiffOperator x} = 0
\end{align*}
Which means that \(u,v\) satisfy the \textit{Laplace's equation} and thus are harmonic functions. 

\begin{theorem}
      If \(u,v\) and have continuous first order partial derivatives and satisfy the Cauchy-Riemann equations then \(\func{f}{z} = \func{u}{x,y} + \func{v}{x,y}\) is analytic with continuous derivative.
\end{theorem}

\section{Linear transformation}
A linear fractional transformation, also called \textit{M{\"o}bius transformation},  is a function in form of 
\begin{equation*}
    w = \func{S}{z} = \dfrac{az + b}{cz + d}
\end{equation*}
with \(ad - bc \neq 0\). If \(c \neq 0\) then \(\func{S}{\infty} = \frac{a}{c}\) and \(\func{S}{-\frac{d}{c}} = \infty\). The inverse of a linear fractional transformation is linear fractional itself and is equal to 
\begin{equation*}
    \func{S^{-1}}{w}= \frac{dw - b}{-cw + a}
\end{equation*}
If \(ad -bc = 1\) we say that \(S\) is normalized and since multiplying \(a,b,c,d\) by a non-zero factor does not change the function we can assume that all M{\"o}bius transformations are normalized. 

\begin{theorem}
    The set of all M{\"o}bius transformations \(\calM\) is a group under composition.
    \begin{equation*}
        \calM = \set<\dfrac{az + b}{cz + d}> {ad - bc = 1, a,b,c,d \in \Complex}
    \end{equation*}
\end{theorem}

\begin{proposition}
    Give 3 different points \(z_1,z_2,z_3\) there exists a linear transformation that takes to \(1, 0 , \infty\) in that order. If non of the point is \(\infty\) then 
    \begin{equation*}
        \func{S}{z} = \dfrac{z - z_2}{z - z_3} \dfrac{z_1 - z_3}{z_1 - z_3}
    \end{equation*}
    if one of them is \(\infty\) then 
    \begin{equation*}
        \func{S}{z} = \dfrac{z - z_2}{z - z_3} , \quad \func{S}{z} = \dfrac{z_1 - z_3}{z - z_3} , \quad \func{S}{z} = \dfrac{z - z_2}{z_1 - z_2}
    \end{equation*}
    for \(z_1 =\infty, z_2 = \infty, z_3 = \infty\) respectively.
\end{proposition}
\begin{proof}
    It is clear that \(S\) as defined above is such a transformation. To prove \(S\) is unique, suppose there is another transformation \(T\) that satisfies the condition. Suppose that neither of the points is \(\infty\) and  
    \begin{equation*}
        \func{T}{z} = \dfrac{az + b}{cz + d}
    \end{equation*}
    for some \(a,b,c,d \in \Complex\) with \(ad - bc \neq 0\). Consider \(TS^{-1}\)
    \begin{align*}
        \func{TS^{-1}}{z} &= \dfrac{a\func{S^{-1}}{z} +b }{c \func{S^{-1}}{z} + d}\\
        &= \dfrac{a \bracket{z - z_2}\bracket{z_1 - z_3}  + b \bracket{z - z_3}\bracket{z_1 - z_2}}{c \bracket{z - z_2}\bracket{z_1 - z_3}  + d \bracket{z - z_3}\bracket{z_1 - z_2}}\\
        &= \dfrac{\bracket{az_1 + bz_1 - az_3 - bz_2} z + az_2z_3 - az_2z_1 + bz_3z_2 - bz_3z_1}{\bracket{cz_1 + dz_1 - cz_3 - dz_2} z + cz_2z_3 - cz_2z_1 + dz_3z_2 - dz_3z_1}
    \end{align*}
    We know that 
    \begin{equation*}
        \func{TS^{-1}}{1} = 1, \quad  \func{TS^{-1}}{1} = 1, \quad  \func{TS^{-1}}{\infty} = \infty, \quad 
    \end{equation*}
    which implies that 
    \begin{equation*}
        \begin{cases}
            &az_2z_3 - az_2z_1 + bz_3z_2 - bz_3z_1 = 0\\
            &cz_1 + dz_1 - cz_3 - dz_2 = 0\\
            &az_1 + bz_1 - az_3 - bz_2 = cz_2z_3 - cz_2z_1 + dz_3z_2 - dz_3z_1
        \end{cases}
    \end{equation*}
    hence 
    \begin{equation*}
        \func{TS^{-1}}{z} = \dfrac{\bracket{az_1 + bz_1 - az_3 - bz_2} z}{az_1 + bz_1 - az_3 - bz_2} = z
    \end{equation*}
    and thus \(TS^{-1}\) is the identity and we must have \(T = S\).
\end{proof}
In general we have the following 
\begin{proposition}
    For any two sets of distinct complex numbers \(z_1,z_2,z_3\) and \(w_1,w_2,w_3\) there exists a linear fractional transformation that takes \(z_i\) to \(w_i\) for \(i = 1,2,3\).
\end{proposition}
\begin{proof}
    We first show that following lemma 
    \begin{lemma}
        Any linear fractional transformation has at most two fixed points unless it is the identity. 
    \end{lemma}
    \begin{prooflemma}
        It is clear that 
        \begin{equation*}
            \frac{az + b}{cz + d} = z \implies cz^2 + \bracket{d-a}z + b = 0
        \end{equation*}
        has at most two solutions unless \(c = 0\) and \(d - a = 0\).
    \end{prooflemma}
    Then consider the following linear fractional transformation (interpreted appropriately if any of them is \(\infty\))
    \begin{equation*}
        S = w_1\frac{\bracket{z - z_2}\bracket{z - z_3}}{\bracket{z_1 - z_2} \bracket{z_1 - z_3}} + w_2\frac{\bracket{z - z_1}\bracket{z - z_3}}{\bracket{z_2 - z_1} \bracket{z_2 - z_3}} + w_3\frac{\bracket{z - z_1}\bracket{z - z_2}}{\bracket{z_3 - z_1} \bracket{z_3 - z_1}}
    \end{equation*}
    Clearly \(S\) the required property. Suppose there exists another linear fractional transformation \(T\) with such property. We have 
    \begin{equation*}
        TS^{-1} w_i = w_i , \quad i = 1,2,3
    \end{equation*}
    which means \(TS^{-1}\) is the identity and we must have \(T = S\).
\end{proof}

\begin{definition}[Cross ratio]
    The cross ratio of \(\bracket{z_1,z_2;z_3,z_4}\) is the image of \(z_1\) under the transformation \(S\) that takes \(z_1,z_2,z_3\) to \(1,0,\infty\). That is 
    \begin{equation*}
        \bracket{z_1,z_2;z_3,z_3} =\frac{z_1 - z_3}{z_1 - z_4} \frac{z_2 - z_4}{z_2 - z_3}
    \end{equation*}
\end{definition}

\begin{proposition}
    Let \(z_1,z_2,z_3,z_4 \in \Complex\) and \(T\) is a linear fractional transformation. Then \(\bracket{Tz_1, Tz_2; Tz_3, Tz_4} = \bracket{z_1, z_2; z_3, z_3}\)
\end{proposition}
\begin{proof}
    Suppose \(S = \bracket{z, z_2; z_3, z_4}\) and \(S' = \bracket{z, Tz_2; Tz_3, Tz_4}\). Note that 
    \begin{equation*}
        \func{ST}{z_2} = 1 , \quad \func{ST}{z_3} = 0 , \quad \func{ST}{z_4} = \infty
    \end{equation*}
    therefore \(S = S'T\) and for all \(z_1 \in \Complex\), \(Sz = S'Tz\).
\end{proof}

\begin{proposition}
    Let \(r,c \in \Reals\) and \(k \in \Complex\), then the equation 
    \begin{equation*}
        r\abs{z}^2 + k\bar{z} + \bar{k}z + c = 0
    \end{equation*}
    represents a line if \(r = 0\), represents a circle if \(r \neq 0\) and \(\abs{k}^2 \geq rc\).
\end{proposition}

The locus of all the points of \(r\abs{z}^2 + k\bar{z} + \bar{k}z + c = 0\), if non-empty, is called a circline.
\begin{proposition}
    circlines correspond to circles on Riemann sphere.
\end{proposition}

\begin{proposition}
    A linear transformation carries circlines to circlines.
\end{proposition}

\begin{proposition}
    The cross ratio is real if and only if the four points lie in a circle or a straight line.
\end{proposition}



\section{Polynomials and rational functions}

\begin{theorem}
      If all zero's of a polynomial \(P\) lie in a half plane (convex polygon), the all zero's of the derivative \(P'\) lie in the same half plane (convex polygon).
\end{theorem}

A \textbf{rational function} \(\func{R}{z}\) is the quotient of two polynomials
\begin{equation*}
      \func{R}{z} = \dfrac{\func{P}{z}}{\func{Q}{z}} = \dfrac{a_m x^m + \dots + a_0}{b_n x^n + \dots + b_0}
\end{equation*}
We can assume that \(P\) and \(Q\) have no factor in common. In that case, the zeros of \(P\) are the zeros of \(R\) and the zeros of \(Q\) are called the \textit{poles} of \(R\). We can further define \(\func{R}{\infty}\) to be \(\func{S_R}{0}\) where \(\func{S_R}{z} = \func{R}{\frac{1}{z}}\). 
\begin{equation*}
      \func{S_R}{z} = z^{n-m} \dfrac{a_m + \dots + a_0 x^m}{b_n + \dots + b_0 x^n}
\end{equation*}
Then if \(n > m\) we say that \(R\) has roots at \(\infty\) with multiplicity of \(n - m\), if \(n < m\) then \(R\) has poles at \(\infty\) with multiplicity of \(m  - n\). If \(n = m\) then \(\func{R}{\infty} = \frac{a_m}{b_n}\). It is now easy to see that the number of roots and poles of \(R\) - including \(\infty\) - is the greater of \(m,n\) and it is called the \textit{degree} of \(R\). 

We can write \(R\) in the following form 
\begin{equation*}
      \func{R}{z} = \func{G}{z} + \func{H}{z}
\end{equation*}
where \(G\) is a polynomial with no constant term, by carrying out the division. Now let \(\beta_1 , \dots , \beta_k\) be distinct poles of \(R\). Then \(\func{S_j}{\xi} = \func{R}{\beta_j + \frac{1}{\xi}}\) is a rational function of \(\xi\) and it can be written as 
\begin{equation*}
      \func{S_j}{\xi} = \func{G_j}{\xi} + \func{H_j}{\xi} \implies \func{R}{z}= \func{G_j}{\dfrac{1}{z - \beta_j}} + \func{H_j}{\dfrac{1}{z - \beta_j}}
\end{equation*}
note that \(\func{G_j}{\frac{1}{z - \beta_j}}\) has pole(s) only at \(\beta_j\). Now consider 
\begin{equation*}
      \func{S}{z} = \func{R}{z} - \func{G}{z} - \sum_{j = 1}^k \func{G_j}{\dfrac{1}{z - \beta_j}}
\end{equation*}
It can have poles only at \(\beta_1 , \dots , \beta_k\) and \(\infty\). \(R\) and \(G_j\) have poles at \(z = \beta_j\), however their difference if finite. \(R\) and \(G\) have poles at \(\infty\) but their difference is finite as well. Hence \(S\) has no poles and thus must be a constant. Incorporating the constant into \(G\) allows us to write \(R\) as 
\begin{equation*}
      \func{R}{z} = \func{G}{z} + \sum_{j = 1}^k \func{G_j}{\dfrac{1}{z - \beta_j}}
\end{equation*}
which is its \textit{partial fraction decomposition}.

\section{Conformal maps}
A \textbf{conformal map} is a transformation that conserves angle and direction. For example every linear \(\Reals^2\) transformation of the form 
\begin{equation*}
      \begin{bmatrix}
            a & -b\\
            b & a
      \end{bmatrix} = \sqrt{a^2 + b^2} \begin{bmatrix}
            \cos \theta & -\sin \theta\\
            \sin \theta & \cos \theta
      \end{bmatrix} 
\end{equation*}
is a conformal mapping since it is a scaling and a rotation, both of which conserve the angle and direction. 
Let \(U\) be an open set in \(\Complex\) and \(\gamma,\eta : \clcl{a}{b} \to U\)  be two complex valued curves. Suppose that \(z_1 = \func{\gamma}{t_1} = \func{\eta}{t_2}\) then the angle between \(\gamma\) and \(\eta\) at \(z_1\) is defined as the angle between \(\func{\gamma'}{t_1}\) and \(\func{\eta'}{t_2}\), provided that they are nonvanishing. Let \(f: U \to \Complex\) be a holomorphic function then by chain rule 
\begin{equation*}
      \ODiff{\func{f}{\func{\gamma}{t}}}{t} = \func{f'}{\func{\gamma}{t}} \func{\gamma'}{t}
\end{equation*}
Viewing \(w = \func{f'}{\func{\gamma}{t}}\) as a \(\Reals^2\) linear transformation, we can deduce that \(f\) is conformal. 
\begin{align*}
      \ODiff{\func{f}{\func{\gamma}{t_1}}}{t} &= \func{f'}{z_1} \func{\gamma'}{t_1}\\
      \ODiff{\func{f}{\func{\eta}{t_2}}}{t} &= \func{f'}{z_1} \func{\eta'}{t_2}
\end{align*}
Thus given that \(\func{f'}{z_1} \neq 0\) the angle between \(\gamma\) and \(\eta\) at \(z_1\) is the same as \(f \circ \gamma\) and \(f \circ \eta\).


\chapter{Formal Power Series}
\section{Algbraic definitions}
\subsection*{Monoid}
A set \(S\) equipped with a binary operation \(\cdot\) is a \textbf{monoid} if it satisfies the following properties:
\begin{itemize}
    \item For all \(a,b,c \in S\), \(a\cdot (b\cdot c) = (a\cdot b) \cdot c\).
    \item There exists \(e \in S\) such that for each \(a \in S\), \(a \cdot e = e \cdot a = a\).
\end{itemize}
\subsection*{Group}
A set \(S\) equipped with a binary operation \(\cdots\) is a \textbf{group} if it satisfies the following properties:
\begin{itemize}
    \item For all \(a,b,c \in S\), \(a\cdot (b\cdot c) = (a\cdot b) \cdot c\).
    \item There exists \(e \in S\) such that for each \(a \in S\), \(a \cdot e = e \cdot a = a\).
    \item For each element \(a \in S\), there exists \(b \in S\) such that \(a \cdot b = b \cdot a = e\).
\end{itemize}
If \(\cdot\) is commutative then the group is an \textit{Abelian group}.
\subsection*{Ring}
A set \(S\) equipped with binary operations \(+\) and \(\cdots\) is a \textbf{ring} if it satisfies the folloing properties:
\begin{enumerate}
    \item \(S\) is an Abelian group under \(+\).
    \item \(S\) is a monoid under \(\cdot\).
    \item \(\cdot\) distributes over \(+\).
\end{enumerate}
If \(\cdot\) is commutative then the ring is called a \textit{commutative ring}.
\subsection*{Integral domain}
It is commutative ring with property that 
\begin{equation*}
    ab =  0 \implies a= 0  \lor b = 0
\end{equation*}

\subsection*{Field}
A commutative ring that each non-zero \(a \in S\) has a multiplicative inverse.

\begin{example}
    A field is an integral domain because if \(ab = 0\) and \(a\) is non-zero then 
    \begin{equation*} 
    a^{-1}ab = b = 0
    \end{equation*}
\end{example}
\section{Introduction}
Let \(K\) be a commutative ring. Then the set of formal power series over \(K\) is 
\begin{equation*}
    \FieldFormalSeries{X} = \set[a_i \in K]{a_0 + a_1x + a_2x^2 + \dots}
\end{equation*}
The addition and multiplication can be generalized from \(\FieldPolynomial{X}\), the set of polynomials in \(K\).
\begin{definition}[Order]
    Let \(S \in \FieldFormalSeries{X}\) then \(\func{\omega}{S}\) is the least \(n \geq 0\) such that \(a_n \neq 0\). Furthermore, \(\func{\omega}{0} = \infty\). 
\end{definition}

\begin{definition}[Summability]
    Let \(I\) be a set of indices then the family \(\set{S_i}_{i \in I}\) is \textbf{summable} if for each \(k \geq 0\), \(\func{w}{S_i} \geq k\) for all \(S_i\) except only a finite number of them. Then we can define \(S\) to be the sum of this family 
    \begin{equation*}
        S = \sum_{i \in I} S_i = \sum_{i \in I} \sum_{j} a^{(i)}_j x^j
    \end{equation*}
    which is sensible since only a finite number of \(a^{(i)}_j\) are non-zero.
\end{definition}


\begin{proposition} \label{th:orderOfMultiplication}
    Let \(K\) be an integral domain, then for any two formal series 
    \begin{equation*}
        \func{\omega}{ST} = \func{\omega}{S} + \func{\omega}{T} 
    \end{equation*}
    Asume that \(\infty + k = \infty\) for any finite number \(k\) and \(\infty + \infty = \infty\).
\end{proposition}

\begin{proof}
    If either of \(S\) or \(T\) is zero then the proposition is true. Otherwise, both \(n = \func{\omega}{S}\) and \(m = \func{\omega}{T}\) are finite. Equivalently \(a_n \neq 0 \) and \(b_m \neq 0\) while \(a_k = 0, b_l = 0\) for \(k < n, l < m\). Therefore, 
    \begin{equation*}
        ST = \sum_i \sum_{j = 0}^i a_{j}b_{i - j} x^i = \sum_i c_i x^i
    \end{equation*}
    for \(i < m + n\), \(c_i = 0\) and \(c_{m+n} = a_nb_m\) which is non-zero as \(K\) is an integral domain which was what was wanted.
\end{proof}

\begin{corollary}
    For an integral domain \(K\), the ring \(\FieldFormalSeries{X}\) is an integral domain.
\end{corollary}
For simplicity from now on assume \(K\) is either an integral domain or a field.

Consider two formal series 
\begin{equation*}
    S = \sum_n a_nx^n , \qquad T = \sum_m b_m x^m
\end{equation*}
with \(b_0 = 0\). Then we can define the composition \(S \circ T\) as follow 
\begin{equation*}
    S \circ T = \sum_n a_n T^n
\end{equation*}

To show that it is well defined we need to show that only a finite number of \(T^n\) have degree less than \(k\) for all \(k \geq 0\). Which is clear implied by the fact that \(\func{\omega}{T} \geq 1\) and by \Cref{th:orderOfMultiplication}, \(\func{\omega}{T^n} \geq n\). It can easily be shown that 
\begin{equation*}
    \begin{cases}
        (S_1 + S_2) \circ T &= S_1 \circ T + S_2 \circ T\\
        (S_1 \cdot S_2) \circ T &= (S_1 \circ T) (S_2 \circ T)
    \end{cases}
\end{equation*}
Moreover, if \(\set{S_i}\) is a summable family of formal series then 
\begin{equation*}
    (\sum_i S_i)\circ T = \sum_i (S_i \circ T)
\end{equation*}

\begin{proposition}
    For three formal series \(S, T,\) and \(U\) with \(\func{\omega}{T},\func{\omega}{U} \geq 1\)
    \begin{equation*}
        (S \circ T) \circ U = S \circ (T \circ U)
    \end{equation*}
\end{proposition}

\begin{proposition}
    A formal series \(S = \sum_i a_i x^i\) has a multiplicative if and only if \(a_0 \neq 0\).
\end{proposition}

\begin{proposition}
    We have that if \(\func{\omega}{T} \geq 1 \)
    \begin{equation*}
        \func{\omega}{S \circ T} = \func{\omega}{S} \func{\omega}{T}
    \end{equation*}
\end{proposition}

For two formal series \(f,g\) we write \(f \equiv g \pmod{T^N}\) if \(a_n = b_n\) for all \(n = 0,1,\dots , N- 1\). Cleary \(f \equiv g \pmod{T^N}\) for all \(N\) is equivalent to \(f = g\).
\begin{proposition}
    Suppose \(f_1 \equiv f_2, g_1 \equiv g_2, h_1 \equiv h_2 \pmod{T^N}\). Then 
    \begin{equation*}
        f_1 + g_1 \equiv f_2 + g_2 \pmod{T^N}, \quad f_1g_1 \equiv f_2g_2 \pmod{T^N}, \quad f_1 \circ h_1 \equiv f_2 \circ h_2 \pmod{T^N}
     \end{equation*} 
\end{proposition}

\section{Formal Derivative}
The derivative of a formal series is defined as 
\begin{equation*}
    \ODiff{S}{x} = a_1 + 2a_2 x + 3a_3 x + \dots 
\end{equation*}
\begin{proposition}
    The derivative has the following properties
    \begin{enumerate}
        \item \((S_1 + S_2)' = S_1' + S_2'\).
        \item The Leibniz rule holds 
        \begin{equation*}
            (S_1 \cdot S_2)' = S_1' S_2 + S_1 S_2'
        \end{equation*}
        \item if \(S\) has multiplicative inverse then 
        \begin{equation*}
            \left(\dfrac{1}{S}\right)' = - \dfrac{S'}{S^2}
        \end{equation*}
        \item The chain rule holds, if \(\func{\omega}{T} \geq 1\) then 
        \begin{equation*}
            (S \circ T)' = (T' \circ S) \circ S'
        \end{equation*}
        \item The higher order derivatives can be defined recursively 
        \begin{equation*}
            S^{(n)} = (S^{(n-1)})', \qquad S^{(0)} = S
        \end{equation*}
        Then, we can write the formal Taylor series of \(S\) as 
        \begin{equation*}
            S = \sum_n \dfrac{\func{S^{(n)}}{0}}{n!} x^n
        \end{equation*}
    \end{enumerate}
\end{proposition}

\section{Compositional inverse}
Given a formal series \(S\) we wish to find a formal series \(T\) such that 
\begin{enumerate}
    \item \(\func{\omega}{T} \geq 1\).
    \item \( S \circ T = X\).
\end{enumerate}

\begin{theorem}
    \(S\) has compositional inverse \(T\), if and only if \(\func{\omega}{S} = 1\). In this case, \(T\) is unique and \(T\circ S = X\) as well.
\end{theorem}


\section{Power series}
A power series is a complex function of the form 
\begin{equation*}
      \func{f}{z} = \sum_{n = 0}^{\infty} a_n\bracket{z - z_0}^n
\end{equation*}

\begin{theorem} 
      For every power series there exists a number \(0 \leq R \leq \infty\), called the radius of convergence, with the following properties 
      \begin{itemize}
            \item The series converges absolutely for every \(z\) with \(\abs{z} < R\) and it is an analytic function. The derivative can be obtained by termwise differentiation and it has the same radius of convergence. The convergence is uniform in \(\abs{z} \leq \rho\) for \(0 \geq \rho \geq R\).
            \item If \(\abs{z} > R\) the series diverges.
      \end{itemize}
\end{theorem}

\begin{proposition}
    Let \(f\) and \(g\) be two power series that converge absolutely in \(\func{B_r}{0}\). Then \(f + g\), \(gf\), and \(\alpha f\) , \(\alpha \in \Complex\) are absolutely convergent in \(\func{B_r}{0}\).
\end{proposition}

\begin{example}
    Let \(\alpha \in \Complex\) 
    \begin{equation*}
        \binom{\alpha}{n} = \dfrac{\alpha \bracket{\alpha - 1} \dots \bracket{\alpha -n + 1}}{n!}, \quad \binom{\alpha}{0} = 1
    \end{equation*}
    then the binomial formal series is defined as 
    \begin{equation*}
        \bracket{1 + T}^{\alpha} = \sum_{n  = 0}^{\infty} \binom{\alpha}{n} T^n
    \end{equation*}
    For non-negative integer \(\alpha\) the radius of convergence is infinite but for other values of \(\alpha\) the radius is 1.
\end{example}

\begin{theorem}[Abel's limit theorem]
    If \(\sum_{n = 0}^{\infty}\) converges, then \(\func{f}{z} = \sum_{n = 0}^{\infty} a_n z^n\) tends to \(\func{f}{1}\) as \(z\) approaches 1 in such a way that \(\frac{\abs{1- z}}{1 - \abs{z}}\) remains bounded.
\end{theorem}

\begin{theorem}
    \begin{enumerate}
        \item  Let \(f\) be a non-constant power series with non-zero radius of convergence. If \(\func{f}{0} = 0\) then there exists \(s > 0\) such that \(\func{f}{z} \neq 0\) for all non-zero \(z\) with \(\abs{z} \leq s\).
        \item Let \(f\) and \(g\) to be two convergent power series. Suppse \(\func{f}{z} = \func{f}{z}\) for all points \(x\) in an infinite set having 0 as a point of accumulation. Then \(f= g\).
    \end{enumerate}
\end{theorem}

Let \(f = \sum a_nz^n\) and \(\phi = \sum c_n z^n\) with \(c_n \in \Reals_+\) be two formal power series. Then \(f\) is dominated by \(\phi\)
\begin{equation*}
    f \prec \phi 
\end{equation*}
if \(\abs{a_n} \leq c_n\) for all \(n\).

\begin{proposition}
    Suppose \(f \prec \phi\) and \(g \prec \psi\) then 
    \begin{equation*}
        f + g \prec \phi + \psi , \quad fg \prec \phi \psi
    \end{equation*}
\end{proposition}

\begin{theorem}
    Let \(f\) be a power series with non-zero constant term and non-zero radius of convergence and let \(g\) be the inverse of \(f\) that is, \(fg = 1\). Then \(g\) has a non-zero radius of convergence.
\end{theorem}

\begin{proposition}
    Let 
    \begin{equation*}
        \func{f}{z} = \sum_{n = 0}^{\infty} a_n z^n , \quad \func{h}{z} = \sum_{n = 1}^{\infty} b_n z^n
    \end{equation*}
    be two power series and \(f\) is absolutely convergent for \(\abs{z} \leq r\) for some \(r > 0\) and let \(s > 0\) be such that 
    \begin{equation*}
        \sum_{n=1}^{\infty} \abs{b_n}s^n \leq r
    \end{equation*}
    Then \(f\circ h\) converges absolutely for \(\abs{z} \leq s\).
\end{proposition}

\section{Analytic functions}
Let \(f\) be defined in a neighbourhood of \(z_0\). \(f\) is said to be analytic if there exists some power series 
\begin{equation*}
    \sum a_n \bracket{z - z_0}^n
\end{equation*}
which converges absolutely for \(\abs{z - z_0} < r\) and for such \(z\)
\begin{equation*}
    \func{f}{z} = \sum a_n \bracket{z - z_0}^n
\end{equation*}

\begin{proposition}
    If \(f\) and \(g\) are analytic on \(U\) then \(f+ g\), \(fg\), \(f/g\) for \(\func{g}{z} \neq 0\) are also analytic. If \(g : U \to V\) and \(f : V \to \Complex\) then \(f \circ g\) is analytic.
\end{proposition}

\begin{theorem}
    Let \(\func{f}{z} = \sum a_n z^n\) with radius of convergence of \(r\). Then \(f\) is analytic on \(\func{B_r}{0}\).
\end{theorem}

\begin{theorem}
    Let \(\func{f}{z} = \sum a_n z^n\) with radius of convergence of \(r\) then 
    \begin{enumerate}
        \item The series \(\sum na_n z^{n-1}\) has the same radius of convergence.
        \item \(f\) is holomorphic on \(\func{B_r}{0}\) and its derivative is equal to \(\sum na_n z^{n-1}\).
    \end{enumerate}
\end{theorem}
We shall see later that every holomorphic function admits a power series. 

\section{Inverse and open mapping}
\(f\) is analytic isomorphism if \(f : U \to V\) is analytic and \(V\) is open and there exists analytic function \(g : V \to U\) such that \(f \circ g = \id_V\) and \(g \circ f = \id_U\). \(f\) is locally analytic isomorphism at \(z_0\) or locally invertible if there exists an open neighbourhood \(U\) containing \(z_0\) such that \(f\) is an analytic isomorphism on \(U\). 

\begin{theorem}
    Let \(\func{f}{z} = a_1z + \dots \) be a formal power series with \(a_1 \neq 0\). We know there exists a unique \(\func{g}{z}\) such that \(\func{f}{\func{g}{z}} = z\) and \(\func{g}{\func{f}{z}} = z\). If \(f\) is convergent a power series, then \(g\) is a convergent power series as well. 
\end{theorem}

\begin{theorem}
    Let \(f\) be an analytic function on open set \(U\) containing \(z_0\). Suppose that \(\func{f'}{z_0} \neq 0\) then, \(f\) is a local analytic isomorphism at \(z_0\).
\end{theorem}

\begin{definition}
    Let \(U\) be an open set then \(f : U \to V\) is open mapping if for each open subset \(U' \subset U\), \(\func{f}{U'}\) is open.
\end{definition}

\begin{theorem}
    Let \(f\) be an analytic function on open set \(U\) such that for each point of \(U\), \(f\) is not constant on a given neighbourhood of that point. Then \(f\) is an open mapping.
\end{theorem}

\begin{theorem}
    Let \(f\) be an analytic function on open set \(U\) and \(f\) is injective. Let \(V = \func{f}{U}\) then, \(f : U \to V\) is an analytic isomorphism and \(\func{f'}{z} \neq 0\) for all \(z \in U\).
\end{theorem}

\section{The local maximum modulus principle}
\begin{definition}
    \(f\) is locally constant at a point \(z_0\) if there exists an open set \(D\) containing \(z_0\) such that \(f\) is constant on \(D\).
\end{definition}

\begin{theorem}
    Let \(f\) be an analytic function on open set \(U\). Let \(z_0 \in U\) be a maximum for \(\abs{f}\), that is 
    \begin{equation*}
        \abs{\func{f}{z_0}} \geq \abs{\func{f}{z}} \quad \forall z \in U
    \end{equation*}
    Then \(f\) is locally constant at \(z_0\).
\end{theorem}

\begin{corollary}
    Above theorem also holds for \(\Re\) instead of \(\abs{\; \cdot \;}\).
\end{corollary}

\begin{theorem}
    Let \(f\) be a non-constant polynomial 
    \begin{equation*}
        \func{f}{z} = a_0 + a_1 z + \dots + a_d z^d
    \end{equation*}
    with \(a_d \neq 0\). Then \(f\) has some complex roots. 
\end{theorem}
\chapter{Complex Integrals}
\section{Integrals over paths}
Path is a sequence of regular curves 
\begin{equation*}
    \gamma = \set{\gamma_1 , \dots , \gamma_n}
\end{equation*}
such that 
\begin{equation*}
    \func{\gamma_j}{b_j} = \func{\gamma_{j+1}}{a_{j+1}}
\end{equation*}

\begin{theorem}
    Suppose \(U\) is a connected open set and \(f\) is holomorphic on \(U\). If \(f' = 0\) then \(f\) is constant on \(U\).
\end{theorem}

\begin{definition}
    If \(f\) is a function on an open set \(\Omega\) and \(g\) is a holomorphic function on \(\Omega\), such that \(g' = f\), then \(g\) is called a \textit{primitive} if \(f\) on \(\Omega\).
\end{definition}

\begin{corollary}
    On an connected open set \(U\), the primitives of are determined up to a constant.
\end{corollary}

\begin{theorem}
    Suppose \(U\) is a connected open set then 
    \begin{enumerate}
        \item If \(f\) is analytic on \(U\) and not constant, the set of zeroes of \(f\) on \(U\) is discrete.
        \item \(f,g\) are analytic on \(U\) and \(S = \set<z>{\func{f}{z} = \func{g}{z}}\) is not discrete then \(f = g\) on \(U\).
    \end{enumerate}
\end{theorem}

Let \(f:\clcl{a}{b} \to \Complex\) be a continuous function with 
\begin{equation*}
    \func{f}{t} = \func{u}{t} + i \func{v}{t}
\end{equation*}
then 
\begin{equation*}
    \int \func{f}{t} \diffOperator t = \int \func{u}{t} \diffOperator t + i \int \func{v}{t} \diffOperator t
\end{equation*}
and by the Fundamental theorem of calculus 
\begin{equation*}
    \func{G}{t} = \int_a^t \func{F}{\tau} \diffOperator \tau
\end{equation*}
is differentiable with \(\func{G'}{t} = \func{F}{t}\).
We can then expand the notion of complex integrability to continuous complex function \(f: U \to \Complex\), where \(U\) is an open subset of \(\Complex\) using regular curves such as \(\gamma:\clcl{a}{b} \to U\). The integral of \(f\) along \(\gamma\) is 
\begin{equation*}
    \int_{\gamma} \func{f}{t} = \int_a^b \func{f}{\func{\gamma}{t}} \func{\gamma'}{t} \diffOperator t
\end{equation*}
This definition of integral is invariant to substitution. To see this, let \(g:\clcl{a}{b} \to \clcl{c}{d}\) be a \(\calC^1\) function and \(\gamma = \psi \circ g\)
\begin{align*}
    \int_{\gamma} f &= \int_a^b \func{f}{\func{\gamma}{t}} \func{\gamma'}{t} \diffOperator t\\
    &= \int_a^b \func{f}{\func{\psi}{\func{g}{t}}} \func{\psi'}{\func{g}{t}} \func{g' }{t} \diffOperator t\\
    &= \int_c^d \func{f}{\func{\psi}{\tau}} \func{\psi'}{\tau} \diffOperator \tau \\
    &= \int_{\psi} f
\end{align*}
We can further expand this notion to paths \(\gamma = \set{\gamma_1, \dots , \gamma_n}\) as follow 
\begin{equation*}
    \int_{\gamma} f = \sum_{i = 1}^n \int_{\gamma_i} f
\end{equation*}

\begin{theorem}
    Let \(f\) be a continuous function on open set \(U\) with primitive \(g\). Let \(\gamma:\clcl{a}{b} \to U\) be a path with \(\alpha= \func{\gamma}{a}\) and \(\beta = \func{\gamma}{b}\) then 
    \begin{equation*}
        \int_{\gamma} f = \func{g}{\beta} - \func{g}{\alpha}
    \end{equation*}
\end{theorem}

\begin{corollary}
    If \(\gamma\) is a closed path in \(U\) then 
    \begin{equation*}
        \int_{\gamma} f = 0
    \end{equation*}
\end{corollary}

\begin{theorem}
    Let \(U\) be a connected open set and \(f\) a continuous function \(U\). If for any closed path \(\gamma\) on \(U\), the integral of \(f\) along \(\gamma\) is zero , \(\int_{\gamma} f = 0 \), then \(f\) has a primitive \(g\) on \(U\).
\end{theorem}

\begin{proof}
    Pick \(z_0 \in U\) and define 
    \begin{equation*}
        \func{g}{z} = \int_{z_0}^z  f
    \end{equation*}
    where the integral is taken over any path from \(z_0\) to \(z\). Let \(\gamma, \eta\) be two curves from \(z_0\) to \(z\) and let \(\eta^-\) be the reverse of \(\eta\). Since 
    \begin{equation*}
        \int_{\gamma} f + \int_{\eta^{-}} f =0 \implies \int_{\gamma} f = \int_{\eta} f 
    \end{equation*}
    and hence \(g\) is well-defined. Consider the difference quotient 
    \begin{equation*}
        \dfrac{\func{g}{z + h} - \func{g}{z}}{h} = \dfrac{1}{h} \int_z^{z+h} f
    \end{equation*}
    where the integral can be taken along any segments. Furthermore, we can write 
    \begin{equation*}
        \func{f}{\xi} = \func{f}{z} + \func{\phi}{\xi}
    \end{equation*}
    where \(\lim_{\xi \to z} \func{\phi}{\xi} = 0\). Then, 
    \begin{align*}
        \dfrac{1}{h} \int_z^{z+h} \func{f}{\xi} \diffOperator \xi &= \dfrac{1}{h} \int_z^{z+h} \func{f}{z} \diffOperator \xi + \dfrac{1}{h} \int_z^{z+h} \func{\phi}{\xi} \diffOperator \xi \\
        &= \func{f}{z} \dfrac{1}{h} \int_z^{z+h} \func{\phi}{\xi} \diffOperator \xi \\
        &\leq \func{f}{z} + \max_{\xi} \abs{\func{\phi}{x}} \xrightarrow{h \to 0} \func{f}{z}\\
        \implies g' =f 
    \end{align*}
\end{proof}

Let \(\gamma:\clcl{a}{b} \to \Complex\) be a regular curve. The \textit{speed} is of \(\gamma\) is defined as \(\abs{\gamma'}^2\) and the length of \(\gamma\) is 
\begin{equation*}
    \func{l}{\gamma} = \int_{a}^b \abs{\func{\gamma'}{t}} \diffOperator t
\end{equation*}
and for a path \(\gamma = \set{\gamma_1, \dots , \gamma_n}\) 
\begin{equation*}
    \func{l}{\gamma} = \sum_{i = 1}^n \func{l}{\gamma_i}
\end{equation*}
and sup norm of \(f\) on a set \(S\) 
\begin{equation*}
    \norm{f} = \sup_{z \in S} \abs{f}
\end{equation*}
and over a curve \(\gamma\) the sup-norm of \(f\) is 
\begin{equation*}
    \norm{f}_{\gamma} = \sup_{t \in \clcl{a}{b}} \abs{\func{f}{\func{\gamma}{t}}}
\end{equation*}

\begin{proposition}{ML inequality}
    Let \(f\) be a continuous function on \(U\) and \(\gamma\) be a path in \(U\) them 
    \begin{equation*}
        \abs{\int_{\gamma} f} \leq \norm{f}_{\gamma} \func{l}{\gamma}
    \end{equation*}
\end{proposition}

\begin{theorem}
    Let \(\set{f_n}\) be a sequence of continuous functions on \(U\) with \(f_n \rightrightarrows f\) then 
    \begin{equation*}
        \lim_{n \to \infty} \int_{\gamma} f_n = \int_{\gamma} f
    \end{equation*}
    and if \(\sum f_n\) is a series of continuous function which converges uniformly on \(U\)
    \begin{equation*}
        \int_{\gamma} \sum f_n = \sum \int_{\gamma} f_n
    \end{equation*}
\end{theorem}

\section{Local primitive for a holomorphic function}
\begin{theorem}[Goursat]
    Let \(R\) be a rectangle, and let \(f\) be a holomorphic function on \(R\) then 
    \begin{equation*}
        \int_{\partial R} f = 0
    \end{equation*}
\end{theorem}

\begin{theorem}
    Let \(U\) be a disc centered at a point \(z_0\) and \(f\) a continuous function on \(U\). Assume that for any rectangle \(R \subset U\)
    \begin{equation*}
        \int_{\partial R} f = 0
    \end{equation*}
    For each point \(z\) in disk define 
    \begin{equation*}
        \func{g}{z} = \int_{z_0}^z f
    \end{equation*}
    where the integral is taken over a rectangle \(R\) whose opposite vertices are \(z_0\) and \(z\). Then, \(g\) is holomorphic on \(U\) and is a primitive for \(f\).
\end{theorem}

\begin{theorem}
    Let \(U\) be a disc and \(f\) be a holomorphic function on \(U\). Then, \(f\) has a primitive on \(U\) and the integral of \(f\) along any closed path in \(U\) is zero.
\end{theorem}

\section{Path integrals for continuous curves}
Knowing that local primitive exists for holomorphic functions allows us to describe their integral along a path in a way which makes no use of differentiability of the path and applys to continuous paths.z

\begin{lemma}
    Let \(\gamma: \clcl{a}{b} \to U\) be a continuous curve in an open set \(U\). Then, there exists some positive number \(r > 0\) such that every point on the curve lies at distance \(\geq r\) from the complement of \(U\).
\end{lemma}

\begin{proof}
    Consider 
    \begin{equation*}
        \func{\phi}{t} = \min_{w \in U^c} \abs{\func{\gamma}{t} - w}
    \end{equation*}
    \(\abs{\func{\gamma}{t} - w}\) is continuous in terms of \(w\) by considering a sufficiently large disc that contains \(U\) we can bound the domain of \(w\) to the intersection of this disc and \(U^c\). Thus, due to compactness of the intersection, \(\abs{\func{\gamma}{t} - w}\) achieves its minimum. Moreover, since \(\func{\phi}{t}\) is continuous and its domain is compact then, it has a minimum and the minimum is non-zero since \(U\) is open.
\end{proof}

\begin{definition}
    Let \(P = \set{a_0, \dots , a_n}\) be a partition of \(\clcl{a}{b}\) and \(\set{D_0, \dots , D_n}\) be a sequence of discs. This sequence is \textit{connected} by curve along the partition \(P\) if \(D_i\) contains \(\func{\gamma}{\clcl{a_i}{a_{i+1}}}\).
\end{definition}

Let \(\epsilon < \frac{r}{2}\) where \(r\) is the same as preceding lemma. Since \(\gamma\) is uniformly continuous, there exists a \(\delta\) such that 
\begin{equation*}
    \forall t,s \in \clcl{0}{1}, \ \abs{t-s} < \delta \implies  \abs{\func{\gamma}{t} - \func{\gamma}{s}} < \epsilon
\end{equation*}
Consider a partition \(P\) on \(\clcl{a}{b}\) such that \(\norm{P} < \delta\). Then, the \(\func{\gamma}{\clcl{a_i}{ a_{i+1}}}\) lies in the disc \(D_i = \func{B_{\epsilon}}{a_i}\). Let \(f\) be a holomorphic function on \(U\). 
\begin{equation*}
    \int_{\gamma} f = \sum_{i = 0}^{n-1 } \int_{\gamma_i} f  \qquad \qquad \gamma_i = \func{\gamma}{\clcl{a_i}{a_{i+1}}}
\end{equation*}
Let \(z_i = \func{\gamma}{a_i}\) and \(g_i\) be a primitive of \(f\) on disc \(D_i\). If each \(\gamma_i \in \calC^1\) then we know that 
\begin{equation*}
    \int_{\gamma} f = \sum_{i = 0}^{n-1} \func{g_i}{z_{i+1}} - \func{g_i}{z_{i}}
\end{equation*} 
By considering the following lemma, we can relax the regularity condition for \(\gamma\) and reduce it to continuity. 
\begin{lemma}
    Let \(U\) be an open set and \(\gamma:\clcl{a}{b} \to U\) be a continuous curver. Let 
    \begin{equation*}
        a = a_0 \leq a_1 \leq \dots \leq a_n = b
    \end{equation*}
    be a partition on \(\clcl{a}{b}\) such that \(\func{\gamma}{\clcl{a_i}{ a_{i+1}}}\) is contained in a disc \(D_i\) which itself is contained in \(U\). Let \(f\) be a holomorphic function on \(U\), \(g_i\) be a primitive of \(f\) on \(D_i\), and \(z_i = \func{\gamma}{a_i}\), then 
    \begin{equation*}\label{eq:continuousPathIntegral}
        \sum_{i = 0}^{n-1} \func{g_i}{z_{i+1}} - \func{g_i}{z_{i}}
    \end{equation*}
    is independent of the choices of partition, discs \(D_i\), and primitives \(g_i\) on \(D_i\) subjected to stated condition. Therefore, \Cref{eq:continuousPathIntegral} depends only on \(\gamma\) and hence the integral on continuous path is well-defined. 

\end{lemma}

\begin{proof}
    
\end{proof}

\begin{definition}
    Let \(\gamma, \eta\) be two paths defined on \(\clcl{a}{b}\). We say that they are \textit{close} together if there exists a partition 
\begin{equation*}
    a = a_0 \leq a_1 \leq \dots \leq a_n = b
\end{equation*}
and for each \(i = 0, \dots , n- 1\) there exists a disc \(D_i\) contained in \(U\) such that 
\begin{equation*}
    \func{\gamma}{\clcl{a_i}{a_{i+1}}}, \func{\eta}{\clcl{a_i}{a_{i+1}}} \subset D_i
\end{equation*}
\end{definition}

\begin{lemma}
    Let \(\gamma, \eta\) be continuous paths on open set \(U\), that are close together and have the same endpoints. Let \(f\) be holomorphic on \(U\) 
    \begin{equation*}
        \int_{\gamma} f = \int_{\eta} f
    \end{equation*}
\end{lemma}

\begin{proof}
    
\end{proof}

\section{Homotopy}
Let \(\gamma, \eta: \clcl{a}{b} \to U\)  be two paths. \(\gamma\) is \textit{homotopic} to \(\eta\), if there eists a continuous function 
\begin{equation*}
    \Psi : \clcl{a}{b} \times \clcl{c}{d} \to U
\end{equation*}
such that 
\begin{equation*}
    \func{\Psi}{t,c} = \func{\gamma}{t} \qquad \func{\Psi}{t,c} = \func{\eta}{t}
\end{equation*}
for all \(t \in \clcl{a}{b}\). Intuitively, \(\Psi\) is a continuous deformation of \(\gamma\) to \(\eta\). \(\Psi\) \textit{leave the end point fixed} if we have 
\begin{equation*}
    \func{\Psi}{a,s} = \func{\gamma}{a} \qquad \func{\Psi}{b,s} = \func{\gamma}{b}
\end{equation*}
for all \(s \in \clcl{c}{d}\). Similarly, we assume that a homotopy of two closed paths is such that each path \(\func{\Psi}{\; \cdot \;, s}\) is a closed path. 

\begin{theorem}
    Let \(\gamma, \eta\) be two homotopic continuous paths on open set \(U\) with the same endpoints and \(f\) be a holomorphic function on \(U\). 
    \begin{equation*}
        \int_{\gamma} f = \int_{\eta} f
    \end{equation*} 
    In particular, if \(\gamma, \eta\) are closed paths and homotopic to a point in \(U\) then 
    \begin{equation*}
        \int_{\gamma} f = \int_{\eta} f = 0
    \end{equation*}
\end{theorem}

\begin{proof}
    Let \(\Psi:\clcl{a}{b} \times \clcl{c}{d} \to U\) be the homotopy of the closed paths \(\gamma,\eta\). Since \(\Psi\) is a continuous function on a compact domain, the image of \(\Psi\) is compact and hence it has a positive distance \(r\) from \(U^c\). Similarly, by considering the uniform continuity we can have partitions \(P = \set{a_1, \dots , a_n}\) on \(\clcl{a}{b}\) and \(Q = \set{c_1, \dots , c_m}\) on \(\clcl{c}{d}\) and
    \begin{equation*}
        S_{ij} = \clcl{a_i }{ a_{i+1}} \times \clcl{c_j }{ c_{j + 1}}
    \end{equation*}
    such that \(\func{\Psi}{S_{ij}}\) is contained in \(D_{ij}\) which itself is contained in \(U\). Let \(\Psi_j\) 
    \begin{equation*}
        \func{\Psi_j}{t} = \func{\Psi}{t,c_j}
    \end{equation*}
    Then \(\Psi_j\) and \(\Psi_{j+1}\) are closed together and by applying the preceding lemma we get 
    \begin{equation*}
        \int_{\gamma} f = \int_{\Psi_0} f = \int_{\Psi_1} f = \dots = \int_{\Psi_m} f = \int_{\eta} f
    \end{equation*}
\end{proof}

A set \(S\) of complex numbers is conex, if for any \(z,w \in S\) the segment \(\clcl{z,w} \subset S\). For examples, discs and rectangles are convex. 
\begin{lemma}
    Let \(S\) be a convex set and \(\gamma, \eta\) continuous closed curves in \(S\). Then, \(\gamma, \eta\) are homotopic in \(S\).
\end{lemma}

\begin{proof}
    \begin{equation*}
        \func{\Psi}{t,s} = s \func{\gamma }{t} + \bracket{1-s} \func{\eta}{t}
    \end{equation*}
\end{proof}

Open set \(U\) is simply connected if it is connected and every closed path is homotopic to a point in \(U\). By the preceding lemma, every convec set is simply connected, as path connectednes implies connectedness.

\section{Existence of global primitives}
\begin{theorem}
    Let \(f\) be a holomorphic function on a simply connected open set \(U\) and let \(z_0 \in  U\). For an \(z \in U\)
    \begin{equation*}
        \func{g}{z} = \int_{z}^{z_0} \func{f}{\xi} \diffOperator \xi
    \end{equation*}
    is independent of the path in \(U\) and \(g\) is a primitive for \(f\).
\end{theorem}

\begin{example}
    Let \(U\) be a simply connected open set not containing \(0\). Pick \(z_0 \in U\) and \(w_0\) such that \(e^{w_0} = z_0\), define 
    \begin{equation*}
        \log z = w_0 +  \int_{z_0}^z \dfrac{1}{\xi} \diffOperator \xi
    \end{equation*}
    Then, \(\log\) is a primitive for \(\frac{1}{z}\). If \(\func{L}{z}\) is another primitive for \(\frac{1}{z}\) on \(U\) such that \(e^{\func{L}{z}}  = z\), then there exists an integer \(k\) such that 
    \begin{equation*}
        \func{L}{z} = \log z + 2\pi i k
    \end{equation*}
\end{example}

\section{Local Cauchy formula}
\begin{theorem}[Local Cauchy formula]
    Let \(\bar{D}\) be a closed disc of positive radius and \(f\) holomorphic on \(\bar{D}\) (open disc \(U\) containing \(\bar{D}\)). Let \(\gamma\) be the circle with is the boundary of \(\bar{D}\). Then, for every \(z_0 \in D\)
    \begin{equation*}
        \func{f}{z_0} = \dfrac{1}{2\pi i} \int_{\gamma} \frac{\func{f}{\xi}}{\xi - z_0} \diffOperator \xi
    \end{equation*}
 \end{theorem}

 \begin{theorem}
     Let \(f\) be holomorphic on open set \(U\) then \(f\) is analytic on \(U\).
 \end{theorem}

 \begin{theorem}
     Let \(f\) be holomorphic on closed disc \(\func{\bar{B}_R}{z_0}\), \(R > 0\). Let \(C_R\) be the circle bounding the disc. Then, \(f\) has a power series expansion 
     \begin{equation*}
         \func{f}{z} = \sum_{n = 0}^{\infty} a_n \bracket{z - z_0 }^n
     \end{equation*}
     whose coefficients are given by the formula 
     \begin{equation*}
         a_n = \dfrac{1}{n!} \func{f^{(n)}}{z_0} = \dfrac{1}{2 \pi i} \int_{C_R} \frac{\func{f}{\xi}}{\bracket{\xi - z_0}^{n+1}} \diffOperator \xi
     \end{equation*}
     Furthermore, if \(\norm{f}_{C_R}\) denotes the sup-norm of \(f\) on \(C_R\) then 
     \begin{equation*}
       \abs{a_n} = \dfrac{\norm{f}_{C_R}}{R^n}  
    \end{equation*}
    in particular, the radus of convergence of the series is greater than \(R\).
 \end{theorem}
 The preceding theorems imply that a function is analytic if and only if it is holomorphic.

 \begin{definition}
     A function \(f\) is entire if it is holomorphic on all \(\Complex\). Hence, entire functions have convergence radius of \(\infty\).
 \end{definition}

 \begin{corollary}
     Let \(f\) be entire and \(\norm{f}_R\) be its sup-norm on the circle of radius \(R\). Suppose that there exists a constant \(C\) and a positive integer \(k\) such that 
     \begin{equation*}
         \norm{f}_R = CR^k
     \end{equation*}
     for arbitrarily large \(R\). Then, \(f\) is a polynomial of degree \(\leq k\).
 \end{corollary}

 \begin{theorem}[Liouville's theorem]
     A bounded entire function os constant.
 \end{theorem}

 \begin{corollary}
     A polynomial over complex number which does not have root in \(\Complex\) is constant.
 \end{corollary}

 \begin{theorem}
     Let \(\gamma\) be a path in an oopen set \(U\) and \(g\) be a continuous function on \(\gamma\). If \(z\) is not on \(\gamma\) define 
     \begin{equation*}
         \func{f}{z} = \int_{\gamma} \dfrac{\func{g}{\xi}}{\xi - z} \diffOperator \xi
     \end{equation*}
     Then \(f\) is analytic on the complement of \(\gamma\) in \(U\) and its derivatives are given by 
     \begin{equation*}
         \func{f^{(n)}}{z} = n! \int_{\gamma} \frac{\func{g}{\xi}}{\bracket{\xi - z}^{n+1}} \diffOperator \xi
     \end{equation*}
 \end{theorem}

 \begin{corollary}
     Let \(f\) be an analytic function on closed disc \(\func{\bar{B}_R}{z_0}\) with \(R > 0\). Let \(0 < R_1 < R\) and denote the sup-norm of \(f\) on the circle of radius \(R\) by \(\norm{f}_R\). For \(z \in \func{\bar{B}_R}{z_0}\)
     \begin{equation*}
         \abs{\func{f^{(n)}}{z}} \leq \dfrac{n! R}{\bracket{R - R_1}^{n+1}} \norm{f}_R
     \end{equation*}
 \end{corollary}

 \begin{theorem}[Morera's theorem]
     Let \(U\) be an open set in \(\Complex\) and let be continuous on \(U\). Assume that the integral of \(f\) along the boundary of every rectangle in \(U\) is zero. Then, \(f\) is analytic.
 \end{theorem}
\chapter{Winding Numbers}
\section{Winding number}
\begin{definition}
    Winding number of a closed path \(\gamma\) with respect to a point \(\alpha \) is 
    \begin{equation*}
        \func{W}{\gamma, \alpha} = \dfrac{1}{2 \pi i } \int_{\gamma} \dfrac{1}{z - \alpha} \diffOperator z = \dfrac{1}{2 \pi i} \int_{a}^b \dfrac{\func{\gamma' }{t}}{\func{\gamma}{t} - \alpha} \diffOperator t
    \end{equation*}
    provided that \(\gamma\) does not pass through \(\alpha\).
\end{definition}

\begin{lemma}
    If \(\gamma\) is a closed path, then \(\func{W}{\gamma, \alpha}\) is an integer.
\end{lemma} 
\begin{proof}
    Consider a the integral and compare it to \(\func{\gamma}{t} - \alpha\).
\end{proof}

\begin{lemma}
    Let \(\gamma\) be a path. Then, the function of \(\alpha\) defined by 
    \begin{equation*}
        \alpha \mapsto \int_{\gamma} \dfrac{1}{z - \alpha} \diffOperator z
    \end{equation*}
    for \(\alpha\) no on the path, is a continuous function of \(\alpha\).
\end{lemma}

\begin{lemma}
    Let \(\gamma\) be a closed path and \(S\) a connected set not intersecting \(\gamma\). Then, 
    \begin{equation*}
        \alpha \mapsto \int_{\gamma} \dfrac{1}{z - \alpha} \diffOperator z
    \end{equation*}
    is a constant for \(\alpha\) in \(S\). If \(S\) is not bounded, then this constant is zero.
\end{lemma}

\begin{definition}
    A closed path \(\gamma\) in \(U\) is homologous to 0 in \(U\) if 
    \begin{equation*}
        \int_{\gamma} \dfrac{1}{z - \alpha} \diffOperator = 0
    \end{equation*}
    for every point \(\alpha\) not in \(U\). Equivalently, 
    \begin{equation*}
        \func{W}{\gamma, \alpha} = 0 \quad \forall \alpha \in U^c
    \end{equation*}
    Furthermore, two closed paths \(\gamma, \eta\) are homologous in \(U\) if 
    \begin{equation*}
        \func{W}{\gamma, \alpha} = \func{W}{\eta, \alpha} \quad \forall \alpha \in U^c
    \end{equation*}
\end{definition}

\begin{theorem}
    \begin{enumerate}
        \item If \(\gamma, \eta\) are closed paths in \(U\) and homotopic then they are homologous. Converse is true if their inside is contained in \(U\).
        \item If \(\gamma, \eta\) are closed paths and close together in \(U\) then they are homologous.
    \end{enumerate}
\end{theorem}

\begin{definition}
    Let \(\gamma_1, \dots , \gamma_n\) be a sequence of curves and let \(m_1, \dots , m_n\) be integers. The formal sum 
    \begin{equation*}
        \gamma = m_1 \gamma_1 + \dots + m_n \gamma_n = \sum_{i = 1}^n m_i \gamma_i
    \end{equation*}
    is called a chain. We define 
    \begin{equation*}
        \int_{\gamma} f = \sum_{i = 1}^n m_i \int_{\gamma_i} f
    \end{equation*}
    and thus 
    \begin{equation*}
        \func{W}{\gamma , \alpha} = \dfrac{1}{2 \pi i } \int_{\gamma} \frac{1}{z - \alpha} \diffOperator z = \sum_{i = 1}^n m_i \func{W}{\gamma_i, \alpha}
    \end{equation*}
\end{definition}

\begin{proposition}
    If \(\gamma, \eta\) are closed chains in \(U\) then 
    \begin{equation*}
        \func{W}{\gamma + \eta, \alpha} = \func{W}{\gamma, \alpha} + \func{W}{\eta, \alpha}
    \end{equation*}
\end{proposition}

\(\gamma\) and \(\eta\) are homologous in \(U\) if 
\begin{equation*}
    func{W}{\gamma, \alpha} = \func{W}{\eta, \alpha} \quad \forall \alpha \in U^c
\end{equation*}

\section{Cauchy's formula}
\begin{theorem}[Cauchy's theorem]
    Let \(\gamma\) be a closed chain an open set \(U\) such that \(\gamma\) is homologous to zero in \(U\). Let \(f\) be holomorphic in \(U\). Then 
    \begin{equation*}
        \int_{\gamma} f = 0
    \end{equation*}
\end{theorem}

\begin{corollary}
    If \(\gamma, \eta\) are closed chains in \(U\) then 
    \begin{equation*}
        \int_{\gamma} f = \int_{\eta} f
    \end{equation*}
\end{corollary}
 
\begin{theorem}
    Let \(U\) be an open set and \(\gamma\) be a closed chain in \(U\) homologous to zero. Let \(z_1, \dots , z_n\) be finite number of distinct points of \(U\). Let \(\gamma_i\) be the boundary of closed disc \(\bar{D}_i\) contained in \(U\), containing \(z_i\), and oriented counter-clockwise. We assume that \(\bar{D}_i\) does not intersect \(\bar{D}_j\) if \(i \neq j\). Let 
    \begin{equation*}
        m_i = \func{W}{\gamma , z_i}
    \end{equation*}
    Let \(U^{\ast}\) be the set \(U\) without \(\set{z_1 ,\dots ,z_n}\). Then \(\gamma\) is homologous to \(\sum m_i \gamma_i \) in \(U^{\ast}\).
\end{theorem}

\begin{theorem}[Cauchy's formula]
    Let \(\gamma\) be a closed chain in open set \(U\) homologous to zero in \(U\). Let \(f\) be an analytic function on \(U\) and \(z_0 \in U\) not on \(\gamma\) 
    \begin{equation*}
        \frac{1}{2 \pi i} \int_{\gamma} \frac{\func{f}{z}}{z - z_0} \diffOperator z = \func{W}{\gamma , z_0} \func{f}{z_0}
    \end{equation*}
\end{theorem}

\begin{proof}
    use analytic
\end{proof}

\begin{proof}
    Dixon, define \(g:  U \times U \)
    \begin{equation*}
        \bracket{w,z} \mapsto \begin{cases}
            \dfrac{\func{f}{w} - \func{f}{z}}{w- z} & w \neq z \\
            \func{f'}{z} & w= z
        \end{cases}
    \end{equation*}
    for each \(w\), \(z \mapsto \func{g}{w,z}\) is analytic. \(g\) is continuous, define a bounded entire function.
\end{proof}

\subsection{Artin's proof}
\begin{lemma}
    Let \(\gamma\) be a path in open \(U\). Then, there exists a rectangular path \(\eta\) with the same endpoints such that \(\gamma\) and \(\eta\) are close together. In particular, \(\gamma\) and \(\eta\) are homologous in \(U\) and for any holomorphic function \(f\) on \(U\) we have 
    \begin{equation*}
        \int_{\gamma} f= \int_{\eta} f
    \end{equation*}
\end{lemma}
This lemma reduces Cauchy's formula to the case when \(\gamma\) is rectangular.

\begin{definition}
    Let \(\gamma_i = \func{\gamma}{\clcl{a_i}{a_{i+1}}}\) then the chain 
    \begin{equation*}
        \gamma_1 + \dots + \gamma_n
    \end{equation*}
    is a subdivision of \(\gamma\). If \(\eta_i\) are obtained by a reparameterization of \(\gamma_i\) the
    \begin{equation*}
        \eta_1 + \dots + \eta_n
    \end{equation*}
    are also subdivision of \(\gamma\).
\end{definition}

\begin{theorem}
    Let \(\gamma\) be a rectangular closed chain in \(U\) and homologous to zero. Then there exists rectangles \(R_1 , \dots , R_N\) contained in \(U\) such that if \(\partial R_i\) is the boundary of \(R_i\) oriented in counter-clockwise then 
    \begin{equation*}
        \sum_{i = 1}^N m_i \partial R_i
    \end{equation*}
    for some integers \(m_i\) is a subdivision of \(\gamma\). 
\end{theorem}

\begin{proof}
    \(m_i = \func{W}{\gamma,\alpha_i}\) , \(\alpha_i \in R_i\). If \(m_i \neq 0\) for \(R_i\) then \(R_i \subset U\). \(\sum m_i \partial R_i\) is a subdivision.
\end{proof} 


\chapter{Application of Cauchy's formula}
\section{Uniform limits of analytical functions}
\begin{theorem}
    Let \(\set{f_n}\) be a sequence of holomorphic functions on a open set \(U\). Assume that for each compact subset \(K\) of \(U\), the sequence converges uniformly on \(K\) and let the limit function be \(f\). Then \(f\) is holomorphic.
\end{theorem}

\begin{theorem}
    Let \(\set{f_n}\) be a sequence of holomorphic functions on a open set \(U\). Assume that for each compact subset \(K\) of \(U\), the sequence converges uniformly on \(K\) and let the limit function be \(f\). Then the sequence of derivatives converges uniformly on every compact subset \(K\), \(f' = \lim f_n'\).
\end{theorem}

\section{Laurant seris}
The series 
\begin{equation*}
    \func{f}{z } = \sum_{n = -\infty}^{\infty} a_n z^n
\end{equation*}
is the Laurant series expansion. Let \(A\) be a set of complex number. Laurant series converges absolutely (uniformly) on \(A\) if the two series 
\begin{equation*}
    \func{f^+}{z} = \sum_{n \geq 0} a_n z^n \qquad \func{f^-}{z} = \sum_{n < 0} a_n z^n
\end{equation*}  
converge absolutely (uniformly). In that case 
\begin{equation*}
    f = f^{-} + f^{-}
\end{equation*}

\begin{theorem}
    Let \(A\) be the annulus \(A = \set<z>{r \leq \abs{z} \leq R}\) for some \(r \leq R\). Let \(f\) be holomorphic on \(A\) and \(r < s < S < R\). Then \(f\) has Laurant expansion 
    \begin{equation*}
        \func{f}{z} = \sum_{n = -\infty}^{\infty} a_n z^n
    \end{equation*}
    where
    \begin{equation*}
        a_n = \begin{cases}
            \frac{1}{2 \pi i} \int_{C_R} \frac{\func{f}{\xi}}{\xi^{n+1}} & n \geq 0 \\
            \frac{1}{2 \pi i} \int_{C_r} \frac{\func{f}{\xi}}{\xi^{n+1}} & n < 0 \\
        \end{cases}
    \end{equation*}
    converges absolutely on \(s \leq \abs{z} \leq S\).
\end{theorem}

\section{Singularity}
Let \(D\) be an open disc cenetered at \(z_0\) and \(U = D/\set{z_0}\). Let \(f\) be analytic on \(U\). Then, \(f\) is said to have isolated singularity at \(z_0\). 
\subsection{Removable singularity}
\begin{theorem}
    If \(f\) is bounded in some neighbourhood of \(z_0\), then one can define \(\func{f}{z_0}\) in a uninque way such that \(f\) is analytic on \(z_0\)
\end{theorem}
\begin{proof}
    Laurant expansion
\end{proof}
This kind of singularity is called Removable singularity.
\subsection{Poles}
If \(f\) has finite negative terms in its Laurant expansion 
\begin{equation*}
    \func{f}{z} = \dfrac{a_{-m}}{\bracket{z - z_0}^m} + \dots + a_0 + a_1 \bracket{z - z_0} + \dots 
\end{equation*}
Then \(f \) is said to have a pole of order \(m\). However, the order of \(f\) at \(z_0\) is \(-m\). 
\begin{equation*}
    \ord_{z_0} fg = \ord_{z_0} f + \ord_{z_0} g
\end{equation*}

\begin{proposition}
    \(f\) has a pole of order \(m\) at \(z_0\) if and only if \(\func{f}{z} \bracket{z - z_0}^m\) is holomorphic at \(z_0\) and has no zero at \(z_0\).
\end{proposition}

\begin{definition}
    \(f\) i defined on \(U/S\), where \(S\) is a discrete set of points which are the poles of \(f\), is mermorphic on \(U\). Thus, it is the quotient of two holomorphic functions in the neighbourhood of a point.
\end{definition}

\subsection{Essential singularity}
When the Laurant expansion has inifinite negative terms. 

\begin{theorem}[Casorati-Weirestrass]
    Let \(0\) be an essential singularity of the function  \(f\). Let \(D\) be a disc cenetered at \(0\) on which \(f\) is holomorphic except at \(0\). Let \(U\) be the complement of \(\set{0}\) in \(D\). Then, \(\func{f}{U}\) is dense in the complex numbers. In other words, the values of \(f\) on \(U\) come arbitrarily close to any complex number. In fact \(f\) takes on every complex value except possible one. 
\end{theorem}

\begin{theorem}
    The only analytic automorphism of \(\Complex\) are functions of the form \(\func{f}{z} = az + b\), where \(a,b\) are constants and \(a \neq 0\).
\end{theorem}
\chapter{Calculus of Residues}
\section{Residue}
Let Laurent expansion of \(f\) at \(z_0\) be 
\begin{equation*}
    \func{f}{z} = \sum_{n = -\infty}^{\infty} a_n \bracket{z - z_0}^n
\end{equation*}
Then, the residue of \(f\) at \(z_0\) denoted by \(\func{\Res}{f,z_0}\) is \(a_{-1}\).
\begin{theorem}
    Let \(z_0\) be an isolated singularity of \(f\), and let \(C\) be a small circle oriented counter-clockwise cenetered at \(z_0\) such that \(f\) is holomorphic on \(C\) and its interior except possibly at \(z_0\). Then 
    \begin{equation*}
        \int_C \func{f}{\xi} \diffOperator \xi = 2\pi i \func{\Res}{f,z_0}
    \end{equation*}
\end{theorem}

\begin{theorem}[Residue formula]
    Let \(U\) be an open set, and \(\gamma\) a closed chain in \(U\) such that \(\gamma\) is homologous to \(0\) in \(U\). Let \(f\) be analytic on \(U\) except at a finite number of points \(z_1, \dots , z_n\). Then 
    \begin{equation*}
        \int_{\gamma} f = 2 \pi i \sum_{i = 1}^n \func{W}{\gamma, z_i} \func{\Res}{f, z_i}
    \end{equation*}
\end{theorem}

\begin{lemma}
    If \(f\) has a simple pole at \(z_0\) and \(g\) is holomorphic at \(z_0\) 
    \begin{equation*}
        \func{\Res}{fg,z_0} = \func{g}{z_0} \func{\Res}{f,z_0}
    \end{equation*}
\end{lemma}

\begin{proof}
    Suppose \(f\) has the following Laurent expansion 
    \begin{equation*}
        \func{f}{z} = \sum_{n = -1}^{\infty} a_n \bracket{z - z_0}^n
    \end{equation*}
    and \(g\) is the following power series expansion 
    \begin{equation*}
        \func{g}{z} = \sum_{n = 0}^{\infty} b_n \bracket{z - z_0}^n
    \end{equation*}
    Then 
    \begin{align*}
        \func{fg}{z} &= \bracket{\sum_{n = -1}^{\infty} a_n \bracket{z - z_0}^n} \bracket{\sum_{n = 0}^{\infty} b_n \bracket{z - z_0}^n}\\
        &= \dfrac{a_{-1}b_0}{z - z_0} + \bracket{a_{-1}b_1 + a_{0}b_0} + \dots 
    \end{align*}
    therefore 
    \begin{equation*}
        \func{\Res}{fg,z_0} = a_{-1}b_0 = \func{g}{z_0} \func{\Res}{f,z_0}
    \end{equation*}
\end{proof}

\begin{lemma}
    Suppose \(\func{f}{z_0} = 0\) but \(\func{f'}{z_0} \neq 0\). Then, \(\frac{1}{f}\) has a pole of order 1 at \(z_0\) with 
    \begin{equation*}
        \func{\Res}{\dfrac{1}{f}, z_0} = \dfrac{1}{\func{f'}{z_0}}
    \end{equation*}
\end{lemma}

\begin{proof}
    Suppose \(f\) has the following power series expansion 
    \begin{equation*}
        \func{f}{z} = \sum_{n = 1}^{\infty} a_n \bracket{z - z_0}^n
    \end{equation*}
    with \(a_1 \neq 0\). Then, 
    \begin{align*}
        \func{\dfrac{1}{f}}{z} &= \dfrac{1}{\bracket{z - z_0} \bracket{a_1 + a_2 \bracket{z - z_0} + \dots }}\\
        &=\dfrac{1}{z - z_0} \bracket{a_1^{-1} + \bigO{z - z_0}}
    \end{align*}
    has residue of \(a_1^{-1} =\bracket{\func{f'}{z_0}}^{-1} \) at \(z_0\).
\end{proof}

\begin{lemma}
    \(f\) has a pole order \(m\) and \(g\) is holomorphic at \(z_0\)
    \begin{equation*}
        \func{\Res}{fg,z_0} = \sum_{i = 0}^{m - 1} \func{g^{(i)}}{z_0} \func{\Res}{(z - z_0)^i \func{f}{z}, z_0}
    \end{equation*}
\end{lemma}

\begin{lemma}
    Let \(f\) be meromorphic at \(z_0\) then 
    \begin{equation*}
        \func{\Res}{\dfrac{f'}{f}, z_0} = \func{\ord_{z_0}}{f}
    \end{equation*}
\end{lemma}

\begin{theorem}
    Let \(U\) be an open set, \(\gamma\) closed chain in \(U\), \(f\) meromorphic on \(U\) with only finite number of negative terms and poles at \(z_0, \dots , z_n\), none of which lie on \(\gamma\). 
    \begin{equation*}
        \int_{\gamma} \func{f'}{f} = 2\pi i \sum_{i = 1}^n \func{W}{\gamma, z_i} \func{\ord_{z_0}}{f}
    \end{equation*}
\end{theorem}

\begin{definition}
    Let \(\gamma\) be a closed path. It has interior and exterior if for all\( \alpha \in \Complex / \gamma\) the winding the number \(\func{W}{\gamma, \alpha} = 0,1\). Then, the interior is defined as all the points with winding number of \(1\) and exterior is all the points with winding number of \(0\).
\end{definition}

\begin{theorem}[Rouche's]
    Let \(\gamma\) be a closed path homologous to \(0\) in \(U\) and assume \(\gamma\) has in interior. Let \(f,g\) be analytic function on \(U\) and 
    \begin{equation*}
        \abs{\func{f}{z} - \func{g}{z}} < \abs{\func{f}{z}}
    \end{equation*}
    for all \(z\) on \(\gamma\). Then \(f\) and \( g\) have the same number of zeros in the interior of \(\gamma\).
\end{theorem}

\begin{theorem}
    Let \(f\) be analytic in neighbourhood of a point \(z_0\) and assumer \(\func{f'}{z_0} \neq 0\). Then, \(f\) is a local analytic isomorphism at \(z_0\).
\end{theorem}

\section{Evaluation of definite integrals}
add the integrals
\part{Measure Theory and Probability Theory}
\chapter{Measure Theory}
\section{Bernoulli sequences}
Let \(\calB\) be the set of all Bernoulli sequences. \(\calB\) is uncountable.
\begin{proposition}
    If we delete a countable subset of \(\calB\), we can index what is left by the points on the real interval \(I = \opcl{0}{1}\). That is, there exists an injective function \(f: I \to \calB\).
\end{proposition}
\begin{proof}
    Each \(\omega \in I\) can be written as 
    \begin{align*}
        \omega &= \sum_{i = 1}^{\infty} \dfrac{a_i}{2^i} \qquad a_i = 0,1\\
        &= 0.a_1a_2\dots
    \end{align*}
    Since \(\omega\) may not have a unique binary representation, we will only consider non-terminating expansion for \(\omega\). Then, by mapping \(1\) to \(H\) and \(0\) to \(T\) we get an injective function from \(I\) to \(\calB\). To show this, suppose \(\calB_{\deg}\) is the set of all Bernoulli sequences that after a certain point degenerates to all tails.
    \begin{lemma}
        \(\calB_{\deg}\) is countable.
    \end{lemma}
    \begin{prooflemma}
        Let \(\calB_{\deg}^k\) be all degenerate Bernoulli sequences where we have only tails after \(k_{\cardinalTH}\) toss. Then, \(\calB_{\deg}^k\) is finite and 
        \begin{equation*}
            \calB_{\deg} = \bigcup_{k = 1}^{\infty} \calB_{\deg}^k
        \end{equation*}
        is the countable union of finite set and hence \(\calB_{\deg}\) is countable.
    \end{prooflemma}

    This concludes the proof.
\end{proof}

\begin{definition}[Borel Principle]
    Suppose \(E\) is a probabilistic event occuring in certain sequences. Let \(\calB_{E}\) denote the subset of \(\calB\) for which that event occurs. Let \(I_E\) be the corresponding subset of \(I\), then 
    \begin{equation*}
        \prob{E} = \func{\mu_L}{I_E}
    \end{equation*}
    where \(\mu_L\) is Lebesgue measure.
\end{definition}

\begin{example}
    Start with \(X\) dollars and at each toss you win \(1\) dollars if heads shows up and ypu lose \(1\) dollars if tail shows up. What is the probability you lose all your original stake?

    For \(\omega \in I\) define the \(k_{\cardinalTH}\) \textit{Radamcher} function, \(R_k\), by 
    \begin{align*}
        \func{R_k}{\omega} &= 2a_k - 1\\
        &= \begin{cases}
            +1 & a_k = 1\\
            -1 & a_k = 0
        \end{cases}
    \end{align*}
    Then, let \(\func{S_k}{\omega}\) be the total amount won or loss at \(k_{\cardinalTH}\) toss.
    \begin{equation*}
        \func{S_k}{\omega} = \sum_{l = 1}^k \func{R_l}{\omega}
    \end{equation*}
    Thus, the event that we lose our stake at \(k_{\cardinalTH}\) is 
    \begin{equation*}
        I_{E_k} = \set[[\Bigg]]<\omega \in I>{\func{S_l}{\omega} > -X \ \mathrm{for} \ l < k, \func{S_k}{\omega} = -X}
    \end{equation*}
    hence 
    \begin{equation*}
        I = \bigcup_{l = 1}^{\infty} I_{E_k}
    \end{equation*}
    is the event that we loss all our money eventually. We will however postpone calculating \(\func{\mu_L}{I_E}\) as it is not finite union of intervals.
\end{example}

\section{Weak law of large numbers}
For some fixed \(\epsilon\) let
\begin{equation*} 
    I_N = \set<\omega \in I>{\abs{\dfrac{\func{s_N}{\omega}}{N} - \dfrac{1}{2}} > \epsilon}
\end{equation*}
where \(\func{s_N}{\omega} = a_1 + a_2 + \dots + a_N\). Then, \(I_N\) represents the event that the number of heads and tails \underline{are not} roughly equal.
\begin{theorem}[Weak law of large numbers]
    WLLN states 
    \begin{equation*}
        \func{\mu_L}{I_N} \to 0 \ \mathrm{as} \ N \to \infty
    \end{equation*}
\end{theorem}
\begin{proof}
    Equivalently, for 
    \begin{equation*} 
        A_N = \set[[\Big]]<\omega \in I>{\abs{\func{S_N}{\omega}} > 2N \epsilon}
    \end{equation*}
    then
    \begin{equation*}
        \func{\mu_L}{A_N} \to 0  \ \mathrm{as} \ N \to \infty
    \end{equation*}
    \begin{lemma}[Chebyshev's inequality]
        Let \(f\) be a non-negative, piecewise constant function on \(\opcl{0}{1}\). Let \(\alpha > 0\) be given. Then, 
        \begin{equation*}
            \func{\mu_L}{\set[[\Big]]<\omega \in I>{\func{f}{\omega} > \alpha}} < \dfrac{1}{\alpha} \int_{0}^1 f \diffOperator x
        \end{equation*}
        where \(\int\) is the Riemann integral.
    \end{lemma}
    \begin{prooflemma}
        Since \(f\) is piecewise constant then there is a parition \(0 = x_1 < \dots < x_k = 1\) such that \(f = c_i\) on \(\opcl{x_i}{X_{i+1}}\) for \(i= 1, \dots k - 1\). Then, 
        \begin{align*}
            \int_{0}^1 f \diffOperator x &= \sum_{i = 1}^{k-1} c_i \bracket{x_{i+1} - x_i} \\
            &\geq  \sum_{c_i > \alpha} c_i \bracket{x_{i+1} - x_i} \\
            &> \alpha  \sum_{c_i > \alpha}  x_{i+1} - x_i = \alpha \func{\mu_L}{\set[[\Big]]<\omega \in I>{\func{f}{\omega} > \alpha}}
        \end{align*}
        which proves the Chebyshev's inequality.
    \end{prooflemma}
    Then, we have 
    \begin{equation*}
        A_N = \set[[\Big]]<\omega \in I>{\abs{\func{S_N}{\omega}} > 2N \epsilon}= \set[[\Big]]<\omega \in I>{\bracket{\func{S_N}{\omega}}^2 > 4N^2 \epsilon^2}
    \end{equation*}
    and hence 
    \begin{align*}
        \func{\mu_L}{A_N} &< \dfrac{1}{4N^2\epsilon^2} \int_0^1 \bracket{\func{S_N}{\omega}}^2 \diffOperator \omega\\
        &= \dfrac{1}{4N^2\epsilon^2} \squareBracket{\sum_i \int_0^1 \bracket{\func{R_i}{\omega}}^2 \diffOperator \omega + \sum_{i\neq j} \int_0^1 \func{R_i}{\omega}\func{R_j}{\omega} \diffOperator \omega }\\
        &= \dfrac{1}{4N^2\epsilon^2} N = \dfrac{1}{4N \epsilon^2}
    \end{align*}
    Therefore, as \(N\) approaches infinity, \(\func{\mu}{A_N}\) approaches zero.
\end{proof}

Now we want to show that for a ``typical'' Bernoulli sequence 
\begin{equation}\label{eq:typicalBernoulli}
    \dfrac{1}{2} - \dfrac{\func{s_N}{\omega}}{N} \to 0 \ \mathrm{as} \ N \to \infty
\end{equation}
and by ``typical'' we mean \Cref{eq:typicalBernoulli} fails on a set of zero proability or the equivalent event has measure zero.
\begin{definition}
    A set \(A \subset \Reals\) has Lebesgue measure zero if for every \(\epsilon > 0\), there exists a countable covering \(\set{A_i}\) of \(A\) by intervals such that 
    \begin{equation*}
        \sum_{i = 1}^{\infty} \func{\mu_L}{A_i} < \epsilon
    \end{equation*}
\end{definition}
\begin{itemize}
    \item Subset of a measure zero are measure zero.
    \item A signle point has a measure zero.
    \item countable union of measure zeros is a measure zero.
\end{itemize}

Let \(N = \set<\omega \in I>{ \frac{\func{s_N}{\omega}}{N} \to \frac{1}{2}}\). \(N\) is called the set of \textbf{normal numbers}. Let \(N^c\) be the complement of \(N\).
\begin{theorem}[Strong law of large numbers]
    SLLN states that \(N^c\) has measure zero.
\end{theorem}

\begin{proof}
    Let \(A_n = \set<\omega \in I>{\abs{\func{S_n}{\omega}} > \epsilon n}\) then 
    \begin{equation*}
        A_n = \set[[\Big]]<\omega \in I>{\bracket{\func{S_n}{\omega}}^4 > \epsilon^4 n^4}
    \end{equation*}
    By Chebyshev's inequality
    \begin{align*}
        \func{\mu_L}{A_n} &< \dfrac{1}{n^4 \epsilon^4} \int_0^1 \bracket{\func{S_n}{\omega}}^4 \diffOperator \omega \\
        &= \dfrac{1}{n^4 \epsilon^4} \bracket{n + 3n(n-1)} \leq \dfrac{3}{\epsilon^4 n^2}
    \end{align*}
    \begin{lemma}
        Given \(\delta > 0\) there exists a sequence \(\epsilon_1, \epsilon_2, \dots \) such that \(\epsilon \to 0\) and 
        \begin{equation*}
            \sum_{n = 1}^{\infty} \dfrac{3}{n^2 \epsilon_n^4} < \delta
        \end{equation*}
    \end{lemma}
    \begin{prooflemma}
        Choose \(\epsilon_n^4 = cn^{-1/2}\) for some constant \(c\). 
        \begin{equation*}
            \sum_{n = 1}^{\infty} \dfrac{3}{n^2 \epsilon_n^4} =\dfrac{3}{c} \sum_{n = 1}^{\infty} n^{-\frac{3}{2}} = \dfrac{3}{c}L < \delta
        \end{equation*}
        if \(c > \dfrac{3L}{\delta}\).
    \end{prooflemma}
    Finally, Let \(B_n = \set<\omega \in I> {\abs{\func{S_n}{\omega}} > \epsilon_n n}\). Then, by our first computation 
    \begin{equation*}
        \func{\mu_L}{B_n} < \dfrac{3}{\epsilon_n^4 n^2}
    \end{equation*}
    and by the lemma 
    \begin{equation*}
        \sum  \func{\mu_L}{B_n} < \dfrac{3}{\epsilon_n^4 n^2} < \delta 
    \end{equation*}
    It remains to show that \(N^c \subset \cup_{n = 1}^{\infty} B_n\) which is obvious. As every \(\omega \in N^c\) is some \(B_n\). Therefore, \(N^c\) has measure zero.
\end{proof}
\section{Measure theory}
\subsection{Measure}
\begin{definition}
    A \textbf{ring} of sets in \(X\) is a non-empty collection \(\scrR\) of subsets of \(X\) satisfying following two properties
    \begin{enumerate}
        \item \(A,B \in \scrR \implies A \cup B \in \scrR\).
        \item \(A,B \in \scrR \implies A - B \in \scrR\).
    \end{enumerate}
\end{definition}
\begin{lemma}
    If \(A,B \in \scrR\), then \(A \cap B \in \scrR\). Moreover, \(R\) is a ring if and only if it is closed under intersection and symmetric difference.
\end{lemma}
\begin{proof}
    \begin{equation*}
        A \cap B = \squareBracket{(A \cup B) - (A - B)} - (B - A)
    \end{equation*}
    \
\end{proof}
\begin{example}
    \(\powerSet{X}\) is a ring.
\end{example}

\begin{definition}
    A \textbf{semi-ring} is a non-empty collection of \(\scrS\) such that 
    \begin{enumerate}
        \item \(\emptyset \in \scrS\).
        \item \(A,B \in \scrS \implies A \cap B \in \scrS\).
        \item \(A,B \in \scrS \implies A - B = \cup_{i = 1}^k C_i\) where \(C_i \in \scrS\) are disjoint.
    \end{enumerate}
    The ring generated by a semi-ring \(\scrS\) is denoted by \(\func{\calR}{\scrS}\).
\end{definition}

\begin{theorem}
    Suppose \(\scrS\) is a semi-ring. Then, \(A \in \func{\calR}{\scrS}\) if and only if \(A = \cup_{i = 1}^k C_i\) for disjoint \(C_i \in \scrS\).
\end{theorem}

\begin{example}
    Let \(\scrR_{\Leb} = \set<A>{A = \bigcup_{i = 1}^n A_i}\) where \(A_i\) are disjoint \(k\)-cells in \(\Reals^k\). \(\scrR_{\Leb}\) is a ring. Why?
\end{example}

\begin{definition}
    Let \(\mu:\scrR \to \ExtReals^+_0\) where \(\scrR\) is a ring of some set \(X\), then \(\mu\) is \textbf{additive} if \(\func{\mu}{A \cup B} = \func{\mu}{A} + \func{\mu}{B}\) whenever \(A,B \in \scrR\) are disjoint.
\end{definition}

\begin{proposition}
    Let \(\scrR\) be a ring of \(X\) and \(\mu: \scrR \to \ExtReals^+_0\) is additive, then 
    \begin{enumerate}
        \item \(\func{\mu}{\emptyset} = 0\).
        \item (Monotonicity) If \(A,B \in \scrR\) with \(A \subset B \implies \func{\mu}{A} \geq \func{\mu}{B}\).
        \item (Finite Addtivity) For disjoint \(A_1, \dots , A_n \in \scrR\), \(\func{\mu}{\cup_{i = 1}^n A_i} = \sum_{i = 1}^n \func{\mu}{A_i}\).
        \item (Lattice property) \(A,B \in \scrR \implies \func{\mu}{A \cup B} + \func{\mu}{A \cap B} = \func{\mu}{A} + \func{\mu}{B}\).
        \item (Finite subaddtivity) If \(A_1, \dots , A_n \in \scrR\), then 
        \begin{equation*}
            \func{\mu}{\bigcup_{i = 1}^n A_i} \leq \sum_{i = 1}^n \func{\mu}{A_i}
        \end{equation*}
    \end{enumerate}
\end{proposition}

\begin{definition}
    \(\mu\) is \textbf{countably additive} on \(\scrR\) if given any countable collection \(\set{A_i} \subset \scrR\) with \(A_i\) mutually disjoint and such that \(A = \cup A_i\) is also in \(\scrR\)
    \begin{equation*}
        \func{\mu}{A} = \sum_{i = 1}^{\infty} \func{\mu}{A_i}
    \end{equation*}
    A countably additive, non-negative set function \(\mu\) on ring \(\scrR\) in \(X\) is called a \textbf{measure}.
\end{definition}

\begin{theorem}
    If \(X = \Reals^n\), \(\scrR = \scrR_{\Leb}\) and \(\mu\) for \(n\)-cells is defined as
    \begin{equation*}
        \func{\mu}{A} = \prod_{i = 1}^n( b_i - a_i)
    \end{equation*}
    where \(A = \set<x \in \Reals^n>{x_i \in \pair{a_i}{b_i}}\)-- \(\pair{a}{b}\) denotes any of  four possibilities, \(\opop{a}{b}, \opcl{a}{b}, \clop{a}{b}\) and \(\clcl{a}{b}\). Then, \(\mu\) is a measure.
\end{theorem}
To prove this theorem, we consider the following lemma 
\begin{lemma}
    Let \(A \in \scrR_{\Leb}\) and let \(\epsilon > 0\). There exists \(F,G \in \scrR_{\Leb}\) such that \(F\) is closed and \(G\) is open, \(F \subset A \subset G\) and 
    \begin{align*}
        \func{\mu}{F} &\geq \func{\mu}{A} - \epsilon\\
        \func{\mu}{G} &\leq \func{\mu}{A} + \epsilon
    \end{align*}
\end{lemma}
\begin{prooflemma}
    
\end{prooflemma}

\begin{proof}
    
\end{proof}

\begin{definition}
Let \(\set{A_n}\) be a sequence of sets in \(X\). Then, 
    \begin{align*}
        \limsup A_n &= \bigcap_{k = 1}^{\infty} \bigcup_{n = k}^{\infty} A_n & \liminf A_n &= \bigcup_{k = 1}^{\infty} \bigcap_{n = k}^{\infty} A_n
    \end{align*}
    \(A_n\) is said to converge to \(A\) if \(\limsup A_n = \liminf A_n = A\). \(A\) is said to be \textbf{limit set} of \(\set{A_n}\). The sequence \(\set{A_n}\) is increasing if \(A_{n} \subset A_{n+1}\) for all \(n\) and it is decreasing if \(A_n \supset A_{n+1}\) for all \(n\).
\end{definition}

It can be readily seen that if \(\set{A_n}\) is increasing/decreasing, then it is convergent to \(\cup A_n\)/ \(\cap A_n\).

\begin{definition}
    Let \(\scrR\) be a ring of subsets in \(X\) and \(\mu:\scrR \to \ExtReals^+_0\) is a set function. For any \(E \in \scrR\), \(\mu\) is said to be \textbf{continuous from below} if for all increasing sequences \(E_n\), \(\func{\mu}{E_n} \to \func{\mu}{E}\). Similarly, \(\mu\) is said to be \textbf{continuous from above} if for all decreasing sequences \(E_n\) such that \(\func{\mu}{E_i} < \infty\) for at least one \(i\), \(\func{\mu}{E_n} \to \func{\mu}{E}\). \(\mu\) is continuous at \(E\) if it is both continuous from below and  above.
\end{definition}

\begin{theorem}
    A measure \(\mu\) is continuous at every \(E \in \scrR\). 
\end{theorem}

\begin{proof}
    
\end{proof}

\begin{proposition}
    Suppose \(\scrR\) is ring of subsets of \(X\) and \(\mu\) is a finite additive function.
    \begin{itemize}
        \item If \(\mu:\scrR \to \ExtReals^+_0\) is continuous from below at every \(E \in \scrR\), then \(\mu\) is a measure.
        \item If \(\mu:\scrR \to \Reals^+_0\) is continuous from above at  \(\emptyset\), then \(\mu\) is a measure.
    \end{itemize}
\end{proposition}

\subsection{Caratheodory extension}

\begin{definition}
    Let \(A\) be a subset of \(X\). A number \(l\geq 0\) is called an \textbf{approximate outer measure} of \(A\) if there exists a covering of \(A\) by countable collection of sets \(\set{A_i}\) with each \(A_i\in \scrR\) such that 
    \begin{equation*}
        \sum_{i = 1}^{\infty} \func{\mu}{A_i} \geq l
    \end{equation*}
\end{definition}
\begin{remark}
    \(l\) is allowed to be \(+\infty\).
\end{remark}

\begin{definition}
    Let \(A\) be a subset of \(X\). The \textbf{outer measure} of \(A\) is the greates lower bound of the set \(\set<l>{l \ \mathrm{is an approximate outer measure}}\). 
    \begin{equation*}
        \func{\mu^{\ast}}{A} =  \inf \set<l>{A \subset \bigcup_{i= 1}^{\infty} A_i, \sum_{i = 1}^{\infty} \func{\mu}{A_i} \leq l}
    \end{equation*}
    If the set is empty, then \(\func{\mu^{\ast}}{A} = + \infty\).
\end{definition}

\begin{remark}
    \(\mu^{\ast}: \powerSet{X} \to \ExtReals^+_0\) is not a measure. However, \(\mu^{\ast}\) is a measure on larger ring of subsets of \(X\).
\end{remark}

\begin{proposition}
    \ 
    \begin{enumerate}
        \item If \(A \in \scrR\), then \(\func{\mu^{\ast}}{A} = \func{\mu}{A}\).
        \item If \(A \subset B\), then \(\func{\mu^{\ast}}{A} \leq \func{\mu^{\ast}}{B}\).
        \item \(\mu^{\ast}\) is countably subadditive. 
        \begin{equation*}
            \func{\mu^{\ast}}{\bigcup_{i=1}^{\infty} A_i}  \leq \sum_{i = 1}^{\infty} \func{\mu^{\ast}}{A_i}
        \end{equation*}
    \end{enumerate}
\end{proposition}

\begin{proof}
    
\end{proof}

\begin{definition}
    A set \(A \subset X\) is \textbf{measurable} with respect to \(\mu\) if for all \(E \subset X\),
    \begin{equation*}
        \func{\mu}{E} = \func{\mu}{E \cap A} + \func{\mu}{E \cap A^c}
    \end{equation*}
    The set of all measurable sets with respect to \(\mu\) is denoted by \(\func{\calM}{\mu}\).
\end{definition}

\begin{proposition}
    Let \(\mu\) be a measure defined on the ring \(\scrR\). Then, \(\func{\calM}{\mu^{\ast}}\) is a ring and \(\scrR \subset \func{\calM}{\mu^{\ast}}\).
\end{proposition}

\begin{proof}
         \(\func{\calM}{\mu^{\ast}}\) is closed under complementing and intersection.
\end{proof}


\begin{definition}
    Let \(\scrS\) be a collection of subsets of a set \(X\). \(\scrS\) is called a \textbf{\(\sigma\)-ring} if
    \begin{enumerate}
        \item \(\scrS\) is a ring.
        \item \(\scrS\) is closed under countable union. That is, given \(\set{A_i} \subset \scrS\), \(\bigcup A_i \in \scrS\).
    \end{enumerate} 
\end{definition}

\begin{theorem}
    \(\func{\calM}{\mu^{\ast}}\) is a \(\sigma\)-ring and the restriction \(\mu^{\ast}\) on \(\func{\calM}{\mu^{\ast}}\) is a measure.
\end{theorem}

\begin{remark}
    The extension of \(\mu\) to \(\mu^{\ast}\) is not necessarily unique. However, by placing certain requirements on \(X\) we can deduce uniqueness.
\end{remark}

\begin{definition}
    Let \(\scrR\) be a ring of subsets in \(X\) and \(\mu:\scrR \to \ExtReals^+_0\) be a measure. \(\mu\) is \textbf{finite} if for each \(A \in \scrR\), \(\func{\mu}{A} < \infty\). \(\mu\) is \textbf{\(\sigma\)-finite} if for each \(E \in \func{\sigma}{R}\), there exists a sequence of subsets \(\set{E_n} \subset \scrR\) such that \(E \subset \cup E_n\) and \(\func{\mu}{E_n} < \infty\) for all \(n\).
\end{definition}

\begin{definition}
    A collection of subsets \(\scrC\) is a \textbf{monotone class} if the limit set of every increasing and decreasing sequence  of \(\scrC\) is in \(\scrC\). The smallest monotone class of a collection \(\scrD\) is denoted by \(\func{\calC}{\scrD}\).
\end{definition}

\begin{theorem}
    If \(\scrR\) is a ring and \(\scrC\) is a monotone class containing \(\scrR\), then \(\func{\sigma}{\scrR} \subset \scrC\). In fact \(\func{\sigma}{\scrR}  = \func{\calC}{\scrR}\).
\end{theorem}

\begin{corollary}
    Let \(\scrR\) be a ring of subsets of \(X\) and \(\mu,\nu: \func{\sigma}{\scrR} \to \ExtReals^+_0\) are two measures. If \(\mu\) and \(\nu\) are finite and equal on \(\scrR\), then they are equal on \(\func{\sigma}{\scrR}\).
\end{corollary}

\begin{definition}
    Let \(\scrD\) be a collection of subset of \(X\) and \(A \subset X\). Then 
    \begin{equation*}
        \scrD|_A = \set<A \cap E>{E \in \scrD}
    \end{equation*}
\end{definition}

\begin{proposition}
    Let \(\scrR\) be a ring. Then, \(\func{\sigma}{\scrR|_A} =\func{\sigma}{\scrR}|_A \).
\end{proposition}

\begin{theorem}
    Suppose \(\scrR\) is a ring of subsets of \(X\) and \(\mu: \scrR \to \ExtReals^+_0\) is a \(\sigma\)-finite measure. The restriction of \(\mu^{\ast}\) to \(\func{\sigma}{\scrR}\) is the only extension of \(\mu\) to \(\scrR\).
\end{theorem}

\subsection{Metric extension}
Let \(\mu:\scrR \to \Reals^+_0\) be a measure on the ring \(\scrR\). For \(A,B \in \powerSet{X}\), we define a \textit{psuedo distance} function on \(\powerSet{X}\)
\begin{equation*}
    \func{d}{A,B} = \func{\mu^{\ast}}{A \triangle B}
\end{equation*}
Note that, \(\func{d}{A,B}\) may be \(+ \infty\) and \(\func{d}{A,B} = 0\) does not imply that \(A = B\). To go around this constraint, we consider the equivalence relation \(\sim\) with \(A \sim B\) when \(\func{d}{A,B} = 0\). Then, \(d\) is metric on the equivalence classes \(\powerSet{X}/\sim\).
\begin{proposition}
    Suppose \(A,B,C \in \powerSet{X}\), then 
    \begin{enumerate}
        \item \(\func{d}{A,B} = \func{d}{B,A}\).
        \item \(\func{d}{A,A} = 0\).
        \item \(\func{d}{A,B} \leq \func{d}{A,C} + \func{d}{C,A}\)
    \end{enumerate}
\end{proposition}
\begin{proof}
    
\end{proof}

Although, \(d\) is not quite a metric, we can still definte the notion of convergence; \(A_i \to A\) if \(\func{d}{A,A_i} \to 0\).

\begin{proposition}
    The Boolean operation in \(\powerSet{X}\) are continuous with respect to \(d\). That is, if \(A_n \to A\) and \(B_n \to B\)
    \begin{align*}
        A_n \cup B_n &\to A \cup B\\
        A_n \cap B_n &\to A \cap B\\
        A_n^c &\to A^c
    \end{align*}
\end{proposition}

\begin{proposition}
    \(\mu^{\ast}\) is continuous in the following sense that for \(A,B \in \powerSet{X}\) where  \(\func{\mu^{\ast}}{A}\) or \(\func{\mu^{\ast}}{B}\) is finite 
    \begin{equation*}
        \abs{\func{\mu^{\ast}}{A} - \func{\mu^{\ast}}{B}} \leq \func{d}{A,B}
    \end{equation*}
\end{proposition}

\begin{definition}
    Let \(\scrM_F\) be the closure of \(\scrR\) in \(\powerSet{X}\). That is, \(A \in \scrM_F\) whenever there exists a sequence \(\set{A_i} \subset \scrR\) such that \(\func{d}{A_i,A} \to 0\) as \(i \to \infty\).
\end{definition}

\begin{theorem}
    \ 
    \begin{enumerate}
        \item \(\scrM_F\) is a ring.
        \item For \(A \in \scrM_F\), \(\func{\mu^{\ast}}{A} < + \infty\).
        \item \(\mu^{\ast}\) is a measure on \(\scrM_F\).
    \end{enumerate}
\end{theorem}

\begin{definition}
    \(A\) is \textbf{measurable set}, \(A \in \scrM\), if there exists a sequence \(\set{A_i} \subset \scrM_F\) such that \(A = \bigcup A_i\).
\end{definition}

\begin{theorem}
    If \(A \in \scrM\), then \(A \in \scrM_F \iff \func{\mu^{\ast}}{A} < + \infty\).
\end{theorem}


\begin{theorem}
    \(\scrM\) is a \(\sigma\)-ring.
\end{theorem}

\begin{theorem}
    If \(A_1,A_2, \dots \) is a countable collection of disjoint sets in \(\scrM\) then 
    \begin{equation*}
        \func{\mu^{\ast}}{\bigcup_{i = 1}^{\infty} A_i} = \sum_{i = 1}^{\infty} \func{\mu^{\ast}}{A_i}
    \end{equation*}
    That is, \(\mu^{\ast}\) is a measure on \(\scrM\).
\end{theorem}

We now investigate the relation between the measurablity in Caratheodory sense and metric sense. Note that, for measurablity in metric sese we assumed that \(\mu\) is a finite measure. Therefore, we assume that \(\mu^{\ast}\) is \(\sigma\)-finite with respect to \(\scrM_F\).

\begin{theorem}
    Let \(\scrR\) be a ring of subset of \(X\) and \(\mu:\scrR \to \Reals^+_0\) be a measure. If \(A \in \scrM_F\), then for every \(E \subset X\)
    \begin{equation*}
        \func{\mu^{\ast}}{E} = \func{\mu^{\ast}}{E \cap A} +  \func{\mu^{\ast}}{E \cap A^c}
    \end{equation*}
\end{theorem}

\begin{theorem}
    Let \(\scrR\) be a ring of subset of \(X\) and \(\mu:\scrR \to \Reals^+_0\) be a measure. If \(\func{\mu^{\ast}}{A} < \infty \) and for every \(E \subset X\)
    \begin{equation*}
        \func{\mu^{\ast}}{E} = \func{\mu^{\ast}}{E \cap A} +  \func{\mu^{\ast}}{E \cap A^c}
    \end{equation*}
    then, \(A \in \scrM_F\).
\end{theorem}
Therefore, from the last two theorems we conclude that if the measure space of Caratheodory extension \((X,\func{\scrM}{\mu^{\ast}},\mu^{\ast})\) is \(\sigma\)-finite, then both methods of extension result in the same extension. 

\subsection{Completion of measure spaces}
\begin{lemma}
    Suppose \(\scrR\) is a ring of subsets of \(X\) and \(\mu:\scrR \to \ExtReals^+_0\) is measure. Furthermore, let \(\mu^{\ast}\) be the outer measure of \(\mu\). If \(\func{\mu^{\ast}}{A} = 0\), then \(A \in \func{\calM}{\mu^{\ast}}\). ESpecially, for every subset \(B \subset A\), \(\func{\mu^{\ast}}{B} = 0\) and \(B \in \func{\calM}{\mu^{\ast}}\).
\end{lemma}

\begin{definition}
    A measure space \((X,\scrF,\mu)\) is \textbf{complete} if every subset of a null set, is in \(\scrF\) and is measure zero. 
\end{definition}

\begin{theorem}
    Every measure space \((X,\scrF,\mu)\) can be uniquely extended to a complete measure space.
\end{theorem}

Let \((X,\overline{\scrF}, \overline{\mu})\) be the extended complete measure described above. We shall investigate how \((X,\overline{\scrF},\overline{\mu})\) is related to Caratheodory extension. Firstly, consider the following covering lemma.
\begin{lemma}
    Suppose \((X,\scrF,\mu)\) is a measure space. For every \(E \subset X\), there exists a \(A \in \scrF\) such that \(E \subset C\) and \(\func{\mu^{\ast}}{E} = \func{\mu}{C}\).
\end{lemma}

\begin{theorem}
    Suppose \((X,\scrF,\mu)\) is a \(\sigma\)-finite measure space. If \((X,\overline{\scrF},\overline{\mu})\) is the completion and \((X,\func{\calM}{\mu^{\ast}},\mu^{\ast})\) is the Caratheodory extension of \((X,\scrF,\mu)\), then \(\overline{\mu} = \mu^{\ast}\) and \(\overline{\scrF} = \func{\calM}{\mu^{\ast}}\).
\end{theorem}

Let \(\mu\) be a measure on the ring \(\scrR\) and \(\scrF = \func{\sigma}{\scrR}\) and let \(\nu\) be the restriction of \(\mu^{\ast}\) to \(\scrF\).
\begin{theorem}
    For every \(A \subset X\), \(\func{\mu^{\ast}}{A} = \func{\nu^{\ast}}{A}\).
\end{theorem}
Therefore, if \((X,\scrF,\nu)\) is \(\sigma\)-finite, then its completion is the same as \((X,\func{\calM}{\mu^{\ast}},\mu^{\ast})\)

\subsection{Lebesgue measure}

\begin{example}
    In the case of Lebesgue measure \(\mu_L\) on \(\scrR_{\Leb}\), since it is a \(\sigma\)-finite measure, then its metric and Caratheodory extensions are equal. The restriction of \(\mu_L^{\ast}\) to \(\func{\calM}{\mu_L^{\ast}}\), is called the \textbf{Lebesgue measure} and it is denoted by \(\lambda_1 = \lambda\). The \(\sigma\)-field \(\func{\calM}{\mu_L^{\ast}}\) is called the \textbf{Lebesgue measurable sets} and it is denoted by \(\Lambda^1 = \Lambda\).
\end{example}

\begin{proposition}
    Every open and closed subset of \(\Reals^n\) is in \(\scrM\).
\end{proposition}

\begin{corollary}
    All countable unions and intersection of closed and open sets are measurable.
\end{corollary}

\begin{definition}
    The Borel sets, \(\scrB\), is the \(\sigma\)-field generated by \(\scrR_{\Leb}\).
\end{definition}

\begin{proposition}
    \(\scrB\) contains all intervals and open sets. Moreover, it is the smallest \(\sigma\)-ring containing the open sets.
\end{proposition}

\begin{theorem}
    If \(A \in \Lambda\), there exists a Borel set \(B \subset A\) such that \(\func{\lambda}{A - B} =0\). That is, \(A\) can be written as \(A = (A - B) \cup B\) where \(B\) is Borel set and \(\func{\lambda}{A - B} = 0\).
\end{theorem}

\begin{theorem}
    For each \(A \subset \Reals\) we have 
    \begin{equation*}
        \func{\lambda^{\ast}}{A} = \inf \set<\func{\lambda}{U}>{A \subset U, U \text{ is open}}
    \end{equation*}
\end{theorem}

\begin{corollary}
    If \(A \in \Lambda\) and if \(\epsilon > 0\) is given, then there exists a Borel set such that \(G \supset A\) and \(\func{\lambda}{G - A} < \epsilon\).
\end{corollary}

\begin{corollary}
    If \(A \in \Lambda\), there exsists a Borel set \(F \subset A\) with \(\func{\lambda}{A - F} < \epsilon\).
\end{corollary}

\begin{corollary}
    If \(\mu\) is a measure on \(\Lambda\) that for each Borel set \(B\), \(\func{\mu}{B} = \func{\lambda}{B}\), then \(\mu = \lambda\).
\end{corollary}

\begin{theorem}
    If \(E\) is a Lebesgue measurable set, then 
    \begin{equation*}
        \func{\lambda}{E} = \sup \set<\func{\lambda}{K}>{K \subset E, K \text{ is compact}}
    \end{equation*}
\end{theorem}

\begin{theorem}
    For each subset \(A \subset \Reals\) and \(c \in \Reals\), \(\func{\lambda^{\ast}}{A + c} = \func{\lambda^{\ast}}{A}\) and \(\func{\lambda^{\ast}}{cA} = \abs{c} \func{\lambda^{\ast}}{A}\). Moreover, if \(A\) is Lebesgue measurable, then \(A + c\) and \(cA\) are Lebesgue measurable as well.
\end{theorem}

\begin{theorem}
    There exists a non-Lebesgue measurable set in \(\Reals\).
\end{theorem}

\subsection{Finite signed measures}

\begin{definition}
    Suppose \((X,\scrF)\) is measurable space. The set function \(\nu: \scrF \to \Reals\) is a \textbf{finite signed measure}  if it is countably additive. That is, for every sequence of disjoint subsets \(\set{A_n}\), \(\sum \func{\nu}{A_n}\) is convergent and 
    \begin{equation*}
        \func{\nu}{\bigcup_{i = 1}^{\infty} A_n} = \sum_{i = 1}^n \func{\nu}{A_n}
    \end{equation*}
    Since the order of right hand side summation does not matter, then the series is absolutely convergent.
\end{definition}

\begin{proposition}
    \ 
    \begin{enumerate}
        \item \(\func{\nu}{\emptyset} = 0\).
        \item \(\nu\) is a finitely additive.
        \item If \(A,B \in \scrF\) and \(A \subset B\), then \(\func{\nu}{B - A} = \func{\nu}{B} - \func{\nu}{A}\).
        \item \(\nu\) is continuous from below at every \(E \in \scrF\).
    \end{enumerate}
\end{proposition}

\begin{definition}
    Let \(\nu\) be a finite signed measure on \(\scrF\). For each \(A \in \scrF\), the signed finite measure \(\nu_A\) is defined as 
    \begin{equation*}
        \func{\nu_A}{E} = \func{\nu}{A \cap E}
    \end{equation*}
\end{definition}

\begin{proposition}
    \(\nu_A\) is a measure if and only if for every subset \(F \subset A\) that \(F \in \scrF\), \(\func{\nu}{F} \geq 0\).
\end{proposition}

\begin{proposition}
    \ 
    \begin{enumerate}
        \item \(\func{\nu_{\emptyset}}{E} = 0\) for all \(E \in \scrF\).
        \item If \(A,B \in \scrF\) and \(A \cap B = \emptyset\), then \(\nu_{A \cup B} = \nu_A + \nu_B\).
        \item If \(A,B \in \scrF\) and \(A \subset B\), \(\nu_{B-A} = \nu_B - \nu_A\).
        \item If \(A,B \in \scrF\), \(\nu_{A \cap B} = (\nu_A)_B\).
        \item If \(A,B \in \scrF\), \(\nu_{A \cup B} + \nu_{A \cap B} = \nu_A + \nu_B\).
    \end{enumerate}
\end{proposition}

\begin{theorem}[Hann-Jordan decomposition]
    If \(\nu\) is a signed finite measure on a \(\sigma\)-field \(\scrF\), then there exists \(A \in \scrF\) such that \(\nu_A \geq 0\) and \(\nu_{A^c} \leq 0\) hence 
    \begin{equation*}
        \nu = \nu_A - (- \nu_{A^c})
    \end{equation*}
    That is, \(\nu\) is the difference of two finite measures.
\end{theorem}

\begin{proof}
    Let \(\scrN = \set<B \in \scrF>{\nu_B \leq 0}\). Then, \(\scrN\) is closed under finite and countable union. Moreover if \(B \in \scrN\) and \(E \in \scrF\), then \(B \cap E \in \scrN\). Consider the following lemma 
    \begin{lemma}
        The set \(\set<\func{\nu}{B}>{B \in \scrN}\) has an smallest element.
    \end{lemma}
    then do some more work.
\end{proof}



\section{Measure theoretic modeling}
%TODO: move the definition
\begin{definition}
    Let \(X\) be a set and \(\scrF\) a ring of subsets of \(X\).
    \begin{enumerate}
        \item \(\scrF\) is a field if \(X \in \scrF\).
        \item \(\scrF\) is a \(\sigma\)-field if \(X \in \scrF\) and \(\scrF\) is a \(\sigma\)-ring.
    \end{enumerate}
\end{definition}

\begin{definition}
    Let \(X\) be a set and \(\scrF\) be a field of subsets of \(X\). Suppose \(\mu\) is a measure defined on \(\scrF\). Then, \(\mu\) is a probability measure if \(\func{\mu}{X} = 1\). In this case, the triplet \((X,\scrF,\mu)\) is a called probability space.
\end{definition}

Let \(X\) be a sample of space of a probabilistic process. A measure theoretic model of the proccess is a \(\sigma\)-field \(\scrF\) of subsets of \(X\) and probability measure \(\mu\) defined on \(\scrF\). So that, for any ``plausible'' event \(E\) in \(X\), we have \(B_E \in \scrF\) and \(\prob{E} =\func{\mu}{B_E}\) where \(B_E\) is the set of points in \(X\) for which in \(E\) occurs.

\begin{definition}
    Given set \(B_1, B_2 , \dots\) in \(\scrF\), then 
    \begin{equation*}
        \set{B_i; \ \mathrm{ i.o.}} = \limsup B_n = \bigcap_{k = 1}^{\infty} \bigcup_{n \geq k} B_n
    \end{equation*}
\end{definition}

\begin{theorem}[First Borel-Cantelli lemma]
    Given a sequence \(B_1, B_2,\dots\) in \(\scrF\) define \(B = \limsup B_n\). Then, \(\sum_{i= 1}^{\infty} \func{\mu}{B_i} < \infty\) implies \(\func{\mu}{B} = 0\).
\end{theorem}

\begin{definition}
    Let \(X\) be a sample space with \(\sigma\)-field \(\scrF\) and probability measure \(\mu\). Two sets \(A_1,A_2 \in \scrF\) are \textbf{independent} if 
    \begin{equation*}
        \func{\mu}{A_1 \cap A_2} = \func{\mu}{A_1}\func{\mu}{A_2}
    \end{equation*}
    More generally, \(A_1, \dots, A_n\) are independent, if for any subset \(I\) of \(\Naturals_n\)
    \begin{equation*}
        \func{\mu}{\bigcap_{i \in I}A_i} = \prob{i \in I} \func{\mu}{A_i}
    \end{equation*}
    Furthermore, a countable collection of sets is independent if every finite subcollection is independent.
\end{definition}

\begin{theorem}[Second Borel-Cantelli lemma]
    Assume \((X,\scrF,\mu)\) is a probability space and let \(A_1,A_2,\dots\) be an independent collection of sets from \(\scrF\). Suppose that \(\sum_{i = 1}^{\infty} \func{\mu}{A_i} \) is not finite, then \(\func{\mu}{\limsup A_n} =1\).
\end{theorem}

\begin{lemma}
    Let \(A_1, A_2 , \dots \) be an independent collection of sets in \(\scrF\). Then, \(A_1^c, A_2^c \dots\) is an independent collection of set in \(\scrF\).
\end{lemma}
\chapter{Integeration}
\begin{definition}
    A \textbf{measure space} is a triplet \((X,\scrF,\mu)\) where \(\scrF\) is a \(\sigma\)-field of subsets of \(X\) and \(\mu\) is a measure defined on \(\scrF\). A \textbf{measurable space} is a pair \((X,\scrF)\).
\end{definition}
\section{Measurable functions}
Let \((X,\scrF)\) and \((Y,\scrS)\) be two measurable spaces. The function \(f:X\to Y\) is measurable if for all \(B \in \scrS\), \(\func{f^{-1}}{B} \in \scrF\). That is, the \(\sigma\)-algebra generated by \(f\), \(\func{\sigma}{f} = \set<\func{f^{-1}}{B}>{B \in \scrS}\) is a subset of \(\scrF\). Moreover, if \(\scrF = \func{\scrB}{X}\) and \(\scrS = \func{\scrB}{Y}\) are the Borel set of \(X\) and \(Y\), respectively, \(f\) is called a \textbf{Borel measurable} function. 

Let \(\ExtReals\) denote the set of the \textit{extended real numbers}, \(\Reals \cup \set{\pm \infty} = \clcl{-\infty}{+\infty}\). We may define addition and multiplication as follows 
\begin{enumerate}
    \item \(\forall a \in R, -\infty < a < \infty\).
    \item \(\forall a \in R, a + (\pm \infty) = \pm \infty\).
    \item \(\forall a \in \Reals^+, a(\pm \infty) = \pm \infty\).
    \item \((-1)(\pm \infty) = \mp \infty\).
\end{enumerate}
The extended Borels sets, \(\func{\scrB}{\ExtReals}\) are collection of subsets having the following form 
\begin{equation*}
    A, \quad A \cup \set{\pm \infty}, \quad A \cup \set{-\infty, + \infty}
\end{equation*}
where \(A\) is a Borel set. The extended Borel set make a \(\sigma\)-field. 

For the rest of this text, we may assume a measurable function \(f: X \to \ExtReals\) where \(\ExtReals\) is equipped with \(\func{\scrB}{\ExtReals}\). 

\begin{lemma}
    Suppose \(f: X \to \ExtReals\) is a function. The followings are equivalent 
    \begin{enumerate}
        \item \(f\) is measurable. 
        \item For all \(a \in \Reals\), \(\set<x \in X>{\func{f}{x} > a} \in \scrF\).
        \item For all \(a \in \Reals\), \(\set<x \in X>{\func{f}{x} \geq a} \in \scrF\).
        \item  For all \(a \in \Reals\), \(\set<x \in X>{\func{f}{x} < a} \in \scrF\).
        \item For all \(a \in \Reals\), \(\set<x \in X>{\func{f}{x} \leq a} \in \scrF\).
    \end{enumerate}
\end{lemma}

\begin{example}
    Let \(f:\Reals^n \to \Reals\) and \(\scrF = \scrM\) the Lebesgue measurable sets. If \(f\) is continuous, then \(f\) is measurable. 
\end{example}

Random variables are measurable functions from a measure space. 

\begin{theorem}
    If \(f\) and \(g\) are measurable functions, then \(\func{\max}{f,g}\) and \(\func{\min}{f,g}\) are also measurable. 
\end{theorem}

\begin{corollary}
    Suppose \(f\) is a measurable function, then 
    \begin{equation*}
        \func{f^+}{x} = \begin{cases}
            \func{f}{x} & \func{f}{x} \geq 0 \\
            0 & \func{f}{x} < 0
        \end{cases}
        \qquad 
        \func{f^-}{x} = \begin{cases}
            -\func{f}{x} & \func{f}{x} \leq 0 \\
            0 & \func{f}{x} > 0
        \end{cases}
    \end{equation*}
    are measurable function. Since \(f = f^+ - f^-\), then every function is the difference of two non-negative measurable functions.
\end{corollary}

\begin{definition}
    Let \(f_i\) be functions of \(X\) to \(\ExtReals\). Then 
    \begin{equation*}
        \func{\inf f_i}{x} = \inf \set{\func{f_i}{x}} \qquad \func{\sup f_i}{x} = \sup \set{\func{f_i}{x}}
    \end{equation*}
\end{definition}

\begin{theorem}
    If \(\set{f_i}\) are measurable functions, then \(\sup f_i\) and \(\inf f_i\) are measurable functions.
\end{theorem}

Furthermore, we may define \(\limsup\) and \(\liminf\) as follows
\begin{align*}
    \limsup f_i &= \lim_{n \to \infty} \sup_{i \geq n} f_i & \liminf f_i &= \lim_{n \to \infty} \inf_{i \geq n}\\
    &= \inf_n \sup_{i \geq n} f_i & \sup_n \inf_{i \geq n } f_i    
\end{align*}
\begin{corollary}
    If \(\set{f_i}\) is a collection of measurable functions, then \(\limsup f_i\) and \(\liminf f_i\) are measurable. 
\end{corollary}

\begin{corollary}
    Suppose \(\set{f_i}\) are measurable and converge pointwise to \(f\). Then, \(f\) is measurable.
\end{corollary}

\begin{remark}
    The restriction of a measurable function \(f:(X,\scrF)\to(Y,\scrS)\) to a measurable set \(A \in \scrF\), is measurable as well. Since 
    \begin{equation*}
        \func{f|_A^{-1}}{B} = A \cap \func{f^{-1}}{B} \in \scrF
    \end{equation*}
    Hence, if a sequence of measurable function \(f_n\) converge pointwise to \( f\) on a measurable set \(A\), then \(f:A \to Y\) is measurable.
\end{remark}

Measurablity of sum/multiplication/inverse of two measurable functions is requires some care. Particulary, to avoid situation like \(+\infty + (- \infty)\).

\begin{theorem}
    Let \(f_i : X \to \Reals\) be some measurable functions. Then, for a continuous function \(G:\Reals^n \to \Reals\), \(\func{G}{f_1, \dots , f_n}\) is measurable. 
\end{theorem}

\begin{definition}
    Suppose \(X\) and \(Y\) are two spaces equipped with their respective Borel sets, \(\func{\scrB}{X}\) and \(\func{\scrB}{Y}\). A function \(f:X \to Y\) is Borel measurable if for all \(S \in \func{\scrB}{Y}\), \(\func{f^{-1}}{S} \in \func{\scrB}{B}\).
\end{definition}

\begin{theorem}
    Every continuous function is Borel measurable.
\end{theorem}
\section{The Lebesgue Integral}
The measurable function \(s:X \to \Reals\) is a \textbf{step} function if it takes on only finite number of values. If the distinct values are \(c_1, \dots, c_n\) and \(E_i = \func{s^{-1}}{c_i}\), then
\begin{equation*}
    s = \sum_{i = 1}^n c_i \indicator{E_i}
\end{equation*}

\begin{theorem}
    \(s\) is a measurable if and only if \(E_i \in \scrF\) for \(i = 1, \dots, n\).
\end{theorem}

\begin{example}
    Let \(E \in \scrF\) then 
    \begin{equation*}
        \func{\indicator{E}}{x} = \begin{cases}
            1 & x \in E\\
            0 & x \notin E
        \end{cases}
    \end{equation*}
    is a step function. Furthermore, let \(s\) be a simple function that takes on values \(c_1, c_2,\dots, c_n\) and let \(E_i = \func{s^{-1}}{c_i}\) for \(i = 1, \dots, n\). Then, 
    \begin{equation*}
        s = \sum_{i = 1}^n c_i \indicator{E_i}
    \end{equation*}
\end{example}

\begin{definition}
    Let \(s:X\to\Reals\) be a (non-negative) simple function and let \(c_1, \dots,c_n\) be distinct non-zero values of \(s\) with \(E_i = \func{s^{-1}}{c_i}\). Let \(E \in \scrF\) and define the \textit{integral of \(s\) over \(E\) with respect to \(\mu\)} as the sum
    \begin{equation*}
        \int_E s  = \sum_{i = 1}^n c_i \func{\mu}{E \cap E_i}
    \end{equation*}
    Note that, the integral might be \(+\infty\) since \(\func{\mu}{E \cap E_i}\) might be \(+\infty\).
\end{definition}

\begin{proposition}
    Let \(s\) and \(r\) be simple non-negative functions and \(E \in \scrF\).
    \begin{enumerate}
        \item \(\int_E s + r  = \int_E s   + \int_E r \).
        \item \(\int_E c s  = c\int_E s  \) for \(c \geq 0\).
        \item If \(s \leq r\), then \(\int_E s  \leq  \int_E r \).
    \end{enumerate}
\end{proposition}

\begin{definition}
    Let \(f:X \to \ExtReals\) be a non-negative measurable function and \(E \in \scrF\). The \textit{integral of \(f\) over \(E\) with respect to \(\mu\)} is defined as 
    \begin{equation*}
        \int_E f \diffOperator \mu = \sup \set<\int_E s >{s\leq f, s\ \mathrm{is\ simple}}
    \end{equation*}
\end{definition}

Let \(s\) be a simple function. We need to check that the new definiton of integral is equivalent to the old one in the case of simple functions. That is 
\begin{equation*}
    \int_E s \diffOperator \mu =  \sum_{i = 1}^n c_i \func{\mu}{E \cap E_i}
\end{equation*}

To justify why simple functions are used to approximate consider the following. 

\begin{theorem}
    Let \(f\) be a non-negative measurable function. Then, there exists a sequence of non-negative simple functions 
    \begin{equation*}
        0 \leq s_1 \leq s_2 \leq \dots \leq f
    \end{equation*}
    such that \(s_i \to f\) pointwise. Moreover, if \(f\) is bounded, \(s_i \rightrightarrows f\).
\end{theorem}

\begin{proof}
    Fix \(n\) and divide the interval \(\clop{0}{n}\) to \(n2^n\) subinterval of length \(2^{-n}\). 
    \begin{equation*}
        I_{n,i} = \set{\dfrac{i-1}{n2^n} \leq x < \dfrac{i}{n2^n}} \qquad i = 1, \dots n2^n
    \end{equation*}
    Then let \(E_{n,i} = \func{f^{-1}}{I_{n,i}}\) and \(F_n = \func{f^{-1}}{\clcl{n}{+\infty}}\). Note that \(E_{n,i}\) and \(F\) are mutually disjoint and cover \(X\).
    \begin{equation*}
        \func{s_n}{x} = \sum_{i = 1}^{n2^n} \bracket{\dfrac{i - 1}{2^n}} \indicator{E_{n,i}} + n \indicator{F_n}
    \end{equation*}
    then \(s_n \leq f\) and \(s_n \leq s_{n+1}\) for all \(n\).
\end{proof}

In contrast to Riemann integral we approximate by dividing the range of the function. Removing the conditions on \(x\)-axis. Gives good approximation without \(f\) having to be continuous.

\begin{proposition}
    Let \(f\) and \(g\) be non-negative measurable functions and \(E,F \in \scrF\). Then 
    \begin{enumerate}
        \item If \(f \leq g\), then \(\int_E f \diffOperator \mu \leq \int_E g \diffOperator \mu \).
        \item If \(E \subset F\), then \(\int_E f \diffOperator \mu \leq \int_F f \diffOperator \mu \).
        \item If \(\func{\mu}{E} = 0\), then \(\int_E f \diffOperator \mu = 0\).
    \end{enumerate}
\end{proposition}

\begin{theorem}[Chebyshev]
    Let \(f\) be a non-negative measurable function and let \(E \in \scrF\) and \(c > 0\). Define \(E_c = \set<x \in E>{\func{f}{x} \geq c}\), then 
    \begin{equation*}
        \func{\mu}{E_c} \leq \dfrac{1}{c} \int_{E} f \diffOperator \mu 
    \end{equation*}
\end{theorem}

\begin{corollary}
    Let \(f\) be a non-negative measurable function with \(\int_E f \diffOperator \mu < \infty\), then
    \begin{equation*} 
        \func{\mu}{\set<x\in E>{\func{f}{x} = + \infty}} = 0
    \end{equation*}
\end{corollary}

\begin{definition}
    If a property holds on a set \(E \in \scrF\) except for a subset of zero measure, we say that the property holds \textbf{almost everywhere} on \(E\).
\end{definition}

\begin{corollary}
    Let \(f\) be a non-negative function and \(E \in \scrF\). 
    \begin{equation*}
        \int_E f \diffOperator \mu = 0 \implies f \equiv 0 \mathrm{\ \alev \ on \ } E
    \end{equation*}
\end{corollary}

\begin{theorem}
    Let \(f\) be a non-negative function and \(A_1, A_2, \dots \) pairwise disjoint from \(\scrF\). 
    \begin{equation*}
        \int_{\bigcup_{i = 1}^{\infty} A_i} f \diffOperator \mu = \sum_{i= 1}^{\infty} \int_{A_i} f \diffOperator \mu 
    \end{equation*}
\end{theorem}

We can use integrals to define measurse. \textit{Gaussian measure} \(\mu_G\) is defined on measurable subsets of \(\Reals\)
\begin{equation*}
    \func{\mu_G}{A} = \dfrac{1}{\sqrt{2\pi}} \int_{A} e^{-x^2/2} \diffOperator\mu_L
\end{equation*}
Moreover, \(\mu_G\) is a probability measure. 

\begin{corollary}
    Let \(f\) and \(g\) be a non-negative functions and \(E \in \scrF\). If \(f = g\) \alev on \(E\)
    \begin{equation*}
        \int_E f \diffOperator \mu = \int_E g \diffOperator \mu 
    \end{equation*}
\end{corollary}

\section{Further properties of integrals}
Let \(\set{f_i}\) be a sequence of measurable functions with 
\begin{equation*}
    0 \leq f_1 \leq f_2 \leq \dots 
\end{equation*}
Then, \(f = \lim_{n \to \infty} f_n\) exists and is measurable. 
\begin{lemma}
    Let \(f\) be a non-negative measurable function on \(X\) and let \(E_1, E_2,\dots \) be a sequence of sets in \(\scrF\) with \(E_1 \subset E_2 \subset \dots\) and \(E = \cup_i E_i\). Then 
    \begin{equation*}
        \int_E f \diffOperator \mu = \lim_{i \to \infty} \int_{E_i} f \diffOperator \mu 
    \end{equation*}
\end{lemma}

\begin{theorem}[Monotone convergence]\label{thm:monotoneConvergence}
    Let \(f\) and \(\set{f_i}\) be described as above. Then for \(E \in \scrF\)
    \begin{equation*}
        \int_E f \diffOperator \mu = \lim_{n \to \infty} \int_E f_n \diffOperator \mu 
    \end{equation*}
\end{theorem}

\begin{remark}
    Let \(f\) be a non-negative measurable function and \(s_n\) be the step functions from the construction. By \ref{thm:monotoneConvergence} 
    \begin{equation*}
        \int_E s_n \diffOperator \mu \to \int_E f \diffOperator \mu 
    \end{equation*}
\end{remark}

\begin{theorem}
    Suppose \(f\) and \(g\) are two non-negative measurable function, \(c > 0\), and \(E \in \scrF\) 
    \begin{enumerate}
        \item \(\int_E f+ g \diffOperator \mu = \int_E f \diffOperator \mu + \int_E g \diffOperator \mu\).
        \item \(\int_E cf \diffOperator \mu = c \int_E f \diffOperator \mu\).
    \end{enumerate}
\end{theorem}

\begin{corollary}
    Let \(\set{f_i}\) be non-negative measurable functions. Then, \(\sum f_i\) is a non-negative measurable function and 
    \begin{equation*}
        \int_E \sum_{i = 1}^{\infty} f_n \diffOperator\mu = \sum_{i = 1}^n \int_E f_n \diffOperator \mu 
    \end{equation*}
\end{corollary}

\begin{lemma}
    The following two conditions are equivalent 
    \begin{enumerate}
        \item \(\int_E \abs{f} \diffOperator \mu < +\infty\).
        \item \(\int_E f^+ \diffOperator \mu < +\infty \) and \(\int_E f^- \diffOperator \mu < + \infty\).
    \end{enumerate}
\end{lemma}
\begin{proof}
    \(\abs{f} = f^+ + f^-\).
\end{proof}

\begin{definition}
    A measurable function \(f\) is \textbf{integrable} over \(E\) if either of the conditions hold. In this case, \(f \in \func{\scrL}{\mu,E}\). If \(E = X\), then \(f \in \func{\scrL}{\mu}\). For \(f \in \func{\scrL}{\mu,E}\) 
    \begin{equation*}
        \int_E f \diffOperator \mu  = \int_E f^+ \diffOperator \mu - \int_E f^- \diffOperator \mu 
    \end{equation*}
\end{definition}

\begin{theorem}
    Suppose \(f,g \in \func{\scrL}{\mu,E}\) and \(c \in \Reals\) 
    \begin{enumerate}
        \item \(cf \in \func{\scrL}{\mu,E}\) and \(\int_E cd \diffOperator \mu = c \int_E f \diffOperator \mu\). 
        \item \(f +g \in \func{\scrL}{\mu,E}\) and \(\int_E f + g \diffOperator \mu = \int_E f \diffOperator \mu + \int_E g \diffOperator \mu\).
        \item If \(f \leq g\), then \(\int_E f\diffOperator \mu = \int_E g \diffOperator \mu\).
    \end{enumerate}
\end{theorem}

\begin{corollary}
    Let \(f \in \func{\scrL}{\mu,E}\), then 
    \begin{equation*}
        \abs{\int_E f \diffOperator \mu} \leq \int_E \abs{f} \diffOperator \mu 
    \end{equation*}
\end{corollary}

\begin{lemma}[Fatou's lemma]
    Assume \(f_1, f_2, \dots\) are non-negative measurable function and let \(f = \liminf f_n\)
    \begin{equation*}
        \int_E f \diffOperator \mu \leq \liminf \int_E f_n \diffOperator \mu 
    \end{equation*}
\end{lemma}

\begin{theorem}[Lebesgue dominated convergence]
    Let \(f_1, f_2, \dots \) be a sequence of measurable functions and let \(E \in \scrF\). Suppose the following assumptions hold 
    \begin{enumerate}
        \item \(\lim_{n \to \infty} \func{f_n}{x}\) exists for all \(x \in E\).
        \item There is a non-negative measurabe function \(g \in \func{\scrL}{\mu,E}\) with \(g \geq \abs{f_n}\) on \(E\) for all \(n\).
    \end{enumerate}
    Then, \(\func{f}{x} = \lim_{n \to \infty} \func{f_n}{x}\) is integrable and 
    \begin{equation*}
        \int_E \lim_{n \to \infty} f_n \diffOperator \mu = \lim_{n \to \infty} \int_E f_n \diffOperator \mu
    \end{equation*}
\end{theorem}

\begin{corollary}
    Let \(\set{f_i}\) be a sequence of functions in \(\func{\scrL}{\mu,E}\) with 
    \begin{equation*}
        \sum_{i = 1}^{\infty} \int_E \abs{f_n} \diffOperator \mu < + \infty 
    \end{equation*}
    Then, 
    \begin{enumerate}
        \item \(\sum f_n\) converges absolutely \alev on \(E\) and is integrable on \(E\).
        \item  
        \begin{equation*}
            \int_E \sum_{n = 1}^{\infty} f_n \diffOperator \mu = \sum_{n = 1}^{\infty }\int_E f_n \diffOperator\mu 
        \end{equation*} 
    \end{enumerate}
\end{corollary}

\section{Lebesgue integral vs Riemann integral}
\begin{theorem}
    Let \(f\) be a bounded Riemann integrable function on \(\clcl{a}{b}\) with Riemann integral \(\int_a^b \func{f}{x} \diffOperator x\). Then, \(f \in \func{\scrL}{\mu_L,\clcl{a}{b}}\) and 
    \begin{equation*}
        \int_a^b f \diffOperator x = \int_{\clcl{a}{b}} f \diffOperator \mu_L
    \end{equation*}
\end{theorem}

\section{Radon-Nikodym theorem}
Suppose \((X,\scrF,\mu)\) is a measure space and \(f:X \to \ExtReals\) is integrable. Define the \textbf{indefinite integer}, \(f\dot \mu:\scrF \to \Reals\), as 
\begin{equation*}
    \func{f \dot \mu}{E} = \int_E f \diffOperator\mu 
\end{equation*}

\begin{proposition}
    \ 
    \begin{enumerate}
        \item \(f,g : X \to \ExtReals\) are integerable, then \((f + g)\dot \mu = f\dot \mu + g \dot \mu\).
        \item If \(c \in \Reals\), then \((cf) \dot \mu = c(f\dot \mu)\).
        \item \(f \geq 0\) if and only if \(f \dot \mu \geq 0\).
        \item \(f \dot \mu = f^+ \dot \mu - f^- \dot \mu\).
        \item If \(E \in \scrF\) and \(\func{\mu}{E} = 0\), then \(\func{f \dot \mu}{E} = 0\).
        \item \(f \dot \mu\) is countably additive.
    \end{enumerate}
    Therefore, \(f \dot \mu\) is a finite sign measure and can be decomposed into 
    \begin{equation*}
        f \dot \mu = f^{+}\dot \mu - f^{-} \dot \mu
    \end{equation*}
\end{proposition}

\begin{definition}
    A measure \(\nu\) is \textbf{absolutely continuous} relative to a measure \(\mu\), denoted by \(v \ll u\) if 
    \begin{equation*}
        E \in \scrF, \ \func{\mu}{E} = 0 \implies \func{\nu}{E} = 0
    \end{equation*}
\end{definition}

Hence, \(f \dot u\) is absolutely continuous relative to \(\mu\).
\begin{lemma}
    Suppose \(\nu\) is finite signed measure on \(\scrF\) and \(E \in \scrF\) such that \(\func{\nu}{E} > 0\). Then, there exists a measurable subset \(G \subset E\) such that \(\nu_G \geq 0\) and  \(\func{v}{G} > 0\).
\end{lemma}

\begin{lemma}
    Suppose \(\mu\) and \(\nu\) are two finite measure on \((X,\scrF)\) such that \(\nu \ll \mu\) and \(\nu \neq 0\). Then, there exists a non-negative integrable function \(f\) such that \(f \dot \mu \leq \nu\) and \(f \dot \mu\). 
\end{lemma}

\begin{theorem}[Radon-Nikodym theorem] 
    Suppose \(\nu\) is finite signed measure and \(\mu\) is a finite measure on \((X,\scrF)\) such that \(\nu \ll \mu\). Then, there exists an integrable function \(f:X \to \ExtReals\) such that \(\nu = f \dot \mu\) and \(f\) is unique a.e.. 
\end{theorem}

\section{Fubini theorem}
Let \((X,\scrM,\mu)\) and \((Y,\scrN,\nu)\) be two measure spaces. Let \(X \times Y\) demote the space 
\begin{equation*}
    X \times Y = \set<(x,y)>{x \in X,y \in Y}
\end{equation*}

\begin{definition}
    \(A \times B \subset X \times Y\) is a \textbf{product set} if \(A \in \scrM\) and \(B \in \scrM\). The smallest \(\sigma\)-field in \(X \times Y\) containing all product sets \(A \times B\) is denoted by \(\scrM \otimes \scrN\). 
\end{definition}

\begin{definition}
    For \(E \subset X \times Y\) and fix \(x \in X\), let \(E_x = \set<y \in Y>{(x,y) \in E}\). \(E_x\) is called the \(x\)-slice of \(E\).
\end{definition}

\begin{proposition}
    IF \(E \in \scrM \otimes \scrN\), then \(E_x \in \scrN\).
\end{proposition}

\begin{corollary}
    LEt \(f: X \times Y \to \ExtReals\) be meaurable with respect to \(\scrM \otimes \scrN\). For fixed \(x_0 \in X\), define \(f_{x_0} : Y \to \Reals\) given by \(\func{f_{x_0}}{y} = \func{f}{x_0,y}\). Then, for eah \(x_0 \in X\), \(f_{x_0}\) is a measurable function on \(Y\).
\end{corollary}

Suppose \(X\) and \(Y\) are \(\sigma\)-finite. We now make a measure on \(\scrM \otimes \scrN\) using \(\mu\) and \(\nu\).

\begin{definition}
    Let \(Z\) ba set and let \(\scrS\) be a collection of subsets of \(Z\). \(\scrS\) is called a \(\lambda\)-system if the following three properties hold. 
    \begin{enumerate}[label = \(\lambda\)\arabic*.]
        \item \(Z \in \scrS\).
        \item If \(E_1 \subset E_2 \subset \dots\) is an increasing sequence with each \(E_n \in \scrS\), then 
        \begin{equation*}
            \bigcup_{i = 1}^{\infty} E_n \in \scrS
        \end{equation*}
        \item If \(E,F \in \scrS\) and \(E \subset F\), then \(F - E \in \scrS\).
    \end{enumerate}
\end{definition}

\begin{definition}
    Let \(\scrP\) be a collection of subsets of \(Z\). \(\scrP\) is called a \(\pi\)-system if the following  property holds. 
    \begin{enumerate}[label = \(\pi\)\arabic*.]

        \item If \(A,B \in \scrP\), then \(A \cap B \in \scrP\).
    \end{enumerate}
\end{definition}

\begin{theorem}[Dynkin \(\pi\)-\(\lambda\) theorem]
    If \(\scrS\) is a \(\lambda\)-system and \(\scrP\) is a \(\pi\)-system with \(\scrP \subset \scrS\), then the smallest \(\sigma\)-field containing \(\scrP\), \(\func{\sigma}{\scrP}\) is contained in \(\scrS\).
\end{theorem}

\begin{proposition}
    If \(E \in \scrM \otimes \scrN\) and \(\phi_E : X \to \Reals\) is defined by \(\func{\phi_E}{x} = \func{\nu}{E_x}\), then \(\phi_E\) is measurable.
\end{proposition}

\begin{definition}
    Let \(E \in \scrM \otimes \scrN\) and define 
    \begin{equation*}
        \func{\pi'}{E} = \int_X \func{\phi_E}{x}\diffOperator \mu 
    \end{equation*}
    to be the product measure on \(E\).
\end{definition}

\begin{proposition}
    \(\pi'\) is a measure.
\end{proposition}

Suppose instead of \(x\)-slices we used \(y\)-slices and denoted the measure by \(\pi''\). 
\begin{theorem}[Fubini, version 1]
    \begin{equation*}
        \pi' = \pi''
    \end{equation*}
\end{theorem}

\begin{definition}
    The measure \(\pi' = \pi''\) is denoted by \(\mu \times \nu\) and is called the product measure on \(\scrM \otimes \scrN\).
\end{definition}

\begin{example}
    Let \(X = Y = \Reals\) and \(\scrM = \scrN = \func{\scrB}{\Reals}\). Also, let \(\mu = \nu =\mu_L\) the Lebesgue measure on \(\Reals\). We claim that, \(\scrM \otimes \scrN = \func{\scrB}{\Reals^2}\) and \(\mu \times \nu = \mu^2_L\), the Borel sets and Lebesgue measure in \(\Reals^2\).
\end{example}

\begin{example}
    We can show that 
    \begin{equation*}
        \func{\scrB}{\Reals^n} \otimes \func{\scrB}{\Reals^m} = \func{\scrB}{\Reals^{m+n}} 
    \end{equation*} 
    and \(\mu^{m}_L \times \mu^{n}_L = \mu^{m+n}_L\).
\end{example}

\begin{theorem}[Fubini, version 2]
    Let \(f:X \times Y \to \Reals\) be a non-negative measurable function. Then 
    \begin{enumerate}
        \item For each \(x_0 \in X\), \(\func{f}{x_0,y}\) is a measurable function of \(y\).
        \item For each \(y_0 \in X\), \(\func{f}{x,y_0}\) is a measurable function of \(x\).
        \item \(\int_Y \func{f}{x,y} \diffOperator \nu\) is a measurable function of \(x\).
        \item \(\int_X \func{f}{x,y} \diffOperator \mu\) is a measurable function of \(y\).
        \item 
        \begin{equation*}
            \int_{X \times Y} \func{f}{x,y} \diffOperator \mu \times \nu = \int_X \int_Y \func{f}{x,y} \diffOperator \nu \diffOperator \mu =  \int_Y \int_X \func{f}{x,y} \diffOperator \mu \diffOperator \nu
        \end{equation*}
    \end{enumerate}
\end{theorem}

\begin{theorem}[Fubini, version 2]
    Let \(f:X \times Y \to \Reals\) be an integrable function. Then 
    \begin{enumerate}
        \item For almost all \(x \in X\), \(\func{f}{x,y}\) is a integrable function of \(y\).
        \item For almost all \(y \in X\), \(\func{f}{x,y}\) is a integrable function of \(x\).
        \item \(\int_Y \func{f}{x,y} \diffOperator \nu\) is equal \alev to an integrable function on \(X\).
        \item \(\int_X \func{f}{x,y} \diffOperator \mu\) is equal \alev to an integrable function on \(Y\).
        \item 
        \begin{equation*}
            \int_{X \times Y} \func{f}{x,y} \diffOperator \mu \times \nu = \int_X \int_Y \func{f}{x,y} \diffOperator \nu \diffOperator \mu =  \int_Y \int_X \func{f}{x,y} \diffOperator \mu \diffOperator \nu
        \end{equation*}
    \end{enumerate} 
\end{theorem}
\section{Random variables, expectation values, and indepedence}
\end{document}