\documentclass[12pt]{book}
\usepackage[a4paper,bindingoffset=0.2in,%
            left=0.75in,right=0.75in,top=1in,bottom=1in,%
            footskip=.25in]{geometry}
\usepackage{fancyhdr}
\setlength{\headheight}{15.2pt}
\usepackage[utf8]{inputenc}
\pagestyle{fancy}

\renewcommand{\chaptermark}[1]{\markboth{\thechapter.\ #1}{}}
\renewcommand{\sectionmark}[1]{\markright{\thesection\ #1}}
\fancyhead[LE,RO]{\textbf{\thepage}}
\fancyhead[LO]{\textbf{\rightmark}}
\fancyhead[RE]{\textbf{\leftmark}}
\fancyfoot{}
\fancypagestyle{plain}
{
    \fancyhf{}
}


\usepackage{amsmath}
\usepackage{amssymb}
\usepackage{hyperref}
\usepackage{mathtools}
\usepackage{xcolor}
\usepackage{enumitem}
\usepackage[ruled,noline]{algorithm2e}

\usepackage{common}
\usepackage{english-theorems}
\setcounter{tocdepth}{1}
\newcommand{\unit}[1]{\ensuremath{\mathrm{#1}}}
\allowdisplaybreaks
\begin{document}
\tableofcontents
\clearpage
\ifodd\value{page}\else
\thispagestyle{empty}
\fi
\part{Digital Communication}
\chapter{Frequency Domain Analysis}
\section{Fourier Series}
For a periodic signal \(\func{x}{t}\):
\begin{align*}
    \func{x_{\pm}}{t} = \sum_{n = -\infty}^{\infty} x_n e^{2\pi j \frac{n}{T_0} t} & & x_n = \dfrac{1}{T_0} \int_{T_0} \func{x}{t} e^{-2 \pi j \frac{n}{T_0} t} \diffOperator t
\end{align*}
and 
\begin{equation*}
    \func{x_{\pm}}{t} = \begin{cases*}
        \func{x}{t} & \(x\)  \text{ is continuous at } \(t\)\\
        \dfrac{\func{x}{t^+} + \func{x}{t^-}}{2} & \(x\) \text{ is discontinuous at } \(t\)
    \end{cases*}
\end{equation*}
for angular frequency \(\omega_0 = 2\pi f_0\):
\begin{align*}
    \func{x_{\pm}}{t} = \sum_{n = -\infty}^{\infty} x_n e^{ j n \omega_0 t} & & x_n = \dfrac{\omega_0}{2\pi} \int_{T_0} \func{x}{t} e^{- j n \omega_0 t} \diffOperator t
\end{align*}
\(f_0 = \frac{1}{T_0}\) is called the \textbf{fundamental frequency} and its \(n_{\cardinalTH}\) is called the \textbf{\(n_{\cardinalTH}\) harmonic}.
\section{Fourier Transform}
For non-periodic signals \(\func{x}{t}\):
\begin{align*}
    \func{X}{f} &= \int_{-\infty}^{\infty} \func{x}{t} e^{-j 2\pi f t} \diffOperator t &  \func{x_{\pm}}{t} &= \int_{-\infty}^{\infty} \func{X}{f} e^{j 2\pi f t}\diffOperator f\\
    \func{X}{f} &= \Fourier{\func{x}{t}}  & \func{x_{\pm}}{t} &= \InvFourier{\func{X}{f}}\\
    \func{X}{\omega} &= \int_{-\infty}^{\infty} \func{x}{t} e^{-j \omega t} \diffOperator t & \func{x_{\pm}}{t} &= \dfrac{1}{2\pi} \int_{-\infty}^{\infty} \func{X}{f} e^{j \omega t}\diffOperator \omega
\end{align*}
\(\func{X}{f}\) is called the \textbf{spectrum} of \(\func{x}{t}\), or the \textbf{voltage spectrum}. From the relationship between the inverse Fourier transform of Fourier transform of a signal we define
\begin{equation*}
    \func{\delta}{t} = \int_{-\infty}^{\infty} e^{2\pi j f t}\diffOperator f = \dfrac{1}{2\pi} \int_{-\infty}^{\infty} e^{j \omega t}\diffOperator f
\end{equation*}
That is, all frequencies in \(\func{\delta}{t}\) are with unit magnitude and zero phase.
\begin{align*}
    \func{\delta}{t} = \InvFourier{1} & & \func{\delta}{f} = \Fourier{1}
\end{align*}
\section{Properties of Fourier transform}
\begin{description}
    \item[Linearity.] For two signals \(\func{x}{t}\) and \(\func{y}{t}\) and complex constants \(a\) and \(b\)
    \begin{align*}
        \Fourier{a \func{x}{t} + b \func{y}{t}} &= \int_{-\infty}^{\infty} \bracket{a \func{x}{t} + b \func{y}{t}} e^{-2\pi j f t} \diffOperator t \\
        &= a  \int_{-\infty}^{\infty}  \func{x}{t} e^{-2\pi j f t} \diffOperator t + b \int_{-\infty}^{\infty} \func{y}{t} e^{-2\pi j f t} \diffOperator t\\
        &= a \Fourier{\func{x}{t}} + b \Fourier{\func{y}{t}}
    \end{align*} 
    \item[Duality.] For any signal \(\func{x}{t}\)
    \begin{equation*}
        \func{x}{f} = \Fourier{\func{\Fourier{\func{x}{t}}}{-\omega}}
    \end{equation*} 
    since 
    \begin{align*}
        \Fourier{\func{\Fourier{\func{x}{t}}}{-\omega}} &= \int_{-\infty}^{\infty} \func{\Fourier{\func{x}{t}}}{-\omega} e^{2 \pi j f \omega } \diffOperator \omega \\
        &= \func{\InvFourier{\Fourier{\func{x}{t}}}}{f} \\
        &= \func{x}{f}
    \end{align*}
    \item [Time shift.] A shift of \(t_0\) in time domain causes a phase shift in the frequency domain.
    \begin{align*}
        \Fourier{\func{x}{t - t_0}} &= \int_{-\infty}^{\infty} \func{x}{t - t_0} e^{-2 \pi j f t} \diffOperator t\\
        &= \int_{-\infty}^{\infty} \func{x}{t} e^{-2 \pi j f (t + t_0)} \diffOperator t \\
        &= e^{-2 \pi j f t_0 } \Fourier{\func{x}{t}}
    \end{align*}
    \item [Scaling. ] Suppose \(a \neq 0\) is real
    \begin{align*}
        \Fourier{\func{x}{at}} &= \int_{-\infty}^{\infty} \func{x}{at} e^{-2 \pi j f t} \diffOperator t\\
        &= \dfrac{1}{a} \func{\sign}{a} \int_{-\infty}^{\infty} \func{x}{t} e^{-2 \pi j f \frac{t}{a}} \diffOperator t \\
        &= \dfrac{1}{\abs{a}} \Fourier{\frac{f}{a}}
    \end{align*}
    \item [Convolution.] For two signals \(\func{x}{t}\) and \(\func{y}{t}\)
    \begin{align*}
        \Fourier{\func{x}{t} \ast \func{y}{t}} &= \int_{-\infty}^{\infty} \bracket{\func{x}{t} \ast \func{y}{t}} e^{-2 \pi j f t} \diffOperator t\\
        &= \int_{-\infty}^{\infty} \int_{-\infty}^{\infty} \func{x}{\tau}e^{-2 \pi j f \tau} \func{y}{t- \tau}  e^{-2 \pi j f (t -\tau)} \diffOperator \tau \diffOperator t\\
        &= \int_{-\infty}^{\infty} \int_{-\infty}^{\infty} \func{x}{\tau}e^{-2 \pi j f \tau} \func{y}{t- \tau}  e^{-2 \pi j f (t -\tau)}\diffOperator t \diffOperator \tau \\
        &= \int_{-\infty}^{\infty}  \func{x}{\tau}e^{-2 \pi j f \tau}  \Fourier{\func{y}{t}} \diffOperator \tau\\
        &= \Fourier{\func{x}{t}} \Fourier{\func{y}{t}}
    \end{align*}
    \item [Parseval's property.]  For two signals \(\func{x}{t}\) and \(\func{y}{t}\) with Fourier transform \(\func{X}{f}\) and \(\func{Y}{f}\)
    \begin{equation*}
        \int_{-\infty}^{\infty} \func{x}{t} \overline{\func{y}{t}} \diffOperator t = \int_{-\infty}^{\infty} \func{X}{f} \overline{\func{Y}{f}} \diffOperator f
    \end{equation*}
    since 
    \begin{align*}
        \int_{-\infty}^{\infty} \func{X}{f} \overline{\func{Y}{f}} \diffOperator f &= \int_{-\infty}^{\infty} \bracket{\int_{-\infty}^{\infty} \func{x}{t} e^{-2\pi j f t} \diffOperator t} \overline{\bracket{\int_{-\infty}^{\infty} \func{y}{t} e^{-2\pi j f t} \diffOperator t} } \diffOperator f \\
        &= \int_{-\infty}^{\infty} \int_{-\infty}^{\infty} \int_{-\infty}^{\infty}  \func{x}{t} e^{-2\pi j f t} \overline{\func{y}{\tau}} e^{2\pi j f \tau} \diffOperator \tau \diffOperator t \diffOperator f\\
        &= \int_{-\infty}^{\infty} \int_{-\infty}^{\infty} \int_{-\infty}^{\infty}  \func{x}{t} \overline{\func{y}{\tau}} e^{2\pi j f (\tau - t)} \diffOperator f \diffOperator \tau \diffOperator t\\
        &= \int_{-\infty}^{\infty}  \int_{-\infty}^{\infty}  \func{x}{t} \overline{\func{y}{\tau}} \func{\delta}{\tau  - t}  \diffOperator \tau \diffOperator t\\
        &= \int_{-\infty}^{\infty} \func{x}{t} \overline{\func{y}{t}} \diffOperator t\\
    \end{align*}
    \item[Rayleigh's propery.]  For any signal \(\func{x}{t}\) with Fourier transform \(\func{X}{f}\)
    \begin{equation*}
        \int_{-\infty}^{\infty} \abs{\func{x}{t}}^2 \diffOperator t = \int_{-\infty}^{\infty} \abs{\func{X}{f}}^2 \diffOperator f
    \end{equation*}
    \item[Autocorrelation.] The time autocorrelation of the signal \(\func{x}{t}\) is defined by 
    \begin{equation*}
        \func{R_{x}}{\tau} = \int_{-\infty}^{\infty} \func{x}{t} \overline{\func{x}{t - \tau}} \diffOperator t = \func{x}{t} \ast \overline{\func{x}{-t}}
    \end{equation*} 
    Then, 
    \begin{equation*}
        \Fourier{\func{R_x}{\tau}} = \abs{\func{X}{f}}^2
    \end{equation*}
    \item[Differentiation.]
    \begin{equation*}
        \Fourier{\dfrac{\diffOperator}{\diffOperator t} \func{x}{t}} = 2\pi j \Fourier{\func{x}{t}}
    \end{equation*}
    \item[Integration.]
    \begin{equation*}
        \Fourier{\int_{-\infty}^t \func{x}{\tau} \diffOperator \tau}=\dfrac{\func{X}{f}}{2\pi j f} + \dfrac{1}{2} \func{X}{0} \func{\delta}{f}
    \end{equation*}
    \item[Moments]
    \begin{equation*}
       \int_{-\infty}^{\infty} t^n \func{x}{t} \diffOperator t =\bracket{\dfrac{j}{2\pi}}^n \evaluate{\dfrac{\diffOperator^n}{\diffOperator f^n} \func{X}{f}}_{f = 0}
    \end{equation*}
    %TODO: uncertainty principle Papoulis Chapter 8.2
\end{description}

\section{Power and Energy}
Define 
\begin{align*}
    \calE_x = \int_{-\infty}^{\infty} \abs{\func{x}{t}}^2 \diffOperator t & & \calP_x = \lim_{T \to \infty} \dfrac{1}{T} \int_{-\frac{T}{2}}^{\frac{T}{2}} \abs{\func{x}{t}}^2 \diffOperator t
\end{align*}
A signal is \textbf{energy-type} if \(\calE_x < + \infty\) and it is \textbf{power-type} if \(0 < \calP_x < +\infty\). A signal can not be both, but it can be neither.
\begin{remark}
    Average power is expressed in units of \(\unit{dBm}\) or \(\unit{dBw}\) as 
    \begin{align*}
        (S)_{\unit{dBw}} &= 10 \log_{10} (S)_{\unit{watts}}\\
        (S)_{\unit{dBm}} &= 10 \log_{10} (S)_{\unit{milli watts}}
    \end{align*}
\end{remark}
\subsection{Energy-type}
Let \(\func{x}{t}\) be a energy-type signal. The \textbf{autocorrelation} of \(\func{x}{t}\) is 
\begin{align*}
    \func{R_x}{\tau} &= \func{x}{\tau} \ast \overline{\func{x}{-\tau}} \\
    &= \int_{-\infty}^{\infty} \func{x}{t} \overline{\func{x}{t - \tau}} \diffOperator t\\
    \implies \calE_x &= \func{R_x}{0}
\end{align*}
By Rayleigh's property 
\begin{equation*}
    \calE_x = \int_{-\infty}^{\infty} \abs{\func{X}{f}}^2 \diffOperator f
\end{equation*}
The Fourier transform exists for 
The \textbf{energy spectral density} \(\func{\calG}{f} = \Fourier{\func{R_x}{\tau}} = \abs{\func{X}{f}}^2\), represent energy per hertz of bandwidth. 
\subsection{Power-type}
Let \(\func{x}{t}\) be a power type signal. The \textbf{time average autocorrelation} function 
\begin{align*}
    \func{R_{x}}{\tau} &= \lim_{T \to \infty} \int_{-\frac{T}{2}}^{\frac{T}{2}} \func{x}{t} \overline{\func{x}{t - \tau}} \diffOperator t\\
    \implies \calP_x &= \func{R_x}{0}
\end{align*}
The \textbf{power spectral density} \(\func{\calS}{f} = \Fourier{\func{R_x}{\tau}}\) and 
\begin{equation*}
    \calP_x = \int_{-\infty}^{\infty} \func{\calS}{f} \diffOperator f
\end{equation*}
\begin{remark}
    The power spectral density does not uniquely determine the signal. As it only retains the magnitude information and all phase information is lost.
\end{remark}
Suppose \(\func{x}{t}\) is a power-type signal passing through a filter with impluse response \(\func{h}{t}\):
\begin{align*}
    \func{y}{t} &= \func{x}{t} \ast \func{h}{t}\\
    \func{R_y}{\tau} &= \lim_{T \to \infty} \int_{-\frac{T}{2}}^{\frac{T}{2}} \func{y}{t} \overline{\func{y}{t - \tau}} \diffOperator t\\
    &= \lim_{T \to \infty} \int_{-\frac{T}{2}}^{\frac{T}{2}}\bracket{\int_{-\infty}^{\infty} \func{h}{u} \func{x}{t - u} \diffOperator u} \bracket{\int_{\infty}^{\infty} \overline{\func{h}{v}} \overline{\func{x}{t - \tau - v}} \diffOperator v }  \diffOperator t\\
    &= \lim_{T \to \infty} \int_{-\infty}^{\infty} \int_{\infty}^{\infty} \int_{-\frac{T}{2}}^{\frac{T}{2}}  \func{h}{u} \overline{\func{h}{v}}  \func{x}{t - u}  \overline{\func{x}{t - \tau - v}}  \diffOperator t \diffOperator u \diffOperator v\\
    &= \int_{-\infty}^{\infty} \int_{\infty}^{\infty}  \func{h}{u} \overline{\func{h}{v}} \lim_{T \to \infty}  \int_{-\frac{T}{2} + u}^{\frac{T}{2} +u} \func{x}{w}   \overline{\func{x}{w + u - \tau - v}}  \diffOperator w \diffOperator u \diffOperator v\\
    &= \int_{-\infty}^{\infty} \int_{\infty}^{\infty}  \func{h}{u} \overline{\func{h}{v}}\func{R_x}{v + \tau - u} \diffOperator u \diffOperator v\\
    &=  \int_{-\infty}^{\infty}   \bracket{\func{R_x}{v + \tau }  \ast \func{h}{v + \tau}} \overline{\func{h}{v}}\diffOperator u \diffOperator v\\
    &= \func{R_x}{\tau }  \ast \func{h}{\tau} \ast \overline{\func{h}{-\tau}}
\end{align*}
Which implies that 
\begin{equation*}
    \func{\calS_y}{f} = \func{\calS_x}{f} \func{H}{f} \overline{\func{H}{f}} = \func{\calS_x}{f} \abs{\func{H}{f}}^2
\end{equation*}
\section{Sampling of bandlimited signals}
\(f_s = 2W\) is the \textbf{Nyquist rate} and \(f_s -  2W\) is \textbf{guard band}.
\section{Bandpass signal}
A \textbf{bandpass signal} has non-zero frequencies around a small neighborhood of some high frequency \(f_0\). That is, \(\func{X}{f} = 0\) for \(\abs{f - f_0} \geq W\) where \(W < f_0\). A \textbf{bandpass system} passes frequencies around some \(f_0\) or equivalently, the impluse response is a bandpass signal. \(f_0\) is called the \textbf{central frequency} even tho it might not be the center of signal's bandwidth. 
\subsection{Analysis of monochromatic signals}
\textbf{Monochromatic} signals are bandpass with \(W = 0\). 
\begin{equation*}
    \func{x}{t} = A \func{\cos}{2\pi f_0 t + \theta}
\end{equation*}
The \textbf{phasor} is defined as \(\hat{X} = A e^{j\theta}\). Consider an LTI system with impluse response \(\func{H}{f}\). Then, the phasor of the output of signal \(\func{x}{t}\) is \(\hat{Y} = A \func{H}{f_0} e^{j\theta}\) and the frequency of the output signal is the same, namely \(f_0\). To obtain the phasor of the input consider the signal 
\begin{align*}
    \func{z}{t} &= A e^{2\pi j f_0 t + j\theta }\\
    &= A \func{\cos}{2\pi f_0 t + \theta} + jA \func{\sin}{2\pi f_0 t + \theta} \\
    &= \func{x}{t} + j \func{x_q}{t} = \func{x}{t} + j \func{x}{t - \frac{\pi}{2}}
\end{align*}
where \(\func{x_q}{t}\) is a \(90^{\circ}\) phase shift version of the original signal-- \(q\) stands for \textit{quadrature}. Then, 
\begin{equation*}
    \hat{X} = \func{z}{t} e^{-2\pi j f_0 t}
\end{equation*}
Note that, \(\func{Z}{f}\) can be obtained from \(\func{X}{f}\) by deleting the negative frequencies and multiplying the positive frequencies by a factor of two. 

\subsection{Analysis of a general bandpass signal}
For a general bandpass signal, let \(\func{Z}{f}\) be the signal obtained from deleting the negative frequencies of \(\func{X}{f}\) and multiplying the positive frequencies by a factor of two. That is, 
\begin{equation*}
    \func{Z}{f} = 2 \func{U_{-1}}{f} \func{X}{f}
\end{equation*}
where \(\func{U_{-1}}{f}\) is the Heaviside step function. \(\func{z}{t}\) is called the \textbf{analytic signal corresponding to \(\func{x}{t}\)} or the  \textbf{pre-envelope of \(\func{x}{t}\)}.
 The inverse Fourier of \(\func{U_{-1}}{f}\) is calculated as follows 
\begin{align*}
    \InvFourier{\func{U_{-1}}{f}} &= \func{\Fourier{\func{U_{-1}}{-\tau}}}{t}\\
    &= \func{\Fourier{1 - \func{U_{-1}}{\tau}}}{t}\\
    &= \func{\delta}{t} - \bracket{\dfrac{1}{2\pi j t} + \dfrac{1}{2}  \func{\delta}{t}}\\
    &= \dfrac{1}{2} \func{\delta}{t} - \dfrac{1}{2\pi j t}\\
    &= \dfrac{1}{2} \func{\delta}{t} + \dfrac{j}{2\pi t}
\end{align*}
Therefore, 
\begin{align*}
    \func{z}{t} &= \func{x}{t} \ast \bracket{ \func{\delta}{t} + \dfrac{j}{\pi t}}\\
    &= \func{x}{t} + j \func{x}{t} \ast \dfrac{1}{\pi t}\\
    &= \func{x}{t} + j \func{x'}{t}
\end{align*}
\(\func{x'}{t}\) is called the \textbf{Hilbert transform of \(\func{x}{t}\)}. Hilbert transform, as derived below, is equivalent to a \(- \frac{\pi}{2}\) shift for positive frequencies and a \(\frac{\pi}{2}\) shift for negative frequencies. 
\begin{equation*}
    \Fourier{\dfrac{1}{\pi t}} = -j \func{\sign}{f} = e^{-j \frac{\pi}{2} \func{\sign}{f}}
\end{equation*}
\(\func{H}{f} = -j \func{\sign}{f}\) is called the \textbf{quadrature filter}. Then, consider the signal \(\func{x_l}{t} = \func{z}{t} e^{-2\pi j f_0 t}\) or equivalently \(\func{X_l}{f} = \func{Z}{f + f_0}\) wher \(f_0\) is the centeral frequency of \(\func{x}{t}\). \(\func{x_l}{t}\) is called the \textbf{lowpass representation of bandpass signal \(\func{x}{t}\)}. In general \(\func{x}{t}\) is a complex-valued signal, hence we can decompose it into real and imaginary parts 
\begin{equation*}
    \func{x_l}{t} = \func{x_c}{t} + j \func{x_s}{t}
\end{equation*}
\(\func{x_c}{t}\) is called the \textbf{in-phase} and \(\func{x_s}{t}\) is called the the \textbf{quadrature} components of \(\func{x}{t}\). Then, 
\begin{align*}
    \func{z}{t} &= \func{x_l}{t} e^{2\pi j f_0 t}\\
    &= \bracket{\func{x_c}{t} \func{\cos}{2 \pi f_0 t} - \func{x_s}{t} \func{\sin}{2 \pi f_0 t}} + j \bracket{\func{x_c}{t} \func{\sin}{2 \pi f_0 t} + \func{x_s}{t} \func{\cos}{2 \pi f_0 t}} 
\end{align*}
hence 
\begin{align*}
    \func{x}{t} &= \func{x_c}{t} \func{\cos}{2 \pi f_0 t} - \func{x_s}{t} \func{\sin}{2 \pi f_0 t}\\
    \func{x'}{t} &= \func{x_c}{t} \func{\sin}{2 \pi f_0 t} + \func{x_s}{t} \func{\cos}{2 \pi f_0 t}
\end{align*}
these two equations are called the \textbf{bandpass to lowpass transformations}.

Define the \textbf{envelope} of \(\func{x}{t}\), \(\func{V}{t}\), as 
\begin{equation*}
    \func{V}{t} = \sqrt{\bracket{\func{x_c}{t}}^2 + \bracket{\func{x_s}{t}}^2}
\end{equation*}
and the \textbf{phase} of \(\func{x}{t}\), \(\func{\Theta}{t}\), as 
\begin{equation*}
    \func{\Theta}{t} = \arctan \dfrac{\func{x_s}{t}}{\func{x_c}{t}}
\end{equation*}
Then, 
\begin{align*}
    \func{x_l}{t} &= \func{V}{t} e^{j \func{\Theta}{t}}\\
    \func{z}{t} &= \func{V}{t} e^{2\pi j f_0 t +  j\func{\Theta}{t}}\\
    \func{x}{t} &= \func{V}{t} \func{\cos}{2\pi f_0 t + \func{\Theta}{t}}\\
    \func{x'}{t} &= \func{V}{t} \func{\sin}{2\pi f_0 t + \func{\Theta}{t}}
\end{align*}

\chapter{Random Signal Theory}
\section{Introduction}
Marcum \(Q\) function 
\begin{equation*}
    \func{Q}{y} = \dfrac{1}{\sqrt{2\pi}} \int_{y}^{\infty} \func{\exp}{-\dfrac{z^2}{2}}
\end{equation*}
and if \(X \sim \func{\NormalDist}{\mu, \sigma^2}\) then 
\begin{equation*}
    \prob{X > a} = \func{Q}{\dfrac{a - \mu}{\sigma}}
\end{equation*}
We have the following upper bounds 
\begin{equation*}
    \func{Q}{x} \leq \dfrac{1}{2}e^{-x^2/2}
\end{equation*}
and 
\begin{equation*}
    \func{Q}{x} < \dfrac{1}{\sqrt{2\pi} x}e^{-x^2 / 2}
\end{equation*}
for all \(x \geq 0\). For lower bound 
\begin{equation*}
    \func{Q}{x} > \dfrac{1}{\sqrt{2 \pi} x} \bracket{1 - \dfrac{1}{x^2}}e^{-x^2 / 2}
\end{equation*}

\(\bar{X}\) has multivariate Gaussian distribution with mean \(\mu\) and covariance matrix \(\Sigma\)
\begin{equation*}
    \func{f_{\bar{X}}}{\bar{x}} = \dfrac{1}{\sqrt{\bracket{2\pi}^n \det \Sigma}} \func{\exp}{- \dfrac{1}{2} (\bar{x} - \mu)^T \Sigma^{-1} (\bar{x} - \mu)}
\end{equation*}

\section{Random Process}
A set of indexed random variables \(\set{X_t}_{t \in T}\) is a random process. We denote a random process as \(\func{X}{t,\omega}\) where \(\omega \in \Omega\) and \(t \in T = \Reals\).  For a specific \(\omega_0\), \(\func{X}{t,\omega_0} = \func{x_0}{t}\) is a time function called \textbf{member function}, \textbf{sample function}, or a \textbf{realization function}. The totality of all sample functions is called an \textbf{ensemble}. For a specific \(t_0\), \(\func{X}{t_0,\omega}\) is a random variable.

\begin{definition}
    A process \(\func{X}{t}\) is described by its \(M_{\cardinalTH}\) \textbf{order statistics} if for all \(m \leq M\) and all \((t_1, \dots, t_m) \in \Reals^m\) the joint PDF of \((\func{X}{t_1} , \dots , \func{X}{t_n})\) is given.
\end{definition}
\subsection{Statistical averages}
The mean of an stochastic process 
\begin{equation*}
    \func{\mu_X}{t} = \expected{\func{X}{t}}
\end{equation*}
The autocorrelation of an stochastic process
\begin{equation*}
    \func{R_{XX}}{t_1,t_2} = \expected{\func{X}{t_1} \func{X}{t_2}}
\end{equation*}
Statistical averages are also called ensemble averages.
\subsection{Stationarity in wide sense}
A stochastic process with constants mean and time invariant autocorrelation is called \textbf{stationary in wide sense}.
\begin{equation*}
    \func{\mu_X}{t} = K, \qquad \func{R_{XX}}{t_1 + t, t_2 + t}=  \func{R_{XX}}{t_1, t_2}
\end{equation*}
for all \(t_1,t_2,t\).
\begin{definition}
    A random process \(\func{X}{t}\) with mean \(\func{\mu_X}{t}\) and autocorrelation \(\func{R_{XX}}{t + \tau,t}\) is called \textbf{cyclostationary} if both the mean and autocorrelation are periodic in \(t\) with some period \(T_0\), that is 
    \begin{equation*}
        \func{\mu_X}{t + T_0} = \func{m_X}{t}
    \end{equation*}
    and 
    \begin{equation*}
        \func{R_{XX}}{t + \tau + T_0, t + T_0} = \func{R_{XX}}{t + \tau, t}
    \end{equation*}
    for all \(t\) and \(\tau\). 
\end{definition}
\subsection{Time averages}
The time average mean and autocorrelation of a random process is defined as 
\begin{align*}
    \angleBracket{\mu_X} &= \lim_{T \to \infty} \dfrac{1}{T} \int_{-T/2}^{T/2} \func{X}{t} \diffOperator t\\
    \angleBracket{\func{R_{XX}}{\tau}} &= \lim_{T \to \infty} \dfrac{1}{T} \int_{-T/2}^{T/2} \func{X}{t} \func{X}{t + \tau} \diffOperator t\\
\end{align*}
Both time averages are random variables and depend on \(\omega\).
Ensemble averages and time averages are equal in mean squared sense. A random variable \(X\) is equal to a constant \(b\) in MS sense if \(\expected{X} = b\) and \(\expected{(X - b)^2} = 0\).
\subsection{Ergodicity}
A wide-sense stationary process is \textbf{ergodic in mean} if \(\angleBracket{\mu_X}\) converges to \(\mu_X\) in mean squared as \(T \to \infty\).
A wide-sense stationary process is \textbf{ergodic in autocorrelation} if \(\angleBracket{\func{R_{XX}}{\tau}}\) converges to \(\func{R_{XX}}{\tau}\) in mean squared as \(T \to \infty\).

\subsection{Power Spectral density of stationary random process}
We can define the random variables for energy \(\scrE_X\) and power \(\scrP_X\) 
\begin{align*}
    \scrE_X &= \int_{-\infty}^{\infty} \func{X^2}{t} \diffOperator t\\
    \scrP_X &= \lim_{T \to \infty} \dfrac{1}{T} \int_{-T/2}^{T/2} \func{X^2}{t} \diffOperator t
\end{align*}
Then, the power content \(\calP_X\) and energy content \(\calE_X\) of a stochastic process \(\func{X}{t}\) are defined as 
\begin{align*}
    \calE_X &= \expected{\scrE_X} = \int_{-\infty}^{\infty} \func{R_{XX}}{t,t} \diffOperator t\\
    \calP_X &= \expected{\scrP_X} =  \lim_{T \to \infty} \dfrac{1}{T} \int_{-T/2}^{T/2} \func{R_{XX}}{t,t} \diffOperator t
\end{align*}
Foe stationary processes if \(\calE_X < \infty\), then \(\func{R_XX}{0} = 0\) and hence \(\func{X}{t}\) is zero almost everywhere.

Let \(\func{X}{t}\) be a random process and \(\func{X_T}{f}\) be the random process from considering the Fourier transforms of truncated sample functions. That is, 
\begin{equation*}
    \func{X_T}{f,\omega} = \Fourier{\func{x_T}{t,\omega}}
\end{equation*}
Then, the \textbf{power density spectrum} or \textbf{power spectral density} is defined as 
\begin{equation*}
    \func{\calG_X}{f} = \lim_{T \to \infty} \dfrac{\expected{\abs{\func{X_T}{f}}^2}}{T}
\end{equation*}
Furthermore,
\begin{equation*}
    \angleBracket{\func{\calG_X}{f}} = \lim_{T \to \infty} \dfrac{\abs{ \int_{-T/2}^{T/2} \func{X}{t} \func{\exp}{-2\pi j ft} \diffOperator t}^2}{T}
\end{equation*}
For an ergodic random process 
\begin{equation*}
    \angleBracket{\func{\calG_X}{f}} \overset{\mathrm{MS}}{=} \func{\calG_X}{f}
\end{equation*}
\begin{theorem}[Wiener-Khinchin]
    If for all finite \(\tau\) and any interval \(I\) of length \(\abs{\tau}\), the autocorrelation function \(R_{XX}\) satisfies the condition 
    \begin{equation*}
        \abs{\int_I \func{R_{XX}}{t + \tau, t} \diffOperator t} < \infty
    \end{equation*}
    Then, 
    \begin{equation*}
        \func{\calG_X}{f} = \Fourier{\lim_{T \to \infty} \dfrac{1}{T} \int_{-T/2}^{T/2} \func{R_{XX}}{t + \tau, t} \diffOperator t}
    \end{equation*}
\end{theorem}
Thus, if \(\func{X}{t}\) is stationary with 
\begin{equation*}
    \int_{-\infty}^{\infty} \abs{\func{R_{XX}}{\tau}} \diffOperator \tau < \infty
\end{equation*}
then 
\begin{equation*}
    \func{\calG_X}{f} = \int_{-\infty}^{\infty} \func{R_{XX}}{\tau} \func{\exp}{-2\pi j f \tau} \diffOperator \tau 
\end{equation*}
If \(\func{X}{t}\) is cyclostationary with 
\begin{equation*}
    \abs{\int_0^{T_0} \func{R_{XX}}{t + \tau , t} \diffOperator t} < \infty 
\end{equation*}
then 
\begin{equation*}
    \func{\calG_X}{f} = \Fourier{\func{\bar{R}_{XX}}{\tau}}
\end{equation*}
where 
\begin{equation*}
    \func{\bar{R}_{XX}}{\tau} = \dfrac{1}{T_0} \int_{-T_0/2}^{T_0/2} \func{R_{XX}}{t + \tau,t} \diffOperator t
\end{equation*}
Moreover, \(\expected{\bracket{\func{X}{t}}^2}\) can be thought of as the average power dissipated by random process across \(1\) ohm resistor.
\begin{equation*}
    \expected{\bracket{\func{X}{t}}^2} = \func{R_{XX}}{0} = \int_{-\infty}^{\infty} \func{\calG_X}{f} \diffOperator f
\end{equation*}

\begin{enumerate}
    \item \(\angleBracket{\func{X}{t}}\) is the DC component.
    \item \(\angleBracket{\bracket{\func{X}{t}}^2}\) is the total average power.
    \item \(\angleBracket{\bracket{\func{X}{t}}}^2\) is the DC power.
    \item \(\angleBracket{\bracket{\func{X}{t}}^2} - \angleBracket{\bracket{\func{X}{t}}}^2\) is the AC power.
    \item \(\sqrt{\angleBracket{\bracket{\func{X}{t}}^2} - \angleBracket{\bracket{\func{X}{t}}}^2}\) is rms value.
\end{enumerate}
\subsection{PSD of a sum process}
Suppose \(\func{Z}{t} = \func{X}{t} + \func{Y}{t}\) is the sum of two jointy stationary process. It can be readily verified that \(\func{Z}{t}\) is stationary and 
\begin{equation*}
    \func{R_{ZZ}}{\tau} = \func{R_{XX}}{\tau} + \func{R_{XY}}{\tau} + \func{R_{YX}}{\tau} + \func{R_{YY}}{\tau}
\end{equation*}
and 
\begin{equation*}
    \func{\calG_Z}{f} = \func{\calG_X}{f} + \func{\calG_Y}{f} + 2 \func{\Re}{\calG_{XY}{f}} 
\end{equation*}
where 
\begin{equation*}
    \func{\calG_{XY}}{f} = \Fourier{\func{R_{XY}}{\tau}} 
\end{equation*}
If \(\func{X}{t}\) and \(\func{Y}{t}\) are uncorrelated and at least one of them is zero mean, then \(\func{R_{XY}}{\tau} = 0\) and 
\begin{equation*}
    \func{\calG_Z}{f} = \func{\calG_X}{f} + \func{\calG_Y}{f}
\end{equation*}
\section{Systems and random signals}
The response of a system to a random signal, is a random signal itself. Therefore, we need tools to investigate the relationships between to random processes.
\begin{definition}
    Two random process \(\func{X}{t}\) and \(\func{Y}{t}\) are \textbf{independent} if for all \(t_1,t_2\), the random variables \(\func{X}{t_1}\) and \(\func{Y}{t_2}\) are independent. Similarly, \(\func{X}{t}\) and \(\func{Y}{t}\) are \textbf{uncorrelated} if for all \(t_1,t_2\), the random variables \(\func{X}{t_1}\) and \(\func{Y}{t_2}\) are uncorrelated.
\end{definition}
\begin{definition}
    The \textbf{cross-correlation} function between two random process \(\func{X}{t}\) and  \(\func{Y}{t}\) is defined as 
    \begin{equation*}
        \func{R_{XY}}{t_1,t_2} = \expected{\func{X}{t_1} \func{X}{t_2}}
    \end{equation*}
    Two processes \(\func{X}{t}\) and  \(\func{Y}{t}\) are \textbf{jointly wide-sense stationary}, if both are stationary and the cross-correlation function depends on \(\tau = t_1 - t_2\).
\end{definition}
\subsection{Response of memoryless channel}
\begin{equation*}
    \func{Y}{t}= \func{g}{\func{X}{t}}
\end{equation*}
\subsection{Response of LTI System}
Suppose stationary process \(\func{X}{t}\) is passed through a LTI system with impulse response \(\func{h}{t}\). Then, the input and output prcesses \(\func{X}{t}\) and \(\func{Y}{t}\) are jointly stationary. Moreover, 
\begin{align*}
    \func{Y}{t} &= \func{X}{t} \ast \func{h}{t}\\
    &= \int_{-\infty}^{\infty} \func{X}{\tau} \func{h}{t - \tau} \diffOperator \tau\\
    \implies  \expected{\func{Y}{t}} &= \int_{-\infty}^{\infty} \expected{\func{X}{\tau}} \func{h}{t - \tau} \diffOperator \tau\\
    \expected{\func{Y}{t}} &= \mu_X \int_{-\infty}^{\infty} \func{h}{\tau} \diffOperator \tau = \mu_X \func{H}{0}
\end{align*}
and 
\begin{align*}
    \func{R_{YX}}{\tau} &= \expected{\func{Y}{t} \func{X}{t - \tau}}\\
    & = \expected{\int_{-\infty}^{\infty} \func{X}{\eta} \func{h}{t - \eta} \func{X}{t - \tau} \diffOperator \eta }\\
    &= \int_{-\infty}^{\infty} \func{R_{XX}}{t - \tau - \eta } \func{h}{t - \eta}\diffOperator \eta \\
    &= \int_{-\infty}^{\infty} \func{R_{XX}}{ \eta - \tau } \func{h}{ \eta}\diffOperator \eta \\
    &= \int_{-\infty}^{\infty} \func{R_{XX}}{  \tau - \eta } \func{h}{ \eta}\diffOperator \eta \\
    &= \func{R_{XX}}{\tau} \ast \func{h}{\tau}\\
    \implies \func{R_{YY}}{\tau} &= \expected{\func{Y}{t + \tau} \func{Y}{t}}\\
    &=  \expected{\int_{-\infty}^{\infty} \func{Y}{t + \tau} \func{X}{t - \eta}\func{h}{\eta} \diffOperator \eta}\\
    &= \int_{-\infty}^{\infty} \func{R_{YX}}{\tau + \eta} \func{h}{\eta} \diffOperator \eta \\
    &= \func{R_{YX}}{\tau } \ast \func{h}{-\tau} \\
    &= \func{R_{XX}}{\tau} \ast \func{h}{\tau} \ast  \func{h}{-\tau}\\
    \implies & \func{\calG_X}{f} = \Fourier{\func{R_{YY}}{t}} = \func{\calG_X}{f} \abs{\func{H}{f}}^2
\end{align*}
\subsection{Special classes of random processes}
\subsubsection{Gaussian random process}
\(\func{X}{t}\) is Gaussian if 
\begin{equation*}
    \func{f_X}{x_1, \dots, x_n ; t_1, \dots, t_n} = \dfrac{1}{\sqrt{(2 \pi)^n \det \Sigma}} \func{\exp}{- \dfrac{1}{2} (\bar{x} - \bar{\mu})^T \Sigma^{-1} (\bar{x} - \bar{\mu})} \qquad \forall n, \forall t_1, \dots, t_n
\end{equation*}
where \(\Sigma = \begin{bmatrix}
    \expected{(\func{X}{t_i} - \expected{\func{X}{t_i}})(\func{X}{t_j} - \expected{\func{X}{t_j}})}
\end{bmatrix}_{i,j}\)
\(    \bar{\mu} = \begin{bmatrix}
        \expected{\func{X}{t_i}}
    \end{bmatrix}_{i}
\).

\(\func{X}{t}\) is zero mean stationary Gaussian if \(\expected{\func{X}{t}} = 0\) and \(\func{R_{XX}}{t_1 + t,t_2 + t} = \func{R_{XX}}{t_1,t_2}\) for all \(t_1,t_2,t\). Then, \(\Sigma =  \begin{bmatrix}
    \expected{\func{R_{XX}}{t_i,t_j}}
\end{bmatrix}_{i,j}\) and \(\bar{\mu} = \bar{0}\).

\begin{theorem}
    For a Gaussian process, \(\func{\mu_X}{t}\) and \(\func{R_{XX}}{t_1,t_2}\) gives a complete statistical description of the process.
\end{theorem}
\begin{corollary}
    For Gaussian processes, WSS and strictly stationarity are equivalent.
\end{corollary}
\begin{theorem}
    The output of an LTI system on a Gaussian input is Gaussian.
\end{theorem}
\begin{theorem}
    A sufficient condition for the ergodicity of the stationary zero-mean Gaussian process is 
    \begin{equation*}
        \int_{-\infty}^{\infty} \abs{\func{R_{XX}}{\tau}} \diffOperator \tau < \infty 
    \end{equation*}
\end{theorem}


\subsubsection{Markoff Sequence}
Suppose \(\func{X}{t}\) is defined for countable indices and it assumes a finte set of values. That is, \(\func{X}{t}\) is discrete-time and discrete-amplitude. \(\func{X}{t}\) is a Markoff chain if 
\begin{equation*}
    \condProb{X_n = a_n}{X_{n-1} = a_{n-1} , \dots , X_1 = a_1} = \condProb{X_{n} = a_n}{X_{n-1}  = a_{n-1}}
\end{equation*}
Let \(\func{p_i}{n} = \prob{X_n = a_i}\) and \(\func{p_{i,j}}{n,m} = \condProb{X_n = i}{X_m = j}\) for \(n > m\) then 
\begin{equation*}
    \func{p_j}{n} = \sum_{i = 1}^N \func{p_{j,i}}{n,m} \func{p_i}{m}
\end{equation*}
If \(\func{p_{i,j}}{n+1,n} = \func{p_{i,j}}{n ,n-1}\) for all \(n\), then \(\set{X_n}\) is called \textbf{homogeneous}. Finally, let 
\begin{equation*}
    \func{P}{k} = \begin{bmatrix}
        \func{p_1}{k} \\
        \vdots \\
        \func{p_N}{k}
    \end{bmatrix}
    \qquad 
    \varphi = \begin{bmatrix}
        \func{p_{1,1}}{n,n - 1} & \dots & \func{p_{1,N}}{n,n- 1}\\
        \vdots & \ddots & \vdots \\
        \func{p_{N,1}}{n,n - 1} & \dots & \func{p_{N,N}}{n,n- 1}
    \end{bmatrix}
\end{equation*}
Therefore 
\begin{equation*}
    \func{P}{k} = \varphi \func{P}{k- 1}
\end{equation*}
A Markov chain is stationary if \(\func{P}{k+1} = \func{P}{k}\) for all \(k\).

\section{Noise in communication systems}
Determined through expriments (thermodynamics and quantum mechanic) noise voltage \(\func{V}{t}\) that appears accross the terminal of a resistor of \(R\) Ohms has a Gaussian distribution with \(\mu_V= 0\) and 
\begin{equation*}
    \expected{\func{V^2}{t}} = \dfrac{(2\pi k T)^2}{3h} R 
\end{equation*}
where \(k\) is Boltzmann constant, \(h\) Plank constant, \(T\) is temperature in Kelvins. Then, 
\begin{equation*}
    \func{\calG_V}{f} = \dfrac{2 R h \abs{f}}{\func{\exp}{h \abs{f}/ (kT)} - 1} 
\end{equation*}
\(\func{\calG_V}{f}\) is flat over \(\abs{f} < 0.1 \frac{kT}{h}\). However, for modeling 
\begin{equation*}
    \func{\calG_V}{f} = 2RkT
\end{equation*}
but in this case \(\expected{\func{V^2}{t}} = \infty\). Yet it is alright since \(\func{V}{t}\) is subjected to filtering and hence \(\expected{\func{V^2}{t}}\) will be finite.

\begin{definition}
    A noise signal having flat power spectral density over a wide range frequency is called white noise.
\begin{equation*}
    \func{\calG_V}{f} = \dfrac{\eta}{2}
\end{equation*}
The factor \(1/2\) is included to indicate that \(\func{\calG_V}{f}\) is a two-sided psd.
\end{definition}

At room temperature, \(\func{\calG_V}{f}\) drops to 90\% of its maximum at about \(f \approx 2 \times 10^12\) Hertz, which is beyond the frequencies employed in the conventionaly communication systems.

\textbf{Available power} is the maximum power that can be delivered to a load from a source having a fixed but non-zero resistence. 
\begin{equation*}
        P_{\max} = \dfrac{\expected{\func{V^2}{t}}}{4R}
\end{equation*}
Available power psd is \(\func{\calG_V}{f} = kT/2\). 

We will assume the thermal noist is stationary, ergodic, zero-mean, white Gaussian noise whose power spectrum is \(N_0/2\) where \(N_0 = kT\).


\chapter{Information Theory}
\begin{remark}
    For a more complete and through treatment refer to the notes on the subject.
\end{remark}
\section{Measure of information}
Information content of a message is inversely proportional to the likelihood of that message. Let \(m_1,m_2, \dots,m_q\) be \(q\) messages with probability \(p_1,p_2, \dots , p_q\) respectively, such that \(p_1 + \dots + p_q = 1\). Then, information content of \(m_k\), \(\func{I}{m_k}\) must satisfy the followings 
\begin{enumerate}
    \item \(\func{I}{m_k} > \func{I}{m_j}\) if \(m_k < m_j\).
    \item \(\func{I}{m_k} \to 0\) as \(p_k \to 0\).
    \item \(\func{I}{m_k} \geq 0\) when \(0 \leq p_k \leq 1\).
\end{enumerate}
Furthermore, for two independent messages \(m_1\) and \(m_2\) 
\begin{equation*}
    \func{I}{m_1, m_2} = \func{I}{m_1} + \func{I}{m_2}
\end{equation*}
One continuous function that satisfies these requirements is \(\func{I}{m_k} = - \log p_k\) where the base of the logarithm determines the unit of information, e.g. base \(e\) is nats, \(2\) is bit, \(10\) is Hartley/decit.

\subsection{Average information content}
For a statistically independent source that emits \(N\) symbols from a \(M\)-symbol alphabet i.i.d. 
\begin{align*}
    I_{tot} &= -N \sum_{i = 1}^M p_i \func{\log}{p_i}\\
    H &= \dfrac{I_{tot}}{N} = -\sum_{i = 1}^M p_i \func{\log}{p_i}
\end{align*}

\begin{proposition}
    For a source with an \(M\)-symbol alphabet, the maximum entropy is attained when the symbols are equiprobabilistic and \(H_{max} = \log M\).
\end{proposition}

Suppose \(r_s\) is the symbol rate of the source, measured in \(\). Then, average information rate \(R\) is 
\begin{equation*}
    R = r_sH 
\end{equation*}

\subsection{Statistically dependent source}
\begin{itemize}
    \item emits a symbol once every \(T_s\) seconds.
    \item A stochastic process.
    \item stationary Markoff.
    \item There are \(n\) states with a transition matrix \(\varphi\), (\(M\) symbols with a residual influence lasting \(q\) symbols can be represented by \(n \leq M^q\) states).
    \item \(\condProb{X_k = s_q}{X_1, \dots , X_{k-1}} = \condProb{X_k = s_q}{S_k}\) where \(S_k\) is a discrete random variable. 
    \item At the begninning it is in one of the \(n\) states with probability \(\func{p_i}{1}\).
    \item \(\func{p_j}{k+ 1} = \sum_{i = 1}^n \func{p_i}{k} p_{ij}\) for all \(j\). \(\func{P}{k + 1} = \varphi \func{P}{k}\).
\end{itemize}
\subsection{Entropy for Markov source}
\begin{itemize}
    \item Assume, discrete finite state ergodic hence stationary.
    \item Entropy of state \(i\) is
    \begin{equation*}
        H_i = - \sum_{j = 1}^M p_{ij} \lg p_{ij}
    \end{equation*} 
    \item Entropy of source 
    \begin{equation*}
        H = \sum_{ i = 1}^n p_i H_i
    \end{equation*}
    \item therefore, \(R = r_sH\).
\end{itemize}
\begin{theorem}
    Let \(G_N = - \frac{1}{N} \sum_i \func{p}{m_i} \lg \func{p}{m_i}\) over all messages of length \(N\). Then, \(G_N\) is monotonice decreasing function of \(N\) and 
    \begin{equation*}
        \lim_{N \to \infty} G_N = H
    \end{equation*}
\end{theorem}
\section{Source encoding}
\begin{definition}
    The ratio of sourse information and the average encoded output bit rate is called \textit{coding efficiency}.
\end{definition}
\subsection{Shannon Algorithm}
Let \(m_1,\dots,m_q\) be arranged in decreasing order of probability \(p_1 \leq \dots \leq p_q\). Let \(F_i = \sum_{k = 1}^{i - 1} p_k\) with \(F_1 = 0\). Let \(n_i = \ceil{-\lg p_i}\). Then, the code 
\begin{equation*}
    c_i = (F_i)_2 \qquad \text{binary fraction of \(F_i\) up to \(n_i\) bits.}
\end{equation*}
has the following properties 
\begin{enumerate}
    \item \(\func{l}{c_i} > \func{l}{c_j} \implies p_i < p_j\).
    \item Codewords are all differnt. In fact, it is an instantaneous code.
    \item \(G_N \leq \hat{H_N} < G_N + \frac{1}{N}\).
    \item The efficiency rate is \(e = \frac{H}{\hat{H_n}}\).
\end{enumerate}
Important parameters in design od encoder/decoder 
\begin{itemize}
    \item rate efficiency
    \item complexity of design
    \item effects of error
\end{itemize}
\section{Communication channel}
Insert image of pg 39
\section{Discrete communication channel}
Consider a discrete memoryless channel. Then, the channel may be described with conditional probability \(\func{p}{y|x}\). The average information rate is \(D_in = r_s \func{H}{X} \) and the average rate of information transmission is 
\begin{equation*}
    D_t = (\func{H}{X} - \func{H}{X|Y}) r_s = r_s \func{I}{X;Y}
\end{equation*}
The capacity of the channel is defined as \(C = \max_{\func{p}{x}} D_t\).

\begin{theorem}
    Let \(C\) be the capacity and \(H\) be the entropy. If \(r_sH \leq C\), then there exists an enconding scheme such that the output of the source can be transmitted over channel with an arbitrary small probability of error. Conversely, it is not possible to transmit information at a rate exceeding \(C\) without a positive frequency.
\end{theorem}

\begin{remark}
    with memory and Gilbert
\end{remark}

\section{Continuous channels}
\begin{remark}
    additive and multiplicative noise
\end{remark}

\begin{itemize}
    \item Modulator and demodulator are techniques to reduce guassian noise effect.
    \item Impulse noise are modeled in the discrete portion.
\end{itemize}

\begin{theorem}[Shannon-Hartley theorem]
    The capacity of a channel with bandwidth \(B\) and additive guassian band-limited white noise is 
    \begin{equation*}
        C = B \func{\lg}{1 + \dfrac{S}{N}}
    \end{equation*}
    where \(S\) and \(N\) are the average signal power and noise power at the output channel. \(N = \eta B\) if two sided spectral density of the noise is \(\frac{\eta}{2}\). 
\end{theorem}

Implications 
\begin{enumerate}
    \item Gives an upperlimit that can be reached
    \item Exchange of \(S/N\) for bandwidth.
    \item Bandwidth compression.
    \item Noiseless channel has infinite capacity. For noisy channels, as bandwidth increases because the noise power increases as well, the capacity approaches a limit.
\end{enumerate}

Communication at transmitting information rate of \(B \func{\lg}{1 + S/N}\) is called \textit{ideal}.
\begin{enumerate}
    \item Most physical channels are approximately gaussian.
    \item Guassian noise provides a lowerbound performace for all other types.
\end{enumerate}
\begin{remark}
    CRT
\end{remark}
\chapter{Baseband Data Transmission}
\begin{remark}
    Most efficient: PAM,PDM,PPM.
\end{remark}

--add image of page 43

\(r_b\) is bit rate, \(T_b\) is bit duration.
\begin{equation*}
    \func{X}{t} = \sum_{k = -\infty}^{\infty} a_k \func{p_g}{t - kT_b}
\end{equation*}
where we can assume that 
\begin{equation*}
    \func{p_g}{0} = 1 \qquad \qquad a_k = \begin{cases}
        -a & d_k = 0\\
        a & d_k = 1
    \end{cases}
\end{equation*}
Then, 
\begin{equation*}
    \func{Y}{t} = \sum_{k = -\infty}^{\infty} A_k \func{p_r}{t - t_d - kT_b} + \func{n_0}{t}
\end{equation*}
where \(A_k = K_c a_k\), \(K_c\) is the normalizing constant that yields \(\func{p_r}{0} = 1\), and \(K_c\func{p_r}{t-t_d}\) is the response of the system to \(\func{p_g}{t}\).

\begin{remark}
    Equalizing filter.
\end{remark}

\(\func{Y}{t}\) is sampled at rate of \(t_m = mT_b + t_d\) and \(m_{\cardinalTH}\) is generated by comparing \(\func{Y}{t_m}\) to some threshold.
\begin{align*}
    \func{Y}{t} &= A_m + \underbrace{\sum_{k \neq m} A_k \func{p_r}{(m -k )T_b}}_{\text{Intersymbol Interference}} + \underbrace{\func{n_0}{t_m}}_{\text{channel noise}}
\end{align*}

The goals are 
\begin{itemize}
    \item minimize errors introduced by noise and ISI.
    \item maximize \(r_b\) for a given bandwidth.
    \item minimize bandwidth for a given \(r_b\).
\end{itemize}

\section{Baseband binary PAM system}
For design purposes we will assume that input data rate, overall bit error probability, characteristics of the channel are given. Channel noise is modeled by a AWGN with known spectral density \(\func{\calG_n}{f}\). Source output is assumed be equiprobable sequence of independent bits.
\begin{remark}
    PAM: specify pulse shapes, \(\func{p_g}{t}, \func{p_r}{t},\func{H_R}{f},\func{H_T}{f}\).
\end{remark}

\subsection{Baseband pulse shaping}
IN the equation for \(\func{Y}{t}\), to remove ISI we must have 
\begin{equation*}
    \func{p_r}{nT_b} = \begin{cases}
        1 & \text{for } n = 0\\
        0 & \text{for } n \neq 0
    \end{cases}
\end{equation*}
\begin{theorem}
    If \(\func{P_r}{f}\) satisfies Nyquist criterion.
    \begin{equation*}
        \sum_{k = -\infty}^{\infty} \func{P_r}{f + \frac{k}{T_b}} = T_b \qquad \qquad \text{for } \abs{f} \leq \frac{1}{2T_b}
    \end{equation*}
    then 
    \begin{equation*}
        \func{p_r}{nT_b} = \begin{cases}
            1 & \text{for } n = 0\\
            0 & \text{for } n \neq 0
        \end{cases}
    \end{equation*}
\end{theorem}

hence, ISI can be removed if and only if bandwidth of \(P_r\), is \(\abs{f} > \frac{r_b}{2}\). ??

In practical systems for \(r_b\) rate the available bandwidth is between \(\frac{r_b}{2}\) to \(r_b\) Hz. and a class of \(\func{P_r}{f}\) with a this bandwidth are \textbf{raised cosine frequence} which are commonly used. (parameter \(\beta\))
\begin{enumerate}
    \item Bandwidth \(= \func{r_b}{2} + \beta\).
    \item larger \(\beta\) implies faster decaying pules, hence less ISI due to timing errors.
    \item \(\func{P_r}{f}\) is real, non-negative and \(\int_{-\infty}^{\infty} \func{P_r}{f} \diffOperator f = 1\).
    \item \(\beta =0\) produces zero ISI at a data rate of \(r_b\).
    \item Practically impossible since time response must be zero prior to a time \(t_0 > 0\). However, a delayed version \(\func{p_r}{t-t_d}\) may be chose so that \(\func{p_r}{t-t_d} = 0\) for \(t < t_0\).
    \item One may want to use the whole bandwidth to get a faster decay of \(\func{p_r}{t}\).
\end{enumerate}
-- add image page 44.


\subsection{Optimum transmitting and receiving filter}
A design constraint 
\begin{equation*}
    \func{P_g}{f} \func{H_T}{f} \func{H_c}{f}\func{H_R}{f} = K_c \func{P_r}{f} e^{-2\pi j ft_d}
\end{equation*}
where \(P_g,H_c,\) and \(P_r\) are assumed to be known. If \(P_r\) is chosen to have zero ISI, then \(P_g\) is a delay version of \(P_r\). Therefore, we need to choose \(H_T\) and \(H_R\) such that thee effect of noise is minimized. Lets derive the probability of error. 
\begin{equation*}
    \func{Y}{t_m} = A_m + \func{n_0}{t_m}
\end{equation*}
where \(\func{n_0}{t_m} \sim \func{\NormalDist}{0,N_0}\). The threshold is assumed to be zero.
\begin{remark}
    To minimize error threshold should be set \(\frac{N_0}{2A} \func{\ln}{\frac{\prob{d_m = 0}}{\prob{d_m =1 }}}\).
\end{remark}
\begin{align*}
    P_e = \prob{\hat{d} \neq d} &= \condProb{\func{Y}{t_m} > 0}{d_m = 0} \prob{d_m = 0} + \condProb{\func{Y}{t_m} < 0}{d_m = 1} \prob{d_m = 1}\\
    &= \dfrac{1}{2} \squareBracket{\prob{\func{n_0}{t_m} < -A} + \prob{\func{n_0}{t_m} > A}}\\
    &= \dfrac{1}{2} \prob{\abs{\func{n_0}{t_m}} > A} 
\end{align*}
Since \(\func{n_0}{t_m}\) is assumed to be zero mean gaussian at the input to \(\func{H_R}{f}\), then 
\begin{equation*}
    N_0 = \int_{-\infty}^{\infty} \func{\calG_n}{f} \abs{\func{H_R}{f}}^2 \diffOperator f
\end{equation*}
Hence 
\begin{align*}
    P_e &= \dfrac{1}{2} \int_{\abs{x} > A} \dfrac{1}{\sqrt{2\pi N_0}} \func{\exp}{-\dfrac{x^2}{2N_0}} \diffOperator x\\
    &= \func{\Phi}{\dfrac{A}{\sqrt{N_0}}} = 1 - \func{Q}{\dfrac{A}{\sqrt{N_0}}}
\end{align*}
Thus to minimize \(P_e\) we should maximize \(\frac{A}{\sqrt{N_0}}\). To do this, we express \(\frac{A^2}{N_0}\) in terms of \(H_T\) and \(H_R\). Recall
\begin{equation*}
    \func{X}{t} = \int_{k = -\infty}^{\infty} a_k \func{p_g}{t - kT_b}
\end{equation*}
equiprobable and independent imples \(\func{X}{t}\) is a random waveform with psd 
\begin{align*}
    \func{\calG_X}{f} &= \dfrac{\abs{\func{P_g}{f}}^2}{T_b} \expected{a_k^2}\\
    &= a^2\dfrac{\abs{\func{P_g}{f}}^2}{T_b} 
\end{align*}
psd of the transmitted signal is 
\begin{equation*}
    \func{\calG_Z}{f} = \abs{\func{H_T}{f}}^2 \func{\calG_X}{f}
\end{equation*}
The average power 
\begin{equation*}
    \calS_T = \dfrac{a^2}{T_b} \int_{-\infty}^{\infty} \abs{\func{P_g}{f}}Y2 \abs{\func{H_T}{f}}^2 \diffOperator f
\end{equation*}
setting \(A_k = k_ca_k\) and \(A = K_ca\)
\begin{equation*}
    \calS_T = \dfrac{A^2}{K_c^2T_b} \int_{-\infty}^{\infty} \abs{\func{P_g}{f}}Y2 \abs{\func{H_T}{f}}^2 \diffOperator f
\end{equation*}
Let \(I = \int_{-\infty}^{\infty} \abs{\func{P_g}{f}}Y2 \abs{\func{H_T}{f}}^2 \diffOperator f\) and \(N_0 = = \int_{-\infty}^{\infty} \func{\calG_n}{f} \abs{\func{H_R}{f}}^2 \diffOperator f\)
\begin{equation*}
    A^2 = K_c^2 T_b S_T I^{-1}
\end{equation*}
thus 
\begin{equation*}
    \frac{A^2}{N_0} = \dfrac{K_c^2 T_b S_T}{IJ}
\end{equation*}
hence we should minimize \(\gamma^2 = IJ\)

\part{Signals and Systems}
\part{Basics}
\chapter{Introduction}
Signals are functions if independent variables that carry information. It can be continuous or discrete and be multi-dimensional.
A system responds to applies input signals, and its response is described in terms of one or more output signals. A continuous time system receives and gives continuous signals and discrete time system receives and gives discrete signals. Systems can be connected together in series, parallel, or feedback loop.

\section{Signal Properties}
\begin{definition}
    The energy of a continuous signal over interval \(\clcl{t_1}{t_2}\) is defined as 
    \begin{equation*}
        \int_{t_1}^{t_2} \abs{\func{x}{t}}^2 \diffOperator t
    \end{equation*}
    and similarly for a discrete signal over interval \(n_1 \leq n \leq n_2\)
    \begin{equation*}
        \sum_{n = n_1}^{n_2} \abs{\squareFunc{x}{n}}^2 
    \end{equation*}
    The power of signal is the time averaged energy of that signal. The total energy of signal is defined to be 
    \begin{equation*}
        E_{\infty} = \lim_{T \to \infty} \int_{-T}{T} \abs{\func{x}{t}}^2 \diffOperator t
    \end{equation*}
    for a continuous time signal and 
    \begin{equation*}
        E_{\infty} =\lim_{N \to \infty}\sum_{n = -N}^{N} \abs{\squareFunc{x}{n}}^2 
    \end{equation*}
    Lastly, the total power of a signal is 
    \begin{equation*}
        P_{\infty} = \lim_{T \to \infty} \dfrac{1}{T} \int_{-T}{T} \abs{\func{x}{t}}^2 \diffOperator t
    \end{equation*}
    and 
    \begin{equation*}
        P_{\infty} =\lim_{N \to \infty} \dfrac{1}{2N + 1} \sum_{n = -N}^{N} \abs{\squareFunc{x}{n}}^2
    \end{equation*}
\end{definition}

\section{System Properties}
\begin{description}
    \item[Memoryless] A system is memoryless if it depends only on the present input. On the other hand, a system has memory if it depends on present and past values. For example, \textit{accumulator} and \textit{delay} are two such systems.
    \begin{equation*}
        \squareFunc{y}{n} = \sum_{k = -\infty}^n \squareFunc{x}{n}, \qquad \squareFunc{y}{n} = \squareFunc{x}{n - 1}
    \end{equation*}
    \item[Invertible] A system is invertible if there exists a system \(\func{y}{t} \to \func{w}{t}\) such that \(\func{w}{t} = \func{x}{t}\), for all \(t\).
    \item[Causal] A system is casual or \textit{nonancitipative} if it depends on past and present value. Mathematicall, a system \(\func{x}{t} \to \func{y}{t}\) is causal if 
    \begin{equation*}
        \func{x_1}{t} = \func{x_2}{t},\quad \forall t \leq t_0 \implies \func{y_1}{t} = \func{y_2}{t} , \quad \forall t \leq t_0
    \end{equation*}
    \item[Stability] Informally means that small change in the input does not converge. That is, bounded input results in bounded output.
    \item[Time invariant] Time shift input results in an identical time shift in the output signal. 
    \begin{equation*}
        \func{x}{t} \to \func{y}{t} \implies \func{x}{t- t_0} \to \func{y}{t - t_0}
    \end{equation*}
    \item[Linearity] A system is linear if
    \begin{equation*}
        a \func{x_1}{t} + \func{x_2}{t} \to a \func{y_1}{t} + \func{y_2}{t}, \quad \forall a \in \Complex
    \end{equation*}
\end{description} 
\begin{proposition}
    A linear system is causal if and only if it satisfies the condition of initial rest 
    \begin{equation*}
        \func{x}{t} = 0, \quad  \forall t \leq t_0 \implies \func{y}{t} = 0 ,\quad  \forall t \leq t_0 
    \end{equation*}
\end{proposition}

\begin{proof}
    Consider a linear system such that the responses to \(\func{x_1}{t}\) and \(\func{x_2}{t}\) are \(\func{y_1}{t}\) and \(\func{y_2}{t}\), respectively. Suppos this system is causal. By linearity,\(\func{x_2}{t} = 0 \implies \func{y_2}{t} = 0\) and hence 
    \begin{equation*}
        \func{x_1}{t} = \func{x_2}{t} = 0,\quad \forall t \leq t_0 \implies \func{y_1}{t} = \func{y_2}{t} = 0 , \quad \forall t \leq t_0
    \end{equation*}
    Suppose the system has the initial rest condition. By linearity, 
    \begin{equation*}
         \func{x_1}{t} - \func{x_2}{t} \to \func{y_1}{t} - \func{y_2}{t}
    \end{equation*}
    Then, if \(\func{x_1}{t} = \func{x_2}{t} , \quad \forall t \leq t_0\) then 
    \begin{equation*}
        \func{x_1}{t} = \func{x_2}{t} \implies \func{x_1}{t} - \func{x_2}{t} = 0 \implies \func{y_1}{t} - \func{y_2}{t} = 0 \implies \func{y_1}{t} = \func{y_2}{t} , \quad \forall t \leq t_0
    \end{equation*} 
 \end{proof}

\chapter{Linear Time Invariant Signals}
\section{Discrete signals}
Let \(\squareFunc{x}{n}\) be a discrete signal then it can be written as
\begin{equation*}
    \squareFunc{x}{n} = \sum_{k = -\infty}^{\infty} \squareFunc{x}{k} \squareFunc{\delta}{n -k} = \squareFunc{x}{n} \ast \squareFunc{\delta}{n}
\end{equation*}
which is called \textit{sifting property}. Consider a discrete LTI system \(\squareFunc{x}{n} \to \squareFunc{y}{n}\), by sifting property
\begin{equation*}
    \squareFunc{y}{n} = \squareFunc{x}{n} \ast \squareFunc{h}{n}
\end{equation*}
where \(\squareFunc{h}{n}\) is the response to \(\squareFunc{\delta}{n}\). Hence a LTI system can be completely characterized by its response to \(\squareFunc{\delta}{n}\).

\section{Continuous signal}
Similarly we can write the sifting propery as 
\begin{equation*}
    \func{x}{t} = \int_{-\infty}^{\infty} \func{x}{\tau} \func{\delta}{t - \tau} \diffOperator \tau = \func{x}{t} \ast \func{\delta}{t}
\end{equation*}
and a continuous LTI system \(\func{x}{t} \to \func{y}{t}\) can be written in terms of its response \(\func{h}{t}\) to unit impluse \(\func{\delta}{t}\)
\begin{equation*}
    \func{y}{t} = \func{x}{t} \ast \func{h}{t}
\end{equation*}

\section{Properties of convolution and LTI}
For simplicity we only bring the continuous, however, the equivalent discrete form also holds.
\begin{description}
    \item [Commutative] \(\func{x}{t} \ast \func{y}{t} = \func{y}{t}\ast \func{x}{t}\).
    \item [Distributive] \(\func{x}{t} \ast \bracket{\func{y}{t} + \func{z}{t}} = \func{x}{t}\ast \func{y}{t} + \func{x}{t}\ast \func{z}{t}\).
    \item [Associative] \(\func{x}{t} \ast \bracket{\func{y}{t} \ast \func{z}{t}} = \bracket{\func{x}{t} \ast \func{y}{t}} \ast \func{z}{t}\).
\end{description}

\begin{itemize}
    \item LTI is memoryless if \(\func{h}{t} = 0\) for \(t \neq 0 \implies \func{h}{t} = K \func{\delta}{t} \implies \func{y}{t} = K \func{x}{t}\).
    \item LTI system is invertible if there exists \(\func{g}{t}\) such that \(\func{h}{t} \ast \func{g}{t} = \func{\delta}{t}\).
    \item LTI system is causal if \(\func{h}{t} = 0\) for \(t  < 0\).
    \item LTI is stable iff 
    \begin{equation*}
        \int_{-\infty}^{\infty} \abs{\func{h}{\tau}} \diffOperator \tau < \infty
    \end{equation*} 
\end{itemize}

\section{Singularity functions}
We can view the unit impluse as a short signal that has integral of \(1\) and hence signals with such property are similar. Consider the LTI system 
\begin{equation*}
    \func{y}{t} = \dfrac{\diffOperator^n \func{x}{t}}{\diffOperator t^n}
\end{equation*}
Then the unit impluse response is \(\func{u_n}{t}\) such that 
\begin{equation*}
    \func{y}{t} = \func{x}{t} \ast \func{u_n}{t} = \bracket{\func{x}{t} \ast \func{u_{n-1}}{t}} \ast \func{u_1}{t}
\end{equation*}
and therefore 
\begin{equation*}
    \func{u_n}{t} = \underbrace{\func{u_1}{t} \ast \func{u_1}{t} \ast \dots \ast \func{u_1}{t}}_n
\end{equation*}
\(\func{u_1}{t}\) is called the unit doublet and it is defined 
\begin{equation*}
   \func{u_1}{t} = \ODiff{\func{\delta}{t}}{t}
\end{equation*}
Similarly for integral 
\begin{equation*}
   \func{u_{-1}}{t} = \func{u}{t} = \int_{-\infty}^t \func{\delta}{\tau} \diffOperator \tau \implies \func{x}{t} \ast \func{u}{t} = \int_{-\infty}^t \func{x}{\tau} \diffOperator \tau
\end{equation*}
for the double integral 
\begin{equation*}
   \func{u_{-2}}{t} = \func{u_{-1}}{t}  \func{u_{-1}}{t} = \int_{-\infty}^t \func{u_{-1}}{\tau} \diffOperator \tau
\end{equation*}
similarly for higher order integrals 
\begin{equation*}
   \func{u_{-n}}{t} = \underbrace{\func{u_{-1}}{t} \ast \func{u_{-1}}{t} \ast \dots \ast \func{u_{-1}}{t}}_n
\end{equation*}
Lastly we can denote \(\func{\delta}{t} = \func{u_0}{t}\) to then arrive at 
\begin{equation*}
   \func{u_r}{t} \ast \func{u_s}{t} = \func{u_{r + s}}{t}
\end{equation*}
for all \(u,s \in \Integers\).
\part{Fourier}
\chapter{Fourier Series Representation}
\section{Eigenfunctions}
For discrete and continuous LTI systems we have 
\begin{align*}
    \func{x}{t} = e^{st} \to \func{y}{t} = \func{H}{s} e^{st}, \; s \in \Complex \\
    \squareFunc{x}{n} = z^n \to \squareFunc{y}{n} = \func{H}{x} z^n, \; z \in \Complex
\end{align*}
A signal for which the output signal of a system is a constant multiple of itself is called the \textbf{eigenfunction} of the system and the constant is called the \textbf{eigenvalue}. To show the above equations (we assume the convergence of the integral and the sum at the last steps.)
\begin{align*}
    \func{y}{t} &= \int_{-\infty}^{\infty} \func{h}{\tau} e^{s \bracket{t - \tau}} \diffOperator \tau = e^{st}  \int_{-\infty}^{\infty} \func{h}{\tau} e^{-s\tau} \diffOperator \tau = \func{H}{s} e^{st}\\
    \squareFunc{y}{n} &= \sum_{k = -\infty}^{\infty} \squareFunc{h}{k} z^{n -k} = z^n \sum_{k = -\infty}^{\infty} \squareFunc{h}{k} z^{-k} = \func{H}{z} z^n
\end{align*}

\section{Fourier series representation}
\subsection{Continuous-time}
Suppose \(\func{x}{t}\) is a continuous-time periodic signal with period \(T\). The value \(\omega_0 = \frac{2\pi}{T}\) then we may be able to write \(\func{x}{t}\) in the form of 
\begin{equation*}
    \func{x}{t} = \sum_{-\infty}^{\infty} a_k e^{jk\omega_0 t}
\end{equation*}
Let \(\func{x}{t}\) be a real signal then \(\overline{x} = x\) hence 
\begin{align*}
    \func{x}{t} &= \sum_{-\infty}^{\infty} \overline{a_k} e^{-jk\omega_0 t}\\
    &=  \sum_{-\infty}^{\infty} \overline{a_{-k}} e^{-jk\omega_0 t} \implies a_k = \overline{a_{-k}}\\
    &= a_0 +\sum_{k = 1}^{\infty} a_k e^{jk\omega_0 t} + \overline{a_{-k}} e^{-jk\omega_0 t} \\
    &= a_0 + 2 \sum_{k = 1}^{\infty}  \func{\Re}{a_k e^{jk\omega_0 t}}, \ a_k = r_k e^{j\theta_k} , a_k = b_k + jc_k\\
    &= a_0 + 2 \sum_{k = 1}^{\infty}  r_k \func{\cos}{\theta_k + k \omega_0 t}\\
    &=  a_0 + 2 \sum_{k = 1}^{\infty}  b_k \func{\cos}{k \omega_0 t} - c_k \func{\sin}{k \omega_0 t}\\
\end{align*}
The last two equation are the \textbf{Fourier series} representation of a real periodic signal \(\func{x}{t}\) \_ this is called synthesis in the context of Fourier series. We also have 
\begin{align*}
    \int_{0}^T \func{x}{t} e^{-jn\omega_0 t} \diffOperator t &= \int_{0}^T \sum_{-\infty}^{\infty} a_k e^{j\bracket{n-k }\omega_0 t}\diffOperator t\\
    &=\sum_{-\infty}^{\infty} a_k \int_{0}^T  e^{j\bracket{n-k }\omega_0 t}\diffOperator t\\
    &= a_n T
\end{align*} 
therefore 
\begin{align*}
    a_n &= \dfrac{\omega_0}{2\pi}  \int_{0}^T \func{x}{t} e^{-jn\omega_0 t} \diffOperator t = \dfrac{\omega_0}{2\pi} \int_{c}^{c +T} \func{x}{t} e^{-jn\omega_0 t} \diffOperator t\\
    a_0 &= \dfrac{\omega_0}{2\pi}  \int_{0}^T \func{x}{t} \diffOperator t 
\end{align*}
we donote \(\int_{c}^{c + T}\) by \(\int_T\).

\subsection{Convergence}
A finite enery real signal \(\func{x}{t}\)
\begin{equation*}
    \int_{T} \abs{\func{x}{t}}^2 \diffOperator t < \infty
\end{equation*}
has a Fourier series representation if it satisfies the \textit{Dirichlet conditions}
\begin{enumerate}
    \item Over any period \(\func{x}{t}\) is absolutely intergrable which implies \(\abs{a_k} < \infty\).
    \item In any finite interval, \(\func{x}{t}\) is of bounded variation \_ finite number of maximas and minimas.
    \item Finite number of discontinuities in any finite interval, and the discontinuities can not be infinite.
\end{enumerate}

\subsection{Discrete-time}
Let \(\squareFunc{x}{n}\) be a discrete signal with period \(N\) and fundamental frequence \(w_0 = \frac{2\pi}{N}\) and let  
\begin{equation*}
    \func{\phi_k}{n} = e^{jk\omega_0 n} \quad k \in \Integers
\end{equation*}
then since 
\begin{equation*}
    \func{\phi_k}{n} = \func{\phi_{rN + k}}{n}
\end{equation*}
we can write \(\squareFunc{x}{n}\) in form of 
\begin{equation*}
    \func{x}{n} = \sum_{k = \angleBracket{N}} a_k \func{\phi_k}{n}
\end{equation*}
where \(\angleBracket{N}\) is a set of \(N\) consecutive integers. Finding \(a_k\) requires solving a linear system 
\begin{equation*}
    a_k = \dfrac{1}{N} \sum_{k = \angleBracket{N}} \squareFunc{x}{n} e^{jk \omega_0 n}
\end{equation*}

\section{Properties of Fourier series}
Suppose \(\func{x}{t}\) is a periodic signal with period \(T\) and \(\omega_0 = \frac{2 \pi}{T}\) then 
\begin{equation*}
    \func{x}{t} \xleftrightarrow{FS} a_k
\end{equation*}
where \(a_k\) is its Fourier series coefficients. Let \(\func{x}{t} \xleftrightarrow{FS} a_k\) and  \(\func{y}{t} \xleftrightarrow{FS} b_k\), then we have the following properties 
\begin{definition}
    \item[Linearity]
    \begin{equation*}
        \alpha \func{x}{t} + \beta \func{y}{t} \xleftrightarrow{FS} \alpha a_k + \beta b_k
    \end{equation*}
    \item[Time shifting]
    \begin{equation*}
        \func{x}{t - t_0}  \xleftrightarrow{FS} e^{-jk\omega_0 t_0} a_k
    \end{equation*}
    \item[Time reversal]
    \begin{equation*}
        \alpha \func{x}{-t} \xleftrightarrow{FS}  a_{-k}
    \end{equation*}
    \item[Time scaling]
    \begin{equation*}
        \alpha \func{x}{ct} \xleftrightarrow{FS} \sum_{k = -\infty}^{\infty} a_k e^{jk \bracket{c\omega_0} t}
    \end{equation*}
    It does not change the coefficients, but changes the whole thing.
    \item[Multiplication]
    \begin{equation*}
        \func{x}{t}\func{y}{t} \xleftrightarrow{FS} c_k = \sum_{l = -\infty}^{\infty}  a_{l} +  b_{k-l}
    \end{equation*} 
    \item[Conjugate symmetry]
    \begin{equation*}
        \overline{ \func{x}{t}} \xleftrightarrow{FS}  \overline{a_{-k}}
    \end{equation*}
    \item[Parseval's relation]
    \begin{equation*}
        \int_T \abs{\func{x}{t}}^2 \diffOperator t = \sum_{k = -\infty}^{\infty} \abs{a_k}^2
    \end{equation*}
    \item[Period convolution]
    \begin{equation*}
        \int_T \func{x}{t}\func{y}{t - \tau} \diffOperator \tau \xleftrightarrow{FS} T\alpha a_k \beta b_k
    \end{equation*}
    \item[Frequency shifting]
    \begin{equation*}
        e^{jM\omega_0 t}\func{x}{t} \xleftrightarrow{FS}  a_{k - M}
    \end{equation*}
    \item[Differentiation]
    \begin{equation*}
        \ODiff{\func{x}{t}}{t}  \xleftrightarrow{FS} jk\omega_0\alpha a_k
    \end{equation*}
    and for discrete-time signals 
    \begin{equation*}
        \squareFunc{x}{n} - \squareFunc{x}{ n - 1} \xleftrightarrow{FS} \bracket{1 - e^{-jk\omega_0}} a_K
    \end{equation*}
    \item[Integeration]          
    \begin{equation*}
        \int_{\infty}^t \func{x}{\tau} \diffOperator \tau \xleftrightarrow{FS} \dfrac{a_k}{jk \omega_0 }
    \end{equation*}
\end{definition}

\section{LTI systems}
Consider a periodic signal \(\func{x}{t}\) with 
\begin{equation*}
    \func{x}{t} = \sum_{k = -\infty}^{\infty} a_k e^{jk \omega_0 t}
\end{equation*}
then for a LTI system \(\func{x}{t} \to \func{y}{t}\) with \(\func{\delta}{t} \to \func{h}{t}\)
\begin{align*}
    \func{y}{t} &= \sum_{k = -\infty}^{\infty} a_k \bracket{e^{jk\omega_0 t} \ast \func{h}{t}}\\
    &= \sum_{k = -\infty}^{\infty} a_k \bracket{\int_{-\infty}^{\infty} e^{jk\omega_0 \bracket{t - \tau}}  \func{h}{\tau} \diffOperator \tau}\\
    &=  \sum_{k = -\infty}^{\infty} a_k e^{jk\omega_0 t}\bracket{\int_{-\infty}^{\infty} e^{-jk\omega_0\tau}  \func{h}{\tau} \diffOperator \tau}\\
    &= \sum_{k = -\infty}^{\infty} a_k \func{H}{jk\omega_0}e^{jk\omega_0 t}
\end{align*}
For a discrete time system \(\squareFunc{\delta}{n} \to \squareFunc{h}{n}\)
\begin{align*}
    \func{y}{n} &= \sum_{k = \angleBracket{N} } a_K \bracket{e^{jk\omega_0 n} \ast \func{h}{n}}\\
    &=  \sum_{k = \angleBracket{N} } a_K \bracket{\bracket{e^{jk\omega_0}}^n \ast \func{h}{n}}\\
    &=  \sum_{k = \angleBracket{N} } a_K \bracket{\sum_{m = -\infty}^{\infty} e^{jk\omega_0 \bracket{n-m}}\func{h}{m}}\\
    &=  \sum_{k = \angleBracket{N} } a_K e^{jk \omega_0 n}\bracket{\sum_{m = -\infty}^{\infty} e^{-jk\omega_0 m}\func{h}{m}}\\
    &=  \sum_{k = \angleBracket{N} } a_K \func{H}{e^{jk \omega_0}} e^{jk \omega_0 n}
\end{align*}

\section{Filtering}
LTI systems the change frequency spectrum are called frequency-shaping filters. Systems that pass, eliminate, or attunate some frequencies are called frequency selective filters. Three types of frequency selective filters include 
\subsection{Lowpass filter}
Passes low frequencies around zero and attunates or eliminates, in the ideal case, high frequencies.
\begin{equation*}
    \abs{H}{j\omega} = \begin{cases}
        1 & \abs{\omega} \leq \omega_c\\
        0 & \text{otherwise}
    \end{cases}
\end{equation*}
--insert diagram 

\subsection{Bandpass filter}
Passes frequencies around certian frequncy and attunates or eliminates, in the ideal case, other frequencies.
\begin{equation*}
    \abs{H}{j\omega} = \begin{cases}
        1 & \omega_{c_1} \leq \abs{\omega} \leq \omega_{c_2}\\
        0 & \text{otherwise}
    \end{cases}
\end{equation*}
--insert diagram 
\subsection{High pass filter}
Passes high frequencies around zero and attunates or eliminates, in the ideal case, low frequencies.
\begin{equation*}
    \abs{H}{j\omega} = \begin{cases}
        1 & \abs{\omega} \geq \omega_c\\
        0 & \text{otherwise}
    \end{cases}
\end{equation*}
--insert diagram 

\begin{example}
    Take a simple RC circuit with input \(\func{V_s}{t}\). We have 
    \begin{equation*}
        \begin{cases}
            &\func{V_R}{t} + \func{V_C}{t} = \func{V_s}{t}\\
            & \func{V_R}{t} = R \func{i}{t} = RC \ODiff{\func{V_C}{t}}{t}
        \end{cases}
    \end{equation*}
Since it is an LTI system, for input \(\func{V_s}{t} = e^{j\omega t}\) and output \(\func{V_C}{t} = \func{H}{j\omega}\) we have 
\begin{align*}
    RC \func{H}{j\omega}j \omega e^{j \omega t} + \func{H}{j \omega} e^{j\omega}&= e^{j \omega} \\
    \implies \func{H}{j \omega} = \dfrac{1}{RC j \omega + 1}
\end{align*}
--insert diagram which is non-ideal lowpass filter. For output \(\func{V_R}{t} = \func{H}{j\omega}\) we have 
\begin{align*}
    RC \func{H}{j\omega}j \omega e^{j \omega t} + \func{H}{j \omega} e^{j\omega}&= RC j \omega e^{j \omega} \\
    \implies \func{H}{j \omega} = \dfrac{j\omega }{RC j \omega + 1}
\end{align*}
--insert diagram which is non-ideal highpass filter.
\end{example}
\begin{example}[First order recursive discrete-time filter]
    Consider 
    \begin{equation*}
        \squareFunc{y}{n} - a \squareFunc{y}{n-1}= \squareFunc{x}{n}
    \end{equation*}
    with input \(\squareFunc{x}{n} = e^{j\omega n}\) the output is of form \(\squareFunc{y}{n} = \func{H}{e^{j\omega}} e^{j \omega n}\) then 
    \begin{align*}
        \func{H}{e^{j\omega}} e^{j \omega n} + \func{H}{e^{j\omega}} e^{j \omega (n - 1)} &= e^{j \omega n}\\
        \implies \func{H}{e^{j\omega}} \bracket{1 + e^{-j \omega}} &= 1\\
        \implies \func{H}{e^{j\omega}} &= \dfrac{1}{1 + e^{-j\omega}}
    \end{align*}
    --insert diagram which is non-ideal lowpass filter for \(0 < a < 1\) and a non-ideal highpass filter for \(-1 < a < 0\).
\end{example}

\begin{example}[Non-recursive discrete-time filter]
    Consider the following FIR
    \begin{equation*}
        \squareFunc{y}{n} = \sum_{k = -N}^{M} b_k \squareFunc{x}{n - k}, \quad b_k \ \text{are weights}
    \end{equation*}
    let \(b_k = \frac{1}{M + N  + 1} \) then 
    \begin{align*}
        \func{H}{e^{j\omega}} &= \dfrac{1}{N + M + 1} \sum_{k = -N}^M e^{j\omega (n-k)}\\
        &= \dfrac{1}{N + M + 1} \dfrac{e^{j\omega (N + 1)} - e^{-j\omega M}}{e^{j\omega } - 1}
    \end{align*}
    which is a lowpass filter that approaches the ideal as \(N + M + 1 \to \infty\).
\end{example}
\chapter{Continuous Fourier Transform}
\section{Fourier transform for a periodic function}
Suppose \(\func{x}{t}\) with \(\func{x}{t} = 0\) for \(\abs{t} \geq T_1\) is a finite duration aperiodic signal. Make a periodic signal \(\func{\tilde{x}}{t}\) a from \(\func{x}{t}\) with period \(T \geq T_1\). Then \(\tilde{x}\) has a Fourier series 
\begin{align*}
    \func{\tilde{x}}{t} &= \sum_k a_k e^{jk\omega_0 t} \qquad \omega_0 = \dfrac{2\pi}{T}\\
    a_k &= \dfrac{1}{T} \int_T \func{\tilde{x}}{t} e^{-jk\omega_0 t} \diffOperator t\\
    &= \dfrac{1}{T} \int_{-T/2}^{T/2} \func{x}{t} e^{-jk\omega_0 t} \diffOperator t \\
    &= \dfrac{1}{T} \int_{-\infty}^{\infty} \func{x}{t} e^{-jk\omega_0} \diffOperator t \\
    &= \dfrac{1}{T} \func{X}{jk\omega_0}
\end{align*}
since for \(\abs{t} \leq T/2, \func{x}{t} = \func{\tilde{x}}{t}\) and for \(\abs{t} > T \geq T_1\), \(\func{x}{t} = 0\). \(\func{X}{j\omega}\) is the \textit{Fourier integral} or \textit{Fourier transform} of \(x\). 
\begin{equation*}
    \func{X}{j\omega} = \int_{-\infty}^{\infty} \func{x}{t} e^{-j\omega t} \diffOperator t
\end{equation*}
Therefore, 
\begin{align*}
    \func{\tilde{x}}{t} &= \sum_k \dfrac{1}{T} \func{X}{jk\omega_0} e^{jk\omega_0 t}\\
    &= \dfrac{1}{2\pi} \sum_k \func{X}{jk\omega_0} e^{jk\omega_0 t} \omega_0
\end{align*}
as \(T \to \infty\), \(\tilde{x} \to x\) and \(\omega_0 \to 0\) hence 
\begin{equation*}
    \func{x}{t} = \dfrac{1}{2\pi} \int_{-\infty}^{\infty} \func{X}{j\omega} e^{j\omega t} \diffOperator \omega
\end{equation*}
For this to converge we need the Dirichlet condition, that is \(\func{x}{t}\) must be 
\begin{enumerate}
    \item absolutely integrable.
    \item bounded variation.
    \item finite number of finite discontinuities.
\end{enumerate}

For periodic signals the Fourier transform is a collection of impulse signals occuring at harmonic
\begin{align*}
    \func{x}{t} &= \dfrac{1}{2\pi} \int_{-\infty}^{\infty} \func{X}{j\omega} e^{j\omega t}\diffOperator \omega
    \intertext{Let \(\func{X}{j\omega} = 2\pi \sum_{k = -\infty}^{\infty} a_k \func{\delta}{\omega - k\omega_0}\)}
    &= \dfrac{1}{2\pi} \int_{-\infty}^{\infty} \bracket{2\pi \sum_{k = -\infty}^{\infty} a_k \func{\delta}{\omega - k\omega_0}} e^{j\omega t}\diffOperator \omega\\
    &= \sum_{k = -\infty}^{\infty} a_k \int_{-\infty}^{\infty} \func{\delta}{\omega - k\omega_0} e^{j\omega t}\diffOperator \omega\\
    &= \sum_{k = -\infty}^{\infty} a_k e^{jk \omega_0 t}
\end{align*}
or equivalently 
\begin{align*}
    \func{X}{j\omega} &= \int_{-\infty}^{\infty} \func{x}{t} e^{-j\omega t} \diffOperator t \\
    &= \sum_{k = -\infty}^{\infty} a_k\int_{-\infty}^{\infty} e^{jk \omega_0 t} e^{-j\omega t} \diffOperator t \qquad \text{(distribution integra)}\\
    &= \sum_{k = -\infty}^{\infty} a_k \func{\delta}{\omega - k \omega_0}
\end{align*}
\section{Properties}
\begin{description}
    \item[Linearity]
    \begin{equation*}
        \Fourier{a \func{x}{t} + b \func{y}{t}} = a \Fourier{\func{x}{t}} + b \Fourier{\func{y}{t}}
    \end{equation*}
    \item[Time shifting]
    \begin{equation*}
        \Fourier{\func{x}{t - t_0}} = e^{-j\omega t_0} \Fourier{\func{x}{t}}
    \end{equation*}
    \item[Conjugate symmetries]
    \begin{equation*}
        \Fourier{\overline{\func{x}{t}}} = \overline{\Fourier{\func{x}{-t}}}    
    \end{equation*}
    Therefore, for a real signal \(x\) 
    \begin{align*}
        & \Fourier{\func{x}{-t}} = \overline{\Fourier{\func{x}{t}}}\\
        \implies & \Fourier{\func{Od}{\func{x}{t}}} = \dfrac{\Fourier{\func{x}{t}} - \overline{\Fourier{\func{x}{t}}}}{2} = i \Im \Fourier{\func{x}{t}}\\
        \implies &  \Fourier{\func{Ev}{\func{x}{t}}} = \dfrac{\Fourier{\func{x}{t}} + \overline{\Fourier{\func{x}{t}}}}{2} = \Re \Fourier{\func{x}{t}}\\
    \end{align*}
    \item[Derivatives]
    \begin{equation*}
        \Fourier{\dfrac{\diffOperator^n}{\diffOperator t^n} \func{x}{t}} = \bracket{j\omega}^n \Fourier{\func{x}{t}}
    \end{equation*}
    \item[Convolution]
    \begin{equation*}
        \Fourier{\func{x}{t} \ast \func{y}{t}} = \Fourier{\func{x}{t}} \Fourier{\func{x}{t}}
    \end{equation*}
    \item[Time scaling]
    \begin{equation*}
        \Fourier{\func{x}{at}} =\dfrac{1}{\abs{a}} \func{\Fourier{\func{x}{t}}}{\dfrac{\omega}{a}}
    \end{equation*}
    \item[Integral]
    \begin{equation*}
        \Fourier{\int_{\infty}^t \func{x}{\tau} \diffOperator \tau} = \dfrac{1}{j\omega} \Fourier{\func{x}{t}} + \pi \func{\Fourier{\func{x}{t}}}{0} \func{\delta}{\omega} 
    \end{equation*}
    \item[Inverse of derivative]
    \begin{equation*}
        \InvFourier{\dfrac{\diffOperator}{\diffOperator \omega} \Fourier{\func{x}{t}}} =-jt \func{x}{t}
    \end{equation*}
    \item[Inverse of frequency shifting]
    \begin{equation*}
        \InvFourier{\func{\Fourier{\func{x}{t}}}{\omega - \omega_0}} = e^{j\omega_0 t} \func{x}{t}
    \end{equation*}
    \item[Multiplication]
    \begin{equation*}
        \Fourier{\func{x}{t} \func{y}{t}} = \Fourier{\func{x}{t}} \ast \Fourier{\func{y}{t}}
    \end{equation*}
    \item[Parseval]
    \begin{equation*}
        \int_{\infty}^{\infty} \abs{\func{x}{t}}^2 \diffOperator t = \dfrac{1}{2\pi} \int_{-\infty}^{\infty} \abs{\Fourier{\func{x}{t}}}^2 \diffOperator \omega 
    \end{equation*}
\end{description}
Fourier transform of some functions 
% \begin{center}
%     \renewcommand{\arraystretch}{1.9}
%     \begin{tabular}{>{$}c<{$} | >{$}c<{$}}
%         \func{x}{t} & \Fourier{\func{x}{t}} \\ \hline 
%         1 & 2\pi \func{\delta}{\omega}\\\hline
%         e^{j\omega_0 t} & 2\pi \func{\delta}{\omega - \omega_0}\\\hline
%         \func{\cos}{\omega_0 t} & \pi \bracket{ \func{\delta}{ \omega - \omega_0} + \func{\delta}{\omega + \omega_0}}\\\hline
%         \func{\sin}{\omega_0 t} & \dfrac{\pi}{j} \bracket{ \func{\delta}{ \omega - \omega_0} - \func{\delta}{\omega + \omega_0}}\\\hline
%         \func{\delta}{t} & 1 \\\hline
%         \dfrac{t^{n-1}}{(n-1)!} e^{-at} \func{u}{t} & \dfrac{1}{(a + j\omega)^n} \quad \Re a > 0
%     \end{tabular}
% \end{center}
\section{Applications}
We can make frequency-selective filtering with variable center frequency (bandpass around \(\omega_c\). Solve ODEs in the following format 
\begin{equation*}
    \sum_{k = 0}^K a_k \dfrac{\diffOperator^k }{\diffOperator t^k} \func{y}{t} = \sum_{m = 0}^M a_m \dfrac{\diffOperator^m }{\diffOperator t^m} \func{x}{t} 
\end{equation*}
since it is an LTI system then \(\func{y}{t} = \func{x}{t } \ast \func{h}{t}\) and
\begin{equation*}
    \func{Y}{j\omega} = \Fourier{\func{y}{t}} = \Fourier{\func{x}{t} \ast \func{h}{t}}= \func{H}{j\omega} \func{X}{j\omega}
\end{equation*}
therefore 
\begin{equation*}
    \func{H}{j\omega} = \dfrac{\sum_{m = 0}^M b_k (j\omega)^m}{\sum_{k=0}^K a_k (j\omega)^k}
\end{equation*}

\chapter{Discrete Fourier Transform}
Let \(\squareFunc{x}{n}\) be a finite duration signal and \(\squareFunc{\tilde{x}}{n}\) be the periodic form of \(\squareFunc{x}{n}\), Then, as \(N \to \infty\) \(\squareFunc{\tilde{x}}{n} \to \squareFunc{x}{n}\). 
\begin{align*}
    \squareFunc{\tilde{x}}{n} &= \sum_{\angleBracket{N}} a_k e^{j\omega_0 kn} \qquad \omega_0 = \dfrac{2\pi}{n} \\
    \implies a_k &= \dfrac{1}{N} \sum_{\angleBracket{N}} \squareFunc{\tilde{x}}{n} e^{-j\omega_0 kn}\\
    &= \dfrac{1}{N} \sum_{n =-N_1}^{N_2} \squareFunc{x}{n} e^{-j\omega_0 kn}\\
    &= \dfrac{1}{N} \sum_{n =-\infty}^{\infty} \squareFunc{x}{n} e^{-j\omega_0 kn} \\
    &= \dfrac{1}{N} \func{X}{e^{j\omega_0 k}} 
\end{align*}
which implies that 
\begin{align*}
    \squareFunc{\tilde{x}}{n} &= \sum_{\angleBracket{N}} \dfrac{1}{N} \func{X}{e^{j\omega_0 k}}  e^{j\omega_0 kn}\\
    \implies \squareFunc{x}{n} &= \lim_{\omega_0 \to \infty} \dfrac{1}{2\pi} \sum_{0}^{\frac{2\pi}{n}} \func{X}{e^{j\omega_0 k}}  e^{j\omega_0 kn} \omega_0\\
    &= \dfrac{1}{2\pi} \int_{2\pi} \func{X}{e^{j\omega}} e^{j\omega n} \diffOperator \omega
\end{align*}
For convergence we need the followings to hold 
\begin{enumerate}
    \item \(\squareFunc{x}{n}\) is absolutely summable 
    \begin{equation*}
        \sum_{n =-\infty}^{\infty} \abs{\squareFunc{x}{n}} < \infty
    \end{equation*}
    or has finite energy 
    \begin{equation*}
        \sum_{n =-\infty}^{\infty} \abs{\squareFunc{x}{n}}^2 < \infty
    \end{equation*}
\end{enumerate}
\section{DFT of periodic signal}
Let \(\squareFunc{x}{n} = e^{j\omega_0 n}\) then 
\begin{align*}
    \func{X}{e^{j\omega}} &= \sum_{n = -\infty}^{\infty} e^{j \omega_0 n} e^{-j \omega n} = 2 \pi \sum_{l = -\infty}^{\infty} \func{\delta}{\omega - \omega_0 - 2\pi l}\\
    \implies \squareFunc{x}{n} &= \sum_{\angleBracket{N}} a_k e^{jk \omega_0 n}\\
    \implies \func{X}{e^{j\omega}} &= 2 \pi \sum_{k = -\infty}^{\infty} a_k \func{\delta}{\omega - k\omega_0}
\end{align*}

\section{Properties}
\begin{description}
    \item[Periodic] It has a period of \(2\pi\).
    \begin{equation*}
        \func{\Fourier{\squareFunc{x}{n}}}{\omega + 2\pi} =\func{\Fourier{\squareFunc{x}{n}}}{\omega}
    \end{equation*} 
    \item[Linear]
    \begin{equation*}
        \Fourier{a\squareFunc{x}{n} + b \squareFunc{y}{n}} = a\Fourier{\squareFunc{x}{n}} + b\Fourier{\squareFunc{y}{n}}
    \end{equation*} 
    \item[Time shifting] \(n_0\) must be an integer.
    \begin{equation*}
        \Fourier{\squareFunc{x}{n - n_0}} = e^{-j\omega n_0} \Fourier{\squareFunc{x}{n}}
    \end{equation*}
    \item[Frequency shifting]
    \begin{equation*}
        \Fourier{e^{j\omega_0 n}\squareFunc{x}{n}} = \func{\Fourier{\squareFunc{x}{n}}}{\omega - \omega_0}
    \end{equation*}
    \item[Conjugate symmetry] 
    \begin{equation*}
        \Fourier{\overline{\squareFunc{x}{n}}}  = \overline{\func{\Fourier{\squareFunc{x}{n}}}{-\omega}}
    \end{equation*}
    and if \(\squareFunc{x}{n}\) is real
    \begin{align*}
        \implies \Fourier{\squareFunc{x}{n}} &= \overline{\func{\Fourier{\squareFunc{x}{n}}}{-\omega}}\\
        \implies \Fourier{\func{Ev}{\squareFunc{x}{n}}} &= \Re \Fourier{\squareFunc{x}{n}}\\
        \implies \Fourier{\func{Od}{\squareFunc{x}{n}}} &= j \Im \Fourier{\squareFunc{x}{n}}
    \end{align*}
    \item[Differencing]
    \begin{equation*}
        \Fourier{\squareFunc{x}{n} - \squareFunc{x}{n - 1}} = \bracket{1 - e^{-j\omega}} \Fourier{\squareFunc{x}{n}}
    \end{equation*}
    \item[Accumulation]
    \begin{equation*}
        \Fourier{\sum_{m = -\infty}^n \squareFunc{x}{m}} = \dfrac{1}{ 1 - e^{-j\omega}} \Fourier{\squareFunc{x}{n}} + \pi \func{\Fourier{\squareFunc{x}{n}}}{0} \sum_{k = -\infty}^{\infty} \func{\delta}{\omega - 2\pi k}
    \end{equation*} 
    \item[Time Rerversal]
    \begin{equation*}
        \Fourier{\squareFunc{x}{-n}} = \func{\Fourier{\squareFunc{x}{n}}}{-\omega}
    \end{equation*}
    \item[Time Expansion]
    \begin{equation*}
        \Fourier{\squareFunc{x_{(k)}}{n}} = \func{\Fourier{\squareFunc{x}{n}}}{k\omega}
    \end{equation*}
    where 
    \begin{equation*}
        \squareFunc{x_{(k)}}{n} = \begin{cases}
            \squareFunc{x}{\frac{n}{k}} & k| n\\
            0 & \text{otherwise}
        \end{cases}
    \end{equation*}
    \item 
    \begin{equation*}
        \InvFourier{j \dfrac{\diffOperator \Fourier{\squareFunc{x}{n}}}{\diffOperator \omega}} = n \squareFunc{x}{n}
    \end{equation*}
    \item[Parsevals']
    \begin{equation*}
        \sum_{n = -\infty}^{\infty} \abs{\squareFunc{x}{n}}^2 = \dfrac{1}{2\pi} \int_{2\pi} \abs{\Fourier{\squareFunc{x}{n}}}^2 \diffOperator \omega 
    \end{equation*}
    \item[Convolution] 
    \begin{equation*}
        \Fourier{\squareFunc{x}{n} \ast \squareFunc{y}{n}} = \Fourier{\squareFunc{x}{n}} \Fourier{\squareFunc{y}{n}}
    \end{equation*}
    \item[Multiplication] 
    \begin{equation*}
        \Fourier{\squareFunc{x}{n} \squareFunc{y}{n}}  = \dfrac{1}{2\pi} \int_{2\pi} \func{\Fourier{\squareFunc{x}{n}}}{\theta} \func{\Fourier{\squareFunc{y}{n}}}{\omega - \theta } \diffOperator \theta
    \end{equation*}
\end{description}

\section{Linear constant coefficient difference equation}
\begin{equation*}
    \sum_{k = 0}^N a_k \squareFunc{y}{n - k} = \sum_{m = 0}^M b_m \squareFunc{x}{n -m}
\end{equation*}
it is the impluse response has the following form 
\begin{equation*}
    \func{H}{e^{j\omega}} = \dfrac{\sum_{m = 0}^M b_m e^{-j\omega m }}{ \sum_{k = 0}^N a_k e^{j\omega k}}
\end{equation*}

Some fourier analysis would not be bad :)
\chapter{Time Frequency Characterization}
\section{Magnitude-phase representation of Fourier Transform}
\(\angle \func{X}{j\omega}\) is the relative phase and \(\frac{1}{2\pi}\abs{\func{X}{j\omega}}^2 \diffOperator \omega \) is the energy of \(\func{x}{t}\) in \(\opop{\omega}{ \omega + \diffOperator \omega}\). Obviously, the magnitude matters but also phase matters depending on the context. 
\section{Magnitude-phase representation of the frequency response of LTI systems}
In an LTI system 
\begin{equation*}
    \func{Y}{j\omega} = \func{H}{j\omega} \func{X}{j\omega} \implies \begin{cases}
        \abs{Y} &= \abs{H} \abs{X} \\
        \angle Y &= \angle H + \angle X
    \end{cases}
\end{equation*}
where the \(\abs{H}\) is called the gain and \(\angle H\) is the phase shift. 
\subsection*{Linear and non-linear phase}
For unit gain all-pass \(\func{H}{j\omega} = 1\) with linear phase \(\angle \func{H}{j\omega} = t_0 \omega\). Then, 
\begin{equation*}
    \func{y}{t} = \func{x}{t - t_0}
\end{equation*}
\subsubsection*{Group delay}
Suppose \(\func{X}{j\omega}\) is zero outside a narrow band around \(\omega = \omega_0\). Then, a non-linear phase can be approximated by a linear phase 
\begin{align*}
    \angle \func{H}{j\omega} &\simeq - \phi - \alpha \\
    \implies \func{Y}{j\omega} &= \func{X}{j\omega} \abs{\func{H}{j\omega}} e^{j\phi} e^{j \omega \alpha}
\end{align*}
\(\alpha\) seconds delay. The group delay is defined as 
\begin{equation*}
    \func{\tau}{\omega} = - \dfrac{\diffOperator}{\diffOperator \omega} \angle \func{H}{j\omega}
\end{equation*}
but this might be discontinuous at \(2\pi k\). So we use the un-wrapped phase 
\begin{equation*}
    \func{\tau}{\omega} = - \dfrac{\diffOperator}{\diffOperator \omega} \squareBracket{\angle \func{H}{j\omega}}
\end{equation*}

\subsection*{Log-magnitude and Bode plots}
\(dB = 20 \log_{10} \abs{\func{H}{j\omega}}\). If \(\omega\)-axis is logarithmic \(\log_{10} \omega\) as well then it is called Bode plot.

\section{Time-Domain properties of ideal frequency selective filter}
a frequency selective filter 
\begin{equation*}
    \abs{\func{H}{j \omega}} = \begin{cases}
        1 & \abs{\omega} \leq \omega_c \\
        0 & \abs{\omega} > \omega_c
    \end{cases}
\end{equation*}
With linear phase (\(\alpha\) seconds delay)
\begin{equation*}
    \angle \func{H}{j\omega} = -\alpha \omega 
\end{equation*}

\section{Time-domain and frequency aspects of non-ideal filters}
add images in pg 175

\section{First and second order CT system}
\subsection*{First order continuous-time system}

\begin{equation*}
    \tau \dfrac{\diffOperator y}{\diffOperator t} + u = \func{x}{t} \implies \func{H}{j\omega} = \dfrac{1}{j \omega \tau + 1}
\end{equation*}
which then implies that 
\begin{align*}
    \func{h}{t} &= \dfrac{1}{\tau} e^{-\frac{t}{\tau}} \func{u}{t}\\
    \func{s}{t} &= \bracket{ 1 - e^{-\frac{t}{\tau}}} \func{u}{t}
\end{align*}

\subsection*{Second order continuous-time system}
\begin{equation*}
    \dfrac{\diffOperator^2 y}{\diffOperator t^2} + 2 \xi \omega_n  \dfrac{\diffOperator y}{\diffOperator t} + \omega_n^2 \func{y}{t} = \omega_n^2 \func{x}{t}
\end{equation*}
then 
\begin{align*}
    \func{H}{j\omega} &= \dfrac{\omega_n^2}{\bracket{j\omega}^2 + 2 \xi \omega_n j \omega + \omega_n^2}\\
    &= \dfrac{\omega_n^2}{\bracket{j\omega - c_1} \bracket{j\omega - c_2}}
\end{align*}
where \(c_1 = - \xi \omega_n + \omega_n \sqrt{\xi^2 - 1}\) and \(c_2 = - \xi \omega_n + \omega_n \sqrt{\xi^2 - 1}\). If \(\xi \neq 1\) then 
\begin{align*}
    \func{H}{j\omega} &= \dfrac{M}{j\omega - c_1} - \dfrac{M}{j\omega - c_2} \qquad \qquad M = \dfrac{\omega_n}{2 \sqrt{\xi^2 - 1}}\\
    \implies \func{h}{t} &= M \bracket{e^{c_1 t} - e^{c_2 t}}\func{u}{t}
\end{align*}
and if \(\xi = 1\) then 
\begin{equation*}
    \func{H}{j\omega} = \dfrac{\omega_n^2}{\bracket{j\omega + \omega_n}^2} \implies \func{h}{t} = \omega_n^2 t e^{-\omega_n t} \func{u}{t}
\end{equation*}
\(\xi\) is called the damping ratio and \(\omega_n\) is the undamped natural frequency. 
\begin{itemize}
    \item For \(0 < \xi < 1\) the response is underdamped which will overshoot and rings in step function. 
    \item For \(\xi = 1\) the response is critically damped which will have the fastest settling time. 
    \item For \(\xi > 1\) the response is overdamped which will imply it has slow settling time.
\end{itemize}

\section{First order and second order discrete-time system}
\section{First order}
\begin{align*}
    \squareFunc{y}{n} - a \squareFunc{y}{n - 1} = \squareFunc{x}{n} \implies \func{H}{e^{j\omega}} &= \dfrac{1}{1 - a e^{j\omega}} \\
    \squareFunc{h}{n} &= a^n \func{u}{n} \\
    \squareFunc{s}{n} &= \dfrac{1 - a^{n+1}}{1 - a} \func{u}{n}
\end{align*}
\section{Second order}
\begin{equation*}
    \squareFunc{y}{n} - 2r \cos \theta \squareFunc{y}{n - 1} + r^2 \squareFunc{y}{n - 2} = \squareFunc{x}{n}
\end{equation*}
\begin{align*}
    \func{H}{e^{j\omega}} &= \dfrac{1}{1 - 2r \cos \theta e^{-j\omega} + r^2 e^{-2j\omega}} \\
    &= \dfrac{1}{\bracket{1 - re^{j\theta} e^{-j\omega}}\bracket{1 - re^{=j\theta} e^{-j\omega}}}
\end{align*}

If \(\theta \neq 0,\pi\) then 
\begin{equation*}
    \func{H}{e^{j\omega}} = \dfrac{A}{1 -re^{j\theta} e^{-j\omega}} + \dfrac{B}{1 -re^{-j\theta} e^{-j\omega}}
\end{equation*}
where
\begin{equation*}
    A = \dfrac{e^{j\theta}}{2j \sin \theta} \qquad B = \dfrac{e^{-j\theta}}{2j \sin\theta}
\end{equation*}
if \(\theta = 0\)
\begin{equation*}
    \func{H}{e^{j\omega}} = \dfrac{1}{\bracket{1 - re^{-j\omega}}^2} \implies \squareFunc{h}{n} = (n + 1)r^n \squareFunc{u}{n} 
\end{equation*}
and for \(\theta = \pi\)
\begin{equation*}
    \func{H}{e^{j\omega}} = \dfrac{1}{\bracket{1 + re^{-j\omega}}^2} \implies \squareFunc{h}{n} = (n + 1)(-r)^n \squareFunc{u}{n} 
\end{equation*}
\chapter{Sampling}
\section{Sampling theorem}
\subsection*{Sampling with impluse train}
Let \(T\) be the sampling period and \(\omega_s = \frac{2\pi}{T}\) be the sampling frequency. The train impluse 
\begin{equation*}
    \func{p}{t} = \sum_{n = -\infty}^{\infty} \func{\delta}{t - nT}
\end{equation*}
and hence the sampled signal
\begin{equation*}
    \func{x_p}{t} = \func{x}{t} \func{p}{t} = \sum_{n = -\infty}^{\infty} \func{x}{nT} \func{\delta}{t - nT}
\end{equation*}
has the Fourier transform 
\begin{equation*}
    \func{X_p}{j\omega} = \dfrac{1}{2\pi} \int_{-\infty}^{\infty} \func{X}{j\theta} \func{P}{j\omega - j \theta} \diffOperator \theta = \dfrac{1}{2\pi} \func{X}{j\omega} \ast \func{P}{j\omega}
\end{equation*}
with 
\begin{equation*}
    \func{P}{j\omega} = \dfrac{2\pi}{T} \sum_{k = -\infty}^{\infty} \func{\delta}{\omega - k \omega_s}
\end{equation*}
and hence 
\begin{align*}
    \func{X_p}{j\omega} &= \dfrac{\omega_s}{2\pi} \func{X}{j\omega} \ast \sum_{k = -\infty}^{\infty} \func{\delta}{\omega - k \omega_s}\\
    &= \dfrac{\omega_s}{2\pi}   \sum_{k = -\infty}^{\infty} \func{X}{j\omega - jk\omega_s}
\end{align*}

\begin{theorem}
    Let \(\func{x}{t}\) be a band-limited signal with \(\func{X}{j\omega} = 0\) for \(\abs{\omega} > \omega_M\). Then, \(\func{x}{t}\) is uniquely determined by its samples \(\func{x}{nT}\) if \(\omega_s > 2 \omega_M\), i.e. we can reconstructed \(\func{x}{t}\) from its samples.
\end{theorem}

\begin{proof}
    Construct \(\func{X_p}{j\omega}\) as above. Note that is has a period of \(\omega_s\). Therefore, by the fact that \(\func{X}{j\omega}\) is band-limited, if \(\omega_M < \omega_s - \omega_M\) or equivalently \(\omega_s > 2 \omega_M\) then we can reconstruct \(\func{X}{j\omega}\) by applying a lowpass filter with \(\omega_M < \omega_c < \omega_s - \omega_M\).
\end{proof}

\begin{definition}
    \(2\omega_M\) is called the Nyquist rate.
\end{definition}

\subsection*{Sampling with zero-order hold}
it is the same as impulse train by we hold the last sample until the new sample. -- add images pg 178
where 
\begin{equation*}
    \func{H_0}{j\omega} = e^{-j\omega T/2} \bracket{\dfrac{2 \sin \frac{\omega T}{2}}{\omega}}
\end{equation*}
and therfore to have \(\func{x}{t} = \func{r}{t}\) we must have 
\begin{equation*}
    \func{H_r}{j\omega} = \dfrac{\func{H}{j\omega}}{\func{H_0}{j\omega}}
\end{equation*}
where \(\func{H}{j\omega}\) is the ideal lowpass needed to convert impulse train to original signal.

\section{Reconstruction of signal from its samples using interpolation}
first-order hold is the linear interpolation. Has the following filter -- inset image pg 178

\section{Effect of undersampling (Aliasing)}
Do the example with a \(\cos \omega_0t\) and conclude that undersampling turns high frequencies into low frequencies. Stroboscopic effect 

\section{Discrete-time processing of continuous time signals}
First we must transform continuous to discrete (C/D conversion , Analog to digital), then process the discrete signal and lastly convert it back to continuous time (D/C conversion, Digital to Analog).
\subsection*{C/D conversion}
it can be effectively by impulse train.
\begin{equation*}
    \func{x_p}{t} = \begin{cases}
        \func{x_c}{nT} & t = nT\\
        0 & \text{otherwise}
    \end{cases} \qquad \qquad \squareFunc{x_d}{n} = \func{x_p}{nT}
\end{equation*}
then 
\begin{align*}
    \func{X_d}{e^{j\Omega}} &= \sum_{k = -\infty}^{\infty} \squareFunc{x_d}{k} e^{-j\Omega k} \\ 
    &= \sum_{k = -\infty}^{\infty} \func{x_p}{nT} e^{-j\Omega k} \\
    &= \sum_{k = -\infty}^{\infty} \func{x_c}{nT} e^{-j\Omega k} \\
    &= \func{X_p}{j\dfrac{\Omega}{T}} \\
    \implies \func{X_d}{e^{j\Omega}} &= \dfrac{\omega_s}{2\pi}   \sum_{k = -\infty}^{\infty} \func{X_c}{j\frac{\Omega}{T} - jk\omega_s}
\end{align*}

\subsection*{D/C Conversion}
we just reverse back. we turn discrete time into impulse train and then apply a lowpass filter to get the continuous signal.

\subsection*{The system}
give a sufficiently band limited input and sampling theorem conditions hold is LTI.
\begin{equation*}
    \func{H_c}{j\omega} = \begin{cases}
        \func{H_d}{e^{j\omega T}} & \abs{\omega} < \dfrac{\omega_s}{2}\\
        0 &\text{otherwise}
    \end{cases}
\end{equation*}

\subsection*{Half-sample delay}
suppose we want to do 
\begin{equation*}
    \func{y_c}{t} = \func{x_c}{t - \Delta}
\end{equation*}
Then 
\begin{equation*}
    \func{Y_c}{j\omega} = e^{-j \omega \Delta} \func{X_c}{j\omega}
\end{equation*}
implying that 
\begin{equation*}
    \func{H_c}{j\omega} = \begin{cases}
        e^{-j\omega \Delta} & \abs{\omega} < \omega_c \\ 
        0 & \text{otherwise}
    \end{cases}
\end{equation*}
and 
\begin{equation*}
    \func{H_d}{e^{j\Omega}} = e^{-j\Omega \frac{\Delta}{T}}
\end{equation*}
given that \(\frac{\Delta}{T}\) is an integer 
\begin{equation*}
    \squareFunc{y_d}{n} = \squareFunc{x_d}{n - \dfrac{\Delta}{T}}
\end{equation*}

\section{Sampling of discrete-time signal}
\subsection*{Implust train signal}
\begin{equation*}
    \squareFunc{x_p}{n} = \begin{cases}
        \func{x}{n} & N | n \\
        0 & \text{otherwise}
    \end{cases}
\end{equation*}
where \(N\) is the sampling period. THen, 
\begin{align*}
    \squareFunc{x_p}{n} &= \squareFunc{x}{n} \squareFunc{p}{n}\\
    &= \sum_{k = -\infty}^{\infty} \squareFunc{x}{kN} \squareFunc{\delta}{n - kN}\\ 
    \func{P}{e^{j\omega}} &= \dfrac{2\pi}{N} \sum_{k = -\infty}^{\infty} \func{\delta}{\omega - k\omega_s}\\
    \implies \func{X_p}{e^{j\omega}} &= \dfrac{1}{2\pi} \int_{2\pi} \func{P}{e^{j\theta}} \func{X}{e^{j(\omega - \theta)}} \diffOperator \theta \\
    &= \dfrac{1}{N} \int_{0}^{2\pi} \func{X}{e^{j(\omega - \theta)}} \sum_{k = -\infty}^{\infty}\func{\delta}{\theta - k\omega_s} \diffOperator \theta\\
    &= \dfrac{1}{N} \sum_{k = 0}^{N-1} \int_{0}^{2\pi} \func{X}{e^{j(\omega - \theta)}} \func{\delta}{\theta - k\omega_s} \diffOperator \theta\\
    &= \dfrac{1}{N} \sum_{k = 0}^{N-1} \func{X}{e^{j(\omega - k\omega_s)}}
\end{align*}
again no aliasing if \(\omega_s > 2\omega_M\).
\subsection*{Discrete-time decimation}
it is unnecessary to keep all the zeros. 
\begin{equation*}
    \squareFunc{x_b}{n} = \squareFunc{x_p}{nN} = \squareFunc{x}{nN}
\end{equation*}
therefore 
\begin{align*}
    \func{X_b}{e^{j\omega}} &= \sum_{k = -\infty}^{\infty} \squareFunc{x_b}{k} e^{-j\omega k}\\
    &= \sum_{k = -\infty}^{\infty} \squareFunc{x_p}{kN} e^{-j\omega k}\\
    &= \sum_{k = -\infty}^{\infty} \squareFunc{x_p}{k} e^{-j\omega/N k}\\
    &= \func{X_p}{e^{j \omega/N}}
\end{align*}
\subsection{upsampling or interpolation}
adding zeros \(\squareFunc{x_b}{n} \to \squareFunc{x_p}{n} \to \squareFunc{x}{n}\) and then apply low-pass filter.
\end{document}